\section[Технократизм. Искусственный интеллект]{Технологический
детерминизм и технократизм. Искусственный интеллект и его влияние на развитие общества}

\subsection{Технократизм и технологический детерминизм}

XX век --- это время развития Большой науки. Научно-техническая революция и Большая наука повлияли не только на облик самой науки, но и изменили облик общества в целом.  
С 60-х годов XX века в развитых странах начинает формироваться технократизм.

\textbf{Технократизм} --- это общество, в котором власть принадлежит научно-техническим специалистам. Социальная структура технократического общества базируется на отношениях уровня образования и квалификации. 
(Тогда как в традиционном обществе структура основана на отношениях происхождения, а в капиталистическом обществе --- на отношениях собственности.)

% Это короткая формула, короткое определение технократизма, но если мы его будем раскрывать, то оно станет более глубоким. Давайте тоже зафиксируем. 

В таком обществе инвестиции в собственное образование всегда имеют прямые положительные последствия для человека.
Например, в Японии и Южной Корее общества обладают явными чертами технократизма, т.е. от образования зависит будущее людей.

% Ярким примером технократических обществ можно назвать, например, Японию и Южную Корею, где от степени престижности твоей школы зависит то, в какой университет ты поступишь, скорее всего. И от престижности университета зависит твоя работа, где ты будешь работать.

Чертами технократизма обладало и общество Соединенных Штатов до 90-х годов, до кризиса системы высшего образования, когда нарушилась прямая связь между уровнем образования и уровнем последующих доходов. 

Основоположником концепции технократизма является Ф.У. Тейлор. Это американский инженер, который в начале XX века перешел к новой организации труда на своих производствах, а впоследствии выдвинул идею, что общество должно быть основано на отношениях квалификации. Согласно Тейлору, власть должна принадлежать инженерам, потому от них зависит благополучие всего общества. 

% Ну, кроме того, с Тейлором связан еще так называемый Тейлоризм --- способ организации производства, когда труд всех задействованных в производстве максимально интенсифицируется. 

Для технократизма характерен \textit{принцип упреждающего проектирования}, согласно которому жизнь общества организуется таким образом, чтобы ограничивать действия людей и предотвращать незаконную деятельность (например, закрытые до оплаты турникеты в метро). Дальнейшее введение ИИ в различные сферы жизни будет и далее рационально ограничивать действия людей.

% Допустим, общественный транспорт. Вы зашли в троллейбус, к вам подошел кондуктор, вы оплатили проезд. Но можно зайти в троллейбус, спрятаться от кондуктора, доехать до своей остановки, можно с ним начать спорить, скандалить. в итоге можно в принципе проехать на троллейбусе, не оплатив проезд. 
% Но когда мы заходим в метро, без того, чтобы мы оплатили проезд через турникет, мы не пройдем. Если раньше у нас стояли турникеты, которые были открыты, и если кто-то проходит без оплаты, они закрываются, то сейчас они наоборот постоянно закрыты и открываются вас, пропускают, когда вы оплатили. 
% Вот это есть упреждающее проектирование. Мы сделали техническую систему таким образом, чтобы она не позволяла вам совершить противоправное действие. Проехать без оплаты. 

% Так вот, технократическое общество, оно стремится любые действия человека ограничить именно таким образом, чтобы он принимал только такие решения, которые являются законными с точки зрения текущего законодательства. И даже если он задумал сделать что-то противоправное, он не мог это сделать. 

% Пока мы только в некоторых сферах можем фиксировать реализацию подобного принципа, но введение искусственного интеллекта в различные сферы экономики, социальной жизни будет действия людей ограничивать. будет манипуляция мнением через контекстную рекламу, через индивидуальную цену товара и так далее, и так далее. 

\textbf{Технологический детерминизм} --- теоретико-методологическая установка в философских и социологических концепциях, исходящая из решающей роли техники и технологии в развитии социально-экономических структур.

% При сравнении разных обществ мы будем главным фактором здесь иметь в виду фактор развития техники и технологий. чем выше уровень развития техники, тем выше уровень развития общества.

Эту концепцию предложил американский социолог науки Торстейн Веблен(<<Инженеры и ценовая система>>, 1921). По Веблену, власть должна принадлежать научно-техническим специалистам в области политики вообще.

Политику можно технологизировать, что успешно происходит в современности. С помощью политической технологии политики влияют на общество, в частности на принятие решений в рамках избирательной кампании. При высокой квалификации политтехнологов личные качества самих политиков утрачивают решающую роль.

% Сейчас у нас существует понятие политической технологии. И соответственно с помощью политической технологии влиять на общество, на принятие решений в рамках избирательной кампании и так далее. политики привлекательной для избирателей это уже технология. Это не про талант человека, это не про его личностные качества, это про то, как его политтехнологи представили избирателям. 
% по сути, мы можем как в промышленности взять заготовку, ее обработать и передать заказчику. Так и здесь политическая технология берет по сути любого человека и делает из него успешного политика. Все зависит от квалификации политтехнологов и количества ресурсов, которые в это готовы вложить. 

Для технократизма и технологического детерминизма количественным измерением уровня развития техники является уровень производства энергии. С появлением атомной энергетики стал важен учет технологии, с помощью которой энергия производится. 

(Впрочем, например, в 2000-х годах считалось, что доля использования возобновляемых источников энергии главным образом определяет уровень развития общества. Однако в последние годы <<зеленая>> энергетика показала свою несостоятельность.)

% , например, чем выше процент в производстве энергии , солнечная энергетика, ветровая энергетика, там еще геотермальная и так далее, тем выше уровень развития общества. То есть сжигание углеводородов, атомная энергетика, это все прошлый век, вот возобновляемая, например, солнечная, это шаг вперед. Сейчас зеленая повестка в энергетике испытывает определенный кризис, потому что есть принципиальные ограничения по генерации ветровой энергии, солнечной энергии, все это не так экологично, как это подавалось ранее. Там про изводство солнечных батарей и панелей, это высокотоксичное производство, ветрогенераторы производят большое количество шума и возле них жить уже невозможно. На их смазку тратится огромное количество продуктов, тем самым загрязняется вокруг них территория нефтепродуктами, ну и так далее. А геотермальные приливные станции, они в принципе недоступны для большинства территорий, только для некоторых. Но так или иначе, еще раз повторю, для технологического детерминизма и технократизма оценка уровня развития технологий количественная зависит от уровня производства энергии и структуры в производстве этой энергии. 

\subsection{Искусственный интеллект}

% Двигаемся дальше. Поговорим о проблеме искусственного интеллекта и его влияние на развитие общества. Сразу скажу, что эта проблема настолько широка и глубока, что в одном под вопросе лекции мы ее, конечно же, не решим. Здесь я намечу только некоторые моменты, которые на экзамене, если этот вопрос вам попадется, можно по вашему усмотрению дальше раскрывать. 

Зарождение дискуссии об искусственном интеллекте связано с именем английского математика Алана Тьюринга, который в 1950 году опубликовал статью <<Может ли машина мыслить?>>.

В статье обсуждалось, можно ли заставить ЭВМ думать подобно человеку. В качестве проверки был предложен ныне знаменитый тест Тьюринга, который сравнивает способности предположительно разумной машины со способностями человека. 

По сути, если проверяющий человек не может понять, общается ли он с человеком или с ЭВМ, то машина успешно проходит тест. Тест Тьюринга в таком классическом виде довольно быстро был подвергнут критике, в дальнейшем были предложены более сложные варианты (например, <<Китайская комната>>). 

Тьюринг приходит к выводу, что искусственный интеллект невозможен, аргументируя это следующими возражениями: 
\begin{itemize} 
    \item \textit{Возражение леди Лавлейс.} (Ада Лавлейс --- математик XIX века и одна из первых программистов). Компьютеры могут выполнять действия лишь в рамках своей программы, т.е. неспособны выйти за ее пределы. С современной точки зрения данное возражение неактуально (например, существуют алгоритмы машинного обучения без учителя).
    \item Аргумент естественного поведения. Невозможно создать исчерпывающий набор правил поведения для искусственного интеллекта, который бы соответствовал бесконечному разнообразию внешней среды. Можно лишь отметить, что и сам человек действует шаблонно, стереотипно, не всегда учитывая каждое изменение.

Современные ИИ относятся к т.н. \textit{слабому}, или прикладному, \textit{искусственному интеллекту}, который может решать ограниченное число специальных задач. Вопрос о возможности существования \textit{сильного ИИ}, который полностью бы имитировал мыслительную деятельность человека, либо в принципе обладал своеобразной мыслительной деятельностью, по сей день остается \textit{открытым}.
\end{itemize}

% делать лишь то, что содержится в их программе. Соответственно, за пределы программы машина выйти не может. однако это возражение было актуально до того момента, пока не появились самообучающиеся алгоритмы искусственного интеллекта.
% Сейчас мы вполне можем говорить о том, что существуют варианты искусственного интеллекта самообучающиеся и направления их самообучения не могут даже сами программисты спрогнозировать. 
% Кроме того, существуют алгоритмы искусственного интеллекта, основанные на принципе черного ящика, когда мы знаем входной сигнал и мы фиксируем выходной сигнал. Но что происходит внутри этого черного ящика с входящим сигналом мы не знаем. 

\subsubsection{Влияние ИИ на развитие общества}

% Каким же образом искусственный интеллект может повлиять на развитие общества? этот момент связан с тем, что повсеместное внедрение искусственного интеллекта формирует новый технологический уклад. Например, повсеместное введение паровой машины в промышленное производство сформировало новый технологический уклад и запустило
% индустриализацию. Повсеместное использование ЭВМ до четвертого поколения включительно это персональный компьютер четвертое поколение привело к цифровизации и формированию нового технологического уклада и привело к разговорам об информационном обществе. 

Повсеместное внедрение искусственного интеллекта приведет к смене технологического уклада. Это приводит к коренному изменению занятости человека в производстве и вообще в экономике. 

Большое количество профессий окажутся невостребованными, при этом возникнет необходимость в новых профессиях. При таком переходном периоде люди будут массову терять работу.

% Огромное количество массовых профессий оказыв ненужными и появляются новые профессии, которые еще только предстоит массово освоить. и вот в этом временном промежутке между моментом, когда какие-то профессии оказались бесполезны и новые профессии еще не стали массовыми, огромное количество людей теряют работу. 

% профессии что это будут за профессии мы можем спрогнозировать. Например, беспилотный транспорт вытеснит массово водителей профессиональных, водители останутся как категория, но это будет уже продуктом элитного потребления. то есть вы хотите, чтобы вас вел искусственный интеллект или живой водитель. Это будет ценовая категория уже более высокая. Будут массово вытеснены низовые категории юристов и экономистов. Бухгалтерский учет будет отдан на откуп искусственному интеллекту. Элементарные юридические вопросы также будут решаться системой поддержки принятия решений без юристов, консультантов. Ну и так далее, вплоть до профессии педагога. Сейчас есть попытки внедрения искусственного интеллекта в сферу образования. Даже несколько ГОСТов принято по этому поводу. 

% Ну, как бы посмотрим, как это будет выглядеть. Естественно, подобные глобальные социальные потрясения беспокоят. Особенно беспокоят людей, которых это косметично. Если искусственный интеллект вытеснит меня с работы. Естественно, в массовое внедрение этой технологии необходимо очень строго контролировать. 

В силу этого введение ИИ необходимо тщательно планировать. Однако это приводит к конфликту интересов в рамках бизнеса, где введение ИИ может существенно снизить расходы и повысить темпы производства. 

% здесь возникает конфликт интересов. Бизнес нацелен на получение прибыли. Даже если мы почитаем гражданский кодекс, любая организация коммерческая, ее основная задача это получение прибыли в пользу выгодоприобретателей. Владельцев компании, акционеров компании и так далее. Все, что ограничивает получение прибыли, является препятствием для развития компании. Соответственно, любые какие-то контролирующие мероприятия по поводу введения искусственного интеллекта будут восприниматься негативно. Это будет снижать прибыль этих компаний. А речь идет о довольно серьезных прибылях. То есть конфликт интересов между интересами получения прибыли и социальной ответственностью. которую в основном несет государство, но коммерческие компании периодически частично заявляют о своей социальной ответственности. 

ИИ является серьезным фактором в конкуренции между странами, поскольку страны, успевшие развить инфраструктуру для искусственного интеллекта, окажутся на новой ступени технологического развития, еще больше увеличив отрыв от развивающихся стран.

% так вот, мы сейчас находимся в ситуации, когда некоторые страны в случае, если они успеют построить у себя , в технологическом развитии навсегда обгонят те страны, которые это сделать не успеют. То есть, по сути, страны, которые успели и не успели, будут находиться всегда в дальнейшем в разных технологических укладах, более высоком и более низком. 

% в глобальном смысле это, конечно, далеко не хорошо, потому что у нас сейчас уже есть ситуация, когда есть развитые страны, есть страны догоняющие, например, страны Африки, Латинской Америки. если повсеместно будет введен искусственный интеллект в развитых странах, они еще дальше в своем развитии уйдут, и этот разрыв будет непреодолим. Так, примерно так. 

\subsubsection{Этапы исследования в области искусственного интеллекта}. 
\begin{itemize}
    \item Первый этап (50-60-е гг. XX века) --- время становления исследовательских программ, формирование круга задач, первые попытки исследований в этой сфере.
    \item Второй этап (60-90-е гг. XX века) --- формирование классической научной дисциплины <<Искусственный интеллект>>.
    \item Третий этап (2000-е годы) --- активное коммерческое использование достижений в сфере ИИ.
\end{itemize}

\section[Становление кибернетики, ее предмет и функции]{Становление кибернетики и различные варианты трактовки ее предмета и функций}

Термин кибернетика встречается в диалогах Платона, где оно означает искусство управления кораблем и его командой. 

В научный оборот данный термин был введен в 1830 году Андре Мари Ампером, французским физиком и естествоиспытателем. Он опубликовал многотомный труд под названием <<Опыт о философии наук>>, где классифицировал все известные научные дисциплины. Согласно Амперу, \textbf{кибернетика} --- это наука об управлении \textit{государством}, которая должна обеспечить гражданам этого государства разнообразные блага. 

В дальнейшем на протяжении XIX века под кибернетикой стал пониматься любой управленческий процесс. Например, согласно Б. Трентовскому, \textbf{кибернетика} --- это наука об управлении \textit{человеческими группами}. Основной целью управления является человек во всей его сложности.  

\begin{quote}
Люди не математические символы и не логические категории, и процесс управления — это не шахматная партия. Недостаточное знание целей и стремлений людей может опрокинуть любое логическое построение. Людьми очень трудно командовать и предписывать им наперед заданные действия. Приказ, если кибернет вынужден его отдавать, всегда должен очень четко формулироваться. Исполняющему всегда должен быть понятен смысл приказа, его цели, результат, который будет достигнут, и кара, которая может последовать за его невыполнением, — последнее обязательно \hfill (Бронислав Трентовский)
\end{quote}

В 1948 году вышла книга Н. Винера «Кибернетика или управление связью в животном и машине». Винер считал кибернетику частью теории информации. По Винеру, \textbf{кибернетика} --- это наука об \textit{общих закономерностях} процессов управления и передачи информации в машинах живых организмах и обществе.  

Наблюдая за развитием определения кибернетики, можно заметить постепенное расширение ее предмета --- от экипажа корабля у Платона до любой системы у Винера.

% Намечая некую линию развития, понимание того, что такое кибернетика, можно сказать, что ее предмет постоянно расширяется. 
% У Платона кибернетика – это управление экипажем корабля, либо воинским каким-то подразделением. 
% У Ампера – это управление государством. 
% У Трентовского – это управление любыми коллективами людей, неважно какими. Связано это с военным делом, государством не имеет значения. 
% У Виннера кибернетика – это наука об управлении в любых возможных системах. То есть мы видим постоянное расширение предмета кибернетики. Согласно Виннеру, кибернетика сама является только частью теории информации. 

\section{Информация как важнейшее понятие науки XX
века}

Понимание информации как ознакомление другого человека с каким-то знанием, сохранялось до середины XX века. В связи с прогрессом технических средств массовых коммуникаций и с появлением электронных вычислительных машин отношение к понятию информации в науке изменилось. 

Прежде всего, начинаются попытки количественного измерения информации. В 1948 году вышла статья «Математическая теория связи» Клода Шеннона и Уоррена Уивера, двух американских математиков, в которой использовались вероятностные методы измерения количества информации. 

Также была предложена \textbf{математическая схема связи}, согласно которой любая схема связи состоит из шести элементов:
\begin{itemize}
    \item источник информации,
    \item передатчик,
    \item линия связи,
    \item приемник,
    \item адресат,
    \item источник помех.
\end{itemize}
Например, в коммуникации лектора и студентов это:
\begin{itemize}
    \item память и мышление лектора,
    \item голосовой аппарат лектора,
    \item электронная линия связи,
    \item слуховой аппарат студентов,
    \item сознание студентов,
    \item шум, потеря данных, нарушения речи и т.д.,
\end{itemize}
соответственно
 
С точки зрения этого подхода, \textbf{информация} --- это сигнал, который уменьшает степень неопределенности у получателя. Мера уменьшения степени неопределенности может быть только вероятностно в силу неодинаковой реакции разных адресатов на одну и ту же информацию. Например, продвинутый ученик, читающий вне занятий учебник, и обычный ученик испытывают разное снижение неопределенности от информации, предоставляемой учителем на уроке. 

Распространение математической теории связи в частности и на социальные процессы привело к информационному повороту в науках. 
В 60-х годах XX века понятие информация и информационный подход как основа методологии получают общий научный статус. Так, коммуникацию вне зависимости от природы системы можно исследовать с точки зрения теории связи и с точки зрения теории информации. 

% Уже упомянутый Норберт Винер также в своей книге о кибернетике размышлял о том, что такое информация и выдвинул идею о том, что информация это некая новая субстанция наряду с материей и энергией. Если системы могут обмениваться материей, энергией и информацией значит информация это некая субстанция, которая обладает собственным бытием. 
% Однако в этой книге 48-го года Винер не смог дать определение что такое информация. информация он буквально написал следующее информация это не энергия и не материя, информация --- это информация. Но это нам как бы ничего не дает. Однако в более поздних работах он определение все-таки дал.

Н. Винер рассуждал о том, что информация является новой субстанцией наряду с материей и энергией, которая обладает собственным бытием. По Винеру, \textbf{информация} --- это обозначение содержания, полученное нами из внешнего мира в процессе приспосабливания к нему нас и наших чувств.

% Другими словами, получая из внешней среды материю и энергию мы вместе с ними получаем и информацию. Но воспринять ее мы можем только в процессе приспособления нашего сознания и наших чувств к этому внешнему миру. 
% мы можем получать огромное количество сигналов из внешнего мира и мы их получаем но мы не воспринимаем их как информацию. 

\textbf{Свойства информации} для получателя или потребителя:
\begin{itemize}
    \item \textit{Полнота}.
    \item \textit{Актуальность} --- соответствие нуждам потребителя в данный момент времени.
    \item \textit{Достоверность} --- свойство информации не иметь скрытых ошибок (при этом достоверная информация со временем может стать недостоверной).
    \item \textit{Доступность} --- свойство информации, характеризующая возможность ее получения данным потребителем.
    \item \textit{Релевантность} --- соответствие информации запросу потребителя.
    \item \textit{Защищенность} --- степень невозможности несанкционированного использования или изменения информации.
    \item \textit{Эргономичность} ---  удобство формы или объема информации с точки зрения потребителя.
\end{itemize}

\section{Формирование синергетики и ее основных понятий}

\textbf{Синергетика} (от др.-греч. --- совместная деятельность) --- междисциплинарное направление научных исследований, задачей которого является изучение природных явлений и процессов на основе принципов самоорганизации систем. 

Синергетика изначально заявлялась как междисциплинарный подход, так как принципы, управляющие процессами самоорганизации, представляются одними и теми же (безотносительно природы систем), и для их описания должен быть пригоден общий математический аппарат. 

С мировоззренческой точки зрения синергетику иногда позиционируют как <<глобальный эволюционизм>> или <<универсальную теорию эволюции>>, дающую единую основу для описания механизмов возникновения любых новаций подобно тому, как некогда кибернетика определялась, как <<универсальная теория управления>>, одинаково пригодная для описания любых операций регулирования и оптимизации: в природе, в технике, в обществе.

Первое использование данного термина связано с докладом профессора Штудгартского университета Германа Хакена «Кооперативные явления в сильно неравновесных и нефизических системах» в 1973 году. Западногерманское издательство Springer в 1975 году заказывает Хакену книгу. Уже в 1977 году монография под названием <<Синергетика>> выходит на немецком и английском языках. В 1978 году книга была переиздана, а вскоре вышла на японском и русском языках. Издательство Springer открывает серию <<Синергетика>>, в которой выходят все новые и новые труды разных авторов.

\subsection{Научные школы (течения) в синергетике}

В синергетике к настоящему времени сложилось уже несколько научных школ. Эти школы окрашены в те тона, которые привносят их сторонники, идущие к осмыслению идей синергетики с позиции своей исходной дисциплинарной области, будь то математика, физика, биология или даже обществознание.

В числе этих школ --- брюссельская школа лауреата Нобелевской премии И.Р. Пригожина, разрабатывающего теорию диссипативных структур. Это открытая система, которая оперирует вдали от термодинамического равновесия. Иными словами, это устойчивое состояние, возникающее в неравновесной среде при условии диссипации (рассеивания) энергии, которая поступает извне.

\subsection{Основные элементы синергетической концепции самоорганизации}

Объектами исследования являются открытые системы в неравновесном состоянии, характеризуемые интенсивным обменом веществом и энергией между подсистемами и между системой с ее окружением (средой).

Различаются процессы организации и самоорганизации. Общим признаком для них является возрастание порядка вследствие протекания процессов, противоположных установлению термодинамического равновесия независимо взаимодействующих элементов среды (также удаления от хаоса по другим критериям). (Организация, в отличие от самоорганизации, может характеризоваться, например, образованием однородных стабильных статических структур.)

Поведение элементов (подсистем) и системы в целом, существенным образом характеризуется спонтанностью --- акты поведения не являются строго детерминированными.

\subsubsection{Основных понятий синергетики}

\textbf{Открытая система} --- система, которая непрерывно взаимодействует с ее средой. Взаимодействие может принять форму информации, энергии, или материальных преобразований на границе с системой, в зависимости от дисциплины, которая определяет понятие. Открытая система противопоставляется понятию изолированная система, которая не обменивается энергией, веществом, или информацией с окружающей средой.

\textbf{Самоорганизация} --- процесс упорядочения (пространственного, временного или пространственно-временного) в открытой системе, за счёт согласованного взаимодействия множества элементов её составляющих.

\textbf{Бифуркация} --- нарушение устойчивости эволюционного режима системы, приводящее к возникновению после точки бифуркации спектра альтернативных сценариев эволюции.

\textbf{Аттрактор} (лат. притягиваю к себе) --- точка или множество точек (замкнутая кривая), к которому стремятся параметры состояния диссипативной системы, конечное состояние диссипативной системы.

\section{Особенности постнеклассической научной картины мира}
\begin{itemize}
    \item Широкое распространение идей и методов синергетики. 
    В синергетике показано, что современная наука имеет дело с очень сложноорганизованными системами разных уровней организации. Синергетика в перспективе может дать науке общий язык для описания, как мира природы, так и общества и человека. Таким образом, она олицетворяет интегративную тенденцию развития современной науки.
    \item Закрепление понятия «информация» в фундаменте естествознания, математики и гуманитарных наук. Информационный подход находит все больше и больше применений в различных сферах научного знания.
    \item  Укрепление парадигмы целостности. Осознание необходимости глобального  всестороннего взгляда на мир. Эта парадигма проявляется:
    \begin{itemize}
        \item в признании единства природы, общества и человека;
        \item человек находится не вне изучаемого объекта, а является частью, познающей целое;
        \item сближение гуманитарных и естественных наук;
        \item сближение разных типов рациональности в научном мышлении.
    \end{itemize}
    \item Укрепление и все более широкое применение принципа коэволюции. Изначально этот термин применялся в биологии для обозначения совместной эволюции различных биологических объектов и уровней их организации. Теперь это понятие охватывает обобщенную картину всех мыслимых эволюционных процессов как материальных, так и духовных систем.
    \item Усиление роли междисциплинарного подхода в исследованиях. Объектом современной науки все чаще становятся сложные системы, исторически развивающиеся системы и, так называемые, человекоразмерные системы (медикобиологические, системы «человек-машина» и т.д.). Изучение подобных систем невозможно в рамках конкретной научной дисциплины из-за предметной узости последней.
    \item Широкое применение философии и ее методов во всех науках. С одной стороны, идет активная рефлексия науки над собственными онтологическими, гносеологическими, аксиологическими и другими основаниями методами философии науки. С другой – современная наука вплотную подошла к проблеме соотношения материи и сознания и ряду других философских проблем.
    \item Методологический плюрализм. Осознание недостаточности единственной методологии для познания всех сторон объекта, особенно сложной системы. Истоком этого процесса является сформулированный Н. Бором принцип дополнительности. Ряд представителей современной философии науки (П. Фейерабенд) доводят эту мысль до признания «методологической анархии».
    \item 8. Ослабление требований к жестким нормативам научного дискурса. Еще В.И. Вернадский писал «научная творческая мысль выходит за пределы логики. Интуиция, вдохновение – основа величайших научных открытий, в дальнейшем опирающихся и идущих строго логическим путем».
    \item Преодоление разрыва субъекта и объекта в познании. Уже на этапе неклассической науки стало очевидно, что, по выражению Э. Шредингера, «субъект и объект едины». Один из основателей квантовой механики В.  Гейзенберг отмечал, что следует уже говорить не о картине природы, складывающейся в естественных науках, а о картине наших отношений с природой. Примером может служить так называемый «антропный принцип» современной космологии, который устанавливает необходимость появления во Вселенной субъекта-наблюдателя.
    \item Усиливающаяся математизация научных теорий и увеличивающийся уровень их абстрактности и сложности. Адекватное использование методов кибернетики, синергетики и информатики в научных исследованиях требует знания математического аппарата этих дисциплин, что особенно актуально для представителей социально-гуманитарных наук. В естественных науках эта тенденция также усиливается из-за изучения принципиально ненаглядных объектов и построения все более сложных моделей, объясняющих действительность.
\end{itemize}