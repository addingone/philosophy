\section{Философия науки, ее объект, предмет и структура. Роль философии в науке }   

\subsection{Философия науки, ее объект и предмет}

% Философия науки (ФН) – философская дисциплина, исследующая структуру научного знания, средство и 
% метода научного познания, а также способы обоснования и развития научного знания.

% В широком смысле, это философская дисциплина, занимающаяся осмыслением места и роли науки в отношении <<человек~—~мир>>.

% В узком смысле — деятельность ученых, посвященная философским и этическим проблемам развития науки.

% Объектом философии науки является наука как социальный и культурный феномен, а её предметом — познавательные структуры и методы, механизмы научного открытия, принципы объяснения и % обоснования научных знаний, а также этические аспекты науки.

Философия науки (ФН) --- философская дисциплина, исследующая структуру научного знания, средство и 
метода научного познания, а также способы обоснования и развития научного знания.

ФН как дисциплина появляется на рубеже 19-20 вв. В западной традиции принято 
использовать понятие эпистемология, но по содержанию это понятие уже, чем ФН, процесс познавательной 
деятельности в науке, ФН шире, структуру.

Объект ФН --- наука как исторически изменчивое единство её важнейших элементов (знание, техника, 
научные институты, типы научной деятельности, субъекты науки). Наука исторична, меняется вместе со 
сменой исторических эпох.
 
Предмет ФН --- наука, рассматриваемая через призму философии данной эпохи и общества. Точка анализа 
науки изменчива исторически, меняется наука и философия науки, который её исследует.

\subsection{Структура философии науки}

\subsubsection{Исследование науки как социокультурного феномена, как вида сознания и знания}

;-)

\subsubsection{Изучение философско-методологических оснований науки}

Онтологические представления о том, 
как устроен мир, гносеологические – методы достижения истинного знания, методологические 
знания -0 какой метод лежит в основе науки.
 
\subsubsection{Анализ проблемы возникновения науки и основных стадий ее развития
}
Существует два подхода к пониманию развития науки. Согласно концепции \textit{континуизма}, наука развивается поступательно и непрерывно, следуя эволюционному ходу. Концепция \textit{дисконтинуизма} утверждает, что развитие науки прерывается научными революциями, после которых она переходит на качественно новый этап с изменёнными фундаментальными представлениями.
    
\subsubsection{Изучение социокультурной динамики науки}

Иначе — определение движущих сил развития науки. Выделяются позиция \textit{экстернализма}, согласно которой основными движущими силами развития науки являются внешние факторы (например, социальный заказ, социальные ожидания от науки), а также позиция \textit{интернализма} - основными движущими силами развития науки являются внутренние факторы (интеллектуальные философские, собственно научные).
    
\subsubsection{Исследование философских проблем областей научного знания} 

Например, проблема жизни. Что такое жизнь с точки зрения различных научных дисциплин? Как эти разные понимания жизни объединяются? Есть ли у физики, биологии, химии, психологии нечто общее в понимании того, что такое жизнь? 

Или проблема происхождения Вселенной. Существовала ли она всегда, либо она появилась в какой-то момент?
        

	\subsection{Функции философии в науке} 

Долгое время обсуждалось, как отделить философию от науки и минимизировать её влияние, вплоть до полного устранения. Однако с конца XIX века, когда философия стала частью профессиональной подготовки ученых, возникла необходимость понять, какую роль она играет в научном образовании. 

Можно выделить несколько функций, которые философия выполняет в рамках научной деятельности.
\begin{itemize}
    \item \textit{Интегративная}. 
    Системное, целостное обобщение разнообразных форм познания, практики и культуры. Создание всеобщего и универсального знания.

    \item \textit{Критическая}. 
    Выяснение границ применимости получаемого человеком знания. Формирует критического сознания.  

    \item \textit{Мировоззренческая}. 
    Разработка определенных моделей реальности, сквозь призму которых ученый смотрит на предмет исследования (например, материализм, идеализм, дуализм)

    \item \textit{Гносеологическая}. 
    Исследование наиболее общих закономерностей познавательного процесса.

    \item \textit{Методологическая}. 
    Создание общенаучных методологических принципов. 
    Например, принцип инструментализма (Галилей) — для получения научного знания необходимо использовать приборы и измерительные инструменты.
    Либо принцип историзма — рассмотрение явления без отрыва от контекста его возникновения, протекания и привязки к определенной эпохе и обществу.
    
    \item \textit{Аксиологическая}. 
    Формирование определенных мировоззренческих и ценностных установок ученого. Это ответ на вопросы: <<зачем заниматься наукой?>>, <<что является в науке главной ценностью?>>, <<какую роль наука выполняет 
    в культуре и в обществе?>>, и т.п.

    \item \textit{Прогностическая}. 
    Разработка идей и представлений, значимость которых обнаруживается на будущих этапах развития науки. 
    Например, принцип атомизма — представление о том, что все твердые тела состоят из мельчайших элементов. Появился еще в античности, долгое время не был востребован в науке, но в 18 веке стал центральной опорой 
    всего корпуса химического знания. 
\end{itemize}
	
	\section {Проблема взаимосвязи философии и науки и основные концепции ее решения}

Еще в период античности возникает вопрос: <<как элементы научного знания взаимосвязаны и взаимодействуют с философией?>>. Рассмотрим концепции взаимосвязи философии и науки.

\subsubsection{Метафизическая концепция}

Появилась в античности усилиями Аристотеля.

Философия - это знание целого, а наука — это знание об отдельных областях действительности.
Философия — это некое состояние ума, которое позволяет рассуждать о предельных основаниях, о бытии, об истине и о ценности, а наука (или эпистема) — это знание, которое можно обосновать с помощью доказательства (например, математическое доказательство). 

Наука включается в рамки философии, так как наука является частным случаем рассуждения о всеобщих основаниях.
Аристотель приводит такое утверждение: все частные <<науки>> возникают из философии (н-р, геометрия, математика, астрономия, биология, география).

\subsubsection{Позитивистская концепция}

Возникла в XIX веке. Основоположник концепции — Огюст Конт. 

Наука — это особый вид познавательной деятельности, являющийся самодостаточным и обладающий приоритетом среди остальных видов познания. 
Это логический вывод из концепции развития общества Конта, где он выделяет три стадии: 
\begin{enumerate}
    \item \textit{теологическая}. Окружающий мир объясняется с точки зрения религиозных представлений (н-р, анимизм, тотемизм, фетишизм);
    \item \textit{метафизическая}. Окружающий мир объясняется с точки зрения философии (период: от античности до XIX века);
    \item \textit{позитивная} (научная). Окружающий мир объясняется с точки зрения науки.
Каждый раз, когда происходит смена этапа развития общества, предыдущий способ объяснения мира не исчезает, но он теряет свой приоритет.
\end{enumerate}

Концепция оказалась продуктивной для развития науки, так как позволила науке выработать собственный строгий методологический инструментарий, благодаря чему в 19-20 веке произошли громкие научные открытия. 
Однако позитивизму не удалось отделить философию от науки — ни в плане научного языка, ни в плане научной методологии, ни в плане мировоззрения. 

\subsubsection{Антиинтеракционизм}

Возник в начале 20 века на фоне Первой Мировой войны (Ж.-П. Сартр, К. Ясперс, Н.А. Бердяев). 

Основные проблемы общества — из-за науки, так как технический прогресс привел к глобальным социальным проблемам (мировая война, массовый голод, эпидемии).
Наука не способна исследовать подлинное бытие, что с точки зрения ряда философских направлений является внутренним миром человека 

\subsubsection{Диалектическая концепция}

Представители — Г. Гегель, Б.М. Кедров, В.С. Степин.

Философия и наука — это разные способы познания, но которые не могут существовать друг без друга. Философия изучает всеобщее, наука изучает частное. При этом философия без науки пуста, но наука без философии слепа.


\section {Наука как предмет прикладных исследований в ХХ веке в контексте научно-технической революции и развития <<большой науки>>}

\subsection{История исследований феномена науки.}

\subsubsection{XIX век. Дискуссия между Фрэнсисом Гальтоном и Альфонсом Декандолем}

Она касалась причин возникновения науки. Гальтон связывал развитие науки с расовыми особенностями, утверждая, что наука была создана исключительно белой расой. Декандоль, напротив, в своей книге «История науки и ученых за два века» (1873) на основе статистического анализа выделил 20 социальных факторов, определяющих развитие науки, таких как развитая система образования, доступ к материальным ресурсам, благоприятный климат и поддержка общественного мнения. Он доказал, что развитие науки зависит от социальных условий, а не от расы.

\subsection{«Большая» и «малая» науки. Основные тенденции развития науки в XX веке. Понятие научно-техничес-кой революции}

С точки зрения Дерека Прайса, основоположника современного прикладного исследования науки, в истории науки было несколько периодов:
\begin{enumerate}
    \item малая наука — разрозненные усилия отдельных людей по познанию окружающего мира (в античности, средневековье и в эпоху Возрождения);
    \item наука — в классическом понимании, возникшая в период нового времени (XVII век);
    \item большая наука (XX век) — характеризуется резким увеличением числа ученых, реализацией больших научных проектов и резким ростом научной информации.
\end{enumerate}
\textbf{Факторами формирования} большой науки являются:
\begin{itemize}
    \item Первая и Вторая мировые войны (высокий спрос на прикладные научные исследования);
    \item возникновение тесной связи науки и государства (глобальные и дорогостоящие научные проекты);
    \item рост роли науки в культуре общества.
\end{itemize}
К \textbf{особенностям большой науки} можно отнести:
\begin{itemize}
    \item резко возросшее число ученых (XVIII век~$\cong 10^3$, XX век~$\cong 10^6$);
    \item рост количества научной информации (в XX веке объем научной информации удваивается с периодом 10-15 лет);
    \item превращение науки в профессиональную деятельность;
    \item научно-техническая революция (в конце 40-х годов XX века);
    \item реализация больших научных проектов.
\end{itemize}

\textbf{Научно-техническая революция} — это коренное преобразование производительных сил общества на основе науки.

\subsection{Комплекс современных дисциплин, изучающих различные аспекты науки}

\subsubsection{Науковедение}

Основоположник — Джон Берналл («Социальная функция науки» — 1938 г.).

Задачей \textit{аналитического} направления науковедения является раскрытие закономерностей развития науки. 

\textit{Нормативное} науковедение занимается разработкой основ совершенствования организации научной деятельности. Нормативное науковедение стремится оптимизировать структуру науки для того, чтобы она работала более эффективно. 

\subsubsection{Социология науки}

Представители — Л. Флек и Р. Мертон.

С точки зрения Л. Флека задача социологии науки — это исследование интеллектуальных коллективов, изучение взаимодействия ученого (как индивида и личности) со своей социальной группой.

С точки зрения Роберта Мертона социология науки изучает универсальные нормы научной деятельности (объективизм, коллективизм и т.д.), или т.н. научный этос. 


\subsubsection{Наукометрия}

Основоположник — В.В. Налимов.

Задача наукометрии — измерить количественные результаты научной деятельности, эффективность как отдельного ученого, так и научного коллектива с целью дальнейшего принятия организационных решений в области науки.

\subsubsection{История науки}

История науки занимается проблемой возникновения и развития феномена науки. Можно выделить несколько уровней исторического исследования феномена науки:
\begin{itemize}
    \item наука в контекст развития общества (н-р, <<почему именно в Древней Греции возникает преднаучное знание?>>);
    \item наука как социокультурного феномена (н-р, история российской науки);
    \item история отдельных наук;
    \item история отдельного открытия, исследователя и т.п.
\end{itemize}

\subsubsection{Психология науки}
Изучает психологические факторы научной деятельности. Среди ее основных задач следующие: 
\begin{itemize}
    \item исследование психологических механизмов производства научных знаний в условиях индивидуальной и коллективной деятельности;
    \item разработка проблем психологической подготовки научных кадров, диагностики и формирования соответственных личностных качеств и установок;
    \item разработка проблем возрастной динамики творчества;
    \item анализ психологических аспектов научных коммуникаций, восприятия и оценки новых идей;
    \item анализ психологических аспектов автоматизации и компьютеризации исследований.
\end{itemize}
По большому счету психология науки занимается психологическим аспектом труда ученых. 

\subsubsection{Hормативно-правовое регулирование научной деятельности}

До конца Второй мировой войны, до так называемого Нюрнбергского трибунала практически не было практики нормативно-правового регулирования труда ученых. 

\texttt{Нюрнбергский кодекс}. Кодекс включает в себя несколько принципов, из которых, например, <<Добровольное согласие испытуемого на медицинский эксперимент>>; <<Интересы индивида преобладают над интересами научного сообщества>>.

\texttt{Хельсинская декларация Всемирной медицинской организации (1964 г.)}. 
Представляет собой набор этических принципов для медицинского сообщества, касающихся экспериментов на людях. Введено разделение на исследования:
\begin{itemize}
    \item c лечебной целью — которые направлены на разработку новых методов лечения или применяются в ситуациях, когда другие способы помочь пациенту невозможны, а его состояние угрожающее.
    \item с теоретической целью — которые изучают особенности работы организма, например, реакции на экстремальные условия (перегрузки, переохлаждение, отсутствие сна), без непосредственного лечебного значения.
\end{itemize}

\texttt{Конвенции о правах человека и биомедицине}. Сформулирован принцип необходимости этических комитетов, осуществляющих независимую экспертизу обоснованности необходимости проведения эксперимента.

\texttt{Декларация о геноме человека и правах человека (ЮНЕСКО)}. В декларации постулируется, что геном человека – основа изначальной общности всех представителей человеческого рода, и он не должен служить источником извлечения доходов. Все геномные исследования должны быть тщательно выверены, их обоснованность не должна вызывать никаких сомнений, и каждое национальное государство обладает своим суверенитетом в деле разрешения или запрещения геномных исследований. 

Наиболее явным образом нормативно-правовое регулирование научной деятельности осуществляется для исследований, направленных на изучение человека (медицинские, физиологические и пр.). Можно ожидать, что со временем начнет регулироваться и научная деятельность как таковая, поскольку современные исследования могут иметь последствия, которые необходимо учитывать в глобальном планировании развития экономики и общества.