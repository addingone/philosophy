 лекции по истории и философии науки. Сегодня мы переходим к самому объемному
разделу нашей дисциплины, который называется «История науки в ее связи с
философией». И здесь, прежде чем начать пятую тему, нам надо затронуть один
важный для понимания целей нашего курса вопрос. Зачем мы занимаемся историей
науки? Почему нам недостаточно познакомиться лишь с современными представлениями
о методологии исследовательской деятельности и ее социальной организации? Это
нужно, чтобы понимать, чем мы занимаемся, когда занимаемся наукой. Сможем ли мы
по-настоящему заниматься наукой, если для нас она будет чем-то не вполне
знакомым? Как можно спросить, сможем ли мы по-настоящему понимать человека, если
толком не знакомы с ним? Мы вроде работаем бок о бок каждый день, знаем его имя,
как он выглядит и так далее, но совершенно не представляем, как с ним быть, чего
от него ожидать, как выстраивать коммуникацию. Точно так же, как про интересного
нам человека, мы хотим знать что-то о его прошлом, причем не просто факты, там-
то тогда-то родился, он столько-то пошел в школу, ходил в какую-то секцию, то мы
хотим узнать о его переживаниях, о его детских воспоминаниях, о его друзьях, его
страхах, кризисах, их преодолении, словом о том, как он формировался, на
основании чего принимал судьбоносные решения и каким образом поступал в сложных
ситуациях. Тогда и информация о датах, цифрах, именах не так важна для понимания
человека. А, собственно, что важно? Важны условия, в которых он рос и
воспитывался, и важны его собственные основания, ориентиры, которыми он
руководствовался в прошлом и руководствуется сейчас. Тогда только мы можем
действительно понять человека и сказать, что это близкий человек, с которым нам
интересно и которому мы готовы уделять много времени, когда мы понимаем
основания его суждений, поступков, выводов. Примерно то же самое имеет место в
случае, когда мы хотим близкого знакомства с делом нашей жизни, буквально с тем,
чему мы посвящаем огромную часть своего времени. Наука не всегда была такой,
какая она предстает сегодня перед нами. Более того, она и сейчас
трансформируется, изобретает новые методы, совершает революционные открытия и
сталкивается с новыми проблемами. Она живет. А понять живое существо в том числе
значит узнать о том, как оно развивалось, на каких основаниях действовало
раньше, на каких и почему действует сейчас. История, лишь по видимости
дисциплина о прошлом, это исследование настоящего в его временной глубине. То,
что здесь и сейчас, не было бы таковым без своей истории, если не только что
возникло. Мы поэтому должны смотреть глубже и организовать наши исследования,
иначе, чем историографическая маркировка тех или иных вех на пути трансформации
научного знания, научной методологии. Сами по себе даты, названия, имена, лишь
точки или пунктирные линии, которыми мы размечаем пространство истории науки. Но
само это пространство не только из них состоит. Оно соткано смыслами. Чтобы
раскрыть для себя эти смыслы, мы должны смотреть глубже и стараться понять
основания того или иного исторического этапа, той или иной культуры, эпохи. Мы
должны научиться у истории науки ставить вопросы и изобретать способы их
решения, вдохновляясь теми проблемами, традицией их обсуждения и ходами мысли,
которые уже до нас и для нас разрабатывали предшественники. Таким образом,
научаясь о истории, мы, самое главное, сможем уверенно выйти к своим основаниям,
которые представляют собой единственный надежный фундамент для наших собственных
суждений, теории и способов поступать. Когда ставится вопрос об основаниях
представлений той или иной эпохи или типа науки, это вопрос о способе
миропонимания, о философских основаниях соответствующего типа культуры. И дело
не только в том, что философия формирует понимание отношений человека к миру,
отталкиваясь от которого каждая отдельная наука углубляется в исследование своей
области реальности. Когда-то между философией и наукой вообще невозможно было
провести границу, они были единым целым. Геометрическими формами обозначались
первоосновы всего сущего, а исследование природы было предметом глубочайших
философских изысканий. Медицинские практики ориентировались на движение небесных
сфер, а бытие описывалось в математических понятиях. Имя этому этапу жизни науки
в её связи с философией — античность. С данной культурно-исторической эпохи мы
начинаем отчёт науки и философии в их совместном становлении в европейской
традиции. Пятая тема курса носит название философской школы «Натурфилософские
программы античности». В рамках данной темы нам предстоит, внимание, две недели
занятий. Античность у нас единственная, занимает на лекциях четыре пары, все
остальные эпохи будут по две. Мы так сделали в нашем курсе, чтобы достаточно
времени уделить самой базовой эпохе, от которой отталкивается вся последующая
европейская философия и наука. Так что разберём четыре экзаменационных вопроса,
к первому из которых мы сейчас перейдём. Второй рассмотрим на второй паре
сегодня, а третий и четвёртый пойдут у нас через неделю на следующем занятии.
Итак, первый вопрос в пятой теме звучит следующим образом. Социально-культурные
условия формирования античной философии и науки. Почему такая именно у нас
логика будет раскрытие вопроса? Мы будем двигаться по нашей схеме изучения
культурно-исторических эпох, который мы разобрали в третьем вопросе предыдущей
темы, перевёрнутая пирамида. Помните, сначала набираем контекст, пишем
социокультурную ситуацию в Древней Греции, времён выделения науки и философии в
специфические виды деятельности. Из этих исторических моментов затем мы
стараемся вывести базовые принципы всего античного мировоззрения и по такому
пути выйдем к фундаментальным основаниям эпохи, антологическим. Так, поняв эти
основания, мы получим ключ, отпирающий дверь к действительному пониманию
философских идей и научных изысканий данного культурно-исторического периода.
Теперь спросим, зачем. В античной философии свёрнута вся последующая философия,
в античной науке свёрнута вся последующая наука. В этот период каким-то чудом
сформировался наш способ мыслить, и какой бы далёкой ни казалось эта эпоха, на
самом деле она в нас сегодняшних глубочайшим образом укоренена. Следовательно,
открывается поразительная перспектива. Изучая основания античного мышления, мы
параллельно сможем прояснить что-то и в своих собственных основаниях, и в
устройстве своего собственного способа мыслить. Итак, начнём с поверхностных
слоёв культуры. Первонаперво нам нужно воссоздать её контекст. Что мы уже знаем
об античности со школы сроков истории и МХК? Припомните, пожалуйста, какие в
целом у вас формировались ассоциации с словом античность. Давайте
последовательно будем отмечать самое общее. Локализуются истоки античной
культуры в Древней Греции, а затем в Древнем Риме. Чётко выделить временные
рамки этой эпохи тяжело. Существует множество вариантов отсчёта, каждый из
которых имеет веские доводы в свою защиту. С начала античной эпохи мы в нашем
курсе условно обозначим с XI века до н.э. С этого момента берёт начало так
называемый Гомеровский период античной истории. Идёт активное формирование
культурной традиции. В своих постоянных чертах оформляются мифы и легенды о
богах и героях. Появляются литературные произведения, посвящённые в частности
странствиям Одиссея и Триянской войне. Это события примерно XIII-XII веков до
н.э. Также с XI столетия до н.э. происходит формирование древнегреческих
полисов, городов-государств, представлявших собой специфические административные
образования, взаимоотношения внутри которых и между которыми также легли в
основании социокультурных особенностей эпохи. Окончание античности тоже
датировать непросто. Обычно отсчёт следующей эпохи Средневековья начинает с
упадка Римской империи, имевшего место в IV-V веках уже н.э. Хотя, например,
величайший философ раннего Средневековья, один из отцов церкви, святой Августин,
жил и творил свои произведения до окончания распада Римской империи, его годы
жизни 354-430 годы. В общем, будем иметь в виду условность такой датировки. На
сегодняшней лекции в плане развития философии Нуки нас будет больше всего
интересовать Древняя Греция, в которой в VII-VI веках до н.э. эти формы
культурной деятельности человека выделили специальную сферу занятий. Римской
традиции первых веков н.э. мы обязательно коснёмся во второй половине занятия на
следующей неделе. Что касается географии, древнегреческой культуры, племена
единого этнического происхождения населяли достаточно обширные территории
Средиземноморского и Черноморского побережья. Древняя Греция или Эллада в те
времена не представляла собой единого государства в той форме, в какой
национальные государства существуют сейчас. Тогда не существовало привычных
межгосударственных границ, а поскольку каждый крупный населённый пункт, полис от
греческого поли, много, но если множество людей, имел в своей власти систему
управления, то фактически по своему статусу представлял собой город-государство.
Тогда сформировались политические отношения, тоже от слова полис, отношения
людей по поводу их совместного проживания, регулирования дел этого сообщества,
внутренней власти и системы управления. Межгосударственные отношения были
буквально межполитическими, то есть выстраивались между полисами. И нередко за
счёт развития дипломатии организовывались союзы таких городов, как для ведения
междоусобных разборок, так и для противостояния внешним врагам, к примеру,
персам, стычки с которыми насушенные на море не были редкостью. Внимательно
посмотрите на карту колоний Древней Греции. Эллины имели колонии в Малой Азии на
территории современной Турции, проникли на восток вплоть до полуострова Крым в
Чёрном море, оставив там знаменитый Херсонес, колонизировали Черноморское
побережье современного Краснодарского края, а также прибрежные земли на
территории современной Грузии. Западные древнегреческие колонизаторы населяли
Сицирию, некоторые территории современной материковой Италии, побережье Испании
и множество островов, окружающих современную Грецию. В рамках этой темы мы будем
обсуждать философские школы, основанные в таких разнообразных местах, как Милет
в Ионии на территории современной Турции, Картон и Элея в современной Южной
Италии, Афины, современная столица Греции и так далее. Как раз на современной
карте я отметила некоторые полюсы, культурные и экономические центры Древней
Греции, в частности гору Олимп, на которой, согласно мифологии, жили боги. Таким
образом, даже уже бросив взгляд на карту, мы можем сделать вывод о том, что
древние греки были отличными мореплавателями и путешественниками, активно
исследовали новые земли, выстраивая с ними экономические и культурные связи за
счет обмена товарами и взаимного обучения. Развитие мореплавания обусловлено
благодатными географическими и климатическими условиями побережья материковой
Греции, изрезано бухтами и заливами вокруг множества мелких и крупных островов,
отсутствуют неблагоприятные морские течения и экстремальные погодные условия. В
полюсах процветали различные ремесла, торговля, занятия искусствами. На обширных
плодородных почвах Средиземного моря развивался седлое земледелие и
скотоводство. Развитие ремесел, искусств в соединении с активным мореплаванием и
колонизацией позволили мощнейшим образом укрепить экономику Древней Греции. За
счет открытия морских путей сообщений появились как восточные, так и в западные
рынки сбыта производимой сельскохозяйственной и ремесленной продукции, а также
предметов роскоши. Надо сказать, что понятие колонизации в случае с древними
греками необходимо отличать от современных наших представлений об этом феномене.
Поскольку в то время не существовало как таковой границ национальных государств
и не было понятия суверенитета, многие земли оставались просто неосвоенными. И
вот на диких побережьях греки высаживались для более удобного осуществления
торговли с другими народами, основывая колонии. Этимологически это слово одного
происхождения с нашим культ культура и означает обработку, освоение. То есть
древнегреческая колонизация с расселением по таким обширным территориям не была
завоеванием чужих земель, а изначально предполагала постройку новых полисов на
неосвоенных землях для абсолютно мирного взаимодействия с другими народами. А к
другим культурам греки относились с огромным почтением, с удовольствием общаясь
с местными жителями, слушая их жизненные истории, знакомясь с их бытом и
мифологическими представлениями. Также и рабовладельческий строй, характерный
для античной Греции, нельзя, на мой взгляд, рассматривать лишь с отрицательной
стороны. Конечно, всякое бывало, и в реальном человеческом обществе во все
времена, независимо от уклада, к сожалению, жестокость неизбывна. Так что
негативные моменты нельзя привязывать ни к конкретной эпохе, ни к определенному
экономическому строю. Но про рабовладение часто говорят о негуманности,
неправильности принуждения и лишении человеческих прав и свобод. Это все верно с
нашей сегодняшней точки зрения. Но для древних греков оно было естественно
сложившейся и глубоко укорененной в культуре традицией, возникшей первоначально
преимущественно по принципу долгового рабства. Когда один человек не был в
состоянии вернуть другому одолженные средства, он даровал господину свою
свободу, становясь его рабом. На самом деле с рабами в ранней античности все
было не так печально, как нам думается. Они могли выкупить себя из рабства и со
временем получить гражданство, то есть обрести статус свободного члена полиса.
Нередкий случай, когда за определенные заслуги господин сам даровал рабу
свободу. Примечательно, что во многом именно благодаря рабовладению граждане
полиса смогли освободить себе время для обучения, развития наук и искусства,
также совершенствования системы государственного управления. Таким образом,
организовав свой быт путем профессионального разделения обязанностей, эллины
преуспели в различных видах деятельности, опережая многих соседей или достойно
соперничая с ними. Это дало не только, как мы бы сейчас сказали, техническое и
военное превосходство. Народ, настолько интересующийся, изобретающий и
совершенствующий формы культурного самовыражения самой своей деятельностью и
идеалами этой деятельности, завоевывал авторитет и внышал уважение. Что еще мы
знаем о культуре Древней Греции? Чем обусловлено ее единство и влияние на столь
обширных территориях? Во-первых, это, безусловно, язык. На древнегреческом
записывались мифы и легенды, создавались поэтические, эпические, драматические
произведения, далась переписка, описывались исторические события, хронология,
генеалогия знатных родов, математические расчеты и исследовательские наблюдения.
Причем желающие могли ознакомиться с этими записями, специально изучать их,
обучаться по книгам. То есть, в отличие от цивилизации, например, Древнего
Египта и Месопотамии, в культуре Древней Греции доступ к передаче знаний,
историческим документам и литературным произведениям имели не только избранный,
обычно узкий круг жрецов и семья проявителя, но все свободные граждане. Так
каждый, в том числе и свободные женщины, могли обучаться тому мастерству,
которое приходилось по душе, от математики до ораторского искусства, от
врачевания до военного дела. И затем уже в свои философские школы женщин будут
принимать наравне с мужчинами, например, Пифагор и Эпикур. Открытый доступ
общества к произведениям культуры в широком смысле к искусству, мифам, истории,
наукам, возможность путешествовать, обучаясь в разных местах и древнегреческий
язык, интегрирующий все эти формы деятельности, безусловно, обеспечивали бурное
развитие всех сфер культуры Древней Греции. Влияние древнегреческого языка на
другие языки, а следовательно и на другие европейские культуры колоссально. Вы
смотрите, я только что употребила слово колоссально, оно древнегреческого
происхождения означает гигантский по размеру. Наверняка вы слышали о колоссе
Родоско. Так даже в, казалось бы, далеком от Греции, нашем Отечестве, в
повседневном обиходе, используется огромное количество слов, происходящих от
древнегреческих корней. До сих пор в обыденной и художественной речи
используются фразеологические выражения, имеющие древнегреческое происхождение.
Даже само словосочетание «крылатые» выражения восходит к Гомеру. «Почивать на
лаврах», «Сизифов труд», «Яблоко раздора», «Петь дифирамбу», «Рок изобилия»,
«Ахиллесова пята», «Ящик пандоры» и так далее. В основном к нам эти отпечатки
древнегреческой культуры пришли через Византию, восточную часть Римской империи,
югром, которая была уже в более поздний период Греция. Возьмем первые попавшиеся
примеры в научном нашем языке, пожалуй, древнегреческих корней не меньше, чем в
обыденной русской речи. Например, «архе» с корнем «арх», обозначающим «древний»,
«главный», «очень». Отсюда «архаичный», «архитектор», «главный строитель». Еще
говорят, например, «архи сложный», значит «очень сложный». Далее. Наши атомы от
«атомос», что значит «неделимый». Наша теория от «феория» – мысленное созерцание
того же корня теорема. Логика, наука логика, с древнегреческого учения о
правильных суждениях или о правильном мышлении. Аризматике от «аритмос» или
«арифмос» в другой транскрипции. «Число», там же корень «ритм» или «рифм», что
является своеобразным языковым указанием на единство музыки со счетом числом.
Кстати, и «музыка» от «музике», что является однокоренным со словом «муза». Муза
– покровительница искусства наук, спутница бога Аполлона. В честь них искусство
в нашем современном смысле называли тогда «мусическими искусствами». А само
слово «искусство» техны понималось более широко, как вообще осмысленная
деятельность человека. Например, искусство землепарца, гончарное искусство,
искусство врачевания и подобное. Искусство как вида человеческой культурной
деятельности, а не в смысле того, что это выражение ощущений и переживаний по
произведению, и как мы сейчас понимаем искусство более узко. Наконец, «психе» –
это душа. Отсюда психология, наука о психике, о душе. Ну, такие примеры можно
приводить еще долго. Это интересное занятие. Вы попробуйте на досуге поискать
другие древнегреческие корни в словах нашего языка, особенно в ваших научных
областях, и почитайте об их происхождении и значении. В таких герменофтических
практиках также должно многое открыться для понимания глубинных истоков
современной науки. Идем дальше. Во-вторых, единство любой культуры
обеспечивается и единством мировоззрения. А что такое мировоззрение? Это то, как
люди видят мир, то есть какие у них представления о возникновении мира, его
структуре и законах, по которым он функционирует. Исторически первой формой
представления о мире является мифологическое объяснение его устройства. Оно
предполагает одушевленный характер всего мира, управления им чьей-то
могущественной волей и сверхчеловеческими способностями. Обычно в мифах также
предполагается хотя бы случавшийся ранее непосредственный контакт божеств или
божественных бессмертных существ с людьми, с миром смертных. Собственно, и мифы
Древней Греции, о которых у каждого из нас есть представления с детства, не
являются в этом плане исключением. Подробнее, мне кажется, о содержании
древнегреческих мифов сейчас нецелесообразно говорить, однако еще несколько
деталей стоит отметить. Самое важное здесь, на мой взгляд, то, что
мифологические представления удовлетворяли человеческую потребность в объяснении
окружающей действительности. Они давали ответы на вопросы о том, как произошел
мир, как он устроен, кем управляется, каким нормам и идеалам необходимо
соответствовать в собственной жизни. И до поры до времени этого мифологического
мировоззрения было достаточно, чтобы существовать и ориентироваться в мире.
Данные особенности мифологического мировосприятия вместе с вниманием древних
греков к языку не могли не отразиться в искусстве Эллады. Для почитания богов
строились величественные храмы из камня в связи с чем развивалась архитектура.
То есть принципы построения зданий используются не только для святилищ, но и для
иных монументальных построек. Снаружи и внутри храмы необходимо было украшать,
поэтому в силу распространения таких наиболее доступных материалов, как камень,
мрамор, глина, металл и сплавы, развитие получали в первую очередь пластические
искусства. Скульптура из камня и бронзы, огончарное мастерство, выкладывание
мозаик, техника росписи фресками и позже камнерезное искусство и,
соответственно, создание барельефа стали основными составляющими
древнегреческого изобразительного искусства. Изображались прежде всего
мифологические персонажи и события с их участием. С другой стороны, устная
передача мифов и легенз способствовала их значительному искажению. Каждый
рассказчик произвольно мог от себя добавлять те или иные детали, допускать
неточности, путать последовательность возникновения богов, в связи с этим со
временем появилась необходимость письменной фиксации взаимосвязи всех
божественных существ древнегреческой мифологии и восстраивания более или менее
единого и непротиворечивого мифологического объяснения мира. Поскольку дела
богов мыслились священными и возвышенными, лежали в основании миропонимания и
несли получающую воспитательную функцию, излагать мифы и легенды надлежала
наиболее высоким стилям в поэтической форме, соблюдении ритма и в сопровождении
музыкальных инструментов. Любовь эллинов к рассказыванию и слушанию историй, к
языку, к речи, ее гармоничному и музыкальному звучанию сыграли ведущую роль в
становлении как словесного творчества, так и музыкальных жанров. Постепенно
совершенствовались музыкальные инструменты и изобретались наиболее подходящие
для того или иного словесного жанра стихотворные ритмы и размеры. Поэты, среди
которых следует обязательно упомянуть Гомера и Лисиона, соревновались между
собой перед публикой выразительности языка, стройности своих сочинений и полноте
передачи в них всех легендарных мифических событий. Собственно, благодаря
сохранившимся фрагментам их произведений мы и имеем современное представление о
древнегреческих мифах и легендах. Распространенными развлечениями с древнейших
времен были также празднества в честь погов с фестивалями и опрядами.
Неотъемлемыми элементами праздников являлись театральные представления,
разыгрывающие поначалу события мифов и сказания о героях Лады, а затем и
драматические произведения с самостоятельным сюжетом, которые постепенно
выделились в отдельный жанр. Спектакли могли быть достаточно длительными, а их
итоги бурно обсуждались зрителями. Самый важный тут момент. Давайте примечанием
отметим, почему так популярны были древнегреческие трагедии, почему собирали
тысячи зрителей в амфитеатрах и на стадионах. Думаю о таких трагедиях, как,
например, Царь Эдип, Антигона, Аристея, все имеют представление. В основе сюжета
лежит социально-этическая опория или парадокс, принципиально неразрешимая
противоречие. Буквально опория означает непроходимость. Слово опора в нашем
языке того же корня, но альфа-приватчевым дает отрицание. То есть опория – это
отсутствие опоры, отсутствие прохода опоры. Так, сталкиваясь с ситуацией, в
которой, в принципе, что бы ты ни выбрал, не получится хорошо для всех, герой
попадает в состояние омехании. Прямой перевод омехании – беспомощность. Такое
состояние ступора, когда мы не можем. Не можем применить никакие механизмы или
техники решения. У нас их много, мы почти все можем, столько всего умеем. Но вот
для этически затруднительных ситуаций, принципиально, нет заранее готовых
решений. И каждый раз нам приходится в конкретике уникальных обстоятельств
сделать выбор на свой страх и риск. Кстати, мы с вами об этом говорили на
третьей теме. Так вот, совместное переживание этого состояния повязанности по
рукам и ногам в таких ситуациях, пронзительное понимание одновременного и
величия человека, и его конечности, смертности, и неумения превозмочь порог
смерти, играло в древнегреческом полисе, городе-государстве роль, простите за
жаргон, таких скреп, на которых держался гражданский дух этого удивительно
мужественного народа. А древние греки и потом римляне, известные своим боевым
духом, побеждали армии других народов численностью до десяти раз превышающей их
войско и лучше вооруженных. Так вот, почему с внешней точки зрения, например,
убийцы своих родителей, Эдип и Арест, все-таки герои? Потому что они мужественно
выдерживают распятие на этой опоре непроходимости, изгоняют в себе пороки, и
благодаря этим чудовищным испытаниям судьбы что-то важное для себя понимают,
оставаясь людьми. Они как бы своей стойкостью, вопреки судьбе и смерти, держат,
выражая словами Мандельштама, место человека во Вселенной. И когда несколько
тысячный коллектив свободных граждан захватывает это переживание, они страдают
вместе со своими героями и понимают, что настоящее человеческое держится только
усилием, стремлением к высшим идеалам и усилием противостояния порокам в себе,
изгнания из собственной души каждым царя Эдипа. Я не случайно здесь произношу
слово «изгнание», оно связано в античном полисе и с такой уникальной практикой,
как острокизм. Голосование, в ходе которого каждый член полиса на общем собрании
писал на черепке или того, кто, по его мнению, наиболее опасен для города, и
того, кто набирал наибольшее количество голосов и сгоняли на 10 лет, как
минимум. Вы понимаете, что этого, естественно, каждый гражданин опасался. Все в
таком небольшом городе друг друга знали и старались для соотечественников
показывать себя с лучшей стороны. Мужчины таким образом буквально соревновались
друг с другом в проявлении добродетелей. Практика острокизма побуждала вести
себя честно, мужественно, справедливо, сдержанно и рассудительно. И древние
греки очень дорожили согласием со своими согражданами в полисе, потому что,
безусловно, изгнание, о котором слухи и сплетни моментально разносятся по
соседним полисам, было страшным позором. А кроме того, потери связей со своими
друзьями и родными, потери своего социального статуса, своего дела и, по сути,
имущества. Очень продуктивная система поддержания порядка, согласитесь. Но
вернемся от нашего углубленного примечания к дальнейшим чертам древнегреческой
культуры, в которых также проявляется соревновательный дух. Богам посвящались,
помимо перов и театральных представлений, многочисленные спортивные состязания
различного уровня и масштаба, известнейшие из которых Олимпийские игры
проводятся в мире по сей день. Спортивные соревнования служили не только в
качестве развлечений и зрелищ, но также позволяли различным полисам
продемонстрировать друг перед другом силу, красоту и способности своих атлетов.
В целом, дух торжества, благородного состязания и возвышенно эстетического
отношения к действительности вдохновлял Эллинов на новые свершения и в силу
такого позитивного соперничества способствовал все большему совершенствованию
различных сфер культурной жизни. Зрелища и праздники, на которые тратились
огромнейшие государственные средства, несоизмеримые по своему объёму даже с
военными расходами, на долгие столетия стали единящим древнегреческое общество
культурным ядром. Как вы понимаете, в любой культурно-исторической локальности
возникает некоторая среда, через которую, как через медиум, каждый отдельный
человек приобщается ко всему своему социуму, чувствует себя его частью и
поддерживает с другими его членами актуальной связи. Таким же медиумом,
например, в средние века выступала церковь. Все члены европейского
средневекового общества в неё ходили, участвовали в обрядах и празднованиях, и
благодаря этому у них поддерживалось такое социокультурное коммуникативное
пространство единых смыслов, в котором рождались и передавались байки и
различные мнения, шутки и серьёзные идеи. Поэтому страшно было отлучение от
церкви. Это означало буквально обрубить для человека возможность общаться, все
его связи с другими людьми. Как если бы нашему современнику, но совсем
заблокировали доступ в интернет и возможность пользоваться мобильной связью, так
что он бы даже электронную почту не мог посмотреть, позвонить родным или
произвести электронную оплату. Вот для нас сегодняшних дома такой средой
являются коммуникативные, цифровые, мобильные системы интернета, особенно
соцсети, мессенджеры и различные платформы, предоставляющие видеоконтент,
возможность трансляции, комментирования, интерпретирования и передачи друг другу
новых актуальных сюжетов. В недавнем прошлом функцию такого медиума выполняли
средства массовой информации, а в нашей стране, в том числе, очень похожем на
древнегреческое общество образом, обязательные коллективные мероприятия вроде
государственных праздников с парадами, субботников и тех же культурных программ,
походов в кино, на концерты, в театр. Так вот, у древних греков подобной средой
тоже были массовые празднества, посещение спортивных и театральных зрелищ, а
также общение на главной площади города, на которой в полисах обычно
располагался рынок. Именно там в основном обсуждались какие-то новости и
политические решения, рождались и передавались мемы, шутки, спледни и так далее.
Кроме того, монархические формы правления архического периода постепенно
сменялись в полисах олигархическими, то есть власть не обязательно передавалась
по наследству. Надо отметить, что древнегреческие правители отнюдь не были
настолько богатыми и богоподобными, как, скажем, персидские или египетские цари,
в полисах власть держалась поэтому в руках сильных духом людей, мужественных,
способных всего себя и всю свою жизнь поставить на карту, рискнуть всем ради
благородного поступка, например, отправиться в совершенно дикие, неизведанные
земли и основать там колонию, организуя при этом своих сподвижников и стараясь
обеспечить для них достойные условия жизни и интересную работу. То есть ценились
прежде всего дела, поступки, а не материальные блага или изнеживающие удобства.
Полисы развивались эффективнее под предводительством Совета, учитывавшего
социально-этические потребности граждан, культурно-экономические и военные
преимущества. Во вновь создаваемых городах на колониальных землях к управлению
гражданскими делами приходили наиболее влиятельные люди, полководцы, стратеги,
знатные и наиболее рассудительные граждане. Они завоевывали у народа авторитет,
покровительство развитию земледелия, ремесел, искусство, торговля и образование,
а также не скупясь на всевозможные развлекательные мероприятия, призванные не
только угодить богам, но и сплотить из сообщества полиса разнообразие в досуг
его членов. Таким образом, к VII веку до н.э. на территориях Древней Греции
складываются уникальные по своей специфике социально-культурные условия,
способствовавшие выделению в отдельной практике особой формы понимания
действительности теоретического мышления. Когда говорят о возникновении
философии и науки в Древней Греции, надо понимать, почему это слово берут в
кавычки. Как мы обсуждали на предыдущих занятиях, вспоминайте, знание как
таковое существовало и передавалось в различные культуры задолго до VII
встаретия до н.э. В Древнем Египте, на Месопотамии и на Востоке, в Древней Индии
и в Древнем Китае известны феномены познания, накопления опыта и передачи
знания. Однако, если можно в этот период говорить о развитии науки, то это была
практическая деятельность, а знание было фактуальным, то есть знанием о фактах,
которые прекращались, прежде всего, в обыденной жизни для тех или иных
практических нужд. И даже, казалось бы, самая теоретическая наука, математика,
использовалась исключительно для применения на практике, для счета измерения
предметов, планирования земельных участков и строительства. Так что мы будем
говорить осторожнее, скорее о том, что феномен теоретического мышления отделился
в эпоху античности от эстетически-мифологического миропонимания. Для понимания
этого ключевого момента давайте, прежде всего, вспомним со второй темы нашего
курса особенности теоретического уровня осмысления. К нему относятся высокая
степень абстракции, необходимая для выделения сущностных повторяющихся черт в
эмпирическом многообразии явлений. Кроме того, если для практической
деятельности характерен непосредственный контакт с объектом, то теоретизирование
происходит именно в уме, не с помощью связи с предметом за счет органов чувств,
но мысленно, независимо от физического ощущения. Так что, когда говорят о
теоретическом мышлении античной культуры, надо понимать, что вовсе не имеют в
виду, что до этого люди не умели обобщать, выделять сущностные причины и явления
или оперировать теми же числами, как предельной абстракцией. Речь идет о том,
что свойственное самой природе человека теоретическое мышление просто не было
систематически развиваемо. Не было до определенного момента самим человеком
замечено как нечто специфическое, как способность, которую развивать необходимо.
Об этом не совсем корректно говорить о том, что наука и философия возникли в
античности и, как в некоторых учебниках пишут, родились из мифов. На самом деле
в Древней Греции к VII веку до н.э. сложился комплекс условий, в которых, во-
первых, мифологические представления не могли дать ответы на интересующие
человека вопросы не о том, как все устроено в мире и каким должен быть человек,
а о том, почему все именно так устроено и зачем человек должен и должен ли быть
таким, как его рисует миф. А во-вторых, культурная и социально-политическая
жизнь побуждала специально развивать и практиковать именно теоретические
способности. Пока вы записываете, я немного прокомментирую, как я вижу эту
ситуацию. На самом деле многие исследователи реконструируют в кавычках
возникновение науки и философии, как именно появление каких-то новых практик в
силу неспособности мифа, как самозамплательную систему отвечать на интересующие
вопросы. Сама формула от мифа к логосу принадлежит советскому антиковеду Тихарио
Харлампевичу Кисиде, греку по национальности, работавшему в институте философии
РАН в Москве. Также подобное понимание характерно для, например, Алексея
Федоровича Лосева, тоже очень известного отечественного мыслителя, исследователя
культуры в ее символическом и эстетическом измерениях. Ну и, например, уже
знакомый нам Мираб Константинович Мамардашвили говорит, что наука не появляется
из простого накопления техник, умений, но обязательно человеку должно стать что-
то непонятно. В мифе все объяснено, и непонятного нет, а проблемность — это уже
черта науки. Но я не совсем с этой трактовкой согласна, поскольку, по-честному,
не знаю, так ли это было, хотя вроде бы логично и обоснованно, что жили себе
столетиями, тысячелетиями люди, пользовались мифологическими представлениями для
ориентирования в мире и не особо задавались вопросами о том, почему все именно
так устроено. А потом такие «стоп, посмотрите, вот непонятно ведь все в мире на
самом деле». С одной стороны, обильная историческая, философская литература
подводит к такому варианту, и нет оснований не доверять нашему гигантам мысли.
Элементарные исторические факты отсчитывают первые, собственно, теоретические,
не художественные, не мифологические произведения с VII-VI веков до н.э. с
Фалесом, Милецкого, Пифагора, Гераклита, Эфесского, Парменида и т.д. Но обратите
внимание, произведения, письменные источники, а как же устная традиция, тем
более в отношении мифа. Поэтому, с другой стороны, мне кажется, не задавались бы
люди вопросами, миф был бы один раз и навсегда. А миф очень текучий, изменчив,
передающийся из уст усталу в слове каждого рассказчика претерпевал изменения,
добавление уточнения, накапливались подробности, развивались все новые витки
сюжетов о богах и героях. Не от того ли, что просто основным, в смысле
привычным, уже сложившимся, средством, инструментом ответа на вопросы был в то
время миф, который можно было трансформировать. Однако, чтобы эту устную
традицию как-то все-таки устаканить, древнегреческие поэты начали в XI-IX веках
до н.э. мифы и легенды записывать. Ведь буква письменного источника тверже
текучести устной речи. Это, по сути, ошибка. Сделать статичным текучее по своей
природе обеспечило настоящий прорыв, который историки и датируют событием в
кавычках возникновения философии. Но на самом деле люди и так уже давно устно
задавались предельными вопросами, пытались для себя что-то понять, удивлялись
чудесно продуманным мироустройству и красоте мира и человека, как и любой
современный человек, но просто не выделяли это занятие в отдельный вид
деятельности и не практиковали специально. В тех же поэмах Гомера уже растворена
вся древнегреческая философия. Большинство философских категорий, встречающихся
в текстах античных мыслителей, слова обыденного языка, которыми и так народ
испокон веков пользовался. Это наводит на мысль о том, что всё-таки, возможно,
философию, науку, теоретические понятия не изобрели и не ввели. Наблюдательные
греки в VII-VI веках до н.э. скорее заметили, обратили внимание на вопрошение и
осмысление как отдельный вид деятельности, наряду с ремеслом, торговлей,
искусством, военным делом и так далее. А в Древней Греции в этот момент
складывалась благоприятная социально-политическая ситуация для осознания
неполноты мифа уже после Гомера Гищода и других поэтов статичного и
зафиксированного явления способности отвечать на предельный вопрос. Давайте
тогда резюмируем всё, что мы тут выше наговорили про начало античной культуры и
уже на более глубоком уровне понимания систематизируем по пунктам эти уникальные
условия того периода в Элладе, которые и побудили греков именно
институционализировать теоретические занятия в качестве самостоятельных видов
культурной деятельности. Развитие искусств, как пластических, так и словесных,
позволило на высоком уровне выражать в письменных и устных формах мифологическое
понимание мира. Во многом этому развитию способствовало поощрение занятий
искусством, а также сложившийся в рамках самого мифотворчества соревновательный
дух. Поэты, скульпторы, атлеты, музыканты, актёры постоянно совершенствовали
свои умения и оттачивали навыки, состязаясь друг с другом. Свобода и
зрелищность, публичность, культура Древней Греции к этому обязывали. В
результате развития словесных речевых жанров активно начали использоваться
художественные приёмы, метафоры и сравнения. К постоянному сопоставлению
обязывала и творческая и соревновательная атмосфера. Как раз сравнение и
сопоставление являются одними из основных теоретических процедур, работающих со
сходствами и различиями. Безусловно, выделение сходств и различий требует
высокого уровня абстрагирования, то есть мысленного одновременного представления
нескольких вещей, событий или феноменов и выделения в них на фоне общего
особенного. Однако к сопоставлению побуждали, как представляются и другие
особенности развития древнегреческой культуры. В силу освоения мореходства и
международной торговли происходило активное знакомство с другими этносами и их
культурами, что побуждало задаваться вопросами о нерушимости устоев собственного
мировоззрения. Другие цивилизации формировались под влиянием других мифов,
соответственно, на иных землях царил свой, отличный от древнегреческого уклад,
формировалось искусство другого типа и распространялись иные представления о
возникновении мира, природе материи и месте человека во Вселенной. Традиции,
обычаи и мифологию других народов греки сравнивали со своими, как мы уже
отметили выше, уважительно относясь к иным культурам и с удовольствием
выслушивая их сказания и предания. Безусловно, тесное взаимодействие действие с
другими народами не пошатнуло мифологические основы древнегреческой культуры,
однако сказалось на широте кругозора эллинов и переосмыслении некоторых моментов
собственного привычного уклада и мировоззрения с учетом опыта других народов.
Полис в качестве специфической формы совместного бытия в древнегреческом
обществе также стимулировал развитие критического и теоретического мышления,
становление олигархических, а затем демократических режимов правления в полисах
постепенно приобщало свободных граждан к вопросам правления, вовлекая в
обсуждение дел полиса. То есть у горожан появилась обязанность помимо
собственных дел заниматься также государственной деятельностью. Общегражданские
собрания проводились на центральной рыночной площади города, которая называлась
Агура. Проводились голосования, направленные на принятие государственных
решений. Каждый при желании мог высказаться в ходе обсуждения вопросов,
касающихся законодательства, полиса, празднеств и других культурных мероприятий,
внутреннего управления, экономических проблем, войны и мира и так далее. Таким
образом, для участия в политической деятельности требовались особые личностные
качества, самостоятельность мышления, критический подход, умение рассуждать,
аргументировать свое мнение, владение ораторским мастерством. В совокупности с
открытым доступом к образованию, что мы тоже уже отмечали, потребность в таком
типе мышления в огромной степени способствовала развитию дисциплин, логики и
риторики, а затем и становлению философского подхода. Все эти условия в целом
дали, как мы бы сейчас сказали, синергетический эффект. То есть совместно
благотворно повлияли на интенсивное развитие рефлексивного и теоретического
мышления, чего в условиях других величайших цивилизаций того же исторического
периода не произошло. Абстрагирование, выделение общего, сравнение,
сопоставление осуществляется умозрительно, то есть не путем практических
манипуляций в мире, но способом воссоздания в своем уме общей формы для
сопоставленных вещей или явлений. Умозрение становится основой, античного
теоретического мышления как метод, с помощью которого можно проникнуть сквозь
видимость к истинной сути вещей. Для чего же был выделен и культивирован этот
специфический способ мыслить? Как мы уже проговорили, мифологическое объяснение
мира, догматическое, по своей сути, не могло объяснить абсолютно все. Человеку
свойственно задаваться вопросами и живая тяга к пониманию никогда не уложится ни
в один статичный конструкт, ни в какие жесткие рамки. Зафиксированная мифология
не давала ответов на многие интересующие человека вопросы. Например, о том, из
чего все состоит, на чем держится единство всего мироздания, что такое добро и
зло, справедливость, мужество, красота, как поступать в противоречивой ситуации,
когда закон велит одно, а сердце подсказывает иное, какова природа человеческой
души и так далее. Ни миф, ни искусство сами по себе ответить на подобные вопросы
были не способны. Однако, несмотря на свою неспособность давать ответы на такие
фундаментальные вопросы, миф и искусство, безусловно, свои собственные функции
выполняли. Поэтому древним грекам вовсе не обязательно было отказываться от
эстетического наслаждения посредством искусства или от мифологического
объяснения вопросов сотворения Вселенной, влияния на человеческие дела. Просто
возникла необходимость осмыслять ряд вопросов другими средствами, чем миф или
искусство. И вот мы с вами сейчас наконец-то подобрались к самым
фундаментальным, глубинным основаниям античной культуры, чтобы ввести
содержатель. Ее онтологические основания мы должны предельно обобщить черты
миропонимания древних греков, о которых сегодня вот уже долго так говорим.
Систематизируем особенности мышления и ценностные ориентиры, которые легли в
основании древнегреческой философии. Итак, фиксируем для себя быстренько. Можно
две колоночки записывать, как представлено на слайде. Любовь к рассказыванию
историй, отразившуюся в мифах, подчеркивала и стремление знать генеалогию, как
происхождение было, так и свою родословную. Отсюда со временем формируется
специальный интерес к поиску причинно-следственных связей и приводит это к тому,
что фундаментальной особенностью древнегреческого типа мышления становится его
рефлексивность, осмысление через последовательную постановку вопросов о
причинах. Древнему греку было интересно все, от новых земель до древних
преданий, от замысловатого движения звезд на небе до отражения окружающих
предметов в мельчайшей капельке воды. Это культура интереса и внимания ко всему,
поэтому в сознании не выстраивалась иерархия окружающих предметов. Не было
различия по степени значимости между всматриванием в ход далеких звезд и
рассматриванием рысинки на траве под ногами, между слушанием дивных сказаний о
богах и интересом к себе смертного человека. Один из лозунгов древнегреческой
мысли, сформулированный первыми философами постулируют все во всем. То есть
отражение одного и того же присутствия, одного и того же события, любого
масштаба, вещи различной величины. Отсюда фундаментальные представления
древнегреческих первых. Мыслили о том, что все, что мы видим, все, что в мире
встречаем, состоит из одного и того же. Следовательно, все во всем может
превращаться. Более того, превращается на наших глазах из куска мрамора в
прекрасную статую, из жидкой массы во вкусный хлеб, из умершего гнющего тела в
плодородный слой, а затем, значит, в нашу пищу, а пища преобразуется в движение
нашего тела и нашего ума и так далее. Все безгранично переходит во все,
поскольку все пронизано чем-то единым. Греки умели восхищаться совершенством и
устройством природы. Стараясь подражать природе в искусстве, они
совершенствовали формы выражения, стремясь к изображению некого законченного
целого. Несмотря на понимание того, что все в мире течет, все изменяется, греки
умели видеть единичность, целостность, воплощенность, полноты в каждой
уникальной вещи. Поэтому целостность и завершенность, ограниченность и
оформленность стали идеалом. Тогда понимание сущего происходило через видение
его границ, его формы, его единичности как завершенности и полноты. В связи с
этим для первых античных исследователей, для всей античной культуры единица не
была числом. В полном смысле слово единица была основой числового ряда, счета,
однако она ценностно превосходила другие числа, составляющиеся путем
суммирования единиц. Важен был предмет в его целостности как своеобразная
единица, которую все и мерили. Также и мир как слаженность целого виделся
упорядоченной самой большой единицей, но тоже единицей. Греки ценили
упорядоченную организацию целого в природе, поэтому в искусстве, а затем в
мышлении в целом. Главнейшими принципами стали гармония и соразмерность.
Ценность имеет целое в единстве своих, пусть и разнородных частей и мерилось все
это философски понятая целостная единица. С ней все соизмерялось точнейшим
образом от пропорции идеальной колонны для храма до разницы в тонах звуков для
благозвучного музыкального произведения. Гармонично упорядоченно двигались
небесные сферы, значит и человек должен стремиться уподобиться этой гармонии
упорядоченной мысли и организуя свою жизнь. Наконец мы готовы ответить на вопрос
об антологических основаниях античной культуры. Напомню, мы их выделяем три. Как
понимается отношение человек-мир, что мыслиться несомненно, и что значит быть в
рамках соответствующего типа мышления. мир представал античному взгляду как
упорядоченная гармонично организованная целая космос. Поскольку же все
отражается во всем, каждая единичность также является полноценным целым. Человек
понимался как микрокосм, то есть как воплощение того же самого порядка, просто
относительно Вселенной более маленького размера. Тогда и во всей человеческой
жизни должно быть стремление к вселенской упорядоченности и соразмерности. Сам
человек должен становиться идеалом совершенства себя как внешне, так и
внутренне. Занятия должны воплощать совершенство, неупорядоченность и
оформленность мышления, а также проговоренность такого мышления, что
обозначается многозначным словом логос, означающим в том числе речь, порядок.
Этическим ориентиром становится добродетель, арите, как неотъемлемое свойство
настоящего правильного разумного человека. Несомненно, для древних греков, не
ставящимся под вопрос, было то, что мы называем вслед за Анатолием Вариановичем
Ахутиным антологической интуицией единого. Для античного человека несомненно,
было то, что все едино. Хотя, обратите внимание, налицо в мире множество
отдельных предметов. здесь демонстрирует свое существо как раз теоретическое
мышление, способное сквозь практическую видимость отдельности и разнородности
умозрительно выявить скрытую от физических глаз связь всего со всем, единства
всего. Раз все едино, для античного мыслителя оказалось очевидным, что все
состоит из чего-то одного. Как бы тогда все могло превращаться во все, например,
как бы мы усваивали пищу, если бы фундаментальным образом не состояли с ней из
одного и того же. Поэтому, задаваясь вопросом о том, из чего все состоит, и
находя разные элогичные ответы первые древнегреческие философы, тем не менее,
были согласны друг с другом в том, что это должно быть что-то единое, одно
начало для всего. Пронизывающий всю Вселенную порядок Логоса не только
обеспечивал своим единством правильную устроенность всего космоса, но и
отражаясь в человеческом уме, как бы настраиваемым на одну волну со Вселенским
порядком, гарантировал человеку способность понимать этот порядок с помощью
разума исследовать законы упорядоченности и соразмерности всего в мире. Так что
этот принцип, замыкающий на себя античное мировосприятие, явился принципом,
легитимирующим саму возможность теоретического познания, что без сомнения важно
для научной мысли любой эпохи. Ведь, чтобы достоверно познавать, нужно
основываться на том, что наше знание не выдуманный плод воображения, но оно
глубинным образом обеспечено благодаря связи наших научных теорий с тем, как мир
устроен на самом деле, наших научных законов с тем, каков настоящий порядок
устройства Вселенной. Конечно, когда мы в следующих наших трех вопросах подробно
познакомимся с идеями античных исследователей, у вас возникнет представление о
плюрализме, то есть множественности направлений. Можно утверждать, что между
мыслителями не было согласия и единства в самых фундаментальных вопросах о
природе материи, первичном, этом же едином, о причинности и так далее.
Действительно, особенно в поздней античности позиции стоиков, пекурейцев и
неоплатоников были непримиримы, они даже демонстративно по-разному стригли
бороды, чтобы подчеркнуть свою принадлежность к тому или иному философскому
направлению. Не искусственно ли мы тогда обобщаем под знаменем единого с большой
буквы всех без разбора представителей античной культуры, тем более столь
длительной и политически неоднородной? Уже у первых атомистов и взявшего их идеи
эпикура в природе базовых начала два атома и пустота их разъединяющая, и у того
же плотина, главного неоплутоника, единая, не только уж единая, распадается на
троицу, проявляясь во Вселенной и также в качестве ума с большой буквы и в
качестве души мира. Вот я сама долгое время пользовалась как штампом этим вроде
бы удачным выражением интуиции единого. Но после тщательного знакомства с
множеством разнообразных первоисточников античных авторов поняла, что правильнее
говорить об интуиции логоса, а не единого. Безусловно, не будет ошибкой
говорить, да если мы сегодня задумаемся над этим, почувствуем очевидность этого,
чем-то единым всё разнородное в нашем мире и правда как-то удерживается вместе.
Душа и тело, например, коллективные, индивидуальные, естественные, искусственные
и так далее. Другое дело, что это непостижимо и почему-то мы упираемся всегда
как минимум вдвоицу, в парадокс, в пары противоположностей. Неприступное это
единое, но всё же даже сам логос логики нас приводит к этому положению. На чём-
то же всё это должно держаться, иначе противоположности схопнулись бы и ничего
бы не было. Так что давайте впредь будем говорить, во-первых, о вкусе античных
мыслителей, к парадоксам, противоположностям, несовместимостям, а во-вторых, о
том, что несомненным для них было именно единство логоса во всём. что можно
своим умом настроиться на понимание фундаментальных законов Вселенной и
человеческого общества, которые, тем не менее, всегда включают в себя
двигательный парадокс, опорю. Наконец, поскольку идеалом был космос, размеренный
и упорядоченный противоположность хаосу, быть в мышлении древнегреческого
человека, по-настоящему быть означало иметь форму, быть целым, завершённым, то
есть имеющим границы, умеренным и при этом соразмерным остальному порядку мира.
Категория меры имеет для понимания античного видения ключевую роль. В
действительности реально быть, быть действовать, значит иметь меру, внутреннюю и
внешнюю соразмерность. Идеал человека быть живой мерой, то есть каждый раз, в
каждой уникальной единичной ситуации как бы примеривать себя к ней, соизмерять
всё и действовать размеренно, умеренно, согласно определённому порядку, а не
имитаться хаотично, беспорядочно действуя и не делать ничего сверх меры. Ничего
сверх меры тоже одна из своеобразных формул древнегреческой культуры. Согласно
такому пониманию, теперь становится ясно, например, почему богатые представители
античного общества, обычно к тому же воскообразованные, одевались скромно, не
пытались чрезмерно украшать свой дом или просто копить свои богатства в виде
золота. Умеренность во всём означало искусство гармонично упорядочивать свою
жизнь, цель которой отнюдь не могла быть мысленна в накопительстве или наоборот
в безмерном расточительстве. Тогда и быть познанным в качестве предмета или
явления мира означало быть включённым в гармоничное целое миро, быть понятым в
качестве оформленного и при этом соразмерного с другими элемента. Всё
беспредельное, а естественно, древним грекам были знакомы математическая
бесконечность и представления о безмерном мыслились существующими лишь
возможности. Их можно было вообразить себе, представить. Однако, поскольку в
мире космосе мы не сталкиваемся с такими вещами, а наоборот наблюдаем
упорядоченность форм и конечность всего внутримирного, то всему беспредельному,
неоформленному, аморфному, безразмерному, неумерному и так далее, античный ум
отказывал в действительном, реальном бытии. То есть, смотрите, при всей
похожести современности на античность, особенно позднюю, что я постараюсь вам
показать на следующей неделе, основополагающим антологическим различием наших
эпох является прежде всего разное отношение к категориям возможного и
действительного. Сегодня нам очевидным и несомненным представляется, что
возможность больше действительности. Ведь мы можем новоображать кучу всего, что
может произойти, чего в реальности непосредственно здесь сейчас нет. И на
основании этого мы поступаем и выстраиваем свою жизнь сплошь и рядом. Например,
живой классик, итальянский современный философ Пауло Лаверно абсолютно верно
подмечает, что, скажем, принимая человека на работу, работодатель смотрит не на
то, что человек реально делает, а доверяет ему, что у того гипотетически есть
способности для выполнения возможных задач. В свою очередь, работодатель обещает
сотруднику тоже возможности карьерного роста и определенных премий или повышения
со временем заработной платы. На деле не факт, что все это оправдается, но мы
полагаемся на такие воображаемые возможности. В Древней Греции надо было на деле
показать свое мастерство, чтобы получить признание, уважение и элементарное
вознаграждение за работу, показать, что ты умеешь, что ты уже реально сделал,
будь то атлетические спортивные трюки или глиняные горшки, песни или лошадиный
упряжь. Мы же на каждом шагу думаем, что у нас много возможностей пойти учиться
туда-то или туда-то, стать тем-то или тем-то, с этим ли человеком сблизиться,
куда поехать отдыхать, какую одежду для кого-то в случае купить. А что на самом
деле, каков настоящий выбор в сфере дел, поступков в мире, а не в голове. И эти
наши воображаемые возможности с удовольствием подпитывают рекламу, пропаганда и
социальные стереотипы. Мы сегодня очень мало смотрим на то, что действительно
здесь, сейчас реально есть. Мало всматриваемся в то, какими мы сами на самом
деле от природы являемся, а не какими нас хотят видеть в этих воображаемых
стереотипах. И, к сожалению, немало людей проживают в жизни, лишь строя планы в
голове и представляя себе, что у них есть возможности. На деле же из-за этого
ничего не успевая по-хорошему воплотить. Так вот, греки видят не так. Они
напоминают нам, что действительно больше возможного. Все наши воображаемые
возможности лишь у нас в голове. Они не имеют воплощения в реальности. А что
реально? Что в мире по-настоящему? Вот это греков интересовало прежде всего.
Конечно, мы отметили, что они первые начали специально практиковать
теоретические занятия, которые как раз именно в нашем воображении целиком
происходят. Но, обратите внимание, для античных исследователей разум
воспринимался как инструмент. Инструмент осмысления действительности, а не в
качестве пространства идолов и мечтаний, которым мы отдаёмся в рабство вместо
жизни в реальности. Безусловно, у человека любой эпохи существует соблазн витать
в облаках и погрязнуть в мечтаниях, не замечая того, что на самом деле
происходит. Однако, сами антологические условия, внимание античности к
настоящему, подлинному, действительному, к природе, как она сама по себе есть и
к человеку, какие поступки он совершает перед лицом сообщества, побуждали, что
называется, заняться делом и воплощать, в том числе, прекрасные произведения
искусства в реальность. Это очень важный момент для понимания всей
древнегреческо-философской научной мысли, поэтому постарайтесь тщательно
запомнить такой ключ к эпохе, антологические основания древнегреческой культуры.
Перейдём теперь в завершение нашего первого вопроса, отталкиваясь от этих самых
фундаментальных оснований эпохи к гносеологическим, этическим и эстетическим.
Интересно здесь прежде всего то, что в античности они были фактически
неразличимы, перетекали друг в друга. Основным источником познания мыслился тот
самый логос, несомненно, антологический принцип, то, что через всё в мире течёт
и всё упорядочивает как в космосе, так и в человеке. Логос — многозначное
понятие. Давайте кратенько здесь раскроем спектр его смыслов и отметим его
логоносеологическое значение для античного способа познания. Во-первых, это
слово или речь. В этом смысле, например, название науки филологии происходит от
греческого «любовь к слову». Во-вторых, не теряя своего значения речи, логос,
как вводилось это понятие первыми древнегреческими философами, мыслился как
некий всеобщий порядок Вселенной, который пронизывает всё своим течением. Логос
и однокоренное с ним лего, как конструктор назван, собираю, означает упорядочно
конструировать. Слушающаяся этого единого порядка природа кампуса образовывает
своё бытие с всё пронизывающим течением космической речи, делающей мир
членораздельным, оформленным. В-третьих, под логосом понимается мысль или
упорядочно оформленное мышление, мышление в понятиях, что на современный манер
можно трактовать как научную теорию. С данным значением как раз коррелирует
название дисциплины логики, которая нацелена на обучение правильности
упорядоченности мыслей и суждения. Сегодня при выяснении темологии название
большинства научных дисциплин, заканчивающихся на логии, типа биология,
психология, геология и подобных, логос в соответствии с данным значением так тут
как учение, что соответствует в частности пифагорейскому оттенку понимания этого
слова. Так, обучение правильному мышлению, настройка своего ума на слышание,
видение, текущего через все единого порядка обеспечивало по мнению
древнегреческих исследователей подключение к логосу как бы программе, которой
запрограммирована вся вселенная. И тогда всматривайся, анализируй, схватывай
порядок и придешь к истинному знанию. Так что истинное знание это логично,
полученное путем умозрительного постижения действительности через рассуждение,
которое обязательно соотносится с тем, как все реально есть, как мир нам
является. Поэтому не удивляйтесь, когда мы начнем рассматривать воззрение разных
античных мыслителей. Они не согласуются друг с другом содержательно. Каждым
авторам предлагаются разные объяснения причин природных свойств материи. Один
будет утверждать, что первое вещество, из которого в разных модификациях
составляются вещи, имеет свойство жидкости, поскольку все течет, изменяется, да
и без влажной среды нет жизни. Другой будет показывать, что в основе всего
огонь, как энергия, которая все в разной степени наполняет своим действием. И
попробуйте на основании современной физики поспорить, что материя в пределе не
энергия. Третий выскажет предположение о том, что материя в пределе число. И все
вещества формируются как отражение комбинаций этих чисел, а главное
соразмерность, гармоничное отношение. И попробуйте, исходя из современных
квантовых представлений о химической связи, поспорить с тем, что соединение не
обязано своим существованием как раз гармоничной согласованности энергетических
уровней, на которых находятся электроны, и которые никак попало образуют связи,
а в строгой порядоченности по определенным квантовым числам. Ну и так далее. Все
эти объяснения реально логично выводятся из определенных основоположений,
которых в рамках своего учения каждый исследователь придерживался. И более того,
звучат равноубедительно даже для нас сегодняшних, хотя и на непривычном языке
выражено то, что мы в своих научных теориях сегодня знаем. Это происходит,
поскольку каждая такая концепция рационально схватывает в действительности то,
что на самом деле имеет место. Другое дело, что каждый схватывает что-то со
своего уникального угла зрения, как бы какую-то более близкую грань в изучаемом
и не может охватить абсолютно все многообразие, выразив в едином
непротиворечивом принципе. Так что конкуренция познавательных конструктов наших
теорий одного и того же это нормально в любое время и античность, несмотря на
стремление к единству и попыткам его схватить выразить, не исключение, а скорее
первая научная эпоха, которая задает такое правило многогранного рассмотрения. И
тут, ребята, до меня на новом уровне доходят идеи Артегии Гассета. Помните, на
прошлой теме я вам цитировала про неизменность истины и изменчивость нашего
знания на примере формулирования Ньютоном законов всемирного тяготения. У Ганса
Георга Гадемера, выдающегося современного разработчика философской герминевтики,
есть подобная идея об истории как истории переосмысления понятий. Мы тоже это
затрагивали на прошлой неделе по поводу разного смыслового наполнения
философских категорий в разной эпохе. И вот банальные вроде бы идеи, но я читаю
античные первоисточники и проводя параллели с современной наукой прямо
прочувствовала на каком-то качестве на новом уровне для себя. Реально, ребят,
смысл знания, прозрение человека в сущность мира, порядка вещей вообще не
меняется. Но язык устаревает, и каждому новому поколению приходится фактически
изобретать новый язык, более понятный, чем те записи прошлого и трактаты
древних. Переоткрывается постоянно одно и то же, потому что логично же,
фундаментально природа не поменялась. Вселенная та же человек, так же
биологически тысячелетия, десятки тысяч лет назад. Но понятая кем-то и
выраженная языком его культуры истина может быть непонятна другому, даже
соотечественнику, скажем, его внуку. Непонятно в силу как раз фиксированности в
каких-то формулировках, уже не работающих для сознания новых поколений в новых
условиях. Конечно, существует прогресс как знания, так и мысли, но некоторые
фундаментальные законы природы и существования человека остаются теми же. Мы
просто сквозь призму своих социокультурных условий переводим их каждый раз на
язык более понятных нам мировоззренческих оснований. Вот поэтому, ребята, мы
изучаем античных авторов. Они уже максимально разработали то мыслительное
пространство для науки и философии, в котором мы до сих пор движемся, из-за
границы которого вряд ли человек выйдет, по крайней мере, пока он человек. По
поводу преобладающих методов познания можно после этого нашего разговора уже
даже не комментировать. Мы только что проговорили о возможном и действительном
вонтологических основаниях. Вспомните теперь со второй темой, чем отличается
наблюдение от эксперимента и соотносите со следующим соображением. Поскольку для
нас современных важнее и больше кажется возможным, в нашей науке с нового
времени процветает эксперимент. В отличие от наблюдения, которое происходит в
естественных условиях, эксперимент создаётся в искусственных. То есть он нацелен
на раскрытие не того, как природа сама по себе естественного действительности
есть, а того, какой она может быть, как она может действовать. То есть цель
раскрытия потенции, возможностей, а не рассмотрения актуальной действительности,
как в наблюдении. Безусловно, каким бы эмпирическим методом мы не пользовались,
теоретическое осмысление будет лежать в основании и для античных исследователей
важно было рационально осмыслить наблюдаемое, сформулировать на логичных
основаниях красивые, предельно простые обобщённые законы, а не просто
зафиксировать описание происходящего, что и так испокон веков до них все делали.
Но самое главное, чтобы вы понимали, я выражу это фразой Анатолия Варьяновича
Ахутина, грекам показалось бы парадоксальным, если бы кто-нибудь решил изучать
естественное, то есть природу и естество неестественными методами. В общем-то,
тоже абсолютно логично. В природе ведь есть только то, что есть, и возникает то,
что возникает, так наблюдай хорошенько за действительностью и познаешь всё. Ведь
возможностей в реальности нет, могло бы быть иным, но не иное, вот такое, какое
есть. И зачем тогда забывать себе голову фантазиями, если мы вот в этом, именно
в таком действительном мире живём, а не в придуманном. Ну а для современных
людей это логические условия такие, что им легко впадать в проживание
ненастоящей жизни, то есть в ненастоящем воображаемом мире возможностей.
Наконец, в этом свете не менее естественно, что знание развивалось обо всём
подряд. Нет, наверное, такого феномена или такого объекта, которого бы не
коснулось этичное любопытство. Так что в эту эпоху было создано знание по всем
областям действительности, зародились практически все наши современные науки, от
физики до истории, от биологии до психологии. Особенностью же эпохи является то,
что не выстраивалась особая иерархия дисциплин, ни в философии, ни в сфере
научного познания, да и между философией и наукой граница не проводилась. Даже
попытавшийся аранжировать по степени значимости для человека дисциплины
Аристотель сам создал головокружительную по своему охвату систему знаний просто
обо всём и равно был захвачен как подробнейшим наблюдением за животными, так и
мысленным созерцанием первопричин всего сущего, как пониманием причин
метеорологических явлений вроде молнии и радуги, так и разбором устройства души.
То есть, можно, безусловно, говорить особенно про позднюю античность после
Аристотеля, что намечается ценностное превознесение метафизики, первой
философии, которая должна прояснять самые фундаментальные первоосновы всего, но
она важнее, поскольку более фундаментальная. И, как Аристотель говорит, без
понимания первых причин любое наше знание будет шатким и фрагментарным. И,
конечно, чтобы упорядоченно мыслить и не совершать ошибок в познании, неплохо
овладеть инструментарием логики. Однако, это просто вспомогательная, буквально,
инструментальная дисциплина. Остальные сферы исследования уже отталкиваются от
определенным образом выстроенной антологической базы, но на которой равна
строится как этика, так и физика, как медицина, так и политика. Что важнее из
этих наук? Вот для античных мыслителей было очевидно и несомненно, что все эти
области одинаково значимы и все их надо развивать. Это дает нам возможность, в
конце концов, плавно перечечь в этические основания и эстетические идеалы.
Принципы разумности, гармоничной упорядоченности, соразмерности, это
одновременно не только антологические принципы, но и этические и эстетические
идеалы. Страшное слово «калок аготия» обозначает одновременно эстетическую и
этическую добротность. Происходит от древнегреческого выражения «калос каи
аготос», буквально «прекрасный и хороший» или «красивый и добрый». Этот принцип
предполагает совместное обязательное сочетание в человеке физической красоты и
добродетельности натуры. То есть был даже такой стереотип, если человек некрасив
внешне, то, скорее всего, он не особо хорош по характеру. Например, Сократа
завистники порицали за его курносость. Это считалось у греков некрасивым, хотя,
на мой субъективный взгляд, вполне обычный, непротивный мужчина, тем более с
отменным здоровьем. Он специально закалялся и регулярно выполнял физические
упражнения, а в Пелопонесской войне принимал участие в качестве гоплита,
пехотинца с тяжелым вооружением, которое носить на себе и которым владеть было
дело исключительных по силе и выносливости воинов. Но своим обидчикам мыслитель
отвечал, что хотя и не подходит к общепринятым канонам красоты, он изо всех сил
старался воплотить в своей жизни, по крайней мере, идеал справедливости,
рассудительности, мужества, внимательности, добродушия и щедрости. Про
умеренность во всем, как принцип жизни, мы уже проговорили в антологических
словах, это, я думаю, тоже особо можно не комментировать. Имеется в виду, что
идеалом античного гражданина было просто при этом опрятно и гармонично
одеваться, не демонстрировать ни умеренность в расходе средств или питаний, а
также уделять внимание равно своему физическому и духовному развитию, чтобы во
всем была пропорциональность чувства, меры и соразмерности, также и эстетический
идеал, который можно, конечно, выразить как красота в простоте, гармонии и
естественности. Очень напоминает, кстати, японский принцип ваби-саби, который
тоже не только эстетический, но и имеет глубокий антологический смысл. Так что
в, казалось бы, совершенно разных и похожих друг на друга культурах мы находим
сходные фундаментальные общечеловеческие основания. Про этический рационализм мы
подробнее обсудим на следующей неделе, когда будем говорить о Сократе и Плутоне.
Но пускай он тут у вас тоже будет в философских основаниях, означает этот
принцип буквально следующий. Если я разумом, а рацию одно да не разум, понимаю,
что такое благо, добро, справедливость, то я принципиально не могу себе
позволить поступить злым или несправедливым образом. Обратите внимание, знание,
гносиологическая категория, неразрывно связано с поступком, а это уровень этики.
Таким образом, подытожим, для античного мышления не характерны разделения таких
областей, как антология, гносиология, этика и эстетика. Поскольку
фундаментальным образом в своей основе всё едино, то ум, красота и добротность
просто разные грани проявления одного бытия. Оно к нам и в каждом человеке
должно являться полнотой, только если всеми оттенками играет, когда человек и
тело своё совершенствует, упражняет, и душу развивает, и разум занимает важными
задачами, рассуждая логично, и поступает по справедливости, по совести. Всё
красиво и стройно. Но, ребят, напоследок примечанием обращу ваше внимание вот на
что. Чтобы у вас не создавались идеализированные впечатления об античности, мы
тут отметили именно идеалы, то есть то, к чему человек стремился. А как вы
догадываетесь, реальные люди сплошь и рядом далеки от идеала, поэтому, несмотря
на высокие этические и эстетические принципы, некорректно было бы превозносить
античность и думать, вот времена-то были, что, мол, все, как на подбор, умные,
добрые, красивые. Реальные античные люди, многие, естественно, как и наши
современники, поддавались соблазнам далеко, не все были симпатичны и
пропорционального телосложения, не всегда могли умерить свой пыл, гордыню,
злогу, нерезко поступали несправедливо и так далее, как и все обычные люди в
любую эпоху. Однако, безусловно, среди них, опять же, как и все времена, были
самые выдающиеся представители нашего рода, которые реально своей жизнью
пытались воплотить эти идеалы, и мы через перерыв начнем, наконец, знакомиться с
теми из великих эллинов, которые посвятили себя исследованиям. На этом завершим
первый вопрос, пятой темы. Всем спасибо огромнейшее за внимание. Если есть
вопросы, с удовольствием отвечу. А ребята, вы тут. Итак, систематизировав в
первом вопросе антологические основания принципа античного мировоззрения,
перейдем ко второму, который в нашем списке к экзамену носит название «Первые
натурфилософские учения античности». Вопрос этот тоже очень объемный, будет как
первый сегодняшний, хотя пунктов в нем мы больше выделяем, просто потому что
множество натурфилософских школ возникло, и понемногу в каждую надо заглянуть.
Чтобы рассматривать сами учения древнегреческих мыслителей, нам прежде всего
нужно разобраться с понятием натурфилософии, которое, к тому же, пригодится вам
в следующих темах. Затем мы должны будем понять, в чем специфика античных
натурфилософских учений, какой круг вопросов объединяет ранних греческих
мыслителей. Следите за заголовочками, как обычно, все пункты сейчас
последовательно разберем. Для начала разберем, что такое натурфилософия вообще.
Существуют две трактовки данного термина. Чаще всего под натурфилософии
понимается философия природы, буквально с латыни «натура природа», то есть это
целостное осмысление природы как общего понятия, через которое, возможно,
становится объяснение отдельных вещей и познание ее фундаментальных законов.
Данное определение предполагает, в частности, что познание природы и ее частей
происходит умозрительно, теоретически, философски, в том смысле, что знание о
природе начинается не с практического взаимодействия с отдельными ее областями и
явлениями, а с построения обобщенного представления благодаря созерцанию
целостной сущности данного понятия и философской рефлексии. Отдельные явления и
области затем изучаются, в том числе эмпирически, в свете такого представления о
природе в целом. Так, натурфилософия, как осмысление категории природы, может
иметь место в рамках любой культурно-исторической эпохи и находит свои
воплощения, например, в метафизике природы нового времени, в стихийном понимании
природы мыслителями романтизма это XIX век, а также в современных
междисциплинарных парадигмах вроде энергетики. Однако, с точки зрения нашей
дисциплины истории и философии науки, натурфилософия представляет собой период
становления науки, берущий свое начало в древнегреческих учениях о первоосновах
и первопричинах мира и заканчивающийся в натурфилософский период в магически-
пантеистических представлениях о природе в эпоху Возрождения. В этом смысле
натурфилософию именно как период развития науки противопоставляют
экспериментальному математическому естественному знанию оформляющемуся с XVII
столетия и последующему развитию науки отграничивающей свою специфику от
философской методологии. Помните на прошлой теме в конце мы делили в
отечественной трактовке историю науки ее на два крупных периода натурфилософский
и период научной рациональности. Так вот в целях нашего курса продуктивно именно
таким способом понимать натурфилософию, поскольку с помощью данного разделения
удобно обобщить особенности нескольких эпох становления науки. Натурфилософский
этап развития науки таким образом включает в себя специфику научного знания и
научного познания в рамках следующих культурно-исторических эпох антично-
середневековья и возрождения. Объединяющими чертами развития науки в данных
периодах являются такие особенности, как преимущественно теоретический
умозрительный подход к познанию. Несмотря на то, что исследователи всегда
наблюдали и проводили опыты, отталкивались они прежде всего от целостных
фундаментальных представлений. Эмпирические методы были вспомогательными и
использовались для уточнения или подтверждения теории. Философское осмысление
природы как целого без четкого разделения на привычную нам область и реальность,
отражающаяся в современном дисциплинарном делении естественных наук. Это
логично. Зачем иерархизировать различные области, если мы изучаем природу как
нечто единое в своем существе? Переплетаться будут все природы, так сказать,
природа камня, ветра, природа превращения и живая природа, природа души. Во всем
будут просвечивать эти единые основания, о чем также следующий пункт.
изоморфность представлений о природе всех вещей вследствие единого основания
понимания категории природы. Изоморфность это сходство различных сущностей по
форме. В данном случае имеется в виду, что в основе любого естественного объекта
или явления видится нечто единое, связывающее его со всем остальным в мире.
метафизическое ориентированность натурфилософского познания. Это означает
преобладание проблематики поиска первоначал и первопричин всего сущего, чем
занимается такой раздел философии как метафизика. О ней мы подробно будем
говорить на следующей теме, когда будем разбирать труды Аристотеля, это он ввел
само понятие метафизики. Несмотря на отмеченные сходства, в рамках каждого
культурно-исторического типа мышления натурфилософия приобретает и уникальные
черты, характерные только для данной эпохи. В связи с этим здесь нам необходимо
для сегодняшней темы выделить особенности именно античной натурфилософии. Для
этого ответим на три вопроса. Как понимается категория природы? Каков основной
источник познания природы? И в чем специфика античного метода познания природы?
Здесь мы также будем опираться и на только что разобранные гносеологические
основания пройдем параллели. В дальнейшем данной схемой можно будет пользоваться
при выделении специфических черт натурфилософии других культурно-исторических
эпох. Итак, прежде всего само понятие природы для древнегреческого сознания это
фюзис. Что это значит? Природа как фюзис природа в широком смысле слова. И
природа вещей, и природа человека, и природа космоса, и природа души. Это
естество. То есть естественное положение вещей такое, как оно нам
непосредственно представляется без специфического ракурса рассмотрения и каких-
то неестественных стесняющих условий. По Аристотелю фюзис противопоставляется
техне, искусству в широком смысле искусственному специальной деятельности
человека, направленной на создание объектов, которых самих по себе в природе не
могло бы возникнуть естественным путем. Аристотелевский пример. Дерево в форме
дерева растет само, а вот мебель из дерева нуждается в искусстве плотника,
столера. Даже если мы воткнем в землю кровать из невыделанных сырых веток,
кровать не вырастет, по природе разовьется опять только дерево, если палки
прорастут. Русское слово природа в этом смысле показательно. Она при родах в
двух смыслах, при рождении, которое происходит само, и при определенных видах
родах того, что возникает. Тогда античная натурфилософия это фактически физика,
исследование того, что, как и благодаря чему есть само. Аристотель, обобщающий в
своих трудах эти представления называет своих предшественников коллег
натурфилософов физиологами, в смысле изучающими порядок природы фюзис. Так что
задача в том, чтобы изучить что и как рождается, в каких видах существуют,
поэтому осмысление движения и преобразования всего сущего устройства Вселенной и
жизни души вещи одного порядка, подлежащие физическому в античном смысле слова
рассмотрению. Основным источником познания природы в Древней Греции, как мы уже
в предыдущем вопросе отметили, про познание вообще, был логос, тот самый
порядок, пронизывающий Вселенную, согласно которому, как и в макрокосмосе, все
было организовано и двигалось определенным образом, так и в человеческом разуме
могло быть все логично отражено, то есть ни миф уже в полной мере не был
источником формирования знаний о природе, ни какие-то другие виды представлений,
например, авторитетные тексты, основным источником было разумное созерцание и
теоретическое осмысление в его соразмерности самой природе, хотя различные
учения о природе формировались, каждый мыслитель все равно подвергал их
осмыслению и соотнесению собственным видением. Античные физиологи от Фалеса до
Лукреция умели начинать сначала с самой природы в ее непосредственной открытости
логическому осмыслению, однако люди, даже в немля упорядоченной вселенской
текучести речи Логоса могут не понимать ее смысл, по мнению, например,
Гераклида, пока не обучаться ясности ее слышания и соразмерному ей проясняющему
рассуждению, в чем, собственно, виделась обучающая функция философии. Для
античных исследователей поэтому не возникала проблема истины, какой она сейчас
стоит перед нами, ведь Логос течет через все, как через природу, так и через
человека, поэтому все возможности для познания открыты, и затуманить его может
только наше собственное нежелание учиться логичности. Тогда как мы можем
охарактеризовать методологию познания природы в эпоху античности? Этот метод
принято трактовать как теоретический, умозрительный, рациональный, логический, в
том смысле, что знание о природе формировалось на основании логически
построенных рассуждений, в рамках которых мыслитель оперировал категориями,
понятиями, выделяя с их помощью сущностные черты изучаемого. Основной первичной
операцией было абстрагирование, то есть выделение из видимого, наблюдаемого
общих закономерностей свойств форм. Предельные абстракции, число и
геометрическое представление, поэтому с ними неизменно переплетаются
натурфилософские учения античности. Однако цель числового и геометрического
выражения представлений о природе недостижения точности или абсолютности, как мы
сейчас подробно посмотрим, греки принимали фундаментальную парадоксальность и
понимали невозможность абсолютного знания. Главное понять упорядоченное
устройство природы в доступных нашему познанию категориях пропорциональности,
соразмерности и гармоничности. Числами и геометрическими фигурами необходимо
уметь оперировать в своем уме в качестве тренировки теоретического мышления или
как объектами, помогающими отвлечь зрение от обыденного наблюдения вещей к их
умственному созерцанию. Короче говоря, математика это как бы язык логоса, того
порядка, которым весь мир дышит и который следовательно человеческому разуму
тоже доступен, если себя соответствующим образом тренировать. И еще одна важная
мыслительная процедура, практиковавшаяся античными авторами, это аналогия. Тоже
однокоренно с логосом. Сопоставление, уподобление, перенесение характеристик,
скользь одних объектов на другие. Буквально аналогия означает равное отношение
равенства логосов. Итак, первые греческие философы начали активно практиковать
теоретическое видение окружающей действительности, задаваясь вопросами, то есть
рефлексируя. И первое, о чем, естественно, было спрашивать, это откуда взялся
мир или с чего все началось. То есть до того, как мифы рассказывают о первых
богах, до них что-то было? Или мир в том или ином виде существовал вечно? Есть
ли у мира начало? Далее в свете интуиция единого логоса спрашивалась о том, из
чего все состоит. То есть было понятно, что все фундаментальным образом из чего-
то одного состоит, но что собой представляет эта единая основа было неочевидно и
в мифах не было объяснено. Наконец, важнейшим был и вопрос о том, почему в мире
все именно так устроено, как нам является. Допустим, все устроили таким образом
боги, однако почему они выбрали и выбирают именно такой порядок вещей, такую
организацию всего, такие цвета, звуки, материалы, формы и так далее. Чем они
руководствуются, делая мир именно таким, с такими красками, такими стихиями,
небесными светилами, особенностями живых существ? Почему не по-другому? Ведь
можно вообразить, что было иначе? Короче говоря, как метко замечает Жиридалёс,
первый философ, натуралист, он говорит о природе, а не о богах. Данный спектр
вопросов образует собой единое смысловое поле проблематики, которое принято
обозначать, как проблема первоначал и первопричин мира. Однако отметим сразу,
что подобные вопросы о начале мира, первооснове, первопричинах всего — это
вечные философские вопросы. Вечные в том смысле, что ими задаётся любой человек
в любую эпоху. Вспомните себя в детстве в вопросе, начинающиеся со слов
«почему?», «как?», «из чего?», «зачем?» никому из нас не чужды. Тем не менее,
несмотря на одну и ту же форму вопросов, в контексте каждой эпохи в определенных
социокультурных условиях ответы даются различные по своему содержанию. Они
оказываются как бы нагружены культурными символами времени, неся также отпечаток
методологии, посредством которой рождались те или иные идеи отвечания на данные
фундаментальные вопросы. К особенностям первых философских сочинений относятся и
то, что они в значительной степени не сохранились, дойдя до нас преимущественно
в изложении и интерпретации более поздних мыслителей. Связано это с тем, что
книги существовали на материальных носителях в небольшом количестве экземпляров,
труды переписывались от руки и таким образом распространялись. Но во время
каких-то войн, природных бедствий или случайных пожаров библиотеки могли быть
разрушены, сохранялись только наиболее популярные, обсуждаемые и цитируемые
произведения. У авторитетных философов естественным образом появлялись ученики и
последователи, которые передавали учения наиболее выдающихся мыслителей. Так
органичным образом складывались философские школы, то есть не специальные
учреждения образования, но спонтанно возникающее сообщества, мыслителей,
учеников и последователей, которые разделяли определенную концепцию. Как мы уже
проговорили в первом сегодняшнем вопросе, отвечать на вопрос о первопричинах и
первоначалах всего можно по-разному. В отсутствии каких-либо ограничений на
свободу мысли сложилось так, что различные мыслители находили отличающиеся друг
от друга ответы на данный вопрос. Все они были достаточно убедительными,
последовательными, логичными, поэтому даже несовпадающие концепции нельзя
называть ложными или выделить только одну из них в качестве единственной
истинной. Различия данных концепций обусловлены тем, что каждый человек видит
мир буквально со своей определенной точки зрения. Никто не может охватить своим
взглядом и учесть абсолютно все. Для разных людей ближайшим и более понятным
оказываются разные вещи, поэтому мы все склонны даже принадлежа одной эпохе,
одной культуре объяснять какие-то вещи по-разному через различные, при этом все
равно умудряться говорить об одном и понимать друг друга. Каждый может через
свое увидеть истину, просто с каждого уникального угла зрения нам будет видеться
несколько иначе. Собственно, такой прорыв к истине нам и важен в философских и
научных идеях. Они подмечают какой-то ход, который, мы говорим, открывает нам
глаза на что-то. Попробуем таким образом познакомиться с различными
направлениями и концепциями древнегреческой мысли. Хронологически первой
философской школы Древней Греции является Милецкая по названию полиса Милет в
Ионии. И первым античным философом считается Фалес Милецкий. Его имя вам
наверняка знакомо со школьных уроков математики. Все когда-то слышали о теореме
Фалеса, о пропорциональных отрезках и параллельных прямых. Без сомнения, это был
выдающий человек с обширными познаниями в различных сферах, путешественник,
получивший образование в процессе своих странствий и изучения культур других
народов египтян, финикийцев, ледийцев. Также известно, что Фалес занимался
торговлей, приумножив свое состояние за счет сдачи в аренду прессов для
оливкового масла благодаря, как бы мы сейчас сказали, экономическому
прогнозированию рынка в совокупности с влиянием климатических факторов на
урожайность. Примечательно то, что во всех сферах своей жизни Фалес
целенаправленно практиковал теоретическое мышление. Например, в 585 году до н.э.
он с точностью предсказал солнечное затмение. Ранее никто этого с достоверностью
не предсказывал. Или сохранили свидетельства, что в Египте Фалес
продемонстрировал свой метод для вычисления высоты пирамиды, которую никто до
него не мог определить, поскольку египтяне не представляли, как можно померить
иначе, чем линейкой. Фалес воспользовался принципом пропорции, измерил длину
тени от пирамиды, которую та отбрасывает в определенный час на ровную
поверхность земли, сопоставив ее с другим, более маленьким предметом, палкой,
длину которой и длину тени, от которой измерить не составляет труда. Так,
сопоставив соотношение длин, он вычислил недостающие, неизвестные значения
высоты пирамиды. Ряд подобных случаев биографии Фалеса буквально обожествил
исследователя, о нем уже при жизни по всей ладе ходили легенды. На самом деле
мыслитель десятилетиями странствовал, обучался, всматривался в действительность
неукасающим интересам, много наблюдал за ходом небесных светил, за погодой,
развитием растений, ростом живых организмов и так далее. А главное, задавался
вопросами о причинах наблюдаемого и старался найти ответы. Это обеспечило ему на
века звание первого в истории философа и ученого. Имя Фалеса неизменно открывало
список семи мудрецов Древней Греции. Считается, что именно ему принадлежит
известнейшее высказывание Гнозиса Автон «Узнай себя или познай самого себя». К
сожалению, сочинения самого Фалеса, как говорят написанные в стихах Гегзаметром,
до нас не дошли, но его учение сохранилось в уцелевших фрагментах его
непосредственных учеников и трудах более поздних мыслителей, ссылавшихся на
произведения милецов. Как философ, пытающийся глубинным образом постичь природу
окружающей действительности в ее единстве и полноте, Фалеса интересен прежде
всего постановкой вопроса о первоматерии или первовеществе, из которой все
состоит и которая им была названа архе, первая древняя иллюзшая в основании. К
VII веку до н.э. у многих народов уже сложились не только чисто мифологические
представления о мире, но, конечно же, и мнения, выводимые из непосредственно
наблюдаемых феноменов, которые мифологизировались. Так достаточно
распространенным было представление о четырех, иногда пяти стихиях, материя и
движение которых образует все видимое в мире многообразие. Данные стихии огонь,
воздух, вода, земля и эфир, как некая высшая спустанция, обожествлялись, с их
силами связывались природные и климатические явления, им приписывались
сверхъестественные свойства. Но с современной точки зрения это аналоги скорее
агрегатных состояний веществ, чем химических элементов, хотя тоже элементами
назывались. Важны в этих первостихиях свойства. Например, землистыми считались
все твердые тела, с водой отождествляя все жидкости, с воздухом все газообразные
тела и так далее. И вот интересно, что Фалес, изучивший воззрение разных народов
на эти стихи, предпринял как раз в свете интуиции Логоса попытку выделить единое
основание для всего. Не считая, что основанием может быть несколько, Фалес
сопоставил известные ему представления с собственными наблюдениями из обыденной
жизни и со своей позиции. Мыслитель увидел воду в качестве первоначала единая
основа всего. без воды все живое вскоре погибает, вода же имеет животворящую
силу, при поливе растения дают хороший урожай, моря и реки дают рыбу, все живое
сдержит влагу, течение жизни похоже на течение воды, и превращение веществ
представляет свой процесс, за течением которых тоже можно наблюдать.
Естественно, вода представлялась в таком качестве в более широком смысле, чем мы
сегодняшние понимаем под этим словом HDO, скорее как влага, вообще как жидкость,
текучая субстанция, которой все наполняется. Другие стихии по мысли фолеса
получались путем превращения воды, замерзая, вода становится твердой, так
поразовались твердые землистые тела, от испарения воды возник воздух, при
чрезмерном перегревании первое вещество переходит в высокоэнергетическое
состояние, становится стихией огнем. На основании данного философского учения
фолес создает, вернее, уточняет имеющиеся мифологические представления о земле,
на которой мы живем, и о небе, в котором наблюдается периодически размеренный
ход светил. Уверенный в том, что все состоит из одной первоматерии, мыслитель
впервые логически выводит, что небесные тела, солнце, луна, звезды, видимые
планеты не божественные сущности, а состоят из того же самого, но они чрезмерно
разогреты, поэтому мы видим их светящимися, горящими. Это на то время просто
революционные идеи. Безусловно, нам сегодняшнее понимание воды в качестве
первого начала кажется ложным, потому что мы привязываемся к словам. Для нас
вода и элемент означают не то же самое, что для древних. Однако обратите
внимание не на содержание учения, а на метод, на логически выстроенное
рассуждение. Тогда становится понятно, что в свете обозначенного нами духа эпохи
и ее антологических оснований из повседневных наблюдений достаточно естественно
было придать именно чему-то жидкому, зримо присутствующему повсеместно, статус
материального, текучего, изменчивого основания мира, из которого все состоит.
учения Фалеса на фоне славы о величии его деяния и его личности привели к нему
множество учеников, жаждущих также постичь порядок мироздания и прикоснуться к
мудрости. Наиболее известными последователями эмилетской школы философии
основанной Фалесом были его непосредственный ученик Анаксимандр и ученик
Анаксимандр Анаксимен. Анаксимандр, судя по сохранившимся отрывкам из его
произведений и свидетельствам о его жизни, был величайшим философом своего
времени, старавшимся глубоко проникнуть мыслью к первосновам. Он многое сделал
своим творчеством для того, чтобы язык философии отделился от художественного
стиля изложения в стихах, а содержание было достаточно самостоятельным от мифа.
В частности, он писал свои труды в прозе. Вслед за своим учителем он задавался
вопросами о первоначалах и, соответственно, пытался рационально объяснить, как
все в мире устроено. Об идеях Анаксимандра сохранилось следующее. Наблюдая
превращение друг в друга четырех стихий, Анаксимандр не счел возможным взять
одну из них за основание, но принял за него нечто от них отличное, чтобы
источник рождения был изобильным. Это особое единое начало всего лежащего
основания мыслитель назвал Аперон беспредельное. Важнейшей категории в культуре
Древней Греции, как мы сегодня уже отмечали, была категория меры или предела.
Мера мыслилась как нечто упорядочивающее, то, благодаря чему все оформилось
именно в таком виде, как это нам предстает. Однако, посмотрите на потрясающую
логичность рассуждения Анаксимандра. В свете вопроса о первичности мера
оказывается вторичной. Она оформляет, ограничивает, а это необходимо делать с
уже чем-то имеющимся. Значит, если ставится вопрос о начале, лежащем в
основании, то есть о том, из чего все состоит, оно должно иметь свойство
неоформленности, неограниченности, чтобы посредством соединения с мерой дать уже
ощущаемые оформленные ограниченные вещи. Также свою первую материю Анаксимандр
наделяет свойствами неуничтожимости и всеобъемлемости. Это потрясающая по своей
логичности и теоретической абстрактности мысль. Анаксимандр славен и многими
другими открытиями и изобретениями. В частности, он первым последовательно
изслужил идею о движущей силе противоположностей. Возражая своему учителю, он
считал, что вечное движение более древнее начало, чем влага, и что благодаря ему
одно рождается, а другое погибает. Из единого выделяются соединенные в нем
противоположности. Рождение происходит не через изменение стихии, а через
обособление благодаря вечному движению противоположностей. О каких
противоположностях идет речь? В мире мы сталкиваемся с ними повсюду. В ощущениях
теплое и холодное, тяжелое и легкое, влажное и сухое, в ходе наблюдений,
рождение и умирание, мужское и женское, большое и маленькое, наконец, в
абстрагирующем мышлении, добро и зло, четное и нечетное, и самое фундаментальное
для максимандра противоположность беспредельного и предела. Эти
противоположности влияют друг на друга и как бы своим неравенством запускают
движение максимандру принадлежит первая формулировка закона сохранения материи.
Из чего все вещи получают свое рождение, в то все они и возвращаются, следуя
необходимости. Все они в свое время наказывают друг друга за несправедливость.
Интересна идея о наказании вещей. Что за несправедливость? Несправедливость в
том, что одна картина занимает собой все пространство. Однако, поскольку все
движется и изменяется, одни и те же вещи не могут постоянно занимать собой все
место. Они должны уходить со сцены в свое время. Так и люди стареют постепенно и
в силу справедливости на их место со временем приходят более молодые. Кстати, о
происхождении человека. Анаксимандр тоже высказал потрясающие идеи, которые
сегодня разделяются большинством ученых в рамках эволюционной теории. В
последствии я говорил о том, что первый человек произошел от живых существ
другого вида, а жизнь вообще родилась в океане, из которого со временем на сушу
вышли существа, покрытые чешуей. Их чешуя лопнула, и вскоре они изменили свой
образ жизни. Поразительно, до какой степени данный маститель умел всматриваться
в природу и замечать, например, сходные черты живых существ разного вида,
предполагая их последовательное возникновение. Аноксимандр также начертил первую
географическую карту известного на тот момент мира, занимался вычислениями,
касающимися движения небесных тел, предложил впервые так называемый небесный
глобус, изображающий движение небесной сферы в кругу Земли, изобрел гномон для
усовершенствования солнечных часов с целью введения измерений и вычислений
посредством фиксации движений Солнца. Аноксимандр постарался уточнить учение
Аноксимандра о первоматерии в свете распространенных тогда идей о логосе как
порядке, которым как бы дышит космос. В сохранившихся отрывках его произведений,
которые были написаны независимословатым стилем и в свидетельствах более поздних
авторов говорится, что Аноксимандр все причины вещей свел к беспредельному
воздуху. Воздух понимался Аноксименом опять же в более широком смысле, однако
наделялся определенными качествами в отличие от Аперона. Для него видимо было
важно показать каким образом происходят трансформации первого вещества, поэтому
он наделил архе вечное и бесконечное свойствами газообразных тел или стихии
воздуха, который как бы вдыхает жизнь во все. В нашем языке, кстати, тоже не
случайно нышать и душа однокоренные слова, а наш нюх одного происхождения с
греческим нус, что означает ум. Так Аноксимен полагал, что из беспредельного
воздуха при разряжении рождается огонь, а при сгущении ветер, туман, вода,
земля, камень. Данные воззрения позволяли наглядно представить смену агрегатных
состояний и превращение вещей друг в друга, та же следует интуиции единого.
Одной из самых многочисленных и влиятельных школ Древней Греции была школа
величайшего населителя Пифагора, имя которого вам также должно быть знакомо
сроков геометрии. Валенекшая в южно-италийском полисе Кратон, пифагорейская
школа распространилась впоследствии благодаря ее ученикам далеко за пределы
одного города, найдя также отклик в сердцах многих выдающихся мыслителей, не
являвшихся непосредственно пифагорейцами. Пифагор, уроженец острова Самос в
Ионии, обучался в Египте и по свидетельствам биографов перенял некоторые идеи,
характерные для древнеегипетской культуры, например, о переселении души, в том
числе в тела растений и животных. Также историки полагают, что пифагор учился у
известного ассирийского мудреца пророка Зоруастро или Зоруастро в другой
транскрипции. Личность мыслителя окутана множеством легенд, однако с
уверенностью можно сказать, что это был мудрый дальновидный человек,
разбиравшийся в людях, умевший убеждать и принимать грамотные политические
решения. За такие способности италийцы верили пифагору и его ученикам управления
полисом. Интересно и то, что пифагорейское учение не распространялось при жизни
пифагора за пределы сообщества его учеников, дававших обет молчания, опознанных
в стенах школы. Однако выдающиеся политические деяния пифагорейцев
способствовали стремительному распространению славы об их тайной школе по всей
Ладе. Также сообщество их славилось особым укладом, у них были общими деньги и
имущества. Учитель сам принимал учеников после беседы, определяя, достоин ли
человек у него обучаться. В том числе наравне с мужчинами у пифагора обучались и
женщины. Одному картонцу Келону было отказано вступить в ряд пифагорейцев по
причине его тиранического нрава. Тогда тот устроил заговор. Собственники Келона
подожгли здание, в котором заседали пифагорейцы. Выбраться во время пожара
удалось лишь двоим самым молодым и сильным ученикам, которые покинули Италию и
поселились на Пелопонесе, решив передавать новым последователям великое учение
пифагора и опубликовать труды. От самого пифагора не сохранилось сочинений,
которые можно было бы приписать ему с достоверностью. До нас дошли отрывки
произведений его последователей, в которых излагались основные положения учения
мыслителя. Однако глубина его философии поражает. Попробуем воспроизвести ее
логику. Размышляя над распространенным тогда пониманием первоначал, как
соединение предела и беспредельного, пифагор не только был захвачен присутствием
других противоположностей во всем правое и левое, мужское и женское, четное и
нечетное, доброе и зло и так далее, но и уловил присутствие единого в основании
этих пар. Если бы не было одного основания до всех противоположностей, основания
фундаментального порядка, согласно которому в мире уже есть разделение, то есть
мы уже сталкиваемся и имеем дело с теми симметричностями, то противоположности
нивелировали бы друг друга, аннигилировали или смешали бы в неразличимую массу.
То есть если бы противоположности и только они были первичны, то, во-первых, это
противоречило бы интуиции единого логоса текущего через всё, а во-вторых,
неясно, что удерживало бы их от смешения и взаимного уничтожения. Поняв смысл
поразительной и непостижимой для человеческого ума задумки природы или
мироздания или высшего разума или Бога назвать как угодно, Пифагор первым назвал
себя философом, а не мудрецом. Пифагор таким образом имел в виду, что человек,
как конечное существо, не может постичь, что предельное основание, благодаря
которому всё существует, потому что оно до нас, не нами создано и не наравне с
нами существует. Его не найти в мире, не схватить своим умом, поэтому он
говорил, что никто не мудр, кроме Бога. Того же, кто желает возвысить свою душу
и стремиться к предельно ясному осмыслению, подобает называть философом, любящим
мудрость или другом мудрости. Слово философия от греческого филия любовь и софия
мудрость на русский язык переводится как любовь к мудрости, буквально
любомудрия. Это любовь к вселенской мудрости, к софии мира, к тому, благодаря
чему всё именно так устроено. По этому поводу замечательный отечественный
мыслитель Владимир Вениаминович Бибихин пишет, что смысл тифагорейского учения в
указании на непостижимый ум, той основы, на которой возникают пары. Когда мы
приходим в мир, то видим сразу правое, левое и другие пары. Увидеть то единое,
которым они выброшены, мы никогда не успеваем принципиально. Тифагорейские пары
указания на нередуцируемую удовольствие на сначала. Нередуцируемую не потому,
что не к чему больше сводить, а потому, что основа пар не наша София. Мы не
знаем и никогда не будем знать, какой Софии создан и почему именно эти, а не
другие морские звезды, которые можно видеть в зоологическом музее. Пифагор ввел
слово философия потому именно, что говорил о Филии, верности такой Софии,
которая другого приобщения к себе, никакой причастности к себе не допускает.
Недоступна эта София, и можно только догадаться об этом и любить ее. Что же это
принципиально меняет? Казалось бы, какая разница, как называть исследователей
первооснов и первопричин мудрец или любящий Софию мудрость, а разница
колоссальная. Представьте себе мудреца, сегодня ведь тоже некоторые люди не
перестают себя выдавать за мудрецов, такой просвещенный говорит, что познал все
законы Вселенной, что в мире 42 ступени различных сущностей, человек это только
четвертое снизу, и надо совершить семь шагов очищения для разных шести типов
людей, но вы поняли, все расписано и посчитано, и нельзя ставить под вопрос. Вот
меня, как философа, всегда такие мудрецы, имеющие, дескать, высшее просветленное
знание, настораживают. Почему? Потому что по определению смертный не может иметь
абсолютно правильного знания обо всем, да и знание о таких предельных вещах
невозможно, не потому что у нас не хватает ума техник напридумывать, но потому
что все эти схемы закрывают собственно человеческое раз и навсегда что-то
определенное о мире и человеке решить, установить, зафиксировать, а наше дело
другое. Во-первых, мир не обязан подстраиваться под наши воображаемые конструкты
и всегда может произойти что-то, что наша схема не описывает. А во-вторых, мы
сами вопрошающие существа и открытые. На каком основании кто-то решает, что все
именно так, что бытие представляет из себя то-то и то-то, как смертные могут
решать, какими быть миру, человеку, чем-то превосходящему нас, предшествующему
нашему рождению. Наш ум, я уже выше говорила, не случайно с нюхом воздуха
математологически связан, а душа с дыханием. Закрывая всю повнуту и все
спонтанное многообразие реальности своими схемами и техниками, такие вот мудрецы
буквально перекрывают душе воздух, дыхание. И в нагромождении их однозначных
объяснений нам становится скучно. Мы задыхаемся, потому что никогда по-честному
не сможем жить в этих закупоренных домиках. Мы ведь творческие существа, а
значит такие, которым обязательно надо новое. Наконец, честнее оставить
открытыми вопросы о том, например, есть ли Бог, неужели мы ему можем указывать,
быть ему или не быть. Возник ли мир или существует вечно? Ведь мы уже упустили
момент начала, родились уже в мире. Что такое материя? А попробуйте ее
определить, если она обладает свойством беспредельности. Если мы не знаем о
таких вещах, может быть, пока не знаем, но вот здесь и сейчас не можем взять и с
наскоку ответить на подобный вопрос. Так давайте, по крайней мере, достойно
поступим, по-честному, сохраним лицо, как настоящие исследователи и мужественно
продолжим держать эти вопросы открытыми. Не спешить впасть в непродуктивные
схемы, повязывающие по рукам и ногам нашу творческую энергию. Так вот, в каком
состоянии дышится свободно? В каком мы счастливы? Не в таком, когда у нас есть
стройное и подробное знание ведания обо всем. А вот любящий не знает, знать и
ведать не пытается, он видит, принимает и понимает. Свободен и открыт, счастлив
в состоянии любви. Ведь что же такое любовь, если не абсолютно доверие и
принятие? Тогда философия принимает все, как оно есть, доверяет устройству мира
и в своем восхищении продуманностью вселенского порядка смиряется с тем, чтобы
фундаментальным образом не знать, как и почему все на самом деле именно так. Мы
любим, например, конкретного человека не потому, что знаем все о нем. Наоборот,
доскональное знание обо всех подробностях его существования скорее будет мешать
его любить. Незнание здесь также имеется в виду не в негативном смысле
необразованности, но в хорошем смысле необязательности модели, конструкции,
прописанной схемы для любимого, принимаемого, видимого. Ведь любим, принимаем,
видим мы не схемы и конструкты, а что-то принципиально не нами созданное, что-то
настоящее, открытое, живое за ними. Хотя без конструкции мы тоже вряд ли можем
обойтись. Другое дело, надо понимать, для чего они нам служат, и каково должно
быть их место в нашей жизни. Но любовь, возможно, только к целому, которая
выбирает в себя парадоксальным образом противоположности в смеси хорошего,
плохого и нас, и мира. Это, безусловно, не значит, что надо заострять внимание
на плохой стороне. Мужество настоящей любви заключается в свободе. Дать
любимому, как конкретному человеку, так и всему миру, быть себе в своей свободе.
А это означает доверить Софии мира, мудрому устройству, чтобы любимое было само
собой устроено, без нашего вмешательства. Тогда мы любим, например, вот этого
конкретного человека, принимаем его полностью, без желания исправить,
переделать, улучшить, залезть в его устройство и контролировать. Наша свобода в
том, чтобы делать свое, показывать пример, тянуться к хорошей стороне, хотя она
не обеспечена, всегда есть и плохая, причем в каждом из нас и то и другое
вложено, независимо от нашего желания. Это удивительно, замечая это все,
парадоксальности и полноте нам дано к такому двигателю из двух неравных половин
фундаментальной парадоксальности всего попробовать как-то аккуратно для
понимания подключаться. Вот Пифагор, видимо, был первым исследителем, который
это четко прояснил. Но подождите, что же в науке ученые не любят? В том-то и
дело, в этом мы все едины. Настоящий исследователь любит свою научную область,
ее методологию, свой предмет изучения. Иначе бы не разговаривал со своими
образцами и приборами, не чувствовал бы их мельчайшего ненастроенность на основе
цели исследования, не сопереживал бы своим респондентам, читая анкеты, не
удивился бы красоте формул, методов, языка. Ученый любит процесс познания, он
захвачен, ему интересно, иначе исследование не настоящее. И его результат в виде
готового знания. Но обратите внимание, в науке мы как бы пользуемся этим
продуктивным состоянием, этим позитивным настроем, чтобы делать свое
исследование. Само наше состояние при этом мы не делаем предметом, мы в его
свете что-то начинаем видеть, понимать и фиксировать в качестве знания.
Философия же, в отличие от науки, можно сказать, наоборот, делает своим
предметом то, благодаря чему оглядывается на само это состояние нашей
человеческой захваченности. Философия угадывает, что состояние нужно
настраивать. И будучи любовью к порядку Софии, призвана нам о таком состоянии
напоминать. Эпифагор был также ученым, на разумных основаниях он старался
создать знания о том, о чем оно нам доступно по поводу природы. Каким же единым
образом можно понять первоосновы всего в мире? Философия Пифагор видел
двоичность предела и беспредельного и понимал логическую невозможность
отождествить одну из стихий с единой первоосновы. Она единая во всех вещах,
похож всепронизывающий порядок Софии или в форме какого-то основы. Тогда на что?
Доступного языка логость течет через наш ум. Ответ Пифагора гениален. Это число.
Число то, что доступно нашему познанию в единой основе всех вещей. Ну, непонятно
вроде, почему вдруг число. Давайте вдумаемся. Число это предельная абстракция,
на которую способен наш разум. Что общего между тремя яблоками и тремя коровами?
То, что их три. Выделяя в обобщающем мышлении одинаковость формы предметов, мы
становимся способны их считать. Так, в пределе мы в нашем уме можем оперировать
числами уже независимо от вещественных коррелятов, то есть математически,
абсолютно абстрактно. То же самое с геометрическими формами. Когда-то поняв
сущность, например, треугольника, мы представляем его в своем уме независимо от
треугольного предмета, а также от того, на бумаге начерчен треугольник, на доске
или на песке. Мы мыслим и в уме оперируем самой идеей фигуры с тремя углами. В
связи с этой способностью нашего теоретического мышления пифагорейское учение и
постулирует число в качестве фундаментальной основы. И число в нашем уме реально
настолько отдельно от вещей мира, что похоже на отдельность единого логоса от
всего доступного нам в империи на опыте. Чувствуете, как близко друг с другом
идут настоящая философия и наука. Благодаря такой аналогии, такому исходному
пункту пифагорейского учения мы обязаны мыслителям данной философской школы
прежде всего развитие математики, строящейся вокруг философских проблем предела
и беспредельного пропорции и отношения подобия и различия форм. Но, естественно,
в свете сказанного пифагорейское учение о числах и геометрических
преобразованиях нельзя понимать, как отвлеченное построение математического
аппарата. Инструментарий этот изобретался и совершенствовался именно благодаря
своей укорененности в философских вопросах о том, как, согласно с какими
принципами и законами все в мире устроено. То есть, не сама по себе математика
была важна, а ее возможности для понимания вселенского порядка логоса, в
соответствии с которым вспоминаете античные антологические основания устроена
как космическая гармония, так и земная, как большое, так и малое. Математика
помогала понять соотношение всего со всем, становилась как бы посредником,
связующим звеном между всеми структурами мира, то есть, своеобразным языком
вселенского порядка. Поэтому, занимаясь математическими абстракциями и, казалось
бы, чисто теоретическими проблемами бесконечной делимости или несоизмеримости
некоторых чисел, античные мыслители на самом деле с одной стороны настраивали
свой ум на чистое восприятие форм, а с другой были захвачены самыми
фундаментальными вопросами нашего бытия и познания, оперируя ими в
математических терминах. В связи с таким особым статусом математики различные
числа и геометрические формы наделялись специфическими свойствами, которые
кажутся непонятными нашему сегодняшнему их восприятию как ценностно одинаково да
и вообще никак не нагруженных человеческим отношением к ним. Единица не
выслелась как непосредственно число. Она была началом и основанием любого счета,
поскольку для того, чтобы начать считать предметы, необходимо выделить каждый в
качестве отдельной единицы. идеал завершенности, полноты, оформленности как раз
соответствует такому ценностному преобладанию единицы как некой самостоятельной
целостности. Неважно какого размера и космос единица как целая, и каждый
отдельный человек единица как завершенность, и каждая отдельная капелька воды.
как пишет один из наших выдающихся философов и историков науки Анатолий
Вальянович Охотин, единицы таким образом для пифагорейцев разновидны и
разнокачественны. Если в области зримых предметов происходит деление единицы, то
она как тело уменьшается и разделяется на меньшие части, но в числовом отношении
она увеличивается, так как место одной вещи занимает теперь несколько вещей.
Помимо единицы ценностно нагружались пифагорейцами и другие числа, в частности,
они полагали число 10 божественным и совершенным, поскольку его составляла сумма
первых четырех цифр 1 плюс 2 плюс 3 плюс 4, она содержала поровну и четных и
нечетных чисел, в конце концов десятками нам от природы видимо удобно считать,
потому что на руках 10 пальцев отсюда пошла наша наиболее распространенная
десятичная система исчисления, хотя вы, наверное, в курсе, что существуют
современные математики и множество других, к примеру, на двоичной построен
принцип действия современных компьютеров. У пифагорейцев числа также
соотносились с геометрическими формами. Единица точка единая целостная, двойка
линия, тройка треугольник, как первая плоская фигура, символизирующая плоскость
вообще, четверка тетраэдр, как первая объемная фигура, символизирующая,
соответственно, трехмерное пространство. По счету углов и ребер геометрических
фигур выводятся все их арифметические характеристики. Равенство сторон,
пропорциональность и симметричность форм мыслятся принципами наивысшего
совершенства. Чем более правильна и проста фигура, тем она божественнее и
прекраснее. Что же может быть идеальнее окружности и сферы? Риторический вопрос.
Поэтому и космос как наиболее совершенное мыслился сферическим, замкнутым, а
движение небесных тел представлялось круговым. Математические исследования
пифагорейцев не были абсолютно умозрительными, находя применение, например, для
описания космоса и для составления теории музыки. Именно пифагорейцам
принадлежит открытие и математическое описание музыкальной гармонии на основании
идеальных числовых пропорций, соотношений. Кто обучался музыке, поймет. Музыка
во многом математика, поскольку для благозвучности необходим точный расчет
интервалов и соотношений высот одновременно извлекаемых звуков. Пифагорейцы
создали основу всей современной теории музыки, исследуя звуки струн, натянутых с
помощью различных грузов, которые в свою очередь пропорционально относились друг
к другу по массе один к двум, два к трём и так далее. Кому интересно подробнее
узнать об этом, почитайте сами, а мы тут подытожим. Поиск пропорций и подобия во
всём характеризуют методологическую направленность пифагорейской школы, гармония
и соотношение. Были интересны этим исследователям во всём, от звучания
музыкального инструмента до устройства Вселенной. Поэтому в заглавии данного
пункта плана я употребила выражение «музыка небесных сфер» отнюдь не
метафорически. Земная музыка для пифагорейцев была способом воспроизводить идеал
космической гармонии, поскольку в своей совершенной упорядоченности космос
буквально так же звучал, как звучат музыкальные инструменты. В школе пифагора
было воспитано множество выдающихся мыслителей и учёных. Заслуги их всех
перечислять интересно, но долго. Поэтому отметим двоих, на мой взгляд, особых
представителей данной философской школы. Алкмен Каратунский был известным
врачом-пифагорейцем, который развил учения о противоположностях применительно к
человеческому телу. Он считал, что сохраняет здоровье равновесие, ибо господство
одной противоположности действует гибелью. К примеру, как избыток пищи, так и её
недостаток вредны для здоровья. То же самое с теплом, холодом, духовным и
физическим развитием и так далее. Это другой взгляд на медицину. В античности
врач-целитель от слова целое, то есть тот, кто помогает человеку быть целым,
целостным, а это как нельзя лучше достигается путём соблюдения баланса и
разумного самоограничения. И действительно, по природе тело стремится к
здоровью, надо только ему не мешать, в том числе стараться, чтобы связанная с
ним душа не тревожилась чрезмерно и с другой стороны не изнеживалась.
Исследовательский интерес к тому, что у человека внутри и как всё устроено в
нашем организме древнегреческим медикам тоже был не чужд. Алкмен впервые
предпринял иссечение, то есть препарирование человеческого тела и благодаря
этому описал, например, как глаза соединены с мозгом с учёт зрительных нервов,
чем доказал в частности, что ощущения поступают в мозг через органы чувств, да и
то, что первенствующая часть души находится в мозге. Алкмен также формулирует
фундаментальное отличие человека от других живых существ. Только он, человек,
понимает, а другие животные ощущают, но не понимают. Филлай, урождённый в
картоне, известен прежде всего тем, что впервые опубликовал основы
пифагорейского учения, которое до этого фактически было тайным и систематически
не распространялось. Филлай интересен тем, что как раз, что благодаря
сохранившимся его произведениям мы много знаем о пифагорейской философской
системе ведения мира с помощью языка чисел и геометрических представлений, а
также о пифагорейской теории музыки. Согласно учению своей школы Филлай понимает
устройство космоса не геоцентрически и даже не гелиоцентрически, но как
Вселенную в центре, в которой находится очаг, огонь, вокруг которого вращаются и
звёзды, и солнце, и другие планеты, в том числе Земля. Вообще я поражаюсь
глубине античной мысли, когда, например, читаю у Филлай, что космос питается
испарениями вылившихся светил, чем не испаряющиеся чёрные дыры современных
астрономов. Когда смеются над воззрениями древних греков и считают их зачатками
науки, значит, не понимают, насколько наши сегодняшние научные представления
повторяют сказанные античными мыслителями. Вроде бы смешно считать, что всё есть
число или всё состоит из некой единой первоматерии, но давайте присмотримся и
вдумаемся на вопрос, из чего всё состоит. Наши современники отвечают, что из
молекул и атомов, а специалисты уточняют из энергии, потому что современная
физика выяснила, материя на микроуровне энергия. Те самые элементарные частицы,
из которых всё состоит, в пределе представляют собой энергетические уровни или
энергетические состояния. А что такое энергетические уровни? Это числа, решение
уравнения Шрёдингера для волновой функции, значит, материя в пределе число. Но
постойте, об этом с конца VI века до н.э. говорят пифагорейцы и вслед за ними
Плутон. Берём примерно более высоком уровне организации материи, если нас сами
элементарные частицы не удовлетворяют. Образ любого живого существа с
определенными особенностями передаётся от родителей детям, как с помощью
генетического кода. А код это шифр, последовательность, которая тоже число или
числовой ряд. Так что мы с вами как ни крути, тоже в пределе число. Пифагорейцы
оказались правы. Другой пример современной физики. Теория струн очень напоминает
пифагорейские струны с их пропорциями. В общем, советую задуматься над таким
современным стереотипом, что наука ступенчато развивается, и в каждую новую
эпоху знания становятся всё точнее, лучше, правильнее. Мне кажется, реальные
люди в каждую эпоху не садятся с грустью дожидаться, пока же наконец пройдём мы
современные со своими новыми теориями и техниками. Они сами стараются всё
исследовать и понять, сами формируют свою понятную им картину мира. Это нам
сегодняшним могут быть непонятны античные первоисточники, если мы не настроились
увидеть смысл в них вложенный глазами оснований этой эпохи. На деле далеко ли мы
сегодняшние ушли от античных мыслей? В начале предыдущей пары я неслучайно
говорила, что вся наука, в том числе сегодняшнее свернуто в идеях
древнегреческих мыслителей. Может быть действительно историчность языка описания
просто создаёт видимость прогресса? Гераклит из Эфеса. Ребята, один из самых
неординарных мыслителей всех времён вообще. Он взял на себя миссию критиковать
тех, кто мыслит нелогично и в своём полисе одёргивал поддающихся стереотипам и
невежественных. Вёл Гераклит себя достаточно резко и эмоционально. По словам
современников, но был знатного происхождения. В юности путешествовал, сам много
исследовал и его уважали за глубину его философии, которой он не учился у кого-
либо из учителей, но всё постиг самостоятельно. В Эфесе к власти пришёл тиран и
изгнал из города друга Гераклита Гермодора. Маслитель призывался граждан
препятствовать этому, но они не стали сопротивляться. Когда же потом в спорной
ситуации к нему играющему с детьми у храма Артемида прибежала взволнованная
толпа сограждан и попросила дать им новые законы, Гераклит остался играть с
детьми, сославшись на то, что город уже во власти дурного государственного
устройства. Афиняне приглушали мыслителя поселиться у них после того, как эфесцы
разобиделись на своего философа за подобное поведение, но он предпочёл остаться
на родине. Даже персидский царь к нему и его друзьям пожелал приобщиться к
мудрости Гераклита и приглашал его переселиться в Персию под свою опеку.
Гераклит себя и с царём держал на равных, переписывался с ним, сохранились
несколько их писем, сообщал о своих идеях царю, но от приглашения отказался,
слишком был свободолюбив. Гераклит не основал своей школы, от его философии до
нас дошло всего 139 фрагментов, в основном афоризмы, но чуть ли не во всех
последующих учениях можно проследить влияние его мысли. Например, Гегель как-то
сказал «нет положения Гераклита, которое я не принял в свою логику». Гераклит
назвал то самое единое начало Логоса и полагал, что этому первичному похожему на
речь порядку подчиняются и боги. Логос один и тот же во всём и для всего. Не
уступая Пифагору в глубине философской мысли, Гераклит говорил, что мудрость
София, которой всё устроено отдельно от мира и настолько иная, что мы себе не
можем её представить. Видим только уже отражение противоположностей в нашей
Вселенной, их движение, их игру. Другое всему отдельное это та половина
Гераклитовской мысли, о которой Сократ сказал, что он её не понял. Чтобы понять,
нужен, пожалуй, говорит он, дозвский ныряльщик, но что она так же подлинно, как
половина Сократа понята, сомнений у него нет. Не поняв, то вторую половину
Сократ отнёсся к ней тем единственным способом, какой достойный ныряльщикам.
Отдельно таково, что в нём нет ничего отвечающего приёмам человеческого
поздания, как под водой нет воздуха для дыхания и надо быть искусным ныряльщиком
не для того, чтобы найти способ пребывания на глубине. Это невозможно, а только
для того, чтобы не сразу задохнуться там. София другое разума, так же, как и
другое неразумие. Для нас другое жизни, смерть, но для гераклита это единое
основание настолько другое, что оно иное и смерти. Это иное мы не можем понять.
Оно одно отдельно. По гераклиту настолько, что оно другое одновременно и богам,
и людям, которые, казалось бы, противоположны друг другу, как смертные и
бессмертные. Захватывают такие прозрения. Другое таково, что не может быть
истолковано и продемонстрировано, стала быть, наша задача не в понимании или
непонимании, главное, в нашем завязавшемся отношении к другому. Не даром
гераклит не хочет ничего описывать и рассказывать. Он говорил, что Гомера и
Археолога надо бы высечь за то, что их мысль вязнет в мифологии, в рассказывании
историй. Так что, например, фрагмент 90. Огонь обменивается на всё в мире и все
и всё на огонь, как золото на вещи и вещи на золото. Эту потрясающую по красоте
лаконичности метафора Гераклида вовсе не обязательно трактовать только как
первую формулировку закона сохранения энергии, хотя это тоже, естественно, есть.
И о космосе и его фрагмент 30. Космос был всегда, есть и будет огонь вечно
живой, зажигающийся соразмерно и гаснущий соразмерно. Не обязательно речь о
концепции пульсирующей Вселенной, одной из концепций развития Вселенной в нашей
современной космологии. Гераклид прямо говорит в 64 фрагменте «Всем сущим правит
огонь, в смысле энергия, которая у мыслителя молния». И еще одна
головокружительная фраза об этом, 11 фрагмент «Все живое пасется молнией» или в
другом переводе «Все ползущее бичом пасется», точнее было бы сказать «мгновенным
ударом». В оригинале там «плэгэ», резкий удар, который во всей греческой
литературе известен как удар молнии Зевса. Так что же, сам наследитель вязнет в
мифологии и решает, что главная сила, управляющая всем во Вселенной молнии, как
бы не так. Давайте присмотримся. Гераклидовский логос, как сосредоточенный
смысл, помните логос от лего, собирая, правит по способу молнии и сам есть
молния. Как мгновенно может править многим? Возможен ли логос как молниеносное
захватывание всего одним? Логос не имеет отношения к обобщающей, абстрагирующей,
рационализации сущего. Он подобен не описи мирового богатства, а его золотому
эквиваленту. Он поэтому не хуже вещей, подобно тому, как золото не хуже товара.
Так что, продав вещи, получив в обмен не их список, а хорошую цену в твердой
валюте, нет причин грустить о них. Огненный логос то золото, которое заранее
знает цену вещам. Золото логоса не условно и схематично, а по существу выбрало в
себя своей непостижимой внезапности все вещи. Золото стоит вещей для вещей, но
их высшая возможность. Они исполняются, узнавая себя в молнии своей тайной и
истинной сути. Молния исполнения вещей, потому что они хотят вторить ей,
тянуться, слиться с ней. То есть, как понять, эта молния всем правит? Это
вертикаль чистого момента теперь по отношению к горизонтали нашей хронологии.
Это вот здесь и вот сейчас как единственное настоящее. В чём оно? В действии. А
это в деле, в действии перевод греческого слова энергии. Вы знаете, как вдруг по
осени разом птицы сбиваются в стаи и пускаются в тысячекилометровый путь? В
общественную государственную жизнь не правит сходный закон. Например, в нашем
Отечестве сообщение, переданное ранее весной 1917 года из отдалённой столицы по
всей стране действовало не своим содержанием, убеждало не обещанием
переустройства жизни на более разумных началах взамен старым и нерациональным.
Сообщение было принято страной как сигнал. Оно вгоняло человека в другое
электризованное состояние. Современники отмечали, что человеческий тип в России
сменился за несколько недель, если не за несколько дней, часов. Все вдруг
поняли, что пробил час. Но длительное до этого существование восприняло импульс.
И далее проблема передачи власти, любой, не только монархической. Молния не
передаётся никак, ни по наследству, никаким другим способом. Молнией можно
только самому быть. Это тяжело настолько же, насколько уместиться в границе,
которая ничто любому что, в чистом различии, где нет пространства. Помните, мы
на третьей теме говорили о призывании, о совпадении с собой? Не могло оторваться
от Гераклита, но приходится. У него ещё множество бездонных фраз. По ссылкам
внизу обязательно сами прочитайте. Мы тут только одним глазком заглянули, но,
надеюсь, молния хотя бы на миг озарила и позволила увидеть, какой величины это
философ. Под стать ему его не менее выдающийся современник с другого конца
Эллады рубежа VI-V веков до н.э. Парменит из Элеи на юге Италии. Сразу отмечу,
что в оригинале на древнегреческом сохранился лишь небольшой отрывочек его поэмы
о природе, но в этих строках спрессована просто вся последующая философия,
столько смыслов, что ни в один перевод не влазит, и никто не может передать весь
размах парменитовской мысли. О его тексте не утихают споры, появляются новые
интерпретации, комментарии, новые развороты, обсуждаемых парменитовом вопрос. Я,
конечно, за несколько минут не смогу вам развернуть весь этот масштаб, мы лишь
коснемся самых базовых моментов, о которых принято говорить про парменида. снова
отошли у вас читать Владимира Вениаминовича Бибихина, этого уникального,
мощнейшего отечественного философа, который во всех своих работах потрясающе
осмысляет античных авторов, идей парменида, много где касается, но самый
головокружительный, захватывающий разбор в чтении философии. Парменитовский
текст в греческом оригинале составлен из точнейшим образом подобранных слов, о
мистическом опыте, постижении самого главного, того единого, что одновременно
любовь, истина, смысл, чему мы все как бы принадлежим. Ни один его перевод уже
не может передать всей этой точности, всего размаха его мысли, но хотя бы в
переводе, что нам осталось от парменида? Он впервые обратил внимание на то, что
та отдельная от мира единая основа всего не есть нечто. Ответ на вопрос что? но
само это есть или быть бытие, которое присуще всему вообще, что было, есть или
может быть возможности, причем в одинаковой степени. В связи с этим говорят, что
параминит ввел в философский язык понятие бытия, хотя, конечно, мы должны
понимать, что эти слова быть, есть в любом языке с момента его возникновения,
есть. Но параминит обратил внимание на то, что самое главное, самое отдельное от
вещей и непохожее на них бытие. Усердно упражнявшийся в зрительном подходе,
многое читавший, видевший, знавший. Параминит как бы отпускает свою мысль,
жаждущую истину. В начале поэму он образно пишет, что его несут кони в
колеснице, стремительно к свету этой истины. Вопрос об основаниях приводит или
приносит мыслителям, благодаря предельному теоретическому абстрагированию, к
видению того, что небытия нет, то есть всё есть и не быть никак невозможно. Он
открывает, что небытие для нас имеет место лишь по видимости. В мире мы
наблюдаем рождение и смерть. до рождения и после смерти нам кажется, что имеет
место небытие. Но параминит вызывает нас внимательнее присмотреться. Тело,
скажем, человека, ведь не появляется из ничего и не исчезает абсолютно в никуда.
Какие-то процессы приводят к возникновению, затем к гибели. Но это, по сути,
лишь качественные резкие изменения того же самого. То есть у нас создается
впечатление, что бытие сменяется небытием, а на самом деле мы наблюдаем
постоянное превращение бытия, смену способов бытия. Просто для нас жизнь
ценностно нагружена по сравнению со смертью. Однако по природе умершее тело
просто переходит в иное качество, разлагается и становится питательным веществом
в почве, благодаря чему что-то тоже может родиться. Сегодня умершие миллионы лет
назад живые организмы для нас топливо, одежда, пластик, то, во что
перерабатывают нефть и уголь. Так что по параминиту можно сказать, что древние
жители планеты не в небытие канули, а преобразовались и никуда не исчезая
сегодня, служат для нас вполне ощутимыми вещами. Так же и рождение существа
происходит не вдруг задолго до появления на свет и создается зародыш, который
питается и растет. До физического своего появления он пребывает просто в другом
качестве, но это тоже бытие. Головокружительная мысль. Во-первых, все уже есть и
прям совсем исчезнуть не может, а во-вторых, весь мир, природа вся как бы творит
себя, постоянно преобразуя то же самое, но перекомбинируя, попробуя новые формы.
Так все развивается. Параминит говорит о бытии, и неделимо оно, коль всецело
подобно. Тут вот не больше его ничуть, а там вот не меньше. То есть везде, во
всем на самом деле одинаковое бытие и в качественном, и в количественном
отношении. Чтобы мы не ткнули, везде упремся в бытие. Нигде не получится найти
ни бытие или ничто, всегда будет обнаруживаться нечто, хотя бы даже это само
наше ищущее стремление. И тут мыслитель в одной своей формулировке сворачивает
основную идею всей последующей западной философии. То, чего нет, нельзя ни
познать, ни удастся, ни изъяснить, и помыслить тоже, что быть. Напоминает
Декарту формулу, я мыслю, следовательно, я существую, но Декарт, конечно, не
воспроизводит Параминита специально, а заново, с нуля, с чистого листа,
осмысляет то же самое, но в своей уникальной культурно-исторической ситуации,
приходя к той же истине. Параминит говорит, можно ли что говорить и мыслить, что
есть. И если уж нам удалось увидеть истину, тогда то же самое мысль и то, о чем
мысль возникает, ибо без бытия, которым ее изрекают, мысли тебе не найти. То
есть и мыслим мы только осущим, и бытии, пусть и встречаются они нам в разных
модусах. С той колеи, по которой София Мира направляет ход всего сущего в
доступном нам мире, нам никогда никаким усилием не соскочить, не перепрыгнуть,
как трамвай едет только по рельсам, и если рельсы не проложены и провода не
проведены, по иному пути не сможет поехать. Также и хоть вот тресните,
элементарные частицы будут именно так собираться в атомы, а атомы вот только вот
так в молекулы. Параминит объясняет и метод своего познания такой истины. Виждь,
то есть вить, узри, однако умом отсущая, то есть отсутствующая, верно, присущая,
присутствующая. Нужно так настроить свой внутренний взор, чтобы он мог увидеть
истину, сквозь мнение и то, чего по видимости вроде бы нет, на самом деле есть,
просто для нас в данный момент предстаёт как отсутствующая. параминитовская
мысль берёт начало традиция противопоставления мнения, как общепринятого
суждения, докста и истины, по-гречески алетая, раскрытость, непотаённость и как
бы выведение на чистоту, чистое видение против видимости. Но параминит обращает
внимание на то, что изучать следует всё, с чем мы имеем дело, в том числе и
людское мнение. В его поэме сказано «Всё должны узнать ты как убедительная
истина, непогрешимое сердце, так и мнение смертных, в которых нет верности
точной». Интересно, зачем же нам это надо, раз есть доступ к светлому пути
истины? Мы целостные существа и существуем не только в теоретическом измерении
умозрительного, но и практически в мире что-то ощущаем и как-то поступаем. Хотя
на самом деле всё едино, мы сами замечающие двойственность, день и ночь,
мужской, женской, правой и левой. Являемся двойственными существами, которым в
познании доступно сияние вечной истины, несмотря на то, что мы конечные
существа, помещенные во время и пространство. Мы способны истину узреть, понять,
прочувствоваться ей, но при этом как бы не можем на её языке говорить и слиться
с ней. Можем только в её свете конструировать себя более или менее удачные
инструменты ориентирования в мире. То есть, например, в современной науке вы же
прекрасно понимаете, вещества это не то же самое, что структурные формулы,
которыми мы их описываем, размножение живых организмов, не то же самое, что
придуманные нами статистические модели, процессы, происходящие с электронами в
металле, не то же самое, что законные формулы физики, не вычисления, не то, что
мы в целом называем знанием. Знание у нас парадоксальным образом никогда не
совпадает с объектом познания, поскольку объект сам в своей полноте существует в
реальности, а знание, мы, люди, о нём создаём как принципиально и изначально не
совпадающую с ним картинку. И парадокс в том, что эта картинка нам нужна, мы без
неё не можем, мы ориентируемся в мире благодаря ей. Однако и в этом
фундаментальный антологический парадокс, из которого фонтанируют все остальные
содержательно различные парадоксы нашей жизни, нашего познания, бытие не
совпадает с сущим. В античности это был фундаментальный парадокс единого и
многого. Бытие едино, одно для всех, как парменинг говорит, и там, и тут его
одинаково, и везде оно одно, но в мире мы имеем дело с многим, с сущим, с
различными существующими предметами, субстанциями, людьми, мнениями и так далее.
Бытие одно единое, а сущие разные, и их много. И самое поразительное, что нас
делают уникальными существами, мы способны эту разницу видеть не между одним
сущим и другим, а антологическую разницу между сущим, тем, что есть, и бытием,
тем, благодаря чему все есть, и единым основанием, и многим, благодаря ему
существующим. Учение о парадоксальности, то есть несовпадающем, противоречащем,
несовозможным, развивал ученик параменида Зенон. Опория Зенона от опория,
буквально, непроходимость, безысходность, ситуация отсутствия опоры, на первой
паре говорили об этом, вспоминайте, описанные им парадоксы, которых, говорят,
насчитывалось более 40, но дошли до нас лишь 9, благодаря их изложению в текстах
Аристотеля представляют собой различные содержательные примеры функционирования
фундаментального антологического парадокса в нашем мышлении. Большинство
посвящено вопросам бесконечной делимости и природы движения. Расскажу о двух
самых известных опоре Зенона, которые называются Ахиллес, черепаха и стрела.
Первое, констатируя то, что даже самый быстроногий бегун, а из известных элинах,
это герой Ахиллес, не догонит оползающую черепаху. Такое вот, очевидно,
медлительное существо. Как это раскрывает Зенон? Внимательно смотрите на каждое
слово формулировки парадокса, чтобы точно понять, о чем здесь речь. Ахиллес, без
сомнения, настолько быстр, что, понятное дело, перегонит черепаху. Но дело здесь
не в том, чтобы ее перегнать, а в том, чтобы фиксировать момент, когда Ахиллес
может констатировать догнал. Рассмотрим ситуацию математически. С каждым шагом
Ахиллес приближается к черепахе, нагоняет ее. Но пока Ахиллес делает следующий
шаг, черепаха, пусть и на очень-очень маленькое расстояние успевает продвинуться
еще вперед. Ахиллес снова вот-вот уже рядом, но черепаха продолжает смещаться. И
получается, что он все время от нее, пусть и на чрезвычайно малую долю
расстояния, но отстает. Ахиллес может быть близко к черепахе, может быть еще
ближе того, что мешает и так без конца. Тут мы имеем дело с проблемой
бесконечной делимости и бесконечно малых, которые как бы фонтанируют прямо из-
под ног у Ахиллеса. И мы никак не можем с точностью указать ту самую точку, в
которой положение Ахиллеса совпадет с положением черепахи. Расстояние все меньше
и меньше, но все равно точка Ахиллеса никак не может совпасть с точкой черепахи.
Ахиллес растеряется, взлыхнется в количестве ступеней приближения. Интересно,
что древние гречки не боялись работать с этими бесконечностями в уме,
всматривались в них и вместо того, чтобы отбросить как бесполезную назрительную
проблему, делали выводы. Какие выводы можно сделать из этой ситуации? Во-первых,
например, что когда мы в ситуации опории, отсутствие опоры, ни напор, ни расчет
не поможет. То есть до погони за черепахой всем понятно, что Ахиллес
быстроногие, значит, догонят по определению. Но когда он пытается отыскать ту
точку, в которой догнал, никакие расчеты и никакие его быстрота, сила усердия не
помогут. При этом он не перестает быть самым быстроногим бегуном, если по-
настоящему им является. Присмотритесь, в этот парадокс мы сплошь и рядом
попадаем, как ученые, когда нас ставят в условия гонки за ноукометрическими
показателями. И вот либо я и так уже исследователь и исполняюсь, применяя свои
исследовательские способности, и тогда я по-настоящему ученый, либо я впадаю в
спешку и суету публикационной машины, думаю, как бы мне выполнить план, где что-
то поскорее сдали, где можно схитрить, чтобы повысить себе индекс Хирша и так
далее. И тогда я в жизни не догоню воскользающий идеал настоящего ученого,
потому что никакое количество публикации не даст мне совпасть с идеей ученого по
призванию, который думает о содержании получаемого знания, а не о своих
показателях результативности. На эту тему есть фильм у Такеши Китана, так и
называется, «Ахиллеса черепаха». Там, правда, не про ученого, а про художника,
но суть та же. Посмотрите, сильная вещь, многому побуждает задуматься.
Возвращаясь к, собственно, опоре, во-вторых, напрашивается вывод о том, что
когда мы пытаемся представить ситуацию математически, мы что-то от самой
реальности неизбежно теряем. Ведь живой Ахиллес и живая черепаха представляют
собой не точки, а тела. В реальности цель нахождения точки догоняния окажется
надуманной по сравнению с целью, например, обогнать. Однако эта проблема отнюдь
не просто умозрительная и говорит о несоответствии реальности и наших попыток
полностью ее математически представить. Это парадоксальность нашего сознания,
которое умеет воспринимать движение, видеть его и понимать, но совершенно не
может составить знание о существе движения. Я имею в виду не о том, как предметы
могут двигаться, а о том, что такое движение в своей сущности. Почему? Об этом
другой парадокс под названием стрела, который можно сформулировать следующим
образом. Несущаяся, движущаяся, летящая стрела в каждый момент времени покоится.
Для нас абсолютно очевидно движение летящей стрелы. Однако, если мы попытаемся
представить ее движение, что мы сделаем? Мы возьмем ее траекторию как
математическую линию и представим эту стрелу в последовательно проходящую каждую
точку этой линии. Однако, в каждой точке получится мысленно, что стрела как бы
зависает, то есть, хоть на малейшую долю секунды она покоится в каждой точке,
которую мы себе представляем. Как же так? Ведь на деле мы видим полеострелы.
Любое наше знание, конструкция, оно что-то регистрирует, фиксирует, чтобы по-
настоящему быть знанием, иначе оно было бы текучим, изменчивым и непостоянным, а
эти характеристики явно противоположны сущности знания. Так ведь? А движение —
это динамика, изменение, непостоянство, причем непрерывное в своем, так сказать,
течение. Не может быть адекватного знания или схематического представления о
движении как таковом, потому что знание как фиксация противоположно в своей
статичности природе движений. Это парадокс. Между статичным и динамичным та же
самая пропасть, что между пределом и беспредельным, мужским и женским, добром и
злым и так далее. Обратите внимание, дело не в том, чтобы ценностно нагружать
один из этих противоположных полюсов. Главное заметить, что они оба в наличии
перед нами всегда в нашей жизни. Мы не можем теоретически выбрать между ними раз
и навсегда избавившись от другого полюса. Парадокс можно только принять,
смириться с таким положением вещей, которое не нами устроено. Но это не означает
сдаться. Напротив, чуткое всматривание в максимум всего, что дано, открытость
всему, даже страшному, неизвестному, неподконтрольному, только позволяют понять
исследуемое во всей его естественной полноте. В вопросе о том, что представляет
собой на микроуровне первое вещество, из которого все состоит, помимо
господствовавшего понимания бесконечной делимости материи, в V-IV веках до нашей
эры формируется еще одно учение атомизм. Левкип и демокрита. Левкип, сведения о
личности которого крайне скудны, был по-видимому учителем демокрита. Данные
мыслители, в отличие от большинства своих современников, полагали, что на
микроуровне все состоит из мельчайших неделимых частиц, разделенных пустотой.
Демокрит считал, что атомы, мельчайшие частицы, которые настолько малы, что мы
их не встречаем глазом, разно качественны по своей форме. То есть, любая вещь
состоит из атомов определенного вида, которые сцепляются друг с другом
посредством крючочков и впадин. Эта идея впоследствии даст рождение в 17-18
веках классической химии с ее представлением молекул, состоящих из соединенных
друг с другом атомов различного вида. Однако, на современном этапе выясняется
несоответствие такого представления действительности. Атомы и элементарные
частицы оказываются в принципе делимыми, да и телами в физическом смысле их
нельзя назвать. В личности атомизм тоже не прижился. С интуицией единого такая
теория с трудом согласуется. Но, с другой стороны, логически, убедительно, легко
доказать, что в действительности нет пустоты. Она мыслится нами лишь в
возможности. И поэтому, собственно, взгляды Левкипа Демокрита на природу первой
материи были опровергнуты. Аристотелем наиболее последовательно. Ну вот, если
пустоты не существует, то между частицами поместилось бы бесконечное количество
других мелких частиц, они были бы, все бы соприкасались без промежутков,
соответственно, это опять была бы единая бесконечно делимая материя. Решение, по
сути, парменидовское. В природе нет пустоты, ничто, не бытия. Спасибо всем, кто
мужественно вынес от начала до конца такую насыщенную лекцию. Что такое, ребята?
Кто-то включился? Уже хотите задавать вопрос? Ну, задавайте. Мы на этом можем
заканчивать. ребята? Пока нет вопросов. Да, было насыщенно очень. Ну, что
поделать, я вам с запасом напоминаю, что безусловно на экзамене мы это все от
вас требовать не будем, будем только самое основное требовать, просто чтобы у
вас контекст был. Вот. Если вопрос... Мы же знаем, что на экзамене мы всегда
напишем меньше, чем было записано. Мы вас будем слушать только 10 минут каждого.
