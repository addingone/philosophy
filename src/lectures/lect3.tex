\section{Коммуникативные аспекты науки} 

\subsection{Понятие коммуникации, статус коммуникации в рамках научной деятельности}

\subsubsection{Понятие коммуникации}

% Например, в английском common, community, communication, однокоренные
% слова, которые означают общий, сообщество, общество и связь, общение, сообщение
% соответственно. 

% Слово коммуникация имеет латинское происхождение. От communico
% делаю общим. 

% Благодаря коммуникации человек развивает мыслительные способности
% объединять свои усилия с другими, чтобы строить цивилизацию. Хотя анатомически
% похожие на нас люди были уже 200 тысяч лет назад. Они обрели культуру, начали
% одеваться, хранить мертвых, оставлять рисунки и так далее. Только 50 тысяч лет
% назад, когда придумали язык и начали общаться. Ну а письменность появилась всего
% 6 тысяч лет назад. Так что это новость в масштабах Вселенной. Ну и огромный наш
% эволюционный скачок. 

Среди определений человеческой коммуникации наиболее
распространены такие варианты, как обмен информацией и общение людей. 

В широком смысле, коммуникация представляет собой связь, взаимодействие,
предполагающее положительное взаимно связывающим некими потоками отношение (???).

Основные отличительные черты человеческой коммуникации:
\begin{itemize}
    \item процесс, который предполагает двух и более участников (при общении с собой --- противопоставление самому себе);
    \item передача опыта, смысла, эмоций, информации и т.д.;
    \item восприятие и интерпретация;
    \item длящийся во времени и требующий времени (для осмысления) процесс;
    \item совместность, осуществление связи между
    людьми, обычно с целью некого взаимного положительного эффекта, предполагающего
    взаимное духовное развитие, обогащение, понимание.
\end{itemize}

\subsubsection{Статус коммуникации в рамках научной деятельности}

В научной среде коммуникация не должна претендовать на основное место в процессе научной работы. Коммуникация и коллективная работа важны наряду с индивидуальной
деятельностью, только в рамках которой возможно творчество. 

Несомненно, обмен
знаниями и совместное планирование исследования, постановка проблемы, поиск
методов ее решения, обсуждение полученных результатов и так далее, крайне важны
для научно-исследовательской работы.

Творчество, производство нового знания — ядро собственно научной деятельности.
Без творческой составляющей, просто из общения как обмена готовыми мнениями научное знание не
могло бы родиться. 

Согласно исследованиям французских мыслителей
XX века Жиля Делёза и Феликса Гваттари, творчество всегда единично. Это означает, что
коммуникация как обмен мнениями может лишь способствовать творчеству, вдохновить
человека, обратить внимание одного на то, что более ясно видится другому с его
точки зрения, но не в силах заменить собой творческий элемент научной
деятельности. 

Макс Вебер утверждал, что помимо коммуникации и творчества,
значимой оказывается и повседневная рутинная работа, не в меньшей
степени обеспечивающая и подготавливающая научное открытие.

Таким образом, коммуникация в научной сфере имеет
достаточно важное значение, однако скорее выполняет вспомогательную и
интегративную функцию, чем играет единственную ведущую роль.


\subsection{Формы и способы коммуникации в научной среде}

Благодаря достижениям техники можно
выделить также и широкий спектр инновационных
возможностей и средств для осуществления профессионального общения.

\begin{table}[h!]
\centering
\begin{tabular}{|p{6cm}|p{6cm}|}
\hline
\textbf{Традиционные} & \textbf{Инновационные} \\ \hline
Личная беседа & Интернет-технологии \\ \hline
Письма & Аудио- и видеоконференцсвязь \\ \hline
Научные публикации & Дистанционные технологии подачи и рецензирования публикаций \\ \hline
Конференции, лекции, теоретические доклады, совещания, съезды & Онлайн-встречи, вебинары \\ \hline
Стажировки, повышение квалификации, совместные праздничные мероприятия & Интерактивные игровые формы: тренинги, модели, неформальные корпоративы \\ \hline
\end{tabular}

\label{table:traditional_vs_innovative}
\end{table}

Личное общение дает больше опыта коммуникации и осуществления
научной деятельности, поскольку при этом участвует несколько каналов восприятия, 
есть возможность физического воспроизведения.

Майкл Полани поднимает важную тему личностного или неявного
знания. Знания обретаются не только в ходе устной или письменной коммуникации.
Ряд знаний, умений,
навыков для профессиональной деятельности обретается лишь в процессе регулярной
практической деятельности, зачастую оседая на бессознательном, интуитивном уровне.

Обычную коммуникацию в целом подразделяют на:
\begin{itemize}
    \item вербальную
        \begin{itemize}
            \item устную: дискуссии, сообщения, участие в конференциях и т.п.;
            \item письменную: статьи, отчеты, книги и т.п;
        \end{itemize}
    \item невербальную: мимика, жесты, взгляд, интонация, поза, темп и тембр речи.
\end{itemize}
Указанные формы выражения играют роль в рамках профессиональной коммуникации, поскольку выражения воспринимаются по-разному в зависимости от сложности конструкций, эмоциональной загруженности,
невербального оформления и соблюдения вежливости.


\subsection{Специфика научного языка}

Для научного языка характерны \textit{термины} (15-20\% научной лексики). Это слова или словосочетания, обозначающие понятия в специальной
области знаний или деятельности, которые стремятся к однозначности и не выражают
экспрессии. Важно, что у терминов есть референт в реальности. 

Большую часть научной лексики (30-40\%) составляют: \textit{абстрактная лексика},
\textit{специальные фразеологизмы}, \textit{клише и стандартные обороты}.

Для научных текстов свойственен академический стиль, имеющий следующие особенности:
\begin{itemize}
    \item логичность, непротиворечивость выражений;
    \item высокий регистр (исключение расхожих фраз, оборотов обыденного языка);
    \item четкость, ясность, умеренная простота конструкций;
    \item точность выражений (отсутствие двузначности, размытых и
    пустых формулировок, полное отражение всей фактической информации);
    \item безличные конструкции (употребление научного <<мы>>);
    \item научная лексика;
    \item аргументированность (доказательность и иллюстративность сформулированных
    положений).
\end{itemize}
Важно уметь отличать его от смежных стилей, соблюдая баланс в рамках официальной научной коммуникации. 

% Нельзя также не сказать хотя бы в двух словах о нормах
% научной коммуникации, которые  Часть этих моментов мы
% рассмотрим подробнее в третьем вопросе сегодняшней темы, потому что они больше
% относятся к этическим аспектам. 

\subsection{Нормы научной коммуникации}

Помимо очевидного соблюдения \textit{вежливости},
\textit{непредвзятости} и \textit{аргументированной критики}, к нормам научной коммуникации относятся культура
цитирования и ответственность за самостоятельность получения и
представление результатов научной деятельности.

\subsubsection{Культура цитирования}

В рамках академического стиля действует следующая культура цитирования: 
\begin{itemize}
    \item принято ссылаться в основном на научные статьи и монографии (до 80\% источников цитирования);
    \item возбраняется ссылаться на неавторитетные и недостоверные источники;
    \item недопустимость плагиата.
\end{itemize}

Генеральная цель науки --- создание нового знания. 
Поскольку новое научное знание  как правило основано на уже созданных представлениях, необходимо соблюдать баланс между опорой на авторитетные источники и творческим элементом собственного исследования. 

\subsubsection{Самостоятельность получения и представление результатов}

% Сравним человека и большие языковые модели.

% Оно
% предполагает, в отличие от животных, прежде всего, рефлексивность в постановку
% вопросов и выход на уровень смысла, то есть осмысление, понимание. Также
% отмечают высокую развитость воображения и речи, когнитивных и коммуникативных
% способностей. 

% При этом интересно, что люди не рождаются с человеческой системой
% восприятия, не владеют языком и не умеют сразу мыслить. Собственно, человеческое
% именно формируется. Изначально у нас есть так называемая лепетная речь. Это
% слоговая непонятная речь детей, которой мы все с двух-трех месяцев начали
% пользоваться. 
% Также в нас заложены потребности выражать свое самоощущение в
% связи с происходящим вокруг и играть, примеривая на себя различные роли.
% Благодаря этим естественным предпосылкам в процессе взросления в среде
% человеческого общения ребенок проходит два фундаментальных кризиса. 

% Появление самосознания и развитие совести. Собственно, умение посмотреть на себя со
% стороны, отделяя я от мира, и внутренний диалог, учитывающий самостоятельность
% других, в горизонте максимально возможного, являются уникальными онтологическими
% особенностями человека, на которых и строится вся специфика его бытия по
% сравнению с другими существами. Развитие мышления и языковые способности
% неразрывно связаны. Переработка изначально лепетной речи в классифицирующие и
% маркирующие вещи события взрослый язык, помогает осваивать абстрактные формы,
% теоретические и оценочные инструменты, логическое, критическое и рефлексивное
% мышление. Безусловно, к творческому, исследующему и преобразующему отношению к
% миру у ребенка есть природная тяга. Однако в процессе взросления они получают
% огромный стимул к совершенствованию именно благодаря развитию речи и
% взаимодействию с другими. И человеческий язык не просто средство передачи
% информации, это многомерная и на большую долю иррациональная система. К тому же
% буквально среда нашего обитания, которая с одной стороны задает схему мышления,
% а с другой открывает возможности понимания смысла различных интерпретаций. Так
% что смоделировать все естество языка через логические операции вряд ли
% получится. На самом деле, задолго до распространения современных генеративных
% нейросетей в дискуссиях о таких мысленных экспериментах, как «Китайская
% комната», «Философские зомби», «Мозг в колбе» и так далее, Джон Сёрль, Сол
% Крипке, Дэвид Чалмерс, Дэниел Дэннет, Ник Бостром и другие аналитические
% философы показали, что антологически машинная речь «речь» может представлять
% собой лишь более или менее удачную имитацию типичного, за которой не стоит опыт,
% понимание, подлинное творчество и здравый смысл. Об этом, опять же, ребят, у нас
% будет еще разговор на последних темах, так что не пугайтесь, что я вам тут
% каких-то имен наговорила, мы позже с ними познакомимся. Но что касается так
% называемых больших языковых моделей, они организованы по принципу токенизации,
% надрабления на морфологические составляющие и комбинирования слов естественного
% языка. Обучение нейросетей для генерации текстов предполагает обработку системы
% огромного массива статей в Википедии, художественной литературы, новостных лент
% и других подобных ресурсов. Например, в обучении ChatGPT3 было включено 600
% гигабайт таких текстов, опубликованных в интернете до 2020 года. Из этих текстов
% выявляются закономерности совместного слова употребления, collocations и
% окончаний с присвоением более высоких коэффициентов в более распространенном
% конструкции. Соответственно, ответ на запрос пользователя, в ответ на запрос
% пользователя, выученная таким образом нейросеть, будет выдавать стереотипные
% фразы общими словами. Искусственно-интеллектуальные системы не понимают смысл ни
% запросов, ни своих ответов, ни той информации, на которой их обучили, ни того
% контекста, в котором ведется поиск. Поэтому актуализируются проблемы
% достоверности выдуваемых сведений, этичности содержания, логичности и
% соответствия здравому смыслу. То есть в отсутствии реального, критического,
% рефлексивного мышления, ответы генеративных нейросетей могут противоречить сами
% себе, быть предвзятыми, предоставлять вымышленные или несоответствующие
% действительности данные. К примеру, в ответ на вопрос, сколько минут следует
% варить 3 яйца, если 5 яиц варятся 10 минут, система выдаст 6 минут, что будет
% ошибкой здравого смысла. Или, к примеру, чат GPT считает, что у лошади 8 ног.
% Две правые, две левые, две передние, две задние. Я недавно спрашивала, Яндекс
% Алиса, например, полагает, что в ноябре только один вторник. Ну, то есть
% понятно, какие проблемы с ними возникают. Таким образом, говоря об
% онтологическом статусе, подчеркнем, что нейронная сеть это математическая
% модель, построенная на перцептронах, которые состоят из трех видов элементов.
% Принимающих сигнал, анализирующих его и выдающих реакцию. То есть какой-то
% ответ. Собственно, почему говорят о ненадежности результатов работы нейросетей и
% о том, что их сложно проверить, что они работают как черный ящик. В многослойных
% системах огромное количество таких перцептронов и не обязательно элементарных.
% На каждом из них, из этих миллиардов элементов, в каждом слое выполняется
% настолько много операций, что поиск узла, в котором, например, произошла ошибка,
% вручную просто нецелесообразен с точки зрения затрат времени и ресурсов. Обычно
% в случаях, когда нейросеть плохо обучилась, разработчики отправляют ее на
% переобучение, корректируя выбор, критерии и так далее. Но, естественно, любая
% выборка конечна и не сможет вместить в себя всю полноту реальности, с которой
% этой системе затем придется иметь дело. У нее нет воображения, поэтому она сама
% не сможет ничего оценивать. Она просто производит операции с блоками информации,
% у нее отсутствует выход в пространство абстрактных форм. Так, несмотря на
% некоторые биологические аналогии с работой нервных клеток, нейросети не являются
% живыми, это лишь инструмент работы с большими данными, пусть и сложный по своей
% архитектуре, и ресурсоемкий в плане воплощения в железе. У ныне распространенных
% искусственных систем отсутствует такая базовая характеристика всего
% одушевленного, как наличие воли. То есть нейросеть ничего не делает сама. За ее
% функционирование и использование могут отвечать только люди, разработчики,
% собственники и пользователи. Таким образом, применение генеративных нейросетей
% оправдано, к примеру, в случаях оптимизации рутинной работы, то есть
% автоматизации каких-то шаблонных действий или индуктивного обобщения. Но нужно
% ограничивать или осознанно контролировать их использование в случаях принятия
% экзистенциальных и этических решений, поскольку нейросети не способны осмыслять.
% И создание уникального творческого продукта, которым, например, является научная
% статья. 
Помимо проблемы плагиата, этот вопрос можно рассмотреть в контексте появившихся недавно нейросетей. Выделим следующие принципы использования нейросетей в научной деятельности и коммуникации:
\begin{itemize}
    \item недопустимо применение для обмана, оскорбления, подлога или искажения данных;
    \item недопустимо использовать при требовании самостоятельной работы;
    \item недопустимо применение в области принятия этических и экзистенциальных решений;
    \item необходимость декларировать использование, указывать объемы и аспекты применения.
\end{itemize}

\subsection{Коммуникация в сфере науки как ценность}

Наряду с ценностями самого научного знания (истинность, рациональность, общедоступность и др.), коммуникация в научной среде сама обладает огромной ценностью, поскольку:
\begin{itemize}
    \item является живой средой формирования и функционирования научного знания
    (научное знание выражается в языке, распространяется в среде);
    \item позволяет делиться опытом (обсуждение обогащает пониманием специфики исследования, позволяет учиться на чужих ошибках, подчеркнуть позитивный опыт);
    \item дает возможность рассмотреть проблему с различных точек зрения;
    \item высвечивает новые перспективы и вопросы;
    \item побуждает участников точно и ясно формулировать свои идеи.
\end{itemize}
Важно осуществлять коммуникацию в сфере науки на международном уровне,
поскольку воспитанные в разных традициях, коллеги из разных стран могут дополнять друг 
друга и совместно находить решения общих проблем, помогать друг другу в научных исследованиях, которые имеют общечеловеческую ценность. 


\section{Наука как социальный институт.}

\subsection{!!! Индивидуальное и социальное. Профессия и призвание}

Для категориальной пары <<индивидуальное и социальное>> существуют синонимы: 
<<уникальное и универсальное>>, <<частное и общественное>>, <<личное и коллективное>>.
 
В каждом человеке одновременно равноправно присутствуют оба эти
полюса противоположностей. Ни социальное само по себе, ни индивидуальное само
по себе не первично, т.о. не определяют человека полностью. Проблемой является понимание, где и как пролегает граница между индивидуальным и социальным в каждом человеке. 

% К примеру,
% реальную память конкретного человека нельзя называть в полной мере ни
% исключительно его собственной, ни абсолютно коллективно сформированной. С одной
% стороны, на восприятие и запоминание событий влияет множество других людей, в
% том числе косвенно, например, через художественное произведение, а с другой
% восприятие и запоминание являются уникальными актами осмысления, которые человек
% может совершать лишь в индивидуальном порядке, сам, в своей голове, в
% собственном состоянии переживания. Тем не менее, благодаря общению с другими,
% человек способен хранить память даже о событиях современником, которых он не
% был. Поэтому, как бы личностно ни была осмыслена, такая, например, историческая
% информация, невозможно говорить и о том, что она сформировалась человеком
% исключительно индивидуально, без влияния контекстуальной среды, которую
% производят другие. 


Прежде всего, когда мы говорим о человеческом
обществе, мы имеем в виду не только популяцию (набор сосуществующих друг с
другом особей одного вида) иначе нам бы не потребовалась социология, достаточно было бы биологии. 

Казалось бы, очевидно, что человеческое
общество невозможно без \textit{другого}, без
отношения к другому. Структура \textit{другого}, во-первых, не дана нам изначально от рождения, а во-вторых, формируется сначала во
внутреннем измерении, затем уже как бы экстраполируется на внешних других. 

% Так,
% каждый из нас во взрослом возрасте может быть другим, сам по отношению к себе.
% Ведь каждому известно явление внутреннего диалога, общения с самим собой,
% которое не было бы возможно не быть в человеческом существе, так сказать,
% дублера-наблюдателя, который, например, винит за тот или иной поступок или
% оправдывает себя перед собой, требует отчета о причинах принятия того или иного
% жизненного решения, анализирует произошедшее с собой в прошлом, планирует
% действия на будущее и много еще о чем с собой разговаривает, смотрит на себя со
% стороны, простирается мыслью, охватывая и свои текущие действия, и временные
% модусы прошлого и будущего, и воображение иных пространств и образов, которых в
% эмпирии здесь и сейчас нет. 

% Если говорить упрощенно, я, как здесь и сейчас
% эмпирически поступающее и нечто ощущающее существо, могу не совпадать с собой,
% как существо мыслящим, которое свои действия, поступки, ощущения и соображения
% рефлексирует. Когда мы были маленькими детьми, изначально мир представлял для
% нас как бы сплошной спектр различных состояний, такой калейдоскоп меняющегося
% положения дел и внутреннего самоощущения в связи с этими изменениями. Мы
% отвечали тому, что нас захватывает всей полнотой своего существа, осваивали
% действительность, представляя и как бы населяя все собой. Могли увидеть в
% обычных предметах метафорически другие вещи и играли во все, примеривали на себя
% любые роли.
% От избытка этого опыта и постепенного научения взрослому языку в
% какой-то момент в возрасте двух-трех лет на нас обрушивается серия открытий,
% прежде всего отдельности себя и мира. И у нас впервые устанавливается
% человеческая система восприятия. Мы понимаем, что мир гораздо больше, чем то,
% что мы видим здесь и сейчас. Для нас включается восприятие хронологического
% времени и самое главное мы обретаем самосознание, то есть начинаем отчетливо
% понимать. Вот я и вот все остальное. Я отдельно, я сам, вот я. Нам становится
% многое непонятно, как устроен мир, на чем это все держится, если я отдельно и
% мир не создан. 
% И начинается период вопросов или почемучек. Мы начинаем задавать
% много вопросов. Вот так в нас устанавливается структура другого. Я сам себе
% другой, потому что вот я, тот, кто эмпирически здесь и сейчас, и тут же я на
% него смотрю со стороны и понимаю, что это я, думаю о я. Это возвратное к себе
% движение и называется рефлексией, буквально складывание удвоения, которое
% начинается с отождествования себя с собой и одновременно понимание различия себя
% и мира. 
% Ну и вот этот внутренний ревизор с двух-трех лет будет в нас развиваться
% и в период от четырех до шести мы переживем второй кризис, когда станет
% окончательно понятно, что у каждого отдельного другого человека свое видение
% мира, свое воображение в голове и прямого доступа мне туда нет. Да еще и видеть
% одни и те же события мира мы можем по-разному. Так необходимо будет развивать
% совесть, буквально совместное ведание, то есть я что-то знаю, вижу, мне что-то
% дано и что-то я должен с этим знанием делать, как-то поступать в его свете так,
% чтобы согласовывать свои действия с внешними другими, чтобы и мне было хорошо и
% им. 
Ассоциальность включает в себя и внутреннее пространство человека в той мере, в какой он
отстоит сам от себя, сам для себя является другим и с этим другим выстраивает
отношения неизбежно совместного бытия. От другого внутри себя уже никак не
отделаться и не отселиться, как это можно было бы устроить физически по
отношению к другим людям. 

% Именно поэтому в частности очень важно быть в единстве
% с собой, в согласии, со своей совестью. если мы осознанно плохо или неправильно
% поступили, соврали, например, то нам придется помнить об этом, чтобы для других
% не раскрылся наш проступок. Но это означает, что мы будем расслаиваться с собой.
% Одна часть нас знает о вранье, а вторая его скрывает и следит за тем, чтобы оно
% не раскрылось. То же самое происходит с завистью, страхом, обидой и многими
% подобными нашими деструктивными переживаниями, которые уводят от единства с
% собой. Поэтому важно их преодолевать в себе. Кроме того, мы представляем в
% голове наши связи, отношения с внешними другими и это очень сильно влияет на
% наши поступки в действительности и на взаимодействие с реальными людьми.

Так что
можно сказать, что каждый носит прежде всего в себе внутреннюю социальность и
как бы через призму внутренних оценок, своих представлений, к тому же меняющихся
у нас в течение жизни, встраиваться во внешние коллективы, группы, коммуникации.
Естественно, наши идеализированные представления о других и о себе, различные
надежды и ожидания часто не оправдываются на деле. Видимо, из-за этого
человеческие отношения такие сложные и в обществе происходят неоднозначные
феномены. Действует прослойка или экран осмысления. Мы не просто повторяем,
воспроизводим тысячелетиями сложившиеся иерархии и все эти связи с другими, мы
можем поставить под вопрос необходимость именно такого социального уклада и
предположить, что другое, более действенное, удобное, правильное, на наш взгляд.

Отсюда возникает история человечества, как трансформация социальных порядков,
опирающаяся на изменения в понимании человека мира и своего места в нем. 

% Приведу пример известного эксперимента с обезьянами. В клетку поместили пять макак,
% подвесили к потолку банан, подставили к нему лестницу. Как только какая-либо из
% макак пыталась забраться за бананом, остальных четверых из шланга поливали
% холодной водой. Неприятная штука, согласитесь? Через некоторое время макаки
% смекнули, что если никого не пускать на лестницу, то никто не будет облит. В их
% сообществе появилась такая норма поведения стаскивать с лестницы, не пускать на
% нее любого, кто попытается взобраться за бананом. Далее в клетке стали по
% очереди менять по одной макаке на новую. Каждый новый член этого сообщества,
% завидев под потолком банан, пытался взобраться за ним. Остальные его стаскивали
% и не пускали. Заменив постепенно всех макак, ученые наблюдали сохранение этого
% порядка, хотя ни одна из этих новых пяти не присутствовала при обливании водой,
% за попытку достать банан, никто за ним не лез, поскольку все друг друга от этого
% удерживали. В принципе, напоминает человеческое общество и передача стереотипов,
% да, ведь? Контрольный вопрос, что отличает человека. 

% В обществе люди всегда
% задаются вопросами о смысле происходящего, о смысле установленных порядков и в
% своей голове прикидывают, как может быть и что будет, если мы так сделаем. это
% не значит, что человек лучше или выше животных, на самом деле, во многом мы
% более ущербный вид, чем остальные, но это говорит о специфике, особенности
% именно человека. Для нас все завязано на осмыслении, понимании. Удивительно, что
% мы не рождаемся уже говорящими на человеческом языке, понимающими смысл из такой
% вот системы восприятия, построенной на другого, на структуре другого, ну или
% структуре различия как такового. 

Что делает
каждого из нас собой? Индивидуальным не является то, что каждый разделяет с другими
внешними людьми, например, верование, представление о мире, моральные нормы, в
которых мы соглашаемся с другими, которые перенимаем от них.Поскольку это вещи
внешние, они могли бы быть другими. 

А вот что в нас не иное? В чем проявляется
индивидуальность? То есть, что в каждом из нас неделимо и от каждого неотделимо?
Своё собственное, не иное. Отличает от всего другого и всех других, но
безусловно, это не собственность в обыденном смысле, поскольку вещи юридической
принадлежности не определяют собственно человека. Их можно полностью поменять,
но от этого не изменится своё. 

Интуитивно кажется, что может быть ближе и
роднее, чем своё собственное тело. Однако, процессы в теле, наши гормоны, гены,
физиологические особенности не детерминируют нас. Многое делается напротив,
вопреки как внешним обстоятельствам, так и внутренним условиям. Все клетки тела
обновляются, меняются каждые 7-10 лет, но мы остаёмся собой. Материи, из которой
мы сделаны, форма, в смысле генетически заложена, единят нас с другими,
обеспечивая семейные сходства и принадлежат к одному роду. Время, в котором мы
живём, пространство, которое занимаем тоже максимально общие универсальные
формы, скорее мы принадлежим им, чем они наша собственность. 

Тогда возможно
собственно своё обеспечивается не чем-то одним, но набором особенностей. В конце
концов уникальность для каждого человека отпечатки пальцев, строения роговицы,
память, точка зрения, особенности характера, способности, интересы, жизненный
опыт поступки. Также и у каждой даже внешне не похожей на другую вазы на деле
уникальное окрашивание, чуть различные составы, форма. Что уж говорить о гораздо большем числе специфических черт у
животных и растений. 

Возьмёмся определить таким образом собственно своё,
например, конкретного кота. Опишем внешность повадки, предпочтения, назовём имя
собственное и все применяемые к нему уменьшено ласкательное и перечислим события
его жизни, с чем имел дело, как на что реагировал, как опыт усвоил и в этом своё
ускользает. Почему? Кота можно приучить к другому. В новой среде он будет вести
себя не так, как обычно. На какие-то вещи даже знакомые спонтанные иначе иногда
реагировать, но при этом останется собой. Все эти вещи окажутся набором
проявлений его индивидуальности, но не ей самой. 

У человека тем более могут
полностью измениться привычки, в экстремальных условиях он выдаст иной раз, чего
даже сам от себя не ожидал, он может изменить имя, скорректировать характер и
внешность. Кроме того, полнота описания невозможны. Детали, нюансы и потенции
можно перечислять бесконечно, бесконечно углубляться в подробности, но это не
исчерпает сущности определяемого. Когда мы теряемся перед этими парадоксами
своего собственного, перед неприступностью того, что уникально, нам может начать
казаться, что все могло бы быть иным, то есть, что все изменчиво и заменяемо. 

Но
на самом деле свое уникальное безусловно есть, обеспечено самим фактом
существования каждого отдельного и изменчивым его не назовешь, потому что если
бы оно изменялось, мы бы не оставались собой. 

% Мы с вами, ребят, специально
% уделяем внимание рассмотрению парадоксов или опорий, потому что они окружают
% самые значимые вещи, смыслы жизненные, такие, без понимания которых нам плохо,
% без понимания которых мы не чувствуем себя полноценно и свободно. Поэтому нужно
% научить с ними дело, чтобы действительно понять, а не сорваться в
% конструирование схемы, в которой для непротиворечивости утверждается одна часть
% и игнорируется другая. Надо научиться спокойно принимать реальность во всей
% полноте, а реальные феномены всегда противоречивы, тогда мы учимся не сужать
% свое зрение, а смотреть как можно шире, потому что только во всем размахе
% реальное можно действительно увидеть. 

Так вот, что означает этот ход принять
парадокс в нашем случае, когда мы ищем свое собственное? Этот шаг принять, что
мы не знаем свое, но незнание своего, невозможность его концептуализации не
означает, что оно нам недоступно, ведь мы и так без нашего желание от рождения
сами свои уникальные есть. То есть, я могу только быть собой, но знание о том,
что делает меня мной, невозможно. Почему знание об уникальном невозможно? Чисто
логически, ребят, смотрите, в знании мы схватываем универсальное, общее, для
ряда похожих, но на деле единичных предметов или явлений, а об уникальном знание
просто не может быть, потому что оно противоположно универсальному и это
нормально. Никто бы из нас, наверное, не хотел, чтобы о нашей индивидуальности
было доступно всем знание и оно бы нас полностью детерминировало, значит,
закрепощало бы, повязывало бы по рукам и ногам, а мы свободны. Кстати,
этимологически свое и свобода слова одного происхождения и не будет в принципе,
даже если кто захочет, ограничивающего знания об индивидах. 

Еще раз смотрите,
какой здесь ход. любая попытка уловить свое является смотрением со стороны,
поэтому неизбежно оборачиваться разделением, противопоставлением себя, самому
себе, а значит, потерей единства, которое только и может хоть как-то приблизить
к своему, потому что индивид это неделимое по определению. 

% выдающийся
% отечественным мыслителем Мирам Константин Шмамар Дашвили по этому поводу
% приводят следующую показательную аналогию. Процитирую вам из его книги «Эстетика
% мышления». Мы ведь в зеркале себя не видим, а видим себя смотрящего в зеркало. И
% точно так же в наших идеологических представлениях, в идеологических частях
% сознания мы не видим себя вовсе. Мы видим кого-то, имеющего о себе, такие-то
% представления и так на себя смотрящего. Это не мы сами. Невозможно засечь себя
% не смотрящим в зеркало. Извне тоже никто не скажет достоверно о нашем собственно
% своем, так как не дано быть за другого. Другой не может влезть в мою шкуру и за
% меня что-то подумать, почувствовать или пережить моим способом. Надеюсь, на этом
% примере разбора парадоксов своего собственного вы видите, где заканчивается
% территория науки и начинается территория философии. 

Есть такие сферы бытия, в
которых знание невозможно или не действует. Даже если мы его составим, например,
об этом конкретном котике, оно не будет иметь ценности, потому что вся прелесть
его в том, что он сам самобытный, самостоятельный и с долей спонтанности. Все
это, естественно, не означает, что о таких вещах не нужно говорить. Напротив,
ведь, конечно, каждый хочет найти себя, понять, что именно его свое собственное.
Сделать это трудно еще и потому что со всех сторон другие спешат нам указать,
кто мы такие и чем должны заниматься. Мы не доверимся социальным стереотипам, не
пойдем в данном случае по пути предлагаемым коллективам, потому что ищем
противоположное, индивидуальное, свое. Тогда берем философский инструментарий и
продолжаем мужественно в такие вещи все равно самостоятельно всматриваться, как
они нам являются. То есть, пусть в своей глубинной сущности они не преступны, но
какими-то сторонами к нам поворачиваются через что мы имеем с ними дело, в чем
они нам доступны, в каких аспектах открываются. Пребывая при своем собственном
или исполняясь в собственном своем, мы совпадаем с собой. В таком состоянии в
нас как бы выключается дублер-наблюдатель, мы перестаем смотреть на себя со
стороны и становится неважно, как мы выглядим, правильно ли делаем то, чему
отдали здесь и сейчас, что было до этого и что будет потом. То есть, исчезают
также страхи и желания. Это состояние можно назвать «меня нет». Оно связано с
определенным отключением от восприятия себя. Например, когда нас что-то
захватило, мы в какой-то момент перестаем замечать, что у нас что-то болит,
когда до этого болело, или долгое время не чувствуем голода. То есть, нам не
мешают представления о себе, контролирование себя, они как-то растворяются и не
отвлекают. И этому сопутствует чувство единения со всем миром, чувствуется
одиночество, даже если человек в этот момент не среди других людей. Мы всем
своим существом пребываем здесь и сейчас, не рассеиваясь ни на сопоставление с
другими, ни на представление неприсутствующих непосредственно перед нами
пространств, не копаемся в прошлом и не строим планы на будущее. Мы целиком
концентрируемся в моменте «теперь». А он единственное, настоящее, реальное,
подлинное. Ведь прошлого уже нет, а будущего еще нет. И даже если мы их
представляем, они лишь возможности у нас в голове. А концентрация на том, что
непосредственно сейчас дает максимальную полноту действительности. Это
счастливое состояние. Не случайно в языке есть поговорка «счастливые часов не
наблюдают». В экстазе полноты исполнения исчезает восприятие времени. Занимаясь
чем, мы перестаем замечать течение времени. Это может быть все, что угодно.
Прогулка и наслаждение пейзажем, пребывание рядом с любимым человеком,
воспитание детей, осмысление чего-то важного, игра, интересная работа, чтение,
просмотр фильма, уборка, слушание музыки и так далее. Общее в этих моментах
состояние захваченности, когда мы отдаемся полностью вплоть до самозабвения.
Парадокс в том, что мы по-настоящему обретаем себя, только отдав себя
захватывающему чему-то исполняемся, то есть достигаем интенсивной полноты своей
жизни, что позволяет нам взять свою максимальную амплитуду в данных условиях.


Таким образом, индивидуальность каждого предполагает разворачивание своего
собственного данного нам от рождения уникального угла зрения. Никто другой в моей
шкуре за меня так видеть, переживать, осмыслять не может. И в свете этого зрения
уникального набора способностей, склонностей. 

Наконец, пометим главный
критерий своего. Это то, что мы не можем не делать, чем бы ни занимались.
Например, кто-то может во всем видеть сложные задачи и исполняться в том, чтобы
их решать, все вокруг оптимизировать. Кто-то не может без коммуникации общаться
с другими, помогать им советом, поддерживать, делиться информацией, выстраивать
связи и в том числе вести с собой диалог. Кому-то обязательно надо что-то новое
придумывать, критерий, ведь даже в обыденных рутинных делах кого-то тянет все
упорядочивать, систематизировать, тщательно, аккуратно выполнять, пусть и
обыденные рутинные процедуры, но доводя любое дело не до совершенства. 

Свое собственное каждый человек может развернуть на
любом содержании, в любом делании. Этим-то призвание и отличается от профессии.

Профессия имеет содержательное наполнение, то есть, что человек делает? 

В то время, как призвание
основано на уникальном способе быть, поступать, видеть, думать, то есть, как
человек делает --- неважно, что именно. 

% Говоря о самореализации, это, напомню,
% согласно пирамиде Маслоу, высшая потребность человека, без удовлетворения
% которой он в полной мере не будет чувствовать себя счастливым, многие начинают
% ошибочно полагать, что стоит мне подобрать профессию по душе, как я буду
% абсолютно счастлив. Но ошибочность такого мнения должны указать следующее
% соображение. Во-первых, дети счастливы без какой-либо профессии, более того,
% постоянно примеривают на себя разные роли и захватывать могут одинаково
% постройка крепости и ухаживание за больными в кавычках куклами. Во-вторых,
% предположим, человеку больше всего приносит смысл производство обуви, научные
% исследования в области химии или продаж товаров. Даже если человек реализуется в
% любимом занятии, в его жизни остается время, когда его отвлекают, приходится
% делать рутинные вещи или нужно переключиться на что-то иное, в том числе на
% отдых. Следовательно, он не будет полностью все время своей жизни счастлив своим
% собственным, если понимает его содержательно. Иначе проходит жизнь человека,
% когда он видит, что его свое собственное не зависит от содержания того, чем он
% занимается, а исполняться можно в каждом проживаемом мгновении. Например,
% креативить и находить нестандартные решения можно во всем, от сочинения музыки
% до паяния микросхем, от разработки нового метода синтеза до участия в играх.
% Так, только разглядев в себе свой способ бытия, можно быть счастливым каждое
% мгновение, концентрируясь на том, что здесь и сейчас, применяя себя, осуществляя
% свое бытие на материале любых содержаний. И вот, когда человек ощутил на себе
% эту истину, я могу исполниться в чем угодно, проявить себя в любом деле, суметь
% применить свои способности даже в том, что не нравится содержанию, тогда он как
% бы укореняется в бытие, перестает быть потерянным или растерянным, неуверенным,
% обретает себя и свое место в мире, а значит, и устойчиво встраивается в
% взаимоотношения с другими людьми. И если в этом ключе глянуть на игру
% социального и индивидуального в научной деятельности, то нетрудно заметить, что
% хорошие, продуктивные коллективы исследователей, которые качественно и
% ответственно подходят к своей работе, выстраиваются только вокруг людей,
% которые, вопреки каким-то внешним негативным обстоятельствам и неудачам,
% мужественно продолжают идти к свету истины, которые стараются на этом
% профессиональном поприще максимально развернуть и применить свои способности, а
% также учиться новому. Конечно, не все, кто приходит в науку, являются со
% стереотипной точки зрения учеными до мозга костей, но это не беда, ведь каждый
% может смотреться в себя, осмыслить свои уникальные склонности и способности,
% которые можно было бы применить в рамках профессии ученого. Кому-то нравится
% кропотливая работа руками, например, в химии проводить точно и аккуратно
% аналитические процедуры или организовать синтез веществ разнообразными
% способами. Кто-то, напротив, меньше любит работать руками или возиться с
% приборами, но его хлебом не кормит, дай только что-нибудь рассчитать, решить
% какие-то сложные интеллектуальные задачи, представить цифры, данные в виде схемы
% таблиц. Третье, больше любит рассуждать теоретически, пытаясь подобрать какие-то
% аналогии, помогающие представить, скажем, механизм реакции или тот или иной
% эффект транспортных свойств твердого тела. Четвертое, любит работать с текстами,
% систематизировать данные, интерпретировать, выражать научные идеи в языке,
% переводить зарубежные статьи, выступать на конференциях. А пятый, мало что,
% спосвящий в этих всех тонкостях, просто хороший управленец, руководитель,
% который может этих четверых организовать, помочь им раздобыть реактивы,
% договориться о проведении серии экспериментов в сторонней организации,
% проконтролировать техника, чтобы вовремя починил прибор, пробить источник
% финансирования, убедить руководителей на производстве, купить в свою группу
% разработку и так далее. Конечно, это идеальный коллектив, в котором каждый
% находит себе место по призыванию, каждый имеет возможность реализовать свои
% способности. В таком коллективе обычно, когда люди заняты каждым своим делом,
% царит атмосфера взаимопонимания и взаимной поддержки, и даже если возникает
% разумево, все спокойно обсуждается и решается разумом. Притом интересно, что
% такие коллективы могут складываться вовсе не обязательно на базе одного отдела
% или одной лаборатории, это могут быть люди из разных структурных подразделений,
% организаций, из разных организаций, из разных городов и даже стран. Однако,
% человек не на месте, когда ему не дают проявлять свое собственное, реализовывать
% свой потенциал, или же он сам не разглядел в себе те способности, которые можно
% в данных условиях продуктивно применить, человек чувствует себя несчастливым и
% частенько возникают всякого рода перекосы. Кто-то становится агрессивным по
% отношению к другим, постоянно провоцирует конфликтные ситуации, пускает сплетни
% или наоборот, он замыкается в себе, теряет интерес к работе и к жизни, впадает в
% депрессию. Поэтому очень важно в социальном плане внуки, чтобы максимально
% исполниться, во-первых, научиться всматриваться в свое собственное, что
% называется, найти себя, а во-вторых, уметь видеть эту социальную ткань вокруг
% себя, чтобы встроиться в нужную именно вам констелляцию отношений. И если на
% текущем рабочем месте вы чувствуете такой социальный дискомфорт, это вовсе не
% значит, что нужно уходить с этой работы или бросать науку. В любой сфере
% коллективы разные и везде вы встретите как деятельных людей, так и тех, кто не
% нашел себя и поэтому мешает другим. Надо искать свое место, выстраивая настоящие
% человеческие связи с теми людьми, которым вы чувствуете тягу, присматривайтесь
% друг к другу, предлагайте интересно, на ваш взгляд, ученым провести совместные
% исследования, связывайте знакомства на конференциях, посещайте какие-то
% обучения, стажировки, научные мероприятия и кто знает, может быть, ваши
% настоящие соратники, с которыми вы образуете удачный коллектив уже рядом. 

\subsection{Социологические исследования научной деятельности}

Исследования в области социологии научной деятельности проходят на трех уровнях:
\begin{itemize}
    \item микроуровень (исследование научных коллективов: их формирование,
    управление ими, распределение ролей, сеть связей вокруг учёного, призвание и профессия);
    \item мезоуровень (как наука существует в государстве, в каких организациях занимаются наукой и т.д.);
    \item макроуровень (какова внутренняя структура научных учреждений и иерархия их соподчинения, как происходит воспроизводство научных кадров).
\end{itemize}

\subsubsection{Теория Роберта Мертона} 

Роберт Мертон считал, что ученые, получившие
признание коллег на ранних этапах научной карьеры, имеют в дальнейшем гораздо
больше шансов к продвижению своих идей, публикации работ, дополнительному
финансированию и т.д., по сравнению с исследователями, которые медленнее
набирают обороты или на этапе обучения не были высоко оценены старшими
коллегами. Эта закономерность получила название \textit{эффекта Матфея}.

Внимательно рассматривая эту теорию, можно заметить, что данный эффект верен лишь на
уровне социального стереотипа. 
В действительности известны прямо противоположные случаи, когда,
вопреки именно неблагоприятным условиям, и в отсутствии раннего проявления
способностей вырастает настоящий ученый.

\subsubsection{Акторная сетевая теория Бруно Латура}

Основная идея Латура в том, что продукт научной деятельности создается как учеными, 
так и множеством сопутствующих акторов (среди которых, н-р, рецензенты, поставщики 
материалов и оборудования, финансирующие структуры, лабораторные животные и т.д.).
Они выстраиваются в удачной или не очень констелляции или сети взаимосвязей. 

\subsection{Понятие социального института. Государственная организация науки} 

Слово институт в широком смысле является синонимом организации (иерархическая правительственная структура, а также процесс создания связей, структурирования).
Примеры социальных институтов: семья, армия, церковь, государство,
образование.

Социальный институт --- исторически сложившаяся или созданная
целенаправленными усилиями форма организации совместной жизнедеятельности людей,
существование которой диктуется необходимостью удовлетворения социальных,
экономических, политических, культурных или иных потребностей общества в целом
или его части.

Под формой организации совместной жизнедеятельности понимаются смысловые
образования, понимаемые и разделяемые членами общества, а также предполагающие
передачу будущим поколениям в ходе воспитания. 


% Не случайно
% мы заговорили об обществе как о неком организме, разные члены общества
% объединяются в группы, производящие, как говорит уже знакомый нам Мираб
% Константинович Мамбардашвили, органы, такие мыслимые формы, благодаря которым
% делается что-то, чего не могло бы быть самого по себе, без усилия осмысления,
% без концентрирования и когерирования усилий нескольких людей через эти формы,
% как через линзы. Соответственно, эти бестелесные органы формы выполняют каждой
% своей социальной функции, подобно функционированию органов и тканей в живом
% организме. В этом смысле существование социальных институтов, как структур,
% обеспечивающих нормальные режимы функционирования социального организма,
% базируются на регуляции при помощи системы ценностей и набора норм, поведения,
% взаимодействия, осуществления той или иной деятельности индивидами, входящими в
% соответствующий социальный институт. Так, разнообразные совокупности идеалов,
% нормы ценностей создают различный тип институциональных отношений, способов и
% форм коллективного взаимодействия. Это, в свою очередь, отражается на характере
% производства научного знания. В этом смысле можно выделить два типа структур,
% которые друг с другом корридируют, как мы выше проговорили, про социальный
% институт. Это и абстрактные формы, и конкретные учреждения, реализующие эти
% формы на деле. 

Рассмотрим внутренние институты науки и примеры реальных учреждений в РФ.
\begin{enumerate}
\item Институт управления наукой --- занимается общим регламентом структурирования и функционирования исследовательской деятельности (Министерство науки и высшего образования).

\item Институт подготовки научных кадров --- институт аспирантуры и
докторантуры, институт научного руководства, институт повышения квалификации (высшие учебные заведения и Институты РАН). 

\item Институт аттестации научных кадров --- контроль за процессами и проведением процедуры аттестации, оценка подготовки кадров высшей квалификации и присваивание официального статуса 
(ВАК РФ). 

\item Институт экспертизы науки --- рецензирование научных публикаций, институт
патентования, институт наукометрии (Роспатент).

\item Институт научной информации --- институты авторства и соавторства, институты
публикации и хранения научных произведений, и т.п. (РИНЦ на портале \texttt{e-library}). 

\item Научные организации --- места, в которых трудятся профессиональные ученые (университеты, Институты РАН, исследовательские и конструкторские отделы предприятий, НИИ и
медицинские научные организации).
\end{enumerate}

Рассмотрим три модели взаимодействия науки и государства в виде следующей таблицы. 

\begin{table}[H]
\centering
\begin{tabular}{|p{4cm}|p{4cm}|p{4cm}|p{4cm}|}
\hline
\textbf{Модель отношения} &
\textbf{Плюсы} &
\textbf{Минусы} &
\textbf{Результат для учёного} \\ \hline
Наука существует абсолютно свободно от власти &
Свобода творчества, на личном интересе учёных &
Нет организации, единства, финансирования &
Нет ориентиров, наука только на свои деньги \\ \hline
Власть подчиняет науку, использует в своих интересах &
Организация, финансирование угодных областей &
Идеологизация, принуждение, необъективность &
Нет свободы, идеологизация ограничивает \\ \hline
Умеренное соотношение, гармоничное сосуществование науки и власти &
Гос. поддержка исследований, управление/нужные государству разработки &
Неполная свобода (выполнение госзаказа), расходы и риски, конфликты с учёными &
Приспосабливаться, используя выгоды от государства и иногда отстаивая больше прав \\ \hline
\end{tabular}
\label{table:science_and_power}
\end{table}

В современных условиях чрезвычайной сложности и
многообразия мировых и локальных процессов, науке со стороны государства
необходима, наряду с правовой, как материальная поддержка, так и
институциональная. Однако, и в данных аспектах следует стремиться к балансу и гармоничному
взаимовыгодному сосуществованию.

\subsection{Социальные функции науки}

Среди социальных функций науки:
\begin{itemize}
    \item теоретическая --- формирование для общества целостной, многогранной картины
    окружающей реальности, благодаря которой люди ориентируются в мире;
    \item образовательная --- обучение, ориентирование в действительности;
    \item интегративная --- объединение людей в рамках одного общества посредством
    представлений о действительности;
    \item утилитарная --- создание разнообразных техник, технологий, устройств, конечная цель которых полезность, принесение блага обществу;
    \item экспертная --- оценочные компетентные суждения по проблемам и их решению,  внутренняя и внешняя экспертиза.
\end{itemize}

\section{Этические аспекты научных исследований}

\subsection{Понятие этики, морали, нравственности, совести} 

Этика --- философская дисциплина, занимающаяся вопросами блага. В рамках этики
исследуются философские основания поведения человека, осуществления выбора,
совершения поступков в категориях добра и зла, свобода и необходимости, мужества
и долга, совести и честности, ответственность, справедливости и так далее. 

Мораль и нравственность являются основными категориями этики наряду с категорией <<благо>>.

\textit{Мораль} --- в широком смысле принятые в обществе представления о хорошем и
плохом, правильном и неправильном, добре и зле, а также совокупность норм
поведения в этих представлениях. 

\textit{Нравственность} --- относящиеся к индивиду внутренние этические интенции, в соответствии с которыми поступает отдельный человек.

% Когда мы в
% раннем детстве проходим стадию обречения самосознания и затем выясняется
% инаковость других людей, что они могут поступать иначе, по-другому оценивать
% одни и те же ситуации, некоторые наши стремления сделать хорошо себе могут им
% навредить или их обидеть. Тогда, естественно, мы вынуждены договариваться об
% универсальных принципах поведения, на которые все должны ориентироваться.
% Скажем, в нашем обществе принято уступать пожилым людям места в общественном
% транспорте. Интересно, что поступок может быть одним и тем же, мы уступили место
% бабушки, но совершен он может быть исходя из трех оснований. Моральным образом
% мы поступаем, если, например, опасаемся, что на нас будут косо смотреть
% окружающие или устно будут укорять, поэтому, дабы не испытывать социальный
% дискомфорт, мы встаем. С другой стороны, у человека может быть просто от природы
% добрый нрав. Он может стремиться заботиться о других, тогда, не задумываясь, как
% это будет выглядеть со стороны, он тут же уступит пожилому место. Это
% нравственный уровень того же самого поступка. Однако, совершение поступка в
% рамках реализации своего естественного, пусть и доброго нрава или привычки
% следования, принятых в данном сообществе правилам, не сильно отличает нас от
% остальных высокоразвитых животных. 

Совесть --- то, что совместно со знанием в плане взаимодействия с ним и дополнительности к нему \textbf{(???)}. \textit{Совесть} --- это внутренний диалог человека, главной функцией которого является постоянное восстановление единства с самим собой, миром и другими.

% Человека отличает от всех остальных то, что
% примерно к пяти годам жизни у него развивается совесть. В нашем примере, если я
% встаю с места и предлагаю его вышедшему пожилому человеку, потому что подумала о
% том, что ему тяжело стоять в отличие от меня, поставила себя на его место,
% пропустила это через себя, то это добросовестный уровень поступка совершенного
% согласно участному осмыслению ситуации другого, а не потому, что я просто
% добренькая сегодня или боюсь порицания окружающих. Вы чувствуете, что во всех
% трех случаях результат один и тот же, мы уступили место, но глубина и
% подлинность собственного бытия разные. Так вот, этические осуждения и
% сопутствующие поступкам размышления не случайный и не лишний элемент системы
% восприятия, а то, что делает ее именно человеческой. Мы затрагивали этот момент
% в предыдущих вопросах сегодня, но напомню, совесть этим логически, то, что
% совпутствует в веданию, устаревшую русскую ведать, знать, то есть, что совместно
% со знанием в плане взаимодействия с ним и дополнительности к нему. Также во
% многих европейских языках это однокоренное слово с сознанием, например, в
% английском conscience и consciousness. Совесть вообще это не только этический
% феномен, но глубже онтологический. Для человека как вида специфично не просто
% иметь знание как некую схему возможного воображаемого, но на его основании
% поступать перед лицом другого, прежде всего себя, как другого самому себе,
% осмысляющего положение, состояние, поступки эмпирического я. Совесть это
% внутренний диалог человека, главной функцией которого является постоянное
% восстановление единства с самим собой, миром и другими, когда в обществе принято
% одно, а сердце подсказывает делать другое. Сразу вспоминается антигона как
% иллюстрация неразрешимости этого противостояния. Но что тогда опираться в этих
% опорях, если то, что мне по нраву, может не соответствовать общественным нормам.
% я единственный, с кем каждый проводит всю свою жизнь, от кого не отселиться, не
% сбежать, не отделаться. 

% Поэтому самым надежным регулятором для человека и
% является эта внутренняя способность к самосогласованию. Но давайте кратенько
% приведу повседневный пример, чтобы вы не пугались, как будто у нас на каждом
% шагу как в античной трагедии. С каждой точки зрения морали, элементарной
% вежливости, по крайней мере в нашей культуре подарки нужно принимать, даже если
% человек, который дарит, нам не нравится. Однако бывают ситуации, когда более
% правильным будет отказаться от подарка, если человек это делает не просто
% капризниче, а понимая, что это обидит дарителя. Если, например, девушка тем
% самым желает пресечь претензии молодого человека на близкие с ней отношения, я
% бы называла это добросовестным поступком. Это честнее с ее стороны, чем из
% вежливости принимать подарки, подавая ему ложные надежды, продлевая страдания и
% ожидания, да еще и само испытывая дискомфорт по поводу вещей от человека,
% который не нравится. И, с другой стороны, если она не просто на эмоциях
% фыркнула, а спокойно объяснила человеку, что, мол, извини, но вот по-честному
% так и так, то юноша должен быть ей благодарен и, несмотря на опечальность от
% неоправданных ожиданий, все-таки уважать девушку за такой поступок по совести.
% Безусловно, столкновение моральных норм и нравственного выбора проявляется во
% многих жизненных ситуациях. 

% Жан-Поль Сартер, известнейший французский философ,
% экзистенциалист и писатель XX века, уделяет огромное внимание в своем творчестве
% проблемам выбора, поступка, свободы и ответственности. Мыслитель убежден, что
% человек открытое существо, которое способно само конструировать себя посредством
% поступка. И в этом смысле существо свободное. Несмотря на то, что в обществе
% существуют моральные нормы, они не могут целиком и полностью определять
% жизненный путь и выбор каждого конкретного человека. В ситуации выбора
% необходимо самостоятельно каждый раз осмыслять, как поступить. Ведь слепо
% следовать принятым принципам полностью невозможно. Думаю, каждому на своем
% жизненном примере это знакомо. И не обязательно, чтобы это был какой-то
% экстремальный случай, например, на войне, когда наиболее остро стоят этические
% проблемы. Приходить ли на работу с симптомами простуды, следовать ли приказу
% начальства, например, он кажется бессмысленным или ведущим к негативным
% последствиям, помогать ли ребенку выполнять домашнее задание и так далее. Все
% это вопрос морали и нравственности, с которыми мы имеем дело буквально на каждом
% шагу решать, которое всегда необходимо, исходя из конкретных, уникальных для
% данной ситуации условий. Делая каждый свой, даже, казалось бы, незначительный
% выбор. Как говорит Сарта, мы на самом деле выбираем свой мир и выбираем будущее
% всего человечества. Задумывайтесь каждый раз над своими поступками, например,
% бросая мусор в городе на улице Невур, но подумайте, что будет, если так поступит
% каждый. Очевидно, будет свалка. Так вот, прежде чем ругать кого-то за
% неприбранные улицы, убедитесь, что вы сами поступаете правильно. Мне кажется,
% это цартовский вопрос, что будет, если каждый так поступит. Полезно себе
% периодически задавать и учить этому приему детей. То есть, задаваясь любыми
% этическими вопросами, мы каждый раз заново решаем, что такое хорошо и что такое
% плохо. Естественно, это не очень удобно. И во все времена мыслители старались
% найти более или менее универсальный ответ, а наиболее глубокие умы понять
% основания блага. А это важнейшие, в том числе, социально-коммуникативные аспекты
% жизни, поскольку от того, как понимается категория блага, будут зависеть не
% только нормы морали, которые тоже очень важны для регуляции социума, но и
% содержание юридических норм, и системы культурных ценностей, и специфика
% устройства общества. 

Жан-Поль Сартр считал, что человек --- открытое существо, которое способно само конструировать себя посредством поступка, и в этом смысле существо свободное. Делая каждый свой выбор, человек выбирает свой мир и выбирает будущее всего человечества.

\subsection{Этические системы и попытка построения «научного этоса»}

На вопрос о том, что есть благо, можно давать разные ответы, в свете каждого из которых
будет выстраиваться определенная этическая система. 

\subsubsection{Утилитаризм}
Под благом понимается польза, т.е. если поступок, вещь, события, условия полезны, приносят удовлетворение, удовольствие, счастье, то это хорошо.

Оцениваются не люди сами по себе, не
действия сами по себе (хорошие они или плохие), но результаты, последствия действий и поступков. 
В связи с этим альтернативное название --- консеквенциализм.

Благополучие нескольких людей перевешивает благополучие одного. То есть, если возникает соответствующая ситуация, согласно утилитаризму пожертвуют меньшинством ради большинства.

Как этическую систему, утилитаризм концептуализировал британский
философ и правовед Иеремия Бентам.

\subsubsection{Аретология}

Согласно этике добродетели, благо определяется добротностью намерений и поступков, личностных
качеств человека. То есть, хорош тот человек, который прежде всего стремится
быть добродетельным, проявлять мужество, мудрость, справедливость, добродушие,
искренность и другие положительные качества. 

На второй план уходят вопросы о том, счастлив ли добродетельный человек, и к хорошим ли результатам приводят его поступки. Истинное счастье обеспечивается уже самим фактом стремления к добродетели, а последствия поступков находятся скорее в руках судьбы.

Представители данного направления: Платон, Луций Анней Сенека, Святой Августин, Фома Аквинский. 

\subsubsection{Деонтология}

Система предполагает благим поступок по долгу, то есть в ситуации выбора наилучшим будет
поступить как должно, даже если это противоречит личному удовольствию, а иногда
и проявлению добродетели. Является основой профессиональной этики.

Систему связывают с именем философа Иммануила Канта.


У ученых существуют гласные и негласные требования профессионального долга. Эти нормы в свое попытался выделить и систематизировать Роберт Мертон. Он вместе с несколькими коллегами разработал концепцию \textbf{научного этоса}, в рамках которой были сформулированы следующие правила осуществления научной коммуникации: 
\begin{itemize}
    \item универсализм --- научное знание должно иметь надличностный
    характер, необходимо исключить уникальное для субъекта культуры конкретной общности;
    \item коллективизм --- плоды исследования должны принадлежать всему научному сообществу,
    а не ограниченному кругу лиц;
    \item бескорыстность --- отсутствие экономических и эгоистических мотивов, получения личной выгоды, стремления к сенсации;
    \item организованный скептицизм --- адекватная критика коллег;
    \item рационализм --- наука должна стремиться к объективной истине, логически доказанной и
    обоснованной;
    \item эмоциональная нейтральность;
\end{itemize}

Соблюдение данных правил сопровождается балансированием между двумя противоположными тенденциями, например:
\begin{itemize}
    \item ученый должен скорее публиковать результаты своих исследований, но и опасаться
    поспешности выводов;
    \item ученый должен быть восприимчив к новым идеям, тенденциям, но при этом не должен поддаваться интеллектуальной моде;
    \item ученый должен стремиться получить знание, которое удостоится высокой оценки колллег, при этом работать, не обращая внимание на мнение других.
\end{itemize} Ученый должен быть восприимчив.
Любая конкретная этика заводит в тупик, поэтому нужно каждый раз задумываться и соотносить действия в конкретной ситуации с общечеловеческим. 

% То есть, когда
% существуют инструкции и предписания обязательно в жизни будет находиться судьбой
% подкидываться такая ситуация, в которой невозможно действовать согласно
% прописанным правилам и в частности профессиональному долгу. Плохо и хорошо,
% добро и зло, надо и не надо. Это предельные идеи, как бы ориентиры и маячки. Их
% нужно иметь в виду. Но каждая конкретная ситуация, складывающаяся в общении, в
% отношениях между людьми, в нужном познании специфично и должна решаться своим
% способом. Не всегда так, как правильно, как принято, как хорошо с точки зрения
% здравого смысла или общественного мнения. В конце концов, мы не боги, мы не
% знаем, как лучше на самом деле. Потому что ведь бывает и зло с точки зрения
% социального стереотипа, в результате оборачивается благом или хотя бы меньшим
% злом, чем могло бы быть. Если присмотреться к тем именам, которые дают благо
% рассмотренные высшие этические теории, то тоже все плывет. Что такое польза? Как
% поступать должно? Благом ли оборачиваются добродетели? Вот мы помогаем человеку,
% а это может не пойти ему на пользу. Может лучше, если он самостоятельно сделает,
% а своими хорошими качествами некоторые люди могут просто доставать, как
% говорится, причиняя добро. Но тут мы снова упираемся в тот же самый вопрос об
% основании блага. Поскольку невозможно раз и навсегда выбрать одну этическую
% теорию, например, из этих трех, а их на самом деле больше, и можно еще
% напридумывать, давая благу разные имена. Давайте подчеркнем в этом плане очень
% важный для понимания момент о различии теории знания и поступка практического
% реального акта. теории сами по себе никогда не могут нам дать ответа на вопрос,
% как поступить вот в этой конкретной ситуации. Во-первых, теория содержит
% обобщенные представления о тенденциях в каких-то наиболее частых закономерностях
% и не может предвидеть абсолютно все детали каждого определенного случая. А в
% реальных поступках все решают нюансы и детали конкретных обстоятельств. Во-
% вторых, даже если мы вдоль и поперег знаем, например, конфликтологию, какие
% бывают типы конфликтов, какие возможны варианты их развития разрешения, это нам
% ровным счетом ничего не говорит о том, как следует поступить, когда я нахожусь в
% конфликте с определенным человеком или, не дай бог, вовлекаюсь в политический
% или военный конфликт. Поступить, сделать выбор, решиться, это личные акты,
% происходящие не на уровне рассудочного знания, а на уровне другой, нашей
% человеческой составляющей, иррациональной воли. Это невычислимый иррациональный
% акт, поэтому, в частности, на данном этапе разработки искусственных
% интеллектуальных систем, они не могут делать этический выбор, могут только
% просчитывать по заданным критериям количественно вероятность вариантов или ту
% или иную их эффективность, то есть, делать только логический выбор. Когда же мы
% выбираем благо сами, мы опираемся не только на нейтральные факты, но прежде
% всего на свое собственное, о котором я вовсе не зря рассказывала в предыдущем
% вопросе. Например, когда мы выбирали вуз и специальность, куда поступить, мы
% взвешивали не только объективные показатели, вроде близости к дому, престижности
% факультета, отзывов от преподавателей в количестве бюджетных мест и так далее,
% мы смотрели прежде всего на свои склонности и способности, что нам нравится
% делать, что мы уже хорошо научились делать, что нам интересно, и,
% соответственно, чем бы, например, мы точно не хотели заниматься. И вот решиться
% на что-то, понять, совершить поступок я могу только самостоятельно, никто за
% меня извне прожить это не сможет, даже если меня принуждают к какому-то выбору,
% принять это или нет, все равно мой индивидуальный акт. Жан-Поль Сартер в
% цитированном выше своем выступлении напоминает, что мы обречены на совершение
% выбора, поскольку, когда даже ничего не выбираем без действия, это уже выбор, мы
% обречены на свободу. Тут удобно, чтобы показать различия уровней теории и
% поступка провести следующую аналогию. Для того, чтобы сориентироваться на
% местности, мы пользуемся географическими картами. Но пройти путь, реальную,
% конкретную траекторию, нам нужно самим вступать своими ногами, видеть своими
% глазами, выдыхать воздух, слышать, что-то при этом переживать. Согласитесь,
% между тем, что мы посмотрели на карту города и реально прошли по улицам, по
% какому-то маршруту, огромная разница. Также теория важна и нужна, поскольку
% позволяет сориентироваться в спектре возможностей. Естественно, никто не умаляет
% ее значения, просто мы смотрим на границы ее применения. А поступать в
% реальности, совершать выбор, нам приходится в опоре на волю, решимость, в
% которой наступает или не наступает, практическим образом в гуще конкретики
% обстоятельств, нашего их понимания или непонимания и по интенции собственной
% совести. Но легче от этого не сильно становится, хотя мы провели важную
% философскую работу и разобрали иллюзию, если у кого она была, что теории могут
% помочь нам в поступках. Пускай этические и любые другие, психологические,
% политические, физические и другие теории не могут сами по себе служить
% основанием поступка. Все равно непонятно, на что опереться, хочется нащупать,
% устойчивое основание. 

% Не претендуя, естественно, на решение вечных философских
% вопросов, я бы хотела поделиться с вами тем, что я в этом плане нашла для себя и
% на себе проверила. Во-первых, случится с нами может все, что угодно, поэтому
% продуктивнее постараться рассмотреть все как благо. Даже если с нами происходит
% что-то плохое, мучительное, тягостное, вдруг все, все это для чего-то нужно,
% чтобы мы что-то поняли, чему-то научились, что-то креативное изобрели для
% преодоления. Может быть так, как случается все во всем мире с нами, это
% наилучший из возможных вариантов, было бы иначе, было бы хуже. С нами могут
% происходить неприятности, мы можем терпеть неудачи, испытывать нужду, лишение,
% боль, но, если получится это принять и осмыслить в ключе вопроса не за что, а
% для чего, то и негативные вещи можно творчески обернуть себе на пользу. Если
% действительно цель любой жизни не сама по себе жизнь, не ее воспроизводство, а
% полнота исполнения, то мы должны стремиться взять свою максимальную амплитуду в
% данных условиях. Условия могут быть разными и наше состояние тоже, но не
% случайно у нас развивается совесть. Это то, что призвано воссоединять нас с
% самими собой в этом максимуме, включая осмысление максимально возможного, учет
% всего возможного и невозможного. И тогда, во-вторых, не думайте, какую этическую
% теорию выбрать, старайтесь выбрать максимум. Такая вот этика максимума. Хорошо,
% когда и польза будет, и я свои положительные качества применю, и чтобы с
% общечеловеческим долгом совпадало и счастье приносило. Если мы постоянно
% стараемся осмыслять, извлекать опыт для себя из всего, даже из ошибок и неудач,
% гибко корректировать в себе моменты, которые не работают, то, в общем-то, мы
% максимально исполняемся в этом, как, собственно, человеческие существа, думающие
% и творческие. И это главное счастье, критерий которого продуктивность. Учусь ли
% я чему-нибудь, получаю ли смысл, расширяю ли душу? Продумайте этот момент,
% примените к себе, и я уверена, у вас не будет ничего невозможного. И вы,
% настраивая себя таким образом, сможете во всей полноте раскрыться, преодолеть
% любые негативные моменты на пути и чувствовать счастье. На этой позитивной ноте
% идем дальше с нашим новым видением блага внуку. Очевидно, в рамках разных
% дисциплинарных направлений имеет место своя этическая специфика. Вот мы часто
% слышим о биомедицинской этике, а для тех, кто занимается, скажем, чистой
% алгеброй, например, обычно полагаем, что этические вопросы не возникают. Но так
% ли это? Существуют ли универсальные для представителя любой научной отрасли
% этические проблемы или они определяются содержанием научных исследований и
% всегда специфичны? 

\subsection{Этические проблемы различных областей науки}

Рассмотрим внутренние этические проблемы научной
деятельности, разделив их спектр на два основных блока. 

\subsubsection{Проблемы морали и нравственности,
возникающие для любого учёного}

Ученые занимаются прежде всего созданием нового знания
реальности, поэтому необходимо оценить само знание. 

Поскольку исследование начинается с целей, следует прояснить, благая ли она, и все ли члены научного коллектива ее понимают и разделяют.

Полнота, качество, глубина проработки, ценности исследования закладываются на каждом
этапе от широты изучения литературы и методологической грамотности до достоверности результатов и тщательности проведения экспериментов. Не допускаются фабрикация и фальсификация данных. 

Наконец, важно грамотно выразить полученное знание, академично представить его в публикациях или выступлениях. Главное требование --- отсутствие плагиата. \textit{Плагиат} --- это заимствование данных, информации и идеи, части текста из опубликованных работ без правильного оформления цитирования.

Вред плагиата:
\begin{itemize}
    \item последствия от преступления в сфере интеллектуальной собственности;
    \item получение денег за труд другого;
    \item отсутствие развития науки;
    \item отсутствие развития плагиатчика, нереализация творческого потенциала.
\end{itemize}

\subsubsection{Примеры вопросов о благе,
характерные для конкретных отраслей научного познания}

\begin{itemize}
    \item Связанные с объектом исследования
    \begin{itemize}
        \item В какой манере и какой доле известной врачу информации о состоянии здоровья и угрозе жизни он должен раскрывать пациенту?
        \item Этично ли проводить испытания лекарств на животных, прежде чем предложить людям новые фармацевтические препараты?
        \item Как способствовать применению результатов исследований во благо, а не во зло?
    \end{itemize}
    \item Связанные с условиями исследования
    \begin{itemize}
        \item Техника безопасности при работе с приборами.
        \item Защита данных, информации.
        \item Соблюдение условий по методикам.
    \end{itemize}
\end{itemize}

\subsection{Виды и аспекты проявления ответственности исследователя}

Ответственность является готовностью иметь дело с \textit{любыми} последствиями своего выбора или поступка. Каждый прежде всего ответственен субъективно перед собой за исполнение своего уникального бытия.

Рассмотрим на примере разной научной деятельности виды ответственности: 
\begin{itemize}
    \item индивидуальная --- касается общей добросовестности ведения научной работы,
    обеспечения достоверности данных, отсутствие плагиата, качество интерпретации
    результатов, выполнение должностных обязанностей и т.д.;
    \item профессионального или экспертного суждения --- влияние на общественное
    мнение через публичное высказывание;
    \item коллективная --- члены научного коллектива совершают индивидуальные поступки от лица коллективного актора перед лицом той или иной структуры;
    \item косвенная --- переживающий ее не совершает действий сам, но оказывается причастен к событию, обычно путем примеривания на себя поступков другого, зависящего от него или автономного;
    \item социально-политическая --- осознание возможных рисков и масштабных последствий использования результатов проведенного исследования на благо или во вред природе, обществу.
\end{itemize}

Рассмотрим позитивный пример Вернера Гейзенберга, который рефлексирует в своем философском
размышлении. В результате обсуждений исследователи приходят к разграничению
между открывателем и изобретателем. Открыватель следует логике развития науки и
даже предполагая возможное использование своего исследования во вред не в
состоянии остановить прогресс науки. Деятельность же изобретателя оказывается более 
этически нагружена, поскольку он направляет применение уже сделанного открытия во благо или во зло.
Как резюмирует Гейзенберг, для индивида, перед которым научно-технический прогресс поставил важную задачу, недостаточно думать лишь об этой
задаче, он должен рассматривать ее решение как составную часть общего развития.

% Рассмотрим пример Вернера Гейзенберга, который рефлексирует в своем философском
% размышлении под названием об ответственности исследователя следующую ситуацию.
% Он и другие немецкие исследователи атомной физики, находившиеся после окончания
% Второй мировой в 1945 году в оккупации со стороны британских военных, услышали
% по радио о применении ядерной бомбы США в Японии. Ведущий исследователь
% расщепления урана, процесса, лежащего в основании действия ядерной бомбы от
% Таган, был в шоке от случившегося, поскольку, по сути, его разработки дела его
% жизни оказались направлены во вред человечеству. Эта группа ученых,
% высокообразованных или интеллигентных людей была в панике от случившегося. Они
% вели беседы друг с другом, осмысляли свою возможную вину и свою ответственность
% за случившееся. В результате обсуждений исследователи приходят к разграничению
% между открывателем и изобретателем. Открыватель следует логике развития науки и
% даже предполагая возможное использование своего исследования во вред не в
% состоянии остановить прогресс науки. В данном примере открывателем является Отто
% Ганн. Деятельность же изобретателя оказывается более этически нагружена,
% поскольку он направляет применение уже сделанного открытия во благо или во зло.
% Так, изобретатели в США применили данное открытие для создания атомной бомбы, в
% то время как Гейзенберг с коллегами, находившимися в нацистской Германии во
% время Второй мировой войны и осознававшие возможность чудовищных посредствий,
% если такое оружие попадет в руки Гитлера, обратили своими усилиями атомные
% разработки в Германии в мирное русло, трудясь над созданием ядерного реактора и
% убедив свое правительство в том, что на данном этапе создание ядерного оружия
% невозможно. Как резюмирует Гейзенберг, для индивиду, перед которым научно-
% технический прогресс поставил важную задачу, недостаточно думать лишь об этой
% задаче, он должен рассматривать ее решение как составную часть общего развития.
% Несомненно, в ходе своей научной работы мы должны ставить вопросы вроде тех, что
% продумывали коллеги Гейзенберга, даже если мы занимаемся фундаментальной наукой,
% что может сделать каждый отдельный человек, чтобы направить прогресс в науке к
% лучшему. И, с другой стороны, мы также постоянно должны задумываться о
% последствиях представления результатов своей научной работы, задавать вопросом о
% том, как необходимо выразить эти результаты, как их интерпретировать, чтобы люди
% нас правильно поняли, чтобы наши выводы не пошли им во вред. То есть, данный вид
% ответственности ученого имеет место не обязательно в сфере военных разработок
% или новейших технологий, любые научные данные могут вызвать социальные
% трансформации или быть использованы при принятии политических решений. Поэтому
% также важно качество интерпретации результатов для неспециалистов, как, к
% примеру, показывают случаи 2009 года в Италии, когда сейсмологи, проведя
% измерения в определенной местности, сообщили, что землетрясение возможно с
% низкой вероятностью, и местные власти, трактовав это как отсутствие угрозы, не
% стали эвакуировать жителей, однако катастрофа все же произошла, тысячи людей
% погибли и остались без домов, в итоге ученых судили и лишили свободы на
% несколько лет. Кто виноват? Ученые, которые должны пояснять, что такое
% вероятностный прогноз, и посоветовать все-таки на всякий случай подготовиться к
% чрезвычайной ситуации, или представители власти, которые в своих решениях могли
% опираться только на предоставленные данные, и сами в сейсмологии не разбираются.
% С одной стороны, кажется, что прописывание этических норм бесполезно, но мне
% кажется, это все-таки нужно, поскольку задает хоть какие-то ориентиры, но, как
% вы понимаете, работать с ними нужно гибко и при необходимости трансформировать.
% Мне кажется, самым действенным является разбирать такие случаи с молодыми
% учеными, чтобы они сами задумывались и развивали добросовестный подход. Таким
% образом, подытоживая наш сегодняшний разговор, современная ситуация ставит перед
% учеными ряд требований, согласно которым необходимо, соответственно, не только
% профессиональной квалификации, но и высоким личностным качествам,
% общечеловеческим идеалам и ценностям. Задумываться об этической стороне своего
% исследования, основать зону своей ответственности, но также уметь устраивать
% коммуникацию с отечественными и зарубежными коллегами и, конечно, заниматься
% наукой по призванию, чтобы она приносила смысл, опыт, новые знания и счастье
% применения своих способностей. 