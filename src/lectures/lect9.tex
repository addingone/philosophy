 Сейчас классический период начинается. Имеется в виду, вот был у нас
натурфилозофский период развития науки, теперь классический период развития
науки. Для некоторых источников, вот таких, которые мы не сильно одобряем, если
вы по ним будете готовиться к экзаменам, по некоторым источникам наука только в
это время и начинается, только в 17 веке. Вы должны понимать, что нет. Вас учат
по такой версии, поэтому можете озвучить ее, но сказать, что... Ну, либо, что
называется, сражайтесь с нами и доказывайте, что все, что было до этого, не
наука, вернее, радикальным образом отличается от того, что мы называем наукой.
Понятно, что... Вот я вам рассказывала в прошлый раз про магию, там еще про что-
то. Понятно, что это не наука, в современном смысле этого слова. Но тем не
менее, если мы посмотрим на историю развития науки дальше, вот каких-то таких,
ну, резкого, знаете, такого, вдруг появилось новое знание. Нет, знание
трансформировалось из-под того, что было раньше. Оно не появилось вдруг из чего-
то пустого места какого-то. Вот новая наука взяла и возникла. Но тем не менее,
эта трансформация очень важна. Так, теперь я, значит, включу вам презентацию,
чтобы у нас было с вами предметно, да, окно. Что окно-то опять мне это самое.
Сейчас, секундочку, подождите. Презентация есть. Видно? Да, заглавный слайд.
Хорошо. Так, ну, вот, значит, становление классического типа рациональности
наука нового времени. Вопросы. Потом их скопируйте. Можете сейчас
сфотографировать. Можете каким-то образом, в общем. Я не буду их зачитывать. Я
вот почему. И мы сразу же переходим к социально-историческим, идеологическим
условиям формирования науки нового времени. И вот вам сразу же страшненькую
картинку. Почему? Потому что буду рассказывать про страшненькие дела. Речь идет
о том, что 1618 год, 1648 год, то есть первая половина, середина 17 века, это
30-летняя война. Знаменитое противостояние протестантов-католиков, которое
переросло в династические войны. Много про них рассказывать не будем. Главное,
что они были общеевропейского масштаба. То есть коалиция французов, англичан,
кого там еще, шведов, против Габсбургов, династии Габсбургов. Это австро-
венгерская империя. Ну, так я в целом рассказываю. Но это не было в чистом виде
войной католиков с протестантами. Почему? Потому что, ну да, Габсбурги это
католики, но в противоположной, что называется, стороне с ними воевали и
католики, и протестанты. В общем, это война за господством над Европой, в ходе
которой Европа едва не исчезла вообще. Наемные армии. Помните, мы говорили, что
они появились в эпоху Возрождения. Вот они содержались не только за счет
государства, но и за счет просто грабежа захваченных городов и сел. Это
тотальная война против мирного населения. Часто солдаты вырезали вообще все
мужское население захваченных городов. Не щетили там ни детей, никого. Ну,
известные моменты этой войны страшные. Это Вафаломеевская ночь, Матенбургская
резня и так далее. Кроме всего этого ужаса, малый ледниковый период, когда
замерз даже Босфор, самое сильное похолодание пришлось как раз на 30-летнюю
войну. То есть Европу помимо этого терзали эпидемии чумы. И, в общем, вот вся
эта ужасающая... Ну, по-настоящему ужасающая. Знаете, я не преувеличиваю. Вот
все это происходило как раз в то время, когда, конечно, люди еще не помышляли о
какой-то новой науке. Дело как бы было даже не в том, что нужно заниматься
наукой. Но вот мировоззрение людей существенно изменилось. Если в эпоху
Возрождения, как вы помните, таким двигателем нового и всякой новизны были такие
люди, подошвленные культом фортуны, то есть фортуна — это нечто такое, что
позволяет поймать удачу за хвост. Она дарит шикарные шансы. Например, там, ну,
купцам, которые ездят там за специи, они за один мешок специи могут дома купить.
Ну, понимаете. А главное, вот, как говорится, сразиться с фортуной. И если она
будет фортуной на твоей стороне, и если ты обладаешь великим виртусом, то, в
принципе, ты тот самый гомо-виртуозо, про которого говорили гуманисты. И значит,
ну, всё прекрасно в твоей жизни, да. А вот в эпоху 30-летней войны, в период
30-летней войны, знаете, все вот эти вот подарки фортуны, все были исключительно
неприятны. Ну, то есть война, понимаете, такая всеобщая. Целое поколение выросло
на ней. Ужасающая демографическая ситуация, настолько она серьёзная моральная
деградация. Про это пишут, вот, можете почитать, Соловьёвка, что толкует нас, и
так далее. Один момент просто скажу, что в Европе в 1650-х годах рост населения
настолько, ну, в смысле, не рост населения, а население сократилось настолько,
что ввели многоженство. Это была Германия. Германия. Ну, в смысле, это
европейская часть территории. Вот, 1650-й год. В Баварии для восстановления
населения церковные власти, церковные власти разрешили многоженство. Вот я вам
зачитываю указ. Так как потребности Священной Римской империи требуют
восстановления мужского населения, уничтоженного мечом, болезнями и голодом, то
каждому мужчине в течение ближайших 10 лет разрешается жениться двумя женами.
Ну, просто, чтобы детей было больше. Вот так вот. Ну, значит, в итоге стояла,
Европа стояла на грани совершенно незавидной судьбы быть задворками Османской
империи, потому что в это время османы, турки, как раз расширяли свою
территорию. Идея фикс Османской империи завоевать Европу и превратить ее
население в мусульман. И они как бы планомерно шли к реализации этой идеи. И
исполнением этих чаяний мешают лишь, знаете, такие дикари, которые в 17 веке
вдруг вмешались в этот расклад. Там, значит, Европа, ну, вот, воюет за
господство, там, ну, ее там, свои вот эта 30-летняя война. Тут османы дерзают
делить мир на части. И вот какие-то дикари, и так сами османы назвали жителей
России. Османскую империю в этот период подтачивают казаки, да-да-да, те самые
наши вот мужики на конях, которых спонсирует Иван Грозный. Так что в ножники
надо кланяться Европе, Ивану Грозному. Вот это прям очень серьезно. И
историографию нужно менять нафиг, потому что на первых страницах должен быть, у
всех учебников Европы, должен быть Иван Грозный, который как раз не позволил в
то время очень легкой добычей была Европа. И если бы вот не... Вот это вот,
извините меня, прослоечка между турками, да, и Европой. Вот тогда... Ну, это не
совсем турки, османы, но не важно. Вот тогда, в общем-то, никаких учебников по
истории Европы бы не было. Ну, вот вы поняли, да, мою мысль, пафос этого
момента, лекции. Ну, тем не менее, значит, все заканчивается, и война тоже
заканчивается, извините. И возникает некая дипломатическая история, связанная с
эсфальским миром. Согласно историографии победителей, ну, как обычно, они же
пишут историю, да, это текст, который выстроил новый европейский порядок на
основе принципа государственного суверенитета. Что это означает? Верховенство,
независимость и самостоятельность государственной власти на территории
государства, независимость международного общения и обеспечение целостности,
неприкосновенности территории. Вот. Но по факту, конечно, никакой мир в Европе
не возник, не случился. И надо сказать, что с тех пор, по сути дела, в
европейской действительности война-то прекращалась, не так вот, чтобы прям
совсем не было войн, она и не прекращалась. Ну, вот там пару лет, может быть, не
было нигде каких-нибудь катаклизмов, а так она вот до сих пор. Потом это в 19
веке наполеоновские войны, весь 18 век до наполеоновских войн, династические
войны, после наполеоновских войн были еще частные какие-то, вот локальные, там
гражданские были потом войны на всей территории Европы, потом они переросли там
в конфликты, потом Первая мировая, между Первыми и Мировой там тоже, сами
понимаете, ну и так далее. Так что вот эта стратегия национальных государств,
которая сложилась, она, в общем-то, не привела к серьезным таким вот, ну как бы
такому стабильности социальной. Что такое государство национального типа? Вот,
пожалуйста, государство национального типа, запоминайте, это, значит, принцип
границы, единые границы, очерчивающие единую территорию. И в пределах этой
территории единая государственная власть. Она может быть разного характера, там,
республика, монархия, все что угодно. Сейчас пока не в этом дело, мы не говорим
про форму правления. Мы говорим про принцип национального государства. Вот этого
типа государства не было пока еще никогда. Да. Были древние государства, но там
тоже не было разговора про жесткую границу. Почему? Потому что, ладно, сейчас не
буду погружаться, немножечко другая история. Были государства феодального типа,
когда земли, когда твое владение, вот если, значит, я феодал, мое владение
разбросаны по всем, ну, тем территориям, до которых я когда-то доходила и их
захватывала. Вот. Единой границы нет. В эпоху возрождения то же самое, плюс
города-коммуны независимые вот эти вот отсеки. Что это типа городов-полисов. И
вот новый тип государства, национальный. Запоминаем, да? Постепенно все-таки
складываются условия для более-менее мирных каких-то возможностей. Но самое
главное, появляется принцип международного права, как координации отношений
между государствами. Но смотрите, вот в чем дело. Это было непотребностью
правящих элит. Вот этот вот мир как таковой, более-менее основанный на
международных каких-то там правилах, на международных юридических каких-то там
закономерностях, вот этот принцип был нужен неправящим элитам, потому что они
как раз династические вот эти вот дома, они сражались за господство. Это нужно
было буржуа. Это нужно было тем, кто в этот период старается наладить просто
свою жизнь, чтобы вести торговлю, чтобы развивать ремесла, ну чтобы жить более-
менее, что называется как-то. Но понятно, ну то есть как всегда, людям, которые
не находятся непосредственно, вине, которые непосредственно воюют, скажем так,
да, вот им-то больше всего война и не нужна. Для нас в данном случае важно
подчеркнуть, что становление буржуазных отношений, опять же, как бы мы сегодня
не ругали там и капитализм, и буржуазные отношения, и фу, и все остальное, но
это всегда все начинается, ну то есть все, любое явление историческое,
культурное, социальное, начинается не с какого-то там чистое зло победило, да,
нет, это традиции, которые спасали, ситуация, которая спасает. Вот все, чего мы
не коснемся, это ситуация, которая крайне важна в то время и в то место, она еще
не переродилась во что-то, что мы сегодня, например, не любим. Так вот,
буржуазные отношения, которые в тот период сложились, они возникают из тех
бургов, по которым мы уже говорили, то есть это города, созданные именно для
реализации профессиональной деятельности, купечества и ремесленничества. Так
вот, эти бурги богатеют все это время, несмотря на какие-то там перипетии
исторические, и в них возникает определенная элита, элита, которая уже имеет
деньги и имеет, соответственно, определенную власть над даже правящими элитами.
Ну, то есть с ней уже приходится считаться как минимум. Так вот, что значит
предполагает, что значит быть буржуа, что это за тип такой? В первую очередь это
человек достойной профессии. Кто считались, вернее, что за профессии считались
достойными? Это судьи, адвокаты, прокуроры, нотариусы, врачи и хирурги,
цирюльники. Ну, то есть все ювисты, условно говоря, плюс врачи и хирурги,
цирюльники. Ну, то есть цирюльники в то время это те, кто, нужно сказать, как
это назвать, ну, что эти травпункты, вот так назовем это. Ну, и плюс ко всему
они еще и бродобрее. Такое вот. Постепенно к этому, значит, вот к этой среде
добавляются купцы-негоцианты. Помните, существуют различия между купцами-
негоциантами и купцами-лавочниками. То есть негоцианты это те, кто оптом куда-то
там едет, закупает, то есть тот, кто рулит, скажем так, под самим процессом
мелкой торговли. Соответственно, они как бы, ну, вот в этой иерархии социальной
поднимаются на воле высокий уровень. Далее. Разбогатевшие имя, если некие, те,
кто имеет мануфактуры уже. Требования к буржуа. У него есть достаточное
благосостояние. Им куплены какие-то земли вокруг города. Непременные условия он
живет в собственном доме. То есть это новый благородный человек. Новый
благородный человек. Но единственное, чего у этого благородного человека не
было, у него не было юридического статуса благородства, потому что по-прежнему
им обладают наследные дворяне, наследные аристократы. И сразу же после, ну, в
смысле, встает вопрос об этих привилегиях. То есть, смотрите, революции, которые
возникнут в 17-м и далее веках, они не продиктованы на самом деле вот какой-то
там нищетой народа. Народ в это время не духоподъемен. Народ в это время живет в
состоянии просто вот выжить бы. И я говорю не случайно, в смысле, я не
преувеличиваю. Почему? Потому что свидетельства об этом говорят. Знаменитый
труд, огромный по масштабам, ну, даже для сегодняшнего дня, там 600 с чем-то
страниц у него, в 30-е годы 17-го века написал Бертон «Анатомия меланхолии».
Анатомия меланхолии. И вот он что отмечает в предисловии. «Что нигде не слышу
новые вести, обычные слухи о войне, о бедствиях, о пожарах, наводнениях,
грабежах, убийствах, резне, метеоритах, кометах, привидениях, чудесах,
призраках, захваченных селениях или осажденных городах, смешение бесчисленных
клятв, ультиматумов, помилований и указов, прошений и тяжб, ходатайств, законов,
воззваний, жалоб и обид. Мы слышим это каждый божий день». То есть, понимаете,
всё, что он перечисляет, исключительно Дорова, всё это под негативным уклоном. И
он напишет, насколько люди переживают многообразие всех форм меланхолии,
многообразие всех форм депрессии в нынешнем смысле этого слова. Ну и переводчик
и автор водной статьи этой книги, Ингер, отмечает, Бертон пишет свою книгу
главным образом для того, чтобы удержать каждого отдельного человека от
беспросветного уныния и не дать ему сойти с ума. Ну такие вот условия. Ну в этих
условиях, знаете, это вам не христианские войны после Средневековья, когда
христиане чувствуют в себе достаточное количество ресурсов, вдохновения, чтобы
начать воевать за свою, как говорится, вот там, ну за свою собору. Так вот,
народ в это время не встает, что называется, на баррикады. Их поднимают. И
поднимают кто? Уже возникший слой крупных буржуа. Вот так надо понимать. Буржуа
– это уже новые лидеры. Лидеры и мнений, и каких-то решений, и постепенно лидеры
социально-политического порядка. То есть это новый благородный человек, я еще
раз повторю, чье благородство уже не связано с приобретаемым хозяйственно-
экономическими… Ой, простите, с приобретаемым… Получаемым наследство статусом,
да, это не наследная аристократия, а благородство связано с приобретаемым
хозяйственно-экономическим каким-то статусом. То есть не с военным делом, вот
это принципиально, да. И буржуазный тип отношений – это капитализм. Так вот,
капитализм. Мы как бы к нему теперь обращаемся. Это важный период, ну, не важный
момент. Да, вы должны понимать, что, например, в историографии, основанной на
марксизме, все основано на экономическом базисе. Вообще экономика – базис, а все
вот эта вот культурная история, про которую я вам рассказываю, и даже
исторические какие-то события, они все как бы надстройка. Но марксистский подход
– это не единственный, и даже в России он не единственный с некоторых пор, как
вы понимаете, с тех пор, как Советский Союз перестал существовать. Вот. Но он
один из важных подходов, мы его тоже как бы учитываем, но учитываем уже вот не в
чистом виде. Поэтому я не буду вам говорить про то, что экономические отношения
– это базис, а все остальное – настройка. Нет, я считаю, что все важно, крайне
важно. Важны любые нюансы, мир называется нелинейный, и иногда выстреливает в
качестве важнейшего фактора вовсе не экономический. Иногда выстреливает и
полностью переворачивает все, не экономика. Иногда это вот такие даже трудно
улавливаемые, но культурные по своему характеру потребности и основания. И может
быть сегодняшние обстоятельства вы как бы тоже в это время учтете, если будете
рассуждать на эту тему. Ну в общем, но в любом случае, конечно, капитализм
крайне важен, и его знание и понимание тоже важно. Капитал. Что такое капитал? В
первую очередь, это средства, используемые для получения прибыли, рост
производства. Рабовладельческое, феодальное общество, то есть не
капиталистическое, основывалось на производстве продукта в единицу времени. Оно
было преимущественно стабильным. Ну вот, то есть ты стабильно, что 10 лет назад,
что сегодня, примерно одинаковое количество продукта производишь в единицу
времени. как бы, ну как сказать, это зависит от условий производства, это
зависит от потребностей. Что называется, больше не надо или больше не могу и
ничего для этого не делаю, чтобы больше было. Капитализм принципиально основан
на том, чтобы в единицу времени, ну как бы последовательно производить все
больше и больше товаров. Соответственно, ну вот и все негативные, не, многие
негативные последствия, которые мы связаны с капитализмом, ну вот они с этим
связаны. Это снова. Но в тот период все-таки капитализм, это не в первую
очередь, как бы, форсирование прибыли. В первых рядах речь идет об этих
идеологических основаниях. Многие исследователи все-таки ставят их на первое
место. И вот здесь можно, ну такую оппозиционную во многом концепцию Марксу
предложить, концепцию Макса Вебера. Речь о том, что именно дух протестантизма
оказывается основанием капитализма. То есть не экономические потребности, а как
раз идейные, культурные, духовные. Речь о чем? Речь о том, что в 16-17 веке
развивается пуританский капитализм. Вот они, представители пуританского
капитализма. Они всегда одеты в черное, всегда в одну и ту же одежду. То есть
они как бы показывают, что мы как бы противостоим вот этой вот роскоши феодалов,
которые, флотиповики там, все на себя напянули, там у них жабо, у них перстни, у
них пряжки изумрудные на сапогах и туфлях. Ну в общем, вот мы не они. А почему?
Да потому что мы как раз новая элита, которая вот по-новому воспринимает и
деньги, и свое богатство. Пастор Ричард Бакстер, Бакстер это сторонник Кальвина,
он говорит о том, что человек должен искать воздаяние в исполнении своих
профессиональных занятий, а Господь, который заранее знает, кто будет избран, а
кто проклят, указывает нам на как бы избранность, вот если ты избран к спасению,
то тебе способствует успех в профессиональном деле. Я как бы, если будут
вопросы, я объясню, с чего он это взял, что называется, на каком месте из
Библии, из Евангелия основаны такие утверждения, и они могли быть только в
католической, на самом деле, среде, потому что православие, разумеется, тоже
знало это место, это первое послание к Оринфянину, но как бы трактовалось
совершенно иначе. Но вот так оно сложилось, как сложилось, что называется, в
Западной Европе, появилась концепция предопределения к спасению, и вот на
основании неё богословы протестантского толка выдвинули эту идею, что успех в
профессии, есть указание на то, что человек в числе избранных. Но богатства,
которые ты получаешь, они не ради наслаждения, они не ради вот самих богатств, в
некотором смысле, на котором можно наслаждаться. То есть, например, сегодня
кушать уже не хлеб, но мясо, не мясо там, не знаю, что там дальше было по тем
временам. Ну, понятно, да? А деньги нужны для того, чтобы их пускать в рост. Они
должны быть пущены в рост. А если, что называется, у тебя остаются деньги от
того, что ты вкладываешь в производство или пускаешь в рост, то ты занимаешься
благотворительностью. Ну, вот вы очертания уже современного капитализма можете
прямо увидеть, да? Увеличение капитала, в принципе, как самоцель. Когда мы
говорим о том, что деньги пущены в рост, то, соответственно, очень скоро и
банкиры, и ростовщики, которых раньше даже в города не пускали, чтобы они жили в
европейских городах, потому что это совершенно недостойная профессия,
недопустимая для христианина, как считалось долгое время, всю эпоху
Средневековья. Но вот, тем не менее, она становится вполне себе достойной,
вполне себе респектабельной, и они тоже становятся буржуа. Это, ну, как бы тоже
достаточно долгий юридический процесс. Ну, в общем, он произошел. Я не буду в
подробности погружаться. Вот таким образом, примерно возникает капитализм.
Техническая основа развития капиталистической формы хозяйствования, развития
машинного производства. Ну, про это, по-моему, у меня в слайде об этом есть. Вот
в этом вот. Нет, нет. Ну, хорошо, вы все запоминаете, да? Сейчас я посмотрю,
может быть, дальше следующий слайд. Ну, да, есть. Вот есть и Франклин,
Бенджамин, и про выбора здесь упоминания есть. Поэтому полюбопытствуйте потом.
Извините, я просто презентацию не смотрела. Ну, вот, мануфактура, да? Немножечко
о них расскажете. Для мануфактуры еще не свойственно, в чистом виде, мануфактура
– это староучное производство. Но на них постепенно появляются машины. Не в 17
веке еще. Все-таки развитие фабрик, а это именно уже мануфактуры, такие, которые
машинизированы, да? Где вот этот процесс продукта, производства продукта,
разложенный на стадии, уже машинизирован. Фабрика называется. Но в любом случае,
процесс, который приводит к машинизации производства, называется промышленная
революция. Это, в самом широком смысле, это 17-19 век. Процесс, который приводит
к полной механизации. Полная механизация производства – это создание
машинизации, конечно же, укрепили сложившуюся ситуацию. Но я о них немножечко
уже сказала. Первая буржуазная революция – это в Нидерландах. Нидерланды, я
напоминаю, еще после лекции говорила, это среда, в которой развивается этот
пуританский или по-другому аскетический капитализм, в первую очередь. Вот именно
там. В Нидерландах. Тогда это двигатель европейской, как бы буржуазной вот этой
вот трансформации. Это двигатель и лидер капитализма, двигатель и лидер
машинизации процессов. Ну вот как бы долгое время Нидерланды были номер один в
Европе. То есть там, понимаете, они тоже участвовали в войнах, но не так
активно. Вот в этих вот династических войнах и тому подобное. Ну, английская
революция, это 17 век, знаменитое смещение короля Якова Стюарта, то есть казни
короля, смещение Якова Карла, конечно, казни короля Карла I. И они так
напугались сами. Вот серьезно. Они пришли в такой ужас, и английское общество,
что бы клялись, никогда больше королей не казнить. И поэтому сегодня в Англии
такая интересная ситуация, половина населения ненавидит королевскую власть, и
конкретно прям монарший дом ненавидит по отдельности. И королеву там Елизавету
Вторую ненавидела, и там Диану тоже, даже, да, Диану. То есть королевскую власть
ненавидит как таковую, считает, что это просто нахлебники, которые сидят на шее
государства. А половина ее просто обожает. Вот как бы средний, вы не найдете
среднего. Да, это вот такой, знаете, травматический стресс, который возник на
фоне английской революции. Ну и Великая французская революция. Ну это уже 18
век, мы о ней как бы не будем сегодня много или вообще, в принципе, даже
говорить. Она достаточно показательна в разных там свидетельствах и кино. Я
думаю, вы фильмы смотрели. Много чего про эту. Ну, конечно, тяжелую ситуацию.
Что нужно знать про всех этих эволюциях? Они никогда не были однородными. Они
никогда не были процессом, в которых вот есть одна лидирующая сила, она в итоге
победила и переустроила социальный строй, извините за тавтологию. Конечно, нет.
Ее начинают одни люди, их потом там смещают или казнят. Ее продолжают другие
люди с другими целями. Тех людей тоже смещают. И тоже эти цели отменяются,
появляются третьи. И в одной эволюции больше, в другой меньше. Ну, в общем, это
такой процесс, который нельзя назвать какой-то логичной трансформацией. Это
всегда дикая стихия, по большому счету. Идейным основанием революции буржуазных
и вообще всех этих трансформаций буржуазных была эпоха просвещения и,
соответственно, просветителей. Кто такие просветители? Вот про это мы вас,
конечно, будем спрашивать. Про революцию мы спросим, но вы знали про их факты,
да? А вот про просвещение спросим более пристально. Просвещение легко
хронологически идентифицировать. Это просто XVIII век. Важно, что это новое
мировоззрение формируется уже не в университетской среде, и уж тем более не в
университетах монастырей, какого-нибудь там школ соборных. Нет, это люди,
которые вот как раз новые интеллектуалы, да, нового типа. Они не в каких-то
академиях там платоновских, да, помните, как это было в Средневековье,
Возрождение. Давайте так. Возрождение – это платоновская академия, такие кружки
по интересам, но они, понимаете, вот как бы все равно в особом месте. Люди
собраны в особое место, похожие чем-то на академии античные. В Средневековье
интеллектуальная элита, то есть те, кто формирует вот этот вот фон
интеллектуальный, это университеты и монастыри. Они между собой как бы даже
сражаются, да, ну как бы за первенство. Два дискурса – университетский и
монастырский. Но мы как бы с вами делаем акцент на университетский, потому что
они действительно были прям во всем победителями, в каждом дискурте нет, такого
не происходило. Но тем не менее, просто вот как бы выводя линию развития науки,
мы как бы выстраиваем ее все-таки не из монастырии, а из университета. В
античности это академии разного рода, да, это гимназии, там, да, вот это, ну
такие вот, прото-социальные институты науки. Что же это в новое время, в
просвещении? Это светские салоны. Светские салоны, в которые люди просто пришли
покушать и потрещать языками. Почему я вот так немножечко уничижительно говорю?
То есть, опять же, я высоко ценю усилия людей, которые в тот период взяли
лидерство в духе, в смысле в мире идей, да, в Европе. Они, конечно, очень, ну
как сказать-то, вы помните это слово «виртус»? Да, в них есть, был этот
«виртус». Но если мы присмотримся к каждому из них по отдельности, это совсем не
философы, это совсем не лидеры мнений вот в таком смысле, знаете,
интеллектуального разума. Это публицисты. Это вот скорее современные медийные
фигуры. Они как бы входят в моду, понимаете? Вот тот, кто моден, тот и
оказывается лидером в интеллектуальном смысле. И это немножко печалит, потому
что это вот как раз начало того, что мы сегодня имеем. Вот не у вас будут
спрашивать о том, как там, и не у меня будут спрашивать о том, как нам там
устроить, да, или как там что-то объяснить, а будут спрашивать у блогеров, да,
про которых мы даже не знаем, что это у нас. С образованием, не говоря уже о
всем остальном. И вот как бы такая тенденция, она появилась, как ни странно,
задолго, да, в долгом появлении интернета, так что не интернет надо в этом
ругать. Или артиста будут слушать. Да, артист выступит, вот по телевизору раньше
выступали артисты, сидел, приглашали, значит, ведущих, приглашали артиста, этот
артист осуждал как-нибудь о политике, об экономике, а там еще о чем-то там
важном, что, на чем выстраивается наш социум. Он ответственный, артист, вы
понимаете, да, это же полный бред. А сегодня, ну там, в этом блоге, там еще что-
то такое. Кто, а кто они такие? Ну, в редких случаях мы знаем, да, в редких
случаях. Я не говорю, что они все глупенькие, ничуть не бывало, но во всяком
случае общий канвал, вы понимаете, да, проблема есть. Проблема возникла в этом
просвещении. Возникли модные люди. Они собирались в светских салонах. Один из
таких светских салонов, это вот салон Марии Терезы Раде, потом она вышла замуж,
стала Жафрен. Мария Тереза Жафрен, хозяйка знаменитого салона, куда в течение
целых 25 лет сходились представители новых интеллектуальных сил Парижа.
Собирались художники, ученые, философы эпохи просвещения, такие как Гидро,
Вольтер, Даламберг, Гальбах и другие. Ну и там иностранные гости. Заслуги Терезы
Жафрен оценены настолько высоко, что вот ее статуя, которую я вам показываю, она
расположена на фасаде мэрии Парижа. То есть она ничего не сделала, кроме того,
как сделала их модными, понимаете? Ну, то есть попасть в этот салон, это было
очень важно для всех, кто хотел быть модным. Ну, как сегодня, да, попасть на
какой-нибудь там канал. Я даже не буду называть именно. Кто хочет быть модным,
идет на этот канал. Вроде как прикрылся, ну и слушай. А у нее 25 лет не
прикрывался канал, в котором начнут, где нужно было тусить, чтобы стать модным
интеллектуалом. Ну вот один из, да что ж такое. Модных из, модный такой вот, ну,
сбор, да. Вот такие люди. Единственное, что на этот, именно в этот, значит,
светский салон женщин не пускали, потому что Мадам Жанхан была страшна, как
смысл ноги. Ну вот поэтому, ну нет, вот тут чисто это. Поэтому, ну, женщины не
часто становились модными в интеллектуальном смысле. Так вот, давайте, что же
они сделали, кроме, если мы начинаем смотреть, что эти там ведро и доломберы
сделали, мы не найдем философских трудов серьезных. Исключением, пожалуйста,
оказывается только Руссо. Да? Вот он писал там прям много основательно. Ну,
трудно сказать, насколько это, прям философия в чистом виде, потому что систему-
то он не создавал. Но, тем не менее, хотя бы много написал. Вот. Остальные
писали статьи, публицисты, публицистические, такие работы им публицистического
характера. и более-менее, и более-менее такое стабильное, что-то производство
идей стабильное, было связано с формированием энциклопедии. Энциклопедия,
давайте переведем это слово, N, V, циклос, круг, педос, воспитуемых, то есть, в
круг, в круг образованных, круг воспитанных, то есть, чтение энциклопедии, это
вхождение в круг избранных с точки зрения идей, воспитанных с точки зрения идей.
Ну, вот эта вот энциклопедия, по которой, может быть, чуть позднее скажем, она
предлагала, ну, то есть, набор знаний по всевозможным, совершенно всевозможным
аспектам, и история, и технологии, и какие-то понятия, но только в новом
идеологическом ключе. Поэтому мы смело можем говорить, что основание, основание,
то, что произвели посвятители, это новую идеологию. Новую идеологию. А если этой
идеологии нужно, что называется, дать имя, то это идеология либерализма. Салон
вообще, это первое воплощение политической партии. Ну, в таком современном
смысле этого слова, потому что партии были и в Дверьмингриме, и в Византии, но
извините мне, вот сравним давайте, Византии и партии. Это партии, которые
вообще-то, ну, как бы, разделились по принципу, за кого они болеют на скачках.
Тогда, ну, как бы, Византия, это большущая роль не только религии, не только там
царской власти, и тому подобное, но еще и скачек на квадригах. Имеется в виду
вот эти вот квадриги, это колесницы, загряженные четырьмя лжедми. Вот эти скачки
на ипподромах, они просто рулили тогда в Византии, долго-долго-долго. Это было
как бы, ну, как сегодня, даже футбол, и, наверное, если мы, наверное, квидич
вместе взяты, то есть давайте с одним всех вообще. То есть это настолько
популярно, что партии политические делились на синих и красных, то есть вот
какие-то флажки, какие-то знаки опознавания были на этих квадригах. Понимаете,
что такое партия, да? А вот теперь партия это по политическим предпочтениям, по
тому, что вы считаете предпочтительным в социальном укладе. Так появляются
политические партии, и вот они возникают именно в таких светских салонах. Ну,
это салон уже 18 века, напоминаю, да? То есть немножечко о идеологии как
таковой. Что такое идеология? Само слово появилось во времена французской
революции. Его вот, трейси, изобрел его даже, прямо скажем так, для того, чтобы
формировать науку идей. Ну, даже почитай, разберите слово на составляющую. Идео,
идея, логия наук, наука идеи. Вот. Но все-таки, по факту, это стало
формированием программы действия, программы действия для достижения определенных
целей каких-то субъектов политики. Эти субъекты политики могут называться
классом, могут называться партией, как сегодня, могут называться общественным
слоем, но, в общем, в таком целостном, социальном пространстве всегда есть такие
выделенные субъекты политики, которые очень нечетко фиксируемы, и вот у них
должна быть какая-то, ну, чувствуется, что должны быть какие-то выразители наших
идей. Выразители этих идей становятся политиками, несущими определенную диалогу.
Но надо отличать партию все-таки того периода от партии в современном смысле
слова. В современном смысле слова мы говорим о массовой политической партии.
Тогда все-таки партия это вот эти светские салоны, в которых это группа
единомышленников, которые собираются, обсуждают какие-то процессы политические,
политические, предлагают решения, а дальше они идут непосредственно уже и
выступают в каком-нибудь парламенте. Все-таки это аристократические группы,
политические клубы в тот период. Политический клуб. Что дает идеология? Почему
каждое государство должно иметь свою идеологию? Ну вот что ж называется тут, уж
не поспоришь. И хотя об этом продолжают спорить, это глупо. Идеология есть, была
и, главное, должна быть у каждого государства. Почему? Потому что она формирует
основу для легитимности власти. Для легитимной власти. Легитимный, правомерный,
то есть согласие с законом. Легитимная власть это та власть, которой доверяют,
которую признают правомерный, правомочный граждане данного государства. То есть
большинство, как минимум. Ну, желательно, все, конечно, таких государств мы еще
не встречали. А вот чтобы большинство признавало, власть легитимная, оно должно
знать, а какие идеи власть-то разделяет. Ну, то есть наши ли это идеи? Вот это и
есть идеология. Поэтому, конечно, идеология, это не какое-то ругательное слово.
В коем случае. Это нормальный процесс осуществления любого государства. Основа
для легитимной власти, для новых политических сил, для буржуазии, которые тогда
формируются. И как раз явились просветители. У всех, у всякой власти есть своя,
как бы, вот эта основа. А просветители, которые формировали идеологическую
основу для новой власти, то есть для буржуазной власти, это так называемая
рациональная форма легитимности. И ее характеристика, ее содержание это
либерализм. В истории либерализма не было работы, которую можно было бы назвать
ключевой. Ну, то есть здесь ее не... Я говорю, это публицисты, которые толком
ничего не писали. Она складывается из многих трудов, и мы выделяем все-таки,
знаете, не в чистом виде либерализм сложился не у тех, кто в этих салонах,
пардон, штаны просиживал. Они просто бросали идеи. Они модные люди, да, они
бросали идеи, а все-таки мыслители уже более серьезные, философы. Вот философы
формировали либерализм как проекты, делогему и политическую культуру. И что это
за философы? Так, сейчас посмотрим. Нет, простите, философы, я вот вам...
Философы... Но вам придется их записать, да? Все-таки это Томас Гоп... На
предыдущем слайде было как раз про это. Да, да, да, там в центре примерно... А,
да, да, да, вот они, товарищи, да. Томас Гоп, Гопс, Джон Лок, Адам Смит,
Монтеске, не помню, как его зовут, посмотрите. Бенджамин Франклин. Вот эти
товарищи, пожалуйста. То есть, их все-таки мы не относим... То есть, что такое
либерализм? Еще раз, это философско-политическое течение, да? Но это еще и
политический проект, и политическая культура. Ну вот, смотрите. А еще должна
быть успешная идеологема. Чувствуете, как много составляющих у идеологии. Ну
вот, смотрите, вопрос, например, вопрос об идеологеме. Это очень важный вопрос.
Он отвечает на вопрос, каким быть, где быть. У меня была идеология Спарты.
Идеологема Спарты, это лаконийский. Лаконийский ты или не лаконийский? Ты вообще
хоть лаконийский, ну, лаконик, это области была, Спарта. Ты человек-то вообще
лаконийский, это слово, кстати, лаконичное потом пошло, да? Вот. Или, например,
идеология, ну, простите меня, что называется, без всякой рекламы, уж тем более
одобрения, идеология фашистской Германии. Постранство для немцев. Все очень
просто. Прямо, и видите, это идеологема, под которой, как бы, народ говорит, да,
нам надо. И он, соответственно, этой власти дает карбанш. Иди и бери. Я почему
специально такой радикальный пример вам привожу, потому что, понимаете, любая
власть, даже делая совершенно, ну, немыслимые, нечеловеческие дела, как это
делала фашистская власть, она все равно основывается на некой легитимной, на
некой легитимности, которую ей дает народ. А народ ее предлагает в форме
определенной идеологемы. Ну, вот, например, идеологема, брошена мыслителями
пространства, значит, для немцев. Рамфурд фюрдойч, не пустой звук, это не просто
лозунг, знаете, популистский лозунг, нет. Ключевая идеологема либерализма,
борьба за свободу. Вот этот клич, борьба за свободу. То есть, по мысли ее
создателей, свобода, это не то, что определяет человека, это то, что должно, как
бы, в борьбе осуществляться. Понимаете, как бы, перевес упал не на слово
свобода, потому что никто не знает что-то такое и не знал, а на слово борьба. По
мысли, например, философы, свобода, это вообще то, что определяет человека. И,
кстати, философы будут тщательно подчеркивать принцип свободы, свободы. А вот в
политике укрепилась акцент на борьбу. И вы можете почувствовать это в полной
мере в современном либерализме. Тогда этого не было, но в современном
либерализме это вполне есть. То есть, уже не важно, против кого мы боремся,
главное, сам принцип этой борьбы. Понятно, что капитализм встал на ноги в ходе
буржуазных революций, когда буржуа были угнетаемыми. Они кричат, надо бороться с
угнетением. Боремся с угнетением и политически, достраивая новые законы. И как
бы, буквально в виде там сражений. Вот. Но постепенно буржуазия же сама стала
угнетателями. Убившие дракона становятся драконом. Да? А как бы, понимаете, она
же не может уже кидать клич борьбу с угнетением. Нет, так уже нельзя, потому что
тогда бороться будут с тобой. Соответственно, ну, например, ну, хорошо, какое-то
долгое время еще этот клич работает, но он уже работает в социалистическом
ключе. Это тоже один из модусов либерализма. Хотя они там были врагами, как
политические партии, но тем не менее, это модус как бы либерализма в широком
смысле этого слова. Потому что базовая идеологема тоже борьба с угнетением. Да?
Но социализм в какой-то там форме случился, и современное общество даже не
социалистического типа, это уже не борьба с угнетателями. А либерализм-то
существует, он будет по-прежнему бороться. Да? Он будет по-прежнему бороться за
что-то. И уже это будет борьба за свободу социальных меньшинств. Да? Вы
понимаете, о чем я говорю. Это свобода, борьба за вот там еще за чью-то свободу
и так далее. То есть он уже до бреда доходит, но либерализм, это, как сказать,
вот это вот, если он прекратит работу этой идеологемы, он остановится. То есть
эта идеологема, это быть против, бороться за что-то. Это идеологема, которая
является перпетум мобили губерализма. то есть она уже как бы уходит в массы.
Политическая культура это то, что разделяет уже не философы, не какие-то даже
там в отдельные. это вот то, что как бы становится принципом такого нормального
поведения обычных людей, скажем так, не философов. То есть политическая культура
быть против становится основой либерализма. То есть смотрите, быть не за, а быть
против. Ну, соответственно, как бы чем я занимаюсь, будучи там политически
активным человеком в духе либерализма? Я нахожу, против кого я воюю. Я не
предлагаю, что сделать, что против, а я не предлагаю, за что я. Я всегда
озвучиваю против чего. Перманент, вернее, претенденты на роль зла перманентно
обновляются. Еще раз повторюсь, что в новое время это позволило состояться
европейскому обществу. Далее это породило кризис того общества, которое,
собственно, и создан был этим, этой идеологией. А теперь мы как бы к
мировоззренческим установкам просвещения. Вот это те понятия, которые вы должны
знать, понимать, которых вы должны уметь рассуждать. То есть, какие, скажем так,
измы, какие максимально широкие принципы формируют мировоззрение просвещения?
Ну, во-первых, рационализм. Это приоритет рассудочно-логического мышления.
Мишления. Приоритет рассудочно-логического мышления. Критического мышления.
Культ разума. Слово рационализм, а слово рацию. Рацию переводится разум.
Достаточно просто. Я сильно сейчас не буду погружаться, по ходу дела это еще,
может быть, как-то проясним. Более сложное, да нет, она не более сложная, она
менее просто привычная. Понятие эвдемонизм. Эвдемонизм. Эвдемонизм вообще-то это
этическое направление, которое признает главным принципом это чувство счастья.
Чувство счастья. Я это говорю не случайно, потому что позднее мы с вами, когда
будем рассуждать, если у нас получится рассуждать о всяких картиях вольности, то
вы обратите внимание, что в некоторых появляется принцип права человека на
счастье. Ну, то есть право на свободу, право на и право на счастье. Почему?
Потому что Вальбах, Гальбах, Дитро, Вальбах, Господи, Вальтер, Гальбах, Дитро,
они в своих вот таких лозунгах, публицистических работах очень часто говорят про
то, что счастье является критерием и категорией практически повседневного опыта.
Понимаете, счастье, оно обретается вот прямо здесь, в этой жизни и прямо сейчас,
сегодня. Вальбах на свое счастье это право каждого человека. Конечно, никто не
сказал, что такое счастье, но чувствуете, им не надо. Никто не скажет, а что
это, свобода это, а что, а вот счастье это, а что, главное ты борись за него,
борьба за счастье. но очевидно, что в нем присутствует, ну, счастье, понимаемое
просветителями, это в основном мотивы какие? Благополучие, комфорт, благополучие
и комфорт, то есть связь с категорией пикурейства, где счастье, это отсутствие
страдания. Ну, опять же, упрекать их за это не приходится, что называется,
посмотрите вокруг, ну, представьте себе в этой ситуации, посмотрите вокруг, а
вокруг, пардон, трупы, которым очень не хочется стать. Соответственно, ну да,
действительно, просто вот это вот, просто нормальная еда, просто нормальная
постель, просто все живы, что называется. Звучит как по-буржуазному, да, как-то
вот, ну, фу, но, опять же, в тех условиях совсем не фу. В тех условиях это
действительно высший принцип существования. Отсюда великая роль утилитаризма, от
слова утилита с пользой. Направление, да, в моральной философии и вообще в
мировоззрении, согласно которому в основе морали лежит принцип пользы, что
полезно для нашего счастья, то и благо. А польза в утилитаризме как философском
направлении это счастье для большинства людей, благосостояние для большинства
людей. Так вот, благосостояние чувствуете, тогда это неоднозначно деньги.
Благосостояние на сегодня. Какого твое благосостояние? Ну, у меня на счету
столько-то, да? А благосостояние тогда это просто состояние, ну, как бы,
безопасности, состояние комфорта и состояние, ну, что-то у тебя есть в качестве
собственности, что ты можешь, считать своим. Поэтому принцип собственности, он
ключевой, конечно, для буржуазии и принцип вот такого утилитарного подхода,
утилитарного благосостояния. Все эти установки формируют проект модерна.
Ключевой идеи, которая, вот это, она формируется на идее бесконечного увеличения
счастья. бесконечного увеличения, сейчас скажу, вот счастье, как движение
сообща, вместе. Каждый будет счастлив только тогда, когда в обществе будет
гармоничная такая, гармоничная страна, когда всем всего будет хватать. Когда
всем всего будет хватать, соответственно, это и будет счастье для каждого.
Хватать же будет, правильно? Вот. но это такое, как бы, такое просто
устремленное будущее, непонятно, когда заканчивающийся процесс, в основе
которого лежит, конечно же, библейский принцип эсхатологический, то есть Библия
в свое время развернула античное, зацикленное время. Поясню. Смотрите, в
античности мы слили время цикличным. Вот идет, развивается время, значит,
двигается, двигается, потом сгорает весь мир и снова начинается все то же самое.
Причем то же самое настолько то же самое, что может быть родиться такая же вот я
и будет сидеть вот так пальцем, показывать экраны, говорить то же самое.
Понимаете, да, что имеется ввиду? Вот, вот эта цикличность времени ее, по сути
дела, разомкнула Библия, потому что там было сказано, что Бог однажды сотворил
все сущее и однажды в таком виде, в котором мы это сущее знаем, оно закончится,
конец там, да, история, конец этого, и наступит некое новое состояние
реальности, которое вот там связано с определенными событиями, и,
соответственно, человек в этом как-то участвует, он должен сейчас проявить
некоторые усилия для того, чтобы вот там потом оказаться в этом новом состоянии
блаженства. То есть эта идея полностью берется на вооружение, но только
содержание ее уже связано не с религиозными моментами, не с Богом, а как бы
берем в свои руки счастье. Тем более, что счастье благосостояние, тут Бог по
большому счету не он его может обеспечить, а вот как бы идея социального
прогресса. Вот здесь у меня на слайде написано с позиции морали, это
актуальность пруденции. Что это такое? Ну, смотрите, пруденция это благоразумие.
Пруден ман рули, правила благоразуменного человека. Или принцип надлежащего по
обстоятельствам. Но это, знаете, как бы вот то, что я сейчас вам говорю про
мировоззрение, про демонизм и утилитаризм, то, знаете, с этим, со звездочкой
тут, более сложно. Почему более сложно? Потому что прочувствовать разницу между
прежними идеалами морали и вот новым идеалом морали не очень просто. Что имеется
в виду? Вот мораль, например, эпохи античности, эпохи средневековья и
возрождения, это все-таки героическая мораль. Ну, возрождение уже в меньшей
степени, но все равно это мораль героическая. Героическая мораль основана на
принципах все-таки самопожертвования. ты жертвуешь. Ты жертвуешь собственной
жизнью или комфортом или еще чем-то. Поэтому это мораль воином. Ты же жертвуешь
воином, ну, то есть в пределе это мораль воином. Воин жертвует комфортом и
собственной жизнью ради чего-то, ради социума, например, ради своих сограждан.
Как бы государство не понималось, но всегда воин жертвует чем-то в первую
очередь жизнью. То есть не обязательно я не буду, что каждый воин погибает, но
он готов на это. Готов. Поэтому как ни крути, воины по праву получали статус
элиты общества. Ну, по праву. Вот что не говорите. Во всеми. То есть вот это вот
средневековый идеал аристократа-воина это нормально. Почему? Потому что тот, кто
жертвует, тот, кто готов пожертвовать ради тебя свою жизнь. Ну, кем ему быть как
не аристократа. серьезно. Надо же понимать, да? И вот какие-то такие вот, ну,
как бы назовем это так, ну, правильные времена, что ли, военные всегда были
элиты. И это совершенно понятно, закономерно и вполне себе обоснованно. Вот.
Тогда как с эпохи нового времени как бы меняются ситуации. Морально это то, что
благоразумно. И согласитесь, может быть, не очень благоразумно жертвовать свою
жизнь вроде других людей. Да? Ну, как бы, ну, чисто вот с какой-то интересной
такой ситуацией, в которой чувствуется какой-то душок. Извините. Я сейчас не
хочу никого убить, потому что ну, вдруг у кого-то такая мораль, да? Но, тем не
менее, уж и не могу не высказаться по этому поводу. Тем более, что вот этот
принцип надлежащего по обстоятельствам, он был известен еще римским юристам,
древнеримским юристам. И они тогда этим принципом на самом деле обосновывали
все, вплоть до каннибализма и там вплоть до убийства родителей. Ну, то есть, это
как бы такой адвокатский прием, который говорит, что ну, в этих условиях это
было морально. Ну, что-нибудь там. В этих условиях это было морально. И вот,
действительно, принцип пруденции, он активизирует эти позиции. Ну, в этих
условиях, извините, это морально. Вот так. Ладно, на этом я пока остановлюсь.
Это тема сложная и про нее надо много говорить. И вот вам картинки, картинки. Не
то, чтобы вы такие глупенькие, что вам только картинки, не-не-не, просто, чтобы
закончить мысль. Вот смотрите, барокко, искусство барокко, это зримое воплощение
нового мировоззрения, где как раз принцип вот такого вот комфорта телесного,
демонизма и, ну, в общем-то, вот так понимаемого благосостояния. Во-первых, это
целлюлит, извините, мировоззрения, это вот рубинсовская, ну, портрет жены, шубка
называется. Просто обратите внимание, вы, наверное, слышали, да, что в эпоху
барокко изображались такие дородные женщины, и это считалось, ну, действительно,
тип женской красоты. Но вы должны понимать, люди, наконец, как бы, ну, не то,
что наконец, люди прочувствовали в полной мере вот эту вот идею кушать сладко,
кушать сладко и спать долго. Ну, и вот вам и целлюлит, что называется. И
целлюлит, представляете, насколько этот образ жизни был привлекательным, что
даже целлюлит стал привлекательным. Поэтому второй, вот, еще тип появившихся
тогда изображений, это натюрморты. Снечь, много еды, еда такая, еда такая. Вот
эти картины с удовольствием покупались, они вешались. Ну, шубку не покупали, она
в частной коллекции, это все-таки портрет жены обнаженной, поэтому нет, Робинс
не продавал. Других, да, там каких, Венера там обнаженная, там, пожалуйста, да,
там вот этого добра хватало. Вот, Робинс нарисовал, да, и другие. А, видите, да,
вот очень, очень распространенный жанр времен 17-го, 18-го века он уже идет на
спад, ну, на термор сохранился как жанр, а вот именно такого типа идет на спад,
потому что, ну, типа уже накушались, накушались не только еды, но накушались уже
и таких вот изображений. Или другой тип, ну, то, что потом станет Бидермайером,
ну, ладно, я сейчас не буду про Бидермайера, так, в 19-е, а про, знаете, вот,
жанровые картины, когда просто у служанка наливает молоко, да, и это красиво,
это же так замечательно, вот просто бы такой уют, спокойствие, разливается от
взгляда на эту служанку. Великолепный художник Вен Вернейер, голландский
художник. Понимаете, да, что я имею в виду? Это вам уже совсем не те
изображения, которым могло похвастаться даже возрождение и уж тем более
средневековье. Значит, что еще обращаю внимание? на понятие секулярное, которое
мы уже упоминали, но вы, пожалуйста, на него обратите внимание. То есть
секуляризм — это естественное состояние вот того мировоззрения, того
мировоззрения, потому что мы сами, Бог уже не участвует так активно в нашем вот,
ну, в формировании нашего блаженства. И оно уже не за пределами, оно не в
посмертии, оно не после страшного суда, оно здесь уже в благосостоянии. И,
собственно говоря, для того, чтобы, ну, как бы вот эта религиозность, она не
требует прежних форм. И секуляризм нового времени, эта тема-то старая, да, но
секуляризм нового времени связан с антиклирикализмом, тоже еще одно понятие.
Антиклирикализм, то есть против клира, против института церкви. Понимаете, да?
Это как бы вот важные моменты, вы, пожалуйста, их учтите, когда будете
рассказывать. Но, понимаете, почему я подчеркну этот момент? Потому что на
экзамене вы приходите нам и рассказываете, что они были атеистами. Они не были
атеистами. Они развивали новый тип религии. Новый тип религии без церкви, без
всяких вот... Ну, то есть религия, которая просто у тебя в сердце, которая
просто делает тебя религиозным, но в то же время она как бы... ну, то есть это
дарит тебе воодушевление, это дарит тебе высокие цели, это дарит тебе даже
определенный дух самопожертвования, но это не связано с Богом. И поэтому именно
в эпоху Нового Времени рождается безумное количество новых религий. Да-да-да. И
это не протестантизм, потому что протестантизм — это всё-таки нормальный тип
религии, это связано с религиозностью в отношении именно трансцендентной
реальности, трансцендентного Бога. Ну, как бы там Бог в обычном смысле этого
слова, чтобы под этим не подразумевалось. А вот в этих новых религиях, например,
религия Абюста Кондо, да, она вообще была основана... Ну, он тут немножечко с
ума сошёл, но тем не менее, вот просто культ какой-то женщины, конкретной
женщины, всё, давайте, культ конкретной женщины. Или культ, например, поэзии,
это Меллармэрин. Или культ... Вот понимаете, то есть как бы религиозная
составляющая, она изымается из обычной религии и переносится на какие-то
объекты, на любые. Самая распространённая секулярная религия, видите, у неё
такое название, секулярная религия. Или, например, внецерковная религия. Такие
названия можно встретить. Ну, вот самый пример, такой для нового времени пример
секулярной религии, это культ науки. Культ науки. Да? Я вот ещё... То есть,
Агюст Конд, до того, как, например, сделать именно определённую женщину, в
которой он там влюбился, сделать объектом культа, первой половине XIX века, что
он предлагает? На место Бога, предметом почитания, преданности, человеческий род
ставим. Коллективное существо, плюс животное, которому мы привязаны, которое нам
служит. Мне, как человеку, животно-ориентированному, извините за такие детали,
подробности, которые, наверное, вам нужны, но тем не менее, всё равно, конечно,
это кажется странным. Но я не готова. Я готова за какую-нибудь, там, несчастную
собаку убить того, кто её мучает. Я, это, кстати, реально, я боюсь потерять
контроль. Если что-то подобное, увижу, не дай богу. Вот. Но я бы точно не стала
ставить животное на место, вот, ну, то есть, поклоняться, культ, да? Ну, вот
видите, такой, обезкомп разработал 84 культовые церемонии, самый был первым
священником, это, наверное, выглядит. Вот такое вот всё. А самый такой
распространённый вид, ах, не туда, вид всё-таки секулярных религий, они связаны
с масонством, да? То есть там что-то такое религиозное, но в то же время это вот
какая-то такая, ну, она же тайная, да? Что-то такое тайное и особое, ну, как бы,
мистическое, но в то же время понятное. А понятное, потому что это наука, наука.
Масонство, это как раз 18 век, это период масонского бума, он происходит как раз
в салонах Франции, а моду туда его занесли на масонство англичане-иммигранты.
Там была в 1725 году основана первая ложа. В России масонов было очень много, мы
любили, что называется, это дело, и европейские ориентированные всякие такие
тайные течения у нас проникали, но нюанс, что называется, обратите, глава
вольных каменщиков Москвы, он же профессор Московского университета Шварц, он
призывал к изучению герметических наук. Его цитата, герметическая философия есть
матерь, она основывается на знании натуры, имеет познание стихий, первой
материи, улучшение металлов и прочее. Знакомые мотивы? Знакомые, если вы слушали
прошлую лекцию. Русские исследователи того периода серпно занимаются алхимией,
единственное, пневматическая магия не прижилась, вот это вот то, что я вам
рассказывала, не прижилась, вот в алхимию как-то хорошо так ушло, все. Елагин
принимает великое делание под руководством графа Калиостро. Калиостро приезжает
в Москву, трансмутация, правда, не состоялась, хотя анонсировалась очень широко,
и вот секретарь Елагина пишет потом, все закончилось безобразной дракой между
Калиостро и секретарем Елагина. Вот так вот как-то, понимаете, после этого
действительно в Европе даже тот же Клёстка, дикари какие-то. Вот эти вот
московские, они дикари. Ну потому что избили. Обещала трансмутацию, не было.
Избили. По-моему, все закономерно, но они считают дикими выходками. Нехорошо так
себя вести. Надо было с почетом провести, значит, достойного человека с почетом
обратно отправить. Я шучу, вы поняли, да? Так вот, еще раз я подчеркну.
Просветители считали, что наука должна стать новой религией европейского
общества. Они называли ее естественной религией. Отсюда, кстати, постоянные
оговорки про храм науки, да, слышали такое? Ученые, первосвященник храма науки.
Они говорили это буквально. Вот мы это сегодня метафора, а тогда это было
буквально. Так вот, роль ученых, о которых мы дальше поговорим, роль
мировоззрения, механицизма, о котором мы дальше поговорим. Знаете, в чем
заключалось по преимуществу? В том, что наука стала новым типом мышления, а не
новой религии. Тогда как именно вот как бы дух просвещения, тесно связанного с
масонством и тому подобное, толкал ее на то, чтобы стать новой наукой и
продолжать все вот то, что продолжало там это вот магия, возрождение и тому
подобное. Понимаете, алхимия и тому подобное. Вот я сейчас перехожу ко второму
вопросу и на этом моменте я ставлю достаточно серьезный встрицательный знак.
Почувствуйте его. Потому что вот отменить эту линию развития науки, когда наука
просто новая религия. Смогли, а философы, настоящие философы, они вот этих,
опять же, я без всякого уничижения, они сделали свою работу, это была важная
работа, но они не философы. Вот кого я сейчас показываю вам. Все-таки. А
философы другие, по которым мы сейчас поговорим. Философы и исследователи.
Давайте сейчас перейдем к философии. У нас есть 15 минут еще. Вот как раз я
надеюсь, что хотя бы часть мы расскажем. Второй вопрос. Гносиологический поворот
в философии нового времени и роль формирования научной рациональности. То есть
сейчас мы будем с вами пытаться понимать, исследовать, как формировалась научная
рациональность, то есть наука в современном смысле этого слова. когда мы ее уже
не связываем с магией, с алхимией, с масонством, с какими-то тайными знаниями,
герметической матерью и тому подобное. Понятно, да? Почему гносиологический
поворот? Потому что акцент уже ставится не на антологии учений о бытии, а на
учений о познании. Гносиология. Но тем не менее, смотрите, общая для всех
философов и публицистов и не публицистов того периода все-таки категория
активности. Она ключевая на тот период вообще для всех людей европейских.
Почему? Потому что она ключевая для буржуа-предпринимателей активность. Для
научного метода, для философии. Силы и вещества они просто наблюдаются, они
активно измеряются, активно. Потребность описать, систематизировать все
природное богатство, вот эти вот категории учета, классификации, порядка. Они в
науке активно развиваются. Классифицируются и упорядочиваются все. Небесные
тела, гармонизуются представления астрономического характера, биологическая
систематика возникает. То есть кладовая природа активно активно человек стоит
рядом и записывает так, что у меня в распоряжении, у меня в распоряжении то-то и
то-то. То есть природа противопоставлена субъекту познания. Человек приобретает
свой статус вот знаете, такого вот распорядителя. Новый статус, новый, высокий
статус человека, он распоряжается своей кладовой. Природа это лишь то, что ему
принадлежит и чем он может распоряжаться. это такая вот форма активного начала.
Чувствуете, человек всегда был весьма значим, но всегда по-разному. Я даже
подчеркну это. Смотрите, античность, космоцентризм, мировоззрение. И там человек
это микрокосмос, отражение макрокосмоса. Макрокосмоса. Макрокосмос отражается в
микрокосмосе. То есть человек значим постольку, поскольку он тоже космос.
Теоцентризм. Да, весь взгляд, все внимание на Бога, но человек тоже крайне
важная фигура. Человек, ну я говорила, что раб Божий это никоим образом никогда
не было уничижительным каким-то, уничижительной характеристикой. Почему? Потому
что это формула римского права. Когда раба продавали, говорили, это раб, не
знаю, Цезарь, например, отныне такой это раб Цезаря. А когда христиане
крестились во имя начислены рабами Божьими, они как будто тем самым говорили, я
больше не раб никого. Больше я не раб ни Цезаря, ни Помпея, я больше не раб
никого. Все, теперь я раб только Бога. То есть это юридическая форма. А так
человек это животное призвано стать Богом. Помните, да, Зианцин выражение? В
эпоху возрождения человек он творец. Ну, творец, конечно, по преимуществу как
маг, но тем не менее гомо-виртуозу. То есть он подчиняется реальность. Какими
формами? Другой разговор, но он подчиняется реальность. Вообще, помните, да,
первый капитализм, поплыть в море мрака, вообще никуда вернуться с кораблями
набитым золотом. Вот так вот. Вот это вот. А в эпоху нового времени не
космоцентризм, не теоцентризм, не антроцентризм, а натурацентризм.
Натурацентризм, то есть человек обращает, ну, как бы природа становится ключевым
объектом исследования, а человек это тот, кто ей распоряжается. человек это тот,
в кладовой природы главное распорядить. Соответственно, категория субъекта
приобретает максимальную значимость. категория субъекта приобретает максимальную
значимость. Это подчеркните, потому что я не случайно тут подзависла. Но
субъектам надо было человеку еще стать. Почему? Потому что я сегодня с двумя
мониторами сижу, не распечатал, поэтому так временами зависают, простите. Ну,
вот, категория активность субъекта. И обратимся сразу же в декарту, которую,
собственно, и вводит нововведение, вводит нововведение, простите, вводит эту
категорию субъекта как того, кто обозначает именно человека. Наверняка вы уже
понимаете, что раньше, до эпохи нового времени, субъектом называли просто
подлежащее предложение. Субъектом может быть все что угодно. Например, стол это
субъект деревянности. так вот даже можно сказать. То есть это то, к чему
деревянность относится, как его характеристики. Или, например, субъектом может
быть космос. Подлежащее предложение такое стоит во всяком случае. А Декарт
сделал так, в смысле так передумал категорию субъекта, что субъектом становится
только человек. Он происходил из обнищавшего дворянского рода, недолго учился в
университете в Пуатье, праздновременную службу, участвовал в битве Хакина. И вот
на войне формируются его предпосылки механицистического взгляда на человека. Я
ввожу понятие механицизм, пока его не объясняю. Механицизм уже будет ключевое
слово, да, для этой эпохи. А почему возникло его понимание человека как
механизма, как такую, ну как бы куклу, которая, да, оживляет Бог, дух, там
подобное, но она как бы устроена как механическая кукла. О том, что слишком
много трупов видел, и более того, он эти трупы препарировал, и сам себя он
называл великолепным мастером препарирования. Он как бы очень, ну как сказать,
проникся этим делом, он хорошо, будучи, ну, мыслителем, хорошо проник в
специфику вот этого вот, всего дела, и развивает идеи о том, что человек-то, по
сути дела, машина, но только машина живая. И он с этими идеями, он пишет там
небольшую работу, обращается к своим учителям, к иезуитам, но он учился, потому
что доунин, все это в иезуитской школе. Но они, мало того, что не поддержали его
начинание, они его как бы сочли опасным, и дальше всю остальную жизнь, он
опасался их преследованием, переселился в Голландию, как вы понимаете,
протестантскую страну, где пишет рассуждение о методе, чтобы хорошо направлять
свой разум и отыскивать истину в науках. Мы его кратко называем рассуждением
метода. А в приложении к этой книге Декарт опубликовал свои исследования по
натур философии, по оптике, геометрии и по математике, по алгебре. Ну, то есть,
все, что мы сегодня связываем в науке с достижениями Декарта, например,
Декартовую систему координат и еще некоторые моменты геометрической
геометризации алгебры, мы, оптические исследования, для него это все-таки
приложение к философии. Его публикации, его идеи получают популярность, вызывают
восхищение кардинала Ришелье, то есть, никто там его, с закатольки-то его не
преследовали. Он предлагает издать труды Декарта во Франции, но Декарт уже
никому там не доверяет, он отправляется в Швецию по приглашению королевы
шведской, а в дороге он постудился, в итоге заболел и скончался, а многие
считают, что его отравили. Но не буду в это погружаться. Так вот, Декарт всю
жизнь искал некой свободы от ограничений, да, свободы для себя, для своей мысли.
И только свободный, говорил он, является самостоятельным, только когда он может
сказать «я сам». Я сам. Самость – это главное для философии Декарта. Я сам. Это
то, что говорит он, объединяет всех людей. У всех есть эта самость. Это и есть
основание свободы. Чувствуете, философы-то всё-таки идут по понятиям, да, не
лезут в понятие свободы, не просто борись за свободу. Борись, а зачем бороться-
то? Ты объясни, да? Просто найди себе уменьшательный, борись, ну извините, нет,
так философ не может мыслить. Это такая мысль недоброкачественная. Философия
всегда обращает внимание на доброкачественные мысли. Вот. Для Декарта основание
свободы – это самость. Самость можно обнаружить тогда, когда сомневаешься. Вот
это ключевая позиция. Ты вот эту самость, самостоятельность, самость можешь
обнаружить тогда, когда сомневаешься. Причем сомневаешься во всем. Смотрите, он
как говорит, можно сомневаться в том, что это помещение там, такой-то формы.
Можно вообще сомневаться, что здесь помещение. Можно сомневаться в своем уме.
Можно сомневаться, а вы существуете или нет вообще? Может, вы мне снитесь? Я
могу сейчас спать, и мне снится, что я просто веду лекцию. Я могу сойти с ума,
не будет казаться, что я преподаватель философии, а вы в это время просто эти не
аспиранты, а эти как санитары, которые наблюдают за моим поведением. Понимаете,
да? Можешь сомневаться во всем, но единственное, в чем я не могу сомневаться,
это в том, что я в этот момент сомневаюсь. Вот. Единственное, в чем не получится
сомниться, в факте самого сомнения. Поэтому это и есть основание самости. Все,
на что указывает мне естественный свет, естественный свет – это разум, да,
разум, вот так они тогда называли, естественный свет. Есть сверхъестественный
свет – это откровение, ну, Библия. А есть естественный свет. Еще раз. Все, на
что мне указывает естественный свет, никоим образом не может быть сомнительным,
поскольку из самого факта моего сомнения вытекает, что я существую. Когита
эргосум. Я существую, потому что сомневаюсь. Мысли, следовательно, существуют.
Ну, все вот это, то, что вы знаете, известное, оно основано на этом понимании
самости, как сомнения. А вот то, что уже никоим образом не, как сказать, то есть
вы во всем сомневаетесь, все, отбрасываете, отбрасываете, отбрасываете, выходит
на что-то, что несомненно. Принцип непосредственной достоверности. Он ставится,
Декарта, на первое место в познании. Это принцип интеллектуальной интуиции. А в
чем мы можем, ну, то есть в чем мы не можем по Декарту сомневаться, это в первую
очередь в математических истинах. Ну, то есть, например, то, что 2 плюс 2 равно
4, мы сомневаться не можем. То, что 2, это вот такое число, которое включает
именно вот, ну, это количество, и не может быть то 2, например, то 2 яблока, то
3 яблока, то 5 яблока. Для нас это настолько очевидная истина, что дальше у нее
уже никуда нельзя двигаться. Понятно? Так вот, для Декарта крайне важно
философское исследование вот этого интуитивно-очевидного, несомненного. И при
этом все-таки откуда мы берем-то вот эти несомненные истины? Вот эти вот идеи
разума, они так еще называются, обязательные, базовые идеи разума, на которых
потом выстраивается все остальное, все здание. Мы их берем только из того, что
Бог не обманывает свое создание в моментах абсолютной достоверности. Вот так
вот. То есть у Декарта все-таки гарантию человеческого познания обеспечивает
Бог. Вернее, Бог есть гарантия человеческого познания, причем не просто абы
какой, а именно добрый Бог. Ну, например, если Бог такой игривый, который любит
подшутить даже, я же не говорю, обманывать, даже подшутить, например, возьмем,
то все рассыпается, вся концепция Декарта рассыпается. Тогда Бог скажет, ну
смотри, интуитивно очевидно и какую-нибудь там ерунду на аккаун. Да? Поняли? Еще
раз. Мыслью следует, не существую. Интуитивно очевидно это несомненное, а
несомненное обеспечивается всемогущим, благим Богом. Это называется, вот это
несомненно, интеллектуальная интуиция. Это прямое непосредственное постижение
сути дела. Математическое знание. Что за вот эти вот врожденные идеи, которые
обеспечены нам Богом и которые есть интеллектуальная интуиция? Еще раз.
Например, идея числа, идея формы, ну фигуры и аксиома равенства. Вот это базовые
вещи, но там есть еще ряд. Ну их хотя бы назовите, будет нормально. Идея числа,
идея формы и аксиома равенства. Ну у меня тут достаточно большие цитаты, они
потом, по-моему, я их даю. Сейчас посмотрю. Да, вот, ну, как бы некоторые
цитаты, вы, пожалуйста, их просто сами потом самостоятельно прочитайте. Хорошо?
Извините. Так вот, благодаря свету разума человек может сделать природу своим
предметом. У него есть эти интуитивные идеи, которые ему позволяют, то есть это
инструменты, которые позволяют человеку и только человеку быть хозяином
вкладовой природы. Сделать природу своим предметом. А слово предмет означает то,
что напротив, воспринимаемое. То есть человек как бы отделился от природы, вот
есть человек, а есть природа. И вот природа, она несколько так отдельно. Природа
— объект, человек — субъект. И в гносеологии Декарт формирует направление
рационализма. Рационализма. То есть наше познание основано на идеях разума, вот
этих вот несомненных идеях разума. На том, в чем сомневаться вообще уже не
приходится. Дальше. Рационализм. Соответственно, смотрите, вот есть мыслящая
субстанция, это человек со своими врожденными идеями, со своим рационализмом, со
своим разумом, да, вот это все. И есть природа протяженная. Имеется в виду, это
то, что не разум, не человек, ну, то есть все остальное. Философия называется
дуализм субстанции или субстанциональный дуализм. Не погружаясь в эту тему, она
довольно интересная, она тоже несет много всяких последствий. Если что, на
семинарах, пожалуйста, проговорите этот момент. Проговорите и момент его метода.
Я его только обозначаю здесь, он достаточно простой, понятный на всяком случае
даже вот в письменном виде, потом я выложу презентацию. Вот. Несколько особняком
стоит его физика. То есть, да, конечно, физика, это физика механицистическая, а
мы про механицизм скажем чуть дальше, да, чуть позднее. но тем не менее, как бы,
она такая, как сказать-то, еще вчерашнего дня. Вот все-таки, как ни крути.
Вадикавта Вселенной это вихри в вихрях. Ну, догадайтесь, вихри где происходят?
Конечно же, в пневме, тонкой материи, в, как сказать, в эфире. Вот. Ну,
знакомая, да, история. Но, все-таки, Декарт рассматривает их механически, эти
вихри. Он не говорит о пневме, которая проникает там в душу и тому подобное.
Наоборот, он уделяет, вот есть душа, а есть материя. И путать их больше не
будем, нельзя. То есть, он противостоит, как бы, вот этому, вот этому магизму,
как может, разделяет душу и материю на две части. Но, опять же, у этого есть
последствия. Но, во всяком случае, понятно, почему он это делает. Это, как бы,
такая идея, но соединение от прежних умозрительных вот этих конструкций. Так
вот, вот этот эфир, он уже становится просто, как бы, субстанцией, в которой
заворачиваются разного рода вихри. Эти вихри и есть водоворот такого эфирного
моря, это и есть движение небесных тел. Они захватывают небесные тела. Поэтому
небесные тела двигаются. Ось вращения походит через Солнце, поэтому всё-таки всё
крутится вокруг Солнца. А каждая планета ещё крутится вокруг себя, потому что
вот эти вихри эфирного моря, её, как бы, всё, обеспечивают всё это движение.
спутники. Спутники двигаются благодаря меньшим вихриям, и они, то есть, окружают
тоже каждую планету. А все тела падают на Землю, потому что подталкиваются в
ходе вот этого движения вихревого мельчайшими невидимыми частицами вихрей,
флюидами. Ну, то есть, чувствуете, там какая-то смесь аристотельской физики и
какими-то вот такими новых идей. Вот, как бы, такая философия и физика Декарта.
Пожалуйста, детали проясните ещё на семинаре, потому что важная фигура. Мы
сейчас отдохнём с вами 5 минут, много не дам никогда. Ну, ладно, 8. Вот у меня
сейчас на часах 11.33, получается, в 11.40. 7. мы с вами встречаемся. Мы
остановились на том, что Декарт ставит саму проблему активность познающего
разума. Там кто-то включен. Ну, всё, спасибо. смотрите, активность познающего
разума. А, кстати, об активности познающего разума. об активности. Дорогие мои
аспиранты, пожалуйста, среди вас кто-нибудь вызоветесь и помните скриншот
посещение. Здравствуйте, да, хорошо сделаю. Спасибо большое. Спасибо. Ну, пока
мои аспиранты, что называется, мои дела, потом будут другие преподаватели еще
несколько, пару лекций читать. вот там как бы другие поработают. Я просто как бы
уж пользуюсь положением. Спасибо еще раз. Так вот, активность познающего разума
сильнее всего. Этот мотив акцентируется философией Фрэнсиса Бэкона. Давайте про
него немножечко. А, в эту самую сейчас включу презентацию. В смысле, она у меня,
есть, да, презентация там? Нормально все? Да, сейчас Декарт. Да, сейчас Декарт.
Все, давайте Бэкона. Лорд-канцлер Бэкон, то есть это не хухра-муха, знаете, и
это очень многое объясняет. Все-таки принцип полезности науки, он выдвигает на
первое место. Но он действительно заботится больше всего о государстве, о людях.
Кстати, человек-то действительно был такой, ну, может быть, мы много о нем не
знаем, но тем не менее, видимо, как бы такой вот, ну, как сказать, забота была
приоритетом его политики. Хотя, напрямую нам об этом ничего не говорит, но
человек умер, замораживая кур. Проводил опыты по замораживанию кур. Ну,
простудился и умер. А для чего он замораживал кур? Он для того, чтобы армия все-
таки питалась не тухлым мясом. Он говорила пользу. Ну, чувствуете, да,
определенная такая вот нотка практичности, она не, что-то, не с бухты-барахты
берется в его работа. И принцип полезности, да, что все действие наиболее
полезно, то есть знание наиболее истинное. И вот как раз принцип утилитаризма в
таком философском воплощении, ну, Баркон Дуз тоже, что называется, очень многое
сделал для этого. но как бы максимально полезно он предлагает работать в
качестве, ну, ученому, в качестве экспериментатора. Он говорит, что
действительно мы будем, то есть как можно быстрее и эффективнее выяснить тайны
природы можно с помощью пыток. Для него эксперименты это пыточная природа. Опять
же, ну, лорд-канцлер Бекон понимает, что говорит, да, и вот представьте себе
человек, ну, конечно, представлять не очень предлагать такое, но тем не менее,
представьте себе пытки. как можно быстрее выпытать у испытуемого какую-то
утаиваемую информацию. То есть нужно правильно ставить вопросы, да, и так, чтобы
ответ был только да или нет. Вы же понимаете, да, так меньше всего возможность
обмана там или еще чего-то. Поэтому вот он и называет природа есть, эксперимент
это испанский сапожок, надо, значит, заточить в него природу, чтобы она выдала
нам все свои тайны. Ну, такая разновидность пыточных. Но надо сказать при этом,
что вот это распространенное мнение, что именно он предлагает
экспериментировать, экспериментировать, еще раз экспериментировать, нет, это не
его позиция. Его позиция не экспериментализм, а эмпиризм. Я объясню сейчас,
погрузимся чуть полнее. Значит, он создает свой проект, он государственный
деятель, создает свой проект великого восстановления наук на основе союза опыта
и рассудка. Это произведение новой Атлантиды. Она издана посмертно. Да, там
очень много всего того, что вы можете встретить раньше других, более ранних
этапах развития науки. Там описываются идеальные государства, орден Соломонова
храма. Да, для него наука все-таки это немножечко религия. А может, и не
немножечко. Целью храма является овладение силами природы. И у него, конечно,
есть вот эти мотивы герметических установок. Маг, там, тот, кто подчиняется
материи. У него магия, кстати, входит в нормальное, ну, как бы, вот это вот, в
науке, нормальный набор наук. То есть, да, видите, я к чему сейчас сделаю
ремарку, почему мы не можем взять и сказать, вот, все, в 17 веке начинается
новый тип науки. А куда Бека наденем? Куда его новую Атлантиду денем? Там магия
еще выше крыши. Куда мы Декарта денемся его вихрями в эфире, эфирными вихрями и
так далее. Ну, нельзя такие границы провести. Это, как бы, слишком схематично
будет. Поэтому мы выстраиваем перед вами историческую картину науки, развития
науки вот так вот последовательно. А в реальности ничего нет вот такого, знаете,
ну, схематичного. Да? Поэтому обращаю еще раз внимание на вот такие вот как бы
двусмысленности в работах, которые, тем не менее, не мешают науке
последовательно осуществлять новый тип рациональности. Продолжаю. Про Бекона.
Так вот, в этом ордене храма, в ордене великого храма, Соломонова храма, есть
такие товарищи, как наблюдатели. Особая группа ученых. Они добывают по всему
свету тайное знание. Они коммерсанты света, так называемые. Вот. Но эти
коммерсанты света, им предлагается развивать новый метод. В противовес
умозрительной философии, говорит он, нужно искать логику изобретений. Нужно
открывать с помощью индукции. то есть умозаключение от частных единичных случаев
к общему, от отдельных фактов к обобщениям. Вы понимаете, что это на самом деле
очень по-новому трактует разного рода даже прежние знания. Попробуйте собрать
индуктивно теорию пневмы. Попробуйте собрать индуктивно теорию пневмы. не
соберете. Но нет таких частных фактов, на основе которых вы выстроите всю вот
эту концепцию. То есть по сути дела вот этим подходом индуктивным он разрушает
умозрительность прежних концепций. Еще раз подчеркну этот важный момент. Когда
он предлагает начать собирать факты природы, их анализировать, выстраивать
таблички, в которых писать их сходства и различия, все стойки, вы становитесь
эмпириком. Тогда как тот же, например, Аристотель или Фичинос Бруно или даже
Схоласты, которые, конечно, наук природы не занимались. Я сейчас говорю про те,
кто природой занимался. Ну, в смысле, не занимались, но меньше степени. Вот. Они
не начинали от элементов. Хотя, конечно, у Аристотеля в его биологии очень много
эмпирических факторов. Но он все-таки, его физика, она не эмпирическая.
Умозрительные, да? Мы создаем умозрительные концепции, которые как бы все
объясняют. Тогда как Бекон предлагает индуктивный метод, а это означает, обращая
внимание на частные, конкретные моменты. Мы сравниваем, говорит он, данные о
схожих предметах, предлагает составлять таблицы. Мы находим общие черты в них, в
результате получаем обобщение, знание обобщенного характера. Конечно, Бекон
знает, что у метода индукции есть пределы применимости. Ну, условно говоря, мы
рассмотрели первого лебедя, второго лебедя, третьего, двадцать первого и так
далее. Мы сказали, что все лебеди белые. И понятно, что у этого индуктивного
обобщения есть уязвимое место ровно до тех пор, пока мы не встретим черного
лебедя. Эта теория будет работой. Но тем не менее, метод индукции все-таки
приводит к общим выводам, которые могут быть равновероятными. Ну, например,
лебеди как белые, так и черные. Ну, сейчас пример не очень подходящий, но тем не
менее, равновероятные выводы. Так вот, чтобы проверить равновероятные выборы, он
предлагает ставить эксперимент. Понятно, да? Когда у него экспериментализм-то
включается? Чтобы проверить равновероятные выборы. Так вот, это позволяет по
Бекону создать знания закона образного характера. И он таки установил некоторые
законнообразные знания. мы не назовем это закон но тем не менее. Сравниваем
многие ситуации. Бекон установил, что теплота связана с движением частиц.
Понимаете? это вот как раз такой эмпирическая закономерность. Он просто наблюдал
некоторые моменты и выводит эмпирическую закономерность. Социологические опросы,
например, они тоже, они же не приводят нас к законам природы социальной, законам
социальной природы. Они говорят нам об эмпирических закономерностях в социуме.
Ну вот, таким образом. Бекон не особенно участвует в споре о том, как же все-
таки мы познаем, благодаря чему? Благодаря опыту или знанию. То есть вот в этом
кнасологическом споре он не участвует, но тем не менее на его работах
основываются многие исследователи сенсуалисты. Давай, вот Бекона значит это то,
что я рассказывала в более-менее так сжатом письменном виде. Вот теперь давайте
обратимся к сенсуализму. То есть ставится вопрос как человек познает? Что
первично? Опыт или интеллект? Разум. Сенсуализм чувствуете от слова чувство
говорит о том, что все-таки первичны именно данные чувств слуха, обоняние и тому
подобное. Не важно. То есть основание надо искать в чувственном опыте. Сознание,
говорит Джон Лок, это чистая доска, на которой опыт пишет свои письмена. Ну, то
есть чистая доска имеется в виду эти восковые вот эти как бы вовращенные
таблички, на которых сознание оттиски оставляет. Да, дорогие мои, это все еще
отголоски теории пневмы. Помните, там пневма как бы в оптический центр попадает,
глаз, да, по оптическому нерву, и оттиски оставляет на внутренней пневме эти
оттиски фантазмы, которые вот и есть знания. Вот у него примерно что-то похожее,
только он использует это немножечко уже более метафорически, как бы саму пневму
уже не упоминает, но тем не менее. И возникает спор с Декартом. Они, кстати, ну,
как бы спорят, ну, как сказать, спор заочный, да, когда публикуешь, когда ты
публикуешь свой рот, и говоришь, а вот Декарт не прав, а он говорит, а вот там.
То есть Лог согласен с Декартом, что самость, субъектность, самостоятельность
это общее для всех людей, когда они в сознании, подчеркивают. Когда они в
сознании, у каждого есть самость, самостоятельность, субъектность. Но все эти
субъекты живут в разных условиях, и будучи одинаковыми, как бы, ну, нулевыми от
рождения, они наполнены разными влияниями, разными оттисками, да, и,
соответственно, они становятся разными людьми. По-разному видят мир, по-разному
познают. И эту, как бы, разность надо ценить, потому что это и есть влияние
общества. Отсюда все идеи о том, что общество выстраивает личность. Чувствуете,
как много закладывалось эпоху нового времени? Рационализм Декарта говорит о том,
что есть некие врожденные идеи, которые позволяют нам самость иметь с самого
начала. Это, как бы, категория личности, которая с вами, ее никто не
выстраивает, она своя. А другая позиция, сенсуализм, говорит, нет, от общества
зависит, что там будет с вашей личностью. Потому что разум чистая доска, нет
никаких врожденных идей. Понимаете? То есть, это вроде бы спор о том, как
человек познает, но в итоге этот спор формирует множество вот серьезных как бы
разветвлений в других областях знания, в других областях мировоззрения. И
постепенно все большую значимость приобретает принцип различия. Вы наверняка
слышали, что новое время связано с культом индивидуализма. То есть, вот этот
индивидуализм. Но это и есть философское обоснование свободы. Нужно иметь
самость как некую самостоятельность. И различие здесь оказывается ключевым
моментом. То есть, философ акцентирует не то, в чем мы похожи, а то, в чем мы
различаемся. Но, конечно, под это должна быть подведена онкологическая база. и
потрясающую базу, очень интересную, но я о ней скажу только кратенько, дает
Лебниц. Это мыслитель, которого я, знаете, про него лучше вообще не говорить,
чем говорить мало. Великолепный, конечно, представитель философии нового
времени. И вообще, он универсальнейший мыслитель нового времени. Он намного
опередил свое время во многих областях знания, поэтому он не всеми-то и признан.
По сферам деятельности он юрист, дипломат, тайный советник юстиции в России,
кстати, наемный алхимик в области общества разнокрейцеров, придворный историк и
так далее. Чем только он не позанимался. Список областей в науке, которые мы
получили в развитии благодаря Лебницу, вообще неохватный. Но я подчеркну одну
важную идею, связанную с категориями тождества и различия. А именно ноль и
единица. Метафизические аллюзии чисел нуля и единицы. Надо сказать, что он
самостоятельно развивал теорию универсального исчисления, согласно которой можно
было бы построить машину, которая могла бы мыслить. Ну как мыслить? Она могла бы
помогать людям выбирать правильные идеи. Ну то есть, например, у нас с вами
спор, мы подошли к этой машине, чик-чик-чик, она нам высчитала, кто прав. И мы
все согласились. Ну типа, ну это же машина, нам не надо дальше спорить. То есть
чувствуете, наше различие важно, но надо все-таки находить форму единства, и он
предлагает создать такую машину. Понимаете, да? Про образ чего это было. Но она
у него не получилась, она, как бы, осталась незавершенным проектом, но в ходе
работы над ней он выходит на принцип нуля и единицы, то есть двоичного
исчисления. Он за него очень радует, говорит, что это очень удобное вычисление,
предлагает его всем домам, там пишет письма, в смысле королевским домам, пишет
письма ученым, говорит, давайте пользоваться нулемой единицей. Но он не хочет
применить ее для этой машины для мысли, то есть пока вот этого соединения не
произошло, ноль единицы и вот эта машина для мысли. Но это не арифмометр, это
как бы проект уже именно думающей машины. но тем не менее Лебниц как-то на этом
нуле и единицы привлекает его очень серьезное внимание. Почему? Потому что в
этот период, тесно общаясь с азуитами, он обнаруживает в их письмах, там разного
рода публикациях, он обнаруживает учение и дзин китайское. Но они же в этот
период активно ездят по миру, активно там участвуют в миссионерской
деятельности, в образовательной деятельности. И вот доносят до Европы разного
рода сведения о других культурах. И, кстати, Лебниц очень ценил китайскую
культуру, считал, что она самая продвинутая. И считал, что Россия, почему он так
интересовался Россией, участвовал активно в ее жизни, научной жизни, он считал,
что Россия это будет главный мостик между вот этой великой китайской культурой и
европейской. Он как бы предлагал такие политические проекты и российским
правителям, и китайским, и так далее. Вот такая у него была идея интересная. И
плюс ко всему вот это вот ноль и единица получает у него метафизическое
значение, потому что Идзин, это книга перемен китайская, она в определенном
смысле, ну, тоже может быть интерпретирована в этой логике двузначной, да,
почему? Потому что, ну, если вы когда-нибудь посмотрели, что это такое, там
такое гадание, которое, ну, типа, король, в смысле, как сказать, орел-решка, ну,
теснули единица, или, то есть там вот эти парные две, как бы, два элемента, как
бы, противоположные, они играют ключевую роль. И он им предлагает дальше
пользоваться этими элементами, тем более, что это очень хорошо сочетается с его
метафизикой. Он, кстати, мечтал, что вот ему медаль дадут, когда он уйдет, и на
ней будет надпись «Все может быть выведено из ничего, все, что нужно, это
единица». Лебедь пришел к выводу, что лучший из возможных миров, наверное, вы
слышали, да, идеи лучшего из возможных миров, то есть, ну, миров может быть
очень много, Бог может состоит любой мир, может быть состоит миллиарды миров,
для него это, что называется, не проблема. Вот. Но мы живем в лучшем из
возможных. Это мир, в котором огромное разнообразие явлений следует из
минимально возможного числа предпосылок. А из нуля единицы можно вывести
бесконечное количество других чисел, умозаключений и так далее. Поэтому его так
порадовала китайская книга перемен, которая, по сути дела, любое состояние
опишет вот из этих двух элементов, парный пример. В метафизическом смысле у
лебницы нуль это и есть тождество. Ну, смотрите, все сливается и превращается в
ноль. Это полная конкурентность. Мы, условно говоря, тождественные, и это
делает, ну, как сказать, если мы абсолютно тождественные, то мы неразличимы,
правильно? То есть мы есть нуль. Но Бог, говорит лебец, когда взывает к бытию,
когда говорит кому-то, кому-то, будь, то он осуществляет различие. То есть
чувствуете, он ноль как бы перестает быть нулём, появляется единица. Всё, что
существует, говорит лебец, существует лишь благодаря различию, неконкурентности.
Поэтому все люди видят мир как бы из центра мироздания. Ну, наверняка, вы это и
по себе замечали. Складывается устойчивое ощущение в ходе жизни, что, ну, я в
центре мира. Не в смысле, я там главный или какая-то ещё, ну, просто я всё вижу
вот из своей позиции, а все остальные это как бы периферия, да? Ну, как бы это
вот, они наверняка, ну, ну, как сказать-то, ну, понятно же, что я наблюдаю мир,
и вот я как бы из него, я в центре этого мироздания. И только, ну, будучи уже
взрослыми, мы понимаем, что ведь точно так же и другой человек воспринимает, он
центр мироздания. Ну, условно говоря, давайте современную лексику применю. Я как
бы реальный игрок, да? А остальные кто, получается, эти самые, да? Списываю
слово. Подскажите мне, не игровые персонажи, как? NPC. NPC, да. Все остальные
NPC. Я как бы, ну, я-то реальный, да, остальные NPC. Вот примерно так же, ну, я
огрубляю очень, преувеличиваю. То есть, каждый так думает. И Ленин в этот момент
очень уловил и говорит, что душа Исмонада, такая вот единственность, которая
наблюдает мир, отражает его. и вся совокупность знаний, это, по сути дела,
синтез знаний разных монад. То есть, все действительно в центре, действительно
все в центре мира. То есть, вот столько, сколько нас, душ, столько и монад,
столько, простите, и центров мира. И получается, что мы все смотрим мир, как бы,
из своей позиции, да? Из своей позиции. Нам видно, несколько иначе мир, чем вот
другим. И в этом наша ценность говорит о. Монада есть перцепцию, аппетитус и
виз, то есть, восприятие, стремление и сила. Вот это и есть устройство
мироздания. А чтобы сочетать все эти взгляды из разного, это означает найти
возможность из миров, возможность этого сочетания. И вот наш мир, это лучший из
таких вариантов. Понимаете? Бытие означает различие, но ведь это означает основу
для противоречия. Мы же по-разному видим мир. Значит, мы в чем-то будем
противоречить друг другу. Значит, в чем-то будет страдание, потому что
противоречие могут дорастать до страданий, до зла. И Ленинс говорит о том, что
надо мужественно принять это обстоятельство, ну то есть то, что в мире есть зло,
потому что зло – это плата за различие. Это плата за вот это неизбежное смещение
смысла, которое возникает благодаря тому, что мы каждый в своем центре, мы
каждый не NPC. А Бог – это трансцендентный субъект, его творение, поэтому
наилучший из миров. Не потому, что в нем все довольны, ничуть не бывало, не
потому, что в нем нет зла, он говорит, это нелепая фантазия. А потому, что все
остальные формы единства различного, единства различающегося, невозможны. Ну,
или они как бы, они возможны для Бога, но невозможны для того, чтобы это было
наше, нашим существованием. Бог же сам есть бесконечное различие, бесконечное
разнообразие, бесконечное богатство. Поэтому никогда не было изначального
тождества, никогда не было изначального нуля, а бытие и сразу же взрыв
разнообразия. Ну, по-моему, очень красивая философия, я не говорю, что там одна
единственная история, но она красива, как мысль. Вот, но скажем честно, все-таки
тому времени такие мысли были еще пока не созвучны, он как бы вперед смотрел.
Для научного познания все-таки 17-18 веков, куда было важнее найти сходство, то
есть тождество явлений в качестве законов природы. Но, смотрите, все равно, я
хочу подчеркнуть, философское мышление уходило в этот момент в разрыв с научным.
Как бы наука, философия разделялись, эмансипировались друг от друга. Вы же
понимаете, что это сыграло науке огромную пользу. Если бы науке до сих пор
занимались философы, ну, мы не знали бы науку в той форме, которую мы знаем
сейчас. Хорошо или плохо, мы оценок не ставим, но тем не менее, определенно
нужно было, чтобы когда-то наука разошлась с философией. И вот в таких вот
идеях, да, в идеях там Леменца, несмотря на то, что они много сделали для науки,
но все-таки философия эмансипировалась от естественно-научных исследований. А
сама философия тем самым продолжает исследовать разум, анализирует, как отражает
разум реальность. Вот. И постепенно выходит к мысли, которые сегодняшние ученые
в тот дом разделяют и даже понимают. Речь идет о том, что реальность все-таки
разумом конструируется, а не просто отражается. Да, не просто пневма, там, глаза
попадают или еще что-то там, какой-то образ. А все-таки мы конструируем
реальность. Ну, так, я вот вам, наверное, не Гоббса все-таки дам, а Гоббс, он,
конечно, крайне важный представитель. Но, тем не менее, ага, дорогие мои, вот
это я прямо сейчас исправлю, вот это я уберу, потому что эта фраза, я только
сейчас поняла, что она принадлежит не Гоббс, она принадлежит Спинозе. Вот так
вот. Не знаю, как это ошибка попала, ну, в смысле, ошибаться я могу, так что все
нормально, главное вовремя заметить. Почему я с ним, что значит разум
конструирует свою реальность? Спиноза тоже самое говорил, что в качестве
достоверного знания мы можем получить только то, что сами создаем. ну, например,
как нам определить круг? Круг. Круг, крытон, это фигура, которую мы можем
очертить, держа веревку, один конец, который закреплен. Вот так, а другой
подвижен. Чувствуете? Вот это круг. Юм, как известно, идет дальше. Коббс и Юм
примерно одинаковые, в смысле мыслит, поэтому Коббс вот здесь оказался. Он
утверждает, что разум сам конструирует причинность. мы не можем утверждать,
говорит философ, что причины и следствия это свойства самой реальности.
Реальность может быть устроена как угодно. То есть сегодня, например, после
дождя земля мокрая, завтра, после дождя земля мокрая, и миллионный день
происходит все точно так же, земля после дождя мокрая. И мы говорим, что дождь
здесь причина намокания земли. Но на какой-нибудь день может случиться по-
другому. Но только Юма это не всемогущество Богу отсылает, а просто к
определенным, ну как типа, давайте, говорит, будем здраво смотреть на свою
позицию. Мы не знаем реальность во всей ее полноте. Мы же не будем жить, ни один
из нас не будет жить там 50 миллионов лет. Скажем, не один из нас, ни один из
нас миллиона жить не будет. Но имеется в виду, что мы же не узнаем то, что
будет, вот, ни один из нас не узнаем, то, что будет там какую-нибудь такую эпоху
или было. Поэтому давайте просто примем, что причинность это наша привычка
разума. Мы ищем вот эти привычки, ориентируемся на них и тем самым конструируем
знания. Во многом на основе его представлений формируется позиция Канта. Кант
очень сложный мыслитель, очень такой, как бы это сказать, красивый, красивый,
да. Почему? Потому что там прям все соединено, все в одну красивую конструкцию,
все как бы, знаете, красивая архитектура, она выстроена, идеально, не
подкопаешься, стоять будет вечно. И все-таки позиция Канта стоит вечно. Как его
не пытались разобучить, как его не пытались оспорить, патиковать, Кант все-таки
стоит столь же твердо его здание, его философию, как и в самом начале своего
появления. Так вот, Кант радикализовал идею субъекта в контексте свободы.
Субъект, говорит он, это тот, кто не предумышленный, не специально, но все-таки
сам формирует, сам конструирует свой предмет. Иными словами, действительно,
только постольку, поскольку разум сам конструирует образ реальности, человек
свободен. Потому что, если не сам, то тогда на него влияют либо там какие-то, в
общем, какие-то воздействия он испытывает, и, соответственно, ну какая же
свобода? Где же свобода личности, если он не сам, не самостоятель? Вот
интересный момент, То есть чувствуете, это все-таки заслоны все от принципа
магии. Если мы возьмем позицию Канта, то никакая магия вообще невозможна.
Вообще, принцип. Как бы его не полюбили, товарищи, возрожденцы. Смотрите дальше,
что он рассуждает. Почему так важен принцип свободы? Наука, а он сам был ученым,
у него был, так называемый, докритический период, когда он был астроном, ну то
есть создавал физические, астрономические концепции. И они до сих пор некоторые
работают. Поинтересуйтесь этим моментом специально, я не буду покружаться, меня
он интересует как философ в данном случае. Вот. И Кант, будучи ученым, прекрасно
знал, что наука существенно продвинулась в понимании таких вот механицистических
процессов. Механика же это представление о линейной причинности, ну условно
говоря, А толкает Б, Б толкает С, С толкает Д, все подчинено вот этой вот
линейной причинности. Причем одна и та же причина всегда порождает одно и то же
следствие. Наука того периода, ну Ньютонская, чуть позднее скажем об этом, она
ведь как сказала, если у частицы та же самая масса, она толкает, не толкает, она
движется с тем же самым импульсом, то это в принципе даже неразличимые частицы,
да ведь? То есть они, все, масса импульсов падает, мы даже не будем говорить там
о том, что, ну там, какие-то цветы, например, различаются у частицы. Вот.
Действия и противодействия, вот что мы должны изучать. А Канн задает вопрос, а
как быть тогда с событиями морального характера? Если мы говорим о выборе и
ответственности, о свободе, да, потому что свободен только тот человек, кто на
самом деле может позволить себе, ну, например, даже пожертвовать собой. Это его
выбор. А если он этого не может делать, почему? Ну, например, потому что боится
смерти, значит, он не свободен. Или, например, он дал милосты на него, потому
что, например, у него такая жалостливая натура. Ну вот он очень эмоционален, но
он плачет от любого, он увидит что-нибудь в несчастье, как он человек, плачет
он, да, вот такое. И он просто не может быть. У него последняя копеечка, вот она
последняя, но он настолько жалостлив, настолько у него вот это вот чувство, как
сказать, сострадания развито, что он это последнюю копеечку отдаст. Так вот, по
канту, это не свободный поступок. Это поступок под действием причин и следствия.
Вот эта вот жалостливость стала причиной его действия. Значит, это не свободный
поступок. Его можно оценивать как угодно хорошо, но он плох тем, что он не
свободен. Тогда как по-настоящему свободный поступок, это когда я поступаю,
например, ну, даю милость мне, не из жалости, да, ну, то есть я как бы понимаю,
что да, действительно, ну, то есть как сказать, меня не толкает собственная
жалостливость. Это мой выбор свободный. Я могу дать, могу не дать. И понимаете,
да, я могу, там, ни под каким дурманом я не нахожусь и жертвую собой не потому,
что у меня какой-то там, не знаю, какое-то воздействие на меня оказали либо
идеологические, либо там, не знаю, там, магические, как угодно. А я жертвую свою
жизнь, например, будучи военным, да, потому что я действительно выбираю защитить
вот тех вот людей, которые за моими спинами. Это мой свободный выбор. Вот
конкретно очень. И поэтому принцип свободы и принцип вот этого вот
самостоятельности решения для Канта очень важен. Но как быть, наука говорит, все
подчиняется причинам и следствиям. Все. Как быть со свободой? Ведь свобода это
риск принять то иное решение. И разум должен быть сам себе законодателем. Это и
есть совершеннолетие мюндишкает разума, согласно Канту. Причем свободны все.
Никто не может восприниматься как средство для достижения цели, ведь каждый сам
реализует свой выбор. Это вот этика Канта основана на категорическом императиве.
Категорический императив. Давайте я оставлю эту тему у нас вопрос не по этике,
но вы, пожалуйста, запомните, категорический императив это крайне важно. Я иду
дальше. Наука не видит возможности для свободы. Казалось бы, ну, давайте
откажемся от науки, чтобы сохранить концептуальную возможность свободы. но Кант,
в чём его действительно открытие, он идёт другим путём, критическим, то есть
путём внимания к нашему разуму. Он соглашается с тем, что причина это наша
способность видеть реальность так, а не иначе. Реальность, как она нам является,
и он вводит понятие феномен. Реальность, как она нам является, феномен. И вот
феномен это то, что, ну, мир для нас. Но мы должны допустить, что реальность
сама по себе, для себя, в себе, ноумен, и она может быть совершенно иной. Ну да,
то есть мы видим, что всё подчиняется причинам и следствиям, всё разворачивается
во времени, а время и есть категория причин и следствия, если так думаться, что
вот, ну, вот реальность такая, какой мы и видим, она, это феномен, это просто
то, как нам является, а реальность сама по себе может быть, например, вне
категории времени, то есть всё есть сразу, всегда и везде. Она может не
подчиняться, тогда она и не будет подчиняться категории причин и следствия.
Вспоминайте этот самый Нолоновский «Интерстелл», да, помните, как там он
изобразил, ну, они там изобразили, вся команда изобразили, вот это состояние
пятого измерения, когда всё сразу есть, да, там же, чувствуете, категория причин
и следствия, всё меняется местами, всё становится, ну, как бы уже не
выстраиваться в линеечку, одно следует за другие и так далее. Невозможно, что
времени-то нет, что будет раньше, что дальше. Ведь причина – это то, что раньше
следствия. А если категория раньше, позже не существует, вот, мы должны
допустить, что мир, реальность сама по себе не та, которую мы воспринимаем.
Познать Ноломен нельзя, говорит Кант. Вот такой запрет. Вещь в себе
непознаваема. Ноломен – это вещь в себе. Вещь в себе непознаваема. Это допущение
Ноломен, которое позволяет не видеть реальность глазами только науки. Понимаете?
Сейчас я посмотрю у нас в данном случае этот статус твоего дела. Да. Есть тут
цитаты кантовские. Достаточно много. Понятие Ноломена взято чисто в
проблематическом значении, остается не только допустимым, но и необходимым. Наш
рассудок приобретает… Ну, в этот случай его читать сложно, я даже не буду
зачитывать. Потом почитайте, ладно? Сами попробуйте разобраться. Если хотите
понять Канта, ну вот, и не вчитываться прям, это сложно, я понимаю, сейчас не до
этого. Хотя вы вступление прочитайте. Вот вступление, там очень много лица.
Критики чистого разума. Вступление. Предисловие критики чистого разума. Много
даст вам. Уже. Даже это уже будет много. Так вот, смотрите, дальше. А как же
разум формирует эту картину реальности? Разум формирует картину реальности с
помощью тех инструментов, которые в него встроены. Понимаете? Я слово
инструменты беру в кавычки. Инструменты, которые встроены в каждый наш разум.
Ваш, мой и все такое. Это и есть то, что Канта называет трансцентральный
субъект. Не буду сейчас покружаться. А что это за инструменты? Это априорные
способности. Априорные. Априорные, то есть доопытные. Доопытные. Нам не нужно
какой-то опыт получить, чтобы эти инструменты приобрести. Что это за
способности? Первое, пространство и время. Вот пространство и время, это не то,
как мир существует сам по себе. Это то, как разум ощупывает действительность.
Это тот инструмент, которым разум может эту действительность нам дать для
осмысления, для понимания. То есть определенная конструкция возникает, вещь для
нас. Канта называет это априорные формы чувственности. Я понимаю, что сейчас я
вам это все не объясню и не сделаю так, чтобы все стало понятно. Я просто
набрасываю пока то, с чем вы будете разбираться. Конечно, спросите еще на
семинарах. Будет там какой-то другая монада вам объяснит, которая видит мир с
другой стороны, да, и расскажет вам, как это. По-другому расскажет. Собидишься,
срезонируйте. Кто-то срезонирует с моими объяснениями, кто-то и так далее.
Смотрите, вот еще раз. Пространство и время это априорные формы чувственности.
То, с помощью, те инструменты, с помощью которых мы формируем вот такой, а не
другой, такой, а не другой вот образ действительности. Помимо форм
чувственности, большую роль играет и способности рассудка, то есть категории.
Категория причинности, необходимости, количество, качество отношений и
модельности. Это априорные формы рассудка. То есть смотрите, когда мы пытаемся о
чем-то судить, разобраться в чем-то, мы хотим, мы невольно ищем, во-первых, а
что было причиной, а каковы следствия. Канн говорит, кто вам сказал, что в
реальности это есть. Тем более, что в реальности у одного события никаким
образом ни одна причинность, даже если так разобраться. Совокупность немыслимого
количества обстоятельств может быть интерпретироваться как причина. Но мы же, мы
так устроены, что мы найдем какую-то одну причину, выделим ее в качестве
главной, назовем это причиной. Ну, вот это как бы такое грубое объяснение.
Априорные формы чувственности и рассудка принадлежат субъекту. Они не
субъективны, в том смысле, что у одних один образ реальности, у другого другой,
у третьего. Они субъектны. Это не индивидуальные особенности восприятия, это то,
в чем мы единообразны. Так нас просто разум устроит, человеческий разум. Канн
говорит о человеке как таковом. Так вот, наука выходит только на мир явлений,
где можно говорить о времени, о форме пространства, о причинах и следствиях.
Анализирует мир тел и отношений. То есть она погружается в явления глубоко, но
все равно будет упираться в самое дно. А это дно есть наш разум. То есть те
конструкции, которые в него встроены, те инструменты, которые в него встроены.
Даже само понятие природы возможно только потому, что так, а не иначе устроен
наш разум. Таким образом, Кант на самом деле додумал до конца концепт
декартовского субъекта, самости, категорию свободы. Субъект это не тот, кто
выбирает, кто является, выступает лишь пассивным началом. Ну, то есть просто
воспринимает. Кант это не субъект, правильно говорит Кант. Вот только если ты
конструируешь реальность в некотором смысле, набрасываешь на него сеть, свою
сеть понимания, то это означает, что ты являешься свободным познающим разумом.
Вообще свободный. Да? Принцип свободного субъекта предполагает, что ты сам
конструируешь свою реальность. Но при этом чувствуете как до этого понять.
Хочешь быть свободным субъектом? Будь. Но только тогда прими следствие этого,
прими неизбежное концептуальное следствие, а именно мир сам по себе, реальность
сама по себе от тебя сокрыта, она непознаваема. Реальность сама по себе в этом
случае, если ты самостоятельный свободный субъект, для тебя непознаваема. Ну,
вот примерно так. Значит, на этом я с философией, что называется, заканчиваю, но
не только потому, что, называется, я все сказала, да нет, ни поим образом,
просто, ну, там, либо говорить дальше о Канте, либо не говорить вообще. Ну, и
плюс ко всему у нас с вами времени, не так уж и много, прямо скажем. Нам надо
разобраться с третьим вопросом, он очень сложный и важный, формирование
экспериментально-математического естествознания в 17-18 веках. Основные черты
научной классической картины мира. Почему это важный вопрос? Потому что вот все,
что мы связываем с современной наукой, это вот как раз вопрос о формировании
экспериментально-математического естествознания. значит, что новый этап, новое
время, это эпоха, когда закончился, я повторюсь, натурфилософский этап развития
науки, натурфилософский этап развития науки связан с методом умозрительности,
метафизические концепции, временами, которые даже уходят в какую-нибудь там
герметизм. А вот метод классической науки, это новый тип рациональности, новый
тип разумности, вот так из этой перевести. Как это осуществляется? Ну, начнем,
конечно, с Кеплера. Это момент такого переходного периода между наукой прежнего
типа, наукой нового времени. Почему так? Ну, потому что в нем мы можем встретить
и то, и другое еще, там и магия есть, и то, и другое. Но, тем не менее, Кеплер
сказал, вот слова его, которые я тут выношу в качестве цитаты. Цитата из его
работы «Новая астрономия», трактат «Новая астрономия». «Моя цель в том, чтобы
показать, что небесная машина должна быть похожа не на божественный организм, а
скорее на часовой механизм». Я показываю, каким образом физическая концепция
должна быть представлена посредством вычисления и геометрии. То есть, он
усиливает то, что сделал Коперник, взял и математически проанализировал, не
герметически, а математически проанализировал устройство Вселенной, предложив
гелиоцентризм. То есть, я еще раз подчеркиваю, не сама идея гелиоцентризма была
прорывной, она была важной, она была значимой, но она сама по себе не двигает
науку сразу же на новые рельсы. По большому счету, какая разница, вокруг чего
крутится, так, если разобраться, для науки. Ну, сейчас сказала крамольную вещь,
конечно, но тем не менее. Ну, само по себе признание того, что мы вокруг Солнца
вращаемся, науку вперед не двинет. Но если ты это сделал, то есть, не только то,
что ты сказал, но как ты сказал, ты обосновал это не герметически, а обосновал
это математически. Вот эту линию продолжает Кеплер. И при этом продолжает ее в
духе механицизма. Запоминает это слово снова и снова. Механицизм это когда мир
похож на механизм. Он даже, видите, и говорит, что это небесная машина. Само по
себе машина мундий, вот это выражение, это античное выражение. Его придумал еще
Овидий, по-моему. То есть, его используют в Средневековье, называя мир машиной.
Метафорически. Метафорически. А вот новая наука начнет это говорить почти
буквально, что, ну почти буквально, совсем буквально, что мир, скажем так,
гораздо больше похож на машину, чем на все остальное. Но тем не менее, конечно,
для самого Кеплера большое значение имело со всеми последствиями теория пневмы,
то есть, теория эфира. Тело Солнца является источником силы, приводящей в
обращение все планеты. Почему? Потому что оно выпускает из себя некую, вот это
вот, некую, как сказать, эфир, испускает эфир, который вот формурует бурный
водоворот, водоворот, он охватывает всю Вселенную, ну немножко похоже на
декартовскую концепцию, это не странно. Дальше, все тела у него обладают душой,
небесные тела. Они реагируют на определенные, то есть, души уже не в смысле,
там, боги какие-то, не в смысле, души, вот какие-то там, магические принципы, а
души в том смысле, что они слышат музыку, вот, и, соответственно, они реагируют
на определенные гармонические пропорции. Душа является системой резонаторов по
его концепции, а вот эти резонаторы, как всегда, любую музыку можно писать
математически, чувствуете, да? То есть, вот такая вот смесь еще, еще как бы
душа, это вот еще души, но души уже описываемые математически в переходный
период. А вот следующий наш герой, Галилеев Галилеев, вот он, пожалуй, первый,
кто сказал, так хорош уже, все, давайте вот это вот все, все вот эти
герметические атовизмы все отбросим. Он был сыном известного музыканта,
теоретика музыки, причем, это семейство Галилеев было крайне известно в Европе,
почему? Потому что они активно топили, извините за аутболизм, топили за музыку
нового типа, за тональную музыку. Очень интересно было бы покрузиться в связь
тональной музыки и нового мировоззрения, но не буду погружаться. Статью я, по-
моему, про это написала, но сейчас погружаться не буду. Так вот, один из сыновей
этого большого семейства Галилео Галилей. И вот он как раз, человек такой,
трезвый, мыслищий, несклонный к излишней метафизичности, вот, он искал простые
математические формулы, которые соответствуют наблюдаемым эффектам.
принципиально нарезная его подхода заключалась в том, что следствие этих форм
должны были проверяться специальным экспериментом. Чувствуете, конкретный опыт,
берешь конкретное положение и проверяешь каким-то экспериментом. То есть не
математика не в целом рулит вселенная, а математика и имеется в виду какие-то
конструкции математические. А он берет, вот на меня катится шарик по доске, он
выстраивает для этого математическое объяснение, какую-то формулу, например,
зависимость высоты доски от ускорения, в смысле зависимость ускорения от высоты
доски, да, массы, понимает, что я имею в виду, формулу какую-то математическую,
да, а затем проверяет ее экспериментом, кидает там с пизанской башни шары,
перышки и тому подобное. Вот. То есть, смотрите, он сознательно выбирает новый
метод и противопоставляет методу умозрения, то есть, аристотливскому методу,
противопоставляет метод Архимеда. Он, кстати, об этом имеется в виду. Галилея
пишет об этом, что надо брать за основу метод Архимеда, потому что Архимед, хотя
и выстроил свое положение в виде аксиом, ну, в стиле Эфлида, но справедливость
этих аксиом, он обосновывал не логикой, не умозрением, а практикой. То есть,
например, вот есть аксиома, что рычаг работает в зависимости от пропорции длины
рукавов, ой, простите, плечей, да, каких рукавов, плечей. Вот. Он выстроил такую
математическую формулу, создал, грубо говоря, какую-то аксиом определенную, а
затем он строит машину, машину, буквально машину, которая подтвердила бы или
опровергла это положение теоретическое. Вот, по сути дела, именно это Галилей
называет механикой, чувствуете, потому что механикой свою работу называют
Архимед, он механик, да, он не какой-нибудь там арестователь, умозрительные
конструкции, он механик. И Галилей, поэтому свою работу тоже называет механикой.
И он призывает механиков-то опираться на конкретные математические расчеты. В
диалоге о движении Галилей резко раскритиковал метод Аристотеля, ну, конечно же,
нашим врагов университетских, потому что университет до сих пор тогда еще живет
по законам, что называется, физики Аристотеля и тому подобное. Вот. Он был
немножечко бунтарем, читал лекции на разговорном итальянском языке, не на
латыни, за исключением сложных доказательств. Их он излагал на латинском, чтобы
математические выводы могли и, ну, как бы все более такие образованные читатели
осознать. в 1632 году была опубликована книга «Диалог о двух главнейших
системах» Птоломейской и Коперниковской, в которой Галилей кроме разговора об
астрономической картине мира много внимания уделяет именно физике. Он
опровергает аристотельскую концепцию движения в этой работе, формулирует принцип
относительности, то есть внутри равномерно движущейся системы все физические
процессы протекают, так же, как и внутри покоящиеся, то есть мы не заметим
движение, если будем равномерно двигаться. Постольку, поскольку он опирался на
Коперниковскую книгу, которая находится в индексе, находилась в индексе
запрещенных книг, туда она попала, почему? Напомню, потому что в этой книге
святая палата узрела магию. Помните? Джордана Бруно Солнце, такой вот магический
новый бог, который в центре мира, в центре мироздания. И Копернин говорит, что
Солнце в центре мироздания, разбираться они не стали. Все, что в центре
мироздания, не все работы, которые там ставят Солнце в центр мироздания, это еще
одна очередная магическая работа, поэтому все в интере запрещенных книг. Но
Галилей в своей работе честно говорит, а он был каноником, он был как бы
церковным человеком. Он говорит о том, что нет, неправильно, Коперник не говорил
ничего про то, что вот там Солнце это новый бог. Коперник просто математически
рассчитал то, что рассчитал, и это очень правильно все это работает. И я
обосновываю на основе принципа Коперника вот как раз мелиоцентризм. И, конечно,
его за это не осудили бы, вот вплоть до какой-то смертной казни и тому подобное.
Ничего подобного и не было. Коперник и, простите, Галилея, инквизиция, тема
интересная вообще на самом деле. На Галилея наехали на самом деле иезуиты, очень
серьезно наехали. Вот если бы иезуиты довели до конца своего расследования, то
вполне возможно Галилей бы поплатился жизнью. Почему? Потому что его-то обвинили
именно в богословском, в богословском, что же называется, в ересе его обвинили.
Дело в том, что Галилей написал книгу о присуществлении, ну то есть это
превращение хлеба и вина в кровь и тело Христа. Причем там не было никаких
антирелигиозных, антицерковных идей. Он просто по-новому трактовал. Чисто вот
религиозно, она даже никак не входит в список научных работ в Галилее. Из-за
этого он столкнулся с иезуитами очень серьезно. Иезуиты пытались его арестовать,
вот так вот, посадить его в тюрьму. И за него вступился его один из друзей, Папа
Римский. Папа Римский был друг Галилея. И чтобы спасти его от этого, что
называется, суда, его подвели под другой суд. Суд о том, что он использовал в
своей работе книгу из индекса запрещенных книг. Галилей сказал, все понял,
осознал, признал, исправлюсь. Вот. То есть, ничего ему не грозило, он понавили
хозяина тюрьмы, не хозяин, а директор, начальник тюрьмы просидел, положенный
сок, вышел из этой тюрьмы совершенно здоровым, но полных сил, никаких не было
над ним пыток и истязаний, ничего это не поддерживает ни один исторический
документ, это опять пытанка очередная. То есть, я не говорю, что он не подсказал
от религии, да, потому что иезуиты тоже религиозные народы, но нет, нет
инквизиции, скажем так, инквизиция в этот случай его как раз спасала. Вот, он
вышел и после уже выхода из тюрьмы написал свою главную, еще одну главную
работу, начало и получил признание очень такое, ну, почти, всеевропейское, это
точно, был очень популярен, и, что называется, не вышел каким-то разбитым
стариком, это неправда. Ну, ладно, я эту тему оставляю немножечко, двигаюсь, вот
к какой теме еще он, Галилея Медведов, положил на час, начало процессу
инструментализации науки, подчеркните это особенно, это крайне важно,
инструментализация науки. То есть, вот это вот двучленное отношение наблюдающий
субъект, наблюдаемый объект, дополнился теперь третьим, прибором, есть научный
прибор. Вообще, в это время к теме прибор, как предмет, с помощью которого мы
видим мир, получает свое очень такое активное развитие. Ведь дело, смотрите, в
чем, долгое время исследующий мир старались не использовать никаких, ни линз,
никаких очков, очки были изобретены в Средневековье, и мы видим много портретов,
когда люди в очках, но вот в качестве исследователей очки или линзы старались не
использовать, потому что считалось, что они вносят искажения. Ну вот одно дело я
своими глазами вижу, а другое дело вот эти через линзы. Ну догадайтесь, во
многом это происходило под теорией чего? Конечно же, теории пневмы, потому что
очки задерживают пневму, искажают. И так-то мы фантазмами видим, да, и так-то мы
там, лучше называется, там столько воздействия пневмы, а тут еще и какие-то там
окуляры. Ну понимаете, да, все же совсем будет все утром. Тогда как вот Галилей,
он по сути дела отбросил, раз уж он бросил концепцию пневмы разного рода, то он
отбросил и вот эти беспокойства. Хотя наверняка вы как раз понимаете, что на
новом этапе эти проблемы снова возникли, потому что и вороны стали сложнее, и мы
уже не знаем, мы точно ли видим реальность или просто преломление в виде каких-
нибудь там следов от частицы в камере Вильсона, да. Ну вот понимаете, это вечная
проблема, но на определенный момент Галилей ее решил и создал очень большое
количество приборов, в том числе механических. все это потребовало, во-первых,
как сказать, само время. Развитие военного дела поставило задачи расчета
оптимальной дальности полета снаряда, артиллерии же, да, уже, всякие другие
динамические проблемы. Кто будет решать? Будет решать ученые, разумеется, все
эти расчеты производить. Вот. Дальше подзорная труба крайне была важна, да,
опять же, ну, военное дело создает техническую науку, а военную, вот эту
подзорную трубу можно же превратить в телескоп, да, соответственно, появляются
специфично научные инструменты, то есть инструменты как таковые, видите, военные
не парились, я там вижу, пневма там мне фантазм дает, или я вижу, я просто лучше
вижу, я вот так вот приглядываюсь, вижу там, кучку людей, а когда я
приглядываюсь, я в этой кучке могу разглядеть уже где там, понимаете, да, вопрос
об ней у военных не стоял, и поэтому они сказали, так, все делаем, и не паримся,
а соответственно, ну, вот это как бы более трезвомыслящее отношение к прибору,
потом и в научный, как бы, дискурс тоже смещается. Изобретение микроскопа сразу
же положило начало микроанализу, да, из крупнейшей открытия в самых разных
областях, итальянский биолог, мальпиги, исследует строение внутренних органов
животных, вписывает структуру растений, роберт Гук усовершенствовал микроскоп,
ладно, про это я не писала, ну, ничего, усовершенствовал микроскоп, приходит к
выводу о клеточном строении растений, даже вводит понятие клетка, тут же нужно
упомянуть биолога Левенгука, очень важная, да, фигура, пожалуйста, уточните,
проанализируйте, поисследуйте эти, о роли этих имен, этих людей, если
изобретение микроскопа сыграло выдающуюся роль в биологии, то для развития
физики даже еще, наверное, большее значение имело изобретение телескопа, первым
вариантом был телескоп, более ранний там, его авторы, Литл Сгейм и Литл Сгейм,
не помню, второй, зрительная труба, а вот следующий шаг, это был Галилеевский
микроскоп, простите, телескоп, и, в общем-то, благодаря нему увидели, ну, Луну в
совершенно новом облике, особую роль играли появление механических часов с более
точными делениями, да, то есть, когда уже там минутная стрелка, не просто час,
но минутная стрелка, для физики это крайне важно, чтобы исследовать динамические
характеристики. Вот, Агаций, это современные исследователи, Эвангра Агаций, он
предлагает даже имя собственное для современной науки, вот чтобы, понимаете, не
путаться с тем, что наука в античности, наука средневековья, наука нового
времени, он предлагает имя для собственной, ну, то есть, собственное имя для
современной науки, называет её технонаука, потому что, говорит он, эта наука, и
приходится с ним согласиться, она уже имеет дело не просто с реальным миром, она
имеет дело с реальным миром, который, преломление, который, через дни, который
мы изучаем, преломившимся через наши приборы, научные приборы, ну, ваши, наши,
мы философы, что называется, в другой парадигме. Понятно, да? Вот тоже обратите
на это внимание. Ну, и заканчивая «Галилея у Галилея», что ещё могу сказать? А,
ну, вот я вспомнила, что одной из важнейших его книг является беседой
доказательства, касающихся двух новых наук. Это была научная сенсация. В ней,
этой книге, он как раз вводит определение силы, скорости, ускорения,
равномерного движения, инерции, средней скорости, среднего ускорения, импульс. И
многие исследователи считают, что теория импульса «Галилея» это переработная
теория импетуса средневековых схоластов, мертонских схоластов. Тоже обратите на
это внимание. Ну, и что я могу сказать? Эта книга была издана через пять лет
после суда. То есть её год, 1638 год издания, а в 1633 году «Галилей» был
осуждён за использование запущенной книги Коперлинга. Ну, поэтому давайте будем
осторожны с этими выводами о том, что Галилея там, он крикнул, последние силы
потратил на крик, и всё-таки она вертится. Ну, это, конечно, никто вообще не
подтверждает ни один документ. То есть, ну, да, был суд, человек, как говорится,
там, пошёл этот суд, с достоинством вышел из него и продолжил работать.
Следующий человек, которого, ну, нельзя не упомянуть, разумеется, в рассуждениях
о науке нового времени, это Исаак Ньютон. Он родился в госсмерти Галилея, в
своём труде «Математические начала натуральной философии» обобщил открытие
Галилеи и добавил к ним третий закон, закон всемирного тяготения. Ну, про
Ньютона рассказывать не так просто, как про Галилея. Почему? Потому что этот
человек более мутный, более скрытный, более тёмный у него там, сама и его наука
более тёмная. Он активно занимался алхимией. То есть Ньютон как раз это такой
учёный, который не перешёл ещё из прежней науки в тип новой науки. Да-да-да, вот
как ни странно. Вроде, эта наука новая, типа называется ньютонянская, но сам-то
Ньютон был алхимиком по преимуществу. Более того, когда ему прислали письмо,
Роберт Гук прислал ему письмо, предложил Ньютону решить ряд проблем вычисления
теории планетных движений. А в ответ Ньютон написал, что в его возрасте
затруднительно заниматься уже всякой ерундой. Его куда больше интересует
алхимия. И вот он просто, в силу того, что он был талантливым человеком,
мыслительно талантливым, он предложил Гуку поставить эксперимент по проверке
теории Коперника. Написал и написал, что называется. Но когда он говорил про его
возраст, такой старый, ему было 37 лет. Но всё-таки, видите, вот это вот как бы
задел-то вот этот вот, его, ну, как это сказать-то, рвение учёного, этот задел,
уловил, и он начал решать задачу, которую ему Бог послал. Вот. И создал вот
новую механику, Мирненскую механику. Было отчётливо сформулировано сам подход к
природе, да, то есть к тому, что нужно, понимаете, какой был подход. Помните,
знаменитая Ньютоновская, я не изобретаю гипотез, я гипотез не измышляю. То есть,
давайте будем анализировать только то, что можно эмпирически наблюдать и
описывать это. Ньютон в предисловии к первому изданию говорит, новейшие авторы,
подобно древним, стараются подчинить явление природы законам математики, но в
данном случае он говорит законам таким умозрительным, а я же намерен заниматься
тем, что относится к силам, притягательным и напирающим. То есть, естествознание
позиционируется как исследование сил, математика в нём не цель, а средство,
метод. Галилей проводит серию опытов с маятником, ну, чтобы проверить выводы
Галилея, убеждается в этих, ну, в том, что Галилей прав, как сказать, обобщает
эти выводы, выводит категории массы как единственной причиной гравитационного
взаимодействия, масса — это нужные свойства вещества, вес — это сила тяжести,
действующая на тело, то есть динамика Ньютон, да, она позволила решать любые
задачи о положении движущегося тела. Вот. В этом важно. Положение движущегося
тела. И реальность стала рассматриваться как совокупность движущихся тел. при
этом пространство и время абсолютно. Это вместилище этих тел. Сон, что абсолютно
это означает, что они никак не влияют на сами движения, да, этих тел. Они
безотносительно, так называемые, к движущемуся телу. Все движения
рассматриваются механическими, то есть они могут быть подвергнуты
количественному анализу. При этом, когда его спрашивали, а его очнити на
спрашивали, ну, а вот это всемирное тяготение, это, благодаря которому яблоко
падает на землю, и луна падает, бесконечно падает на землю, и другие небесные
тела находятся в состоянии вот этого всемирного тяготения к другим телам, как
его понимать? Но вот тут Ньютон ускользал от ответа и придумал эту концепцию. Я
не измышляю гипотез. Потому что он был бы вынужден сказать, что вот это
всемирное тяготение им понимается самим, всё-таки магическим. Как вот эта вот
всеобщая связь у Джордана Бруно. Да? Для него, вот как бы, он же не случайно был
химиком, это была сфера его интереса, главная сфера его интереса, по большому-то
счёту. Вот. Для него абсолютное пространство, вот это вот, в котором
осуществляется тяготение, это, по его выражению, «sensorum dei», то есть
чувствилище Бога. Это некое, скажем так, ну вот, тело Бога. То есть, если мы с
религиозной позиции отнесёмся к Ньютону, ну, не обратимся к Ньютону, то станет
понятно, что, конечно, он еретик. Ну, то есть, он разделяет позиции пантеизма,
пространство — это божественная самореальность, некатегория времени, отсюда и
дальнодействие, принцип дальнодействия. То есть, например, вот, упавшее яблоко
на землю действительно перераспределяет весь вот этот вот расклад сил во
Вселенной, причём мгновенно, мгновенно, сразу же, не там, не на волнах эфир, а
вот мгновенно и сразу же, просто Вселенная знает. Почему знает? Потому что это и
есть тело Бога, чувствовище Бога. Как он так вообще, вот, ну, как он так,
разрешил себе мыслить? Не странно, потому что, ну, он был действительно тем
самым еретиком. Это признание безапостасного Бога, так называемого, не будут
окружаться, это и есть Бог-керметистов, безапостасный Бог. И про это, конечно,
современные исследователи очень хорошо знают, что Ньютон был арианином,
безапостасным Богом, верящим. Вот. Но, понимаете, в чём дело? Он же это открыто
не сказал в тех вот, в тех письмах, которые он писал. Это сегодня только из его
архивов это выяснили. А он просто сказал, я гипотезно измышляю. Почему он так
сделал? Ну, что он работал тогда в Тринити-колледже, колледже Святой Троицы.
Понимаете, что нельзя было ему говорить о том, что вот я верю в Бога, то есть
вовсе не Святую Троицу. Понимаете, да? Была бы проблема. Но его, в основном
прикали за принцип дальнодействия и за то, что он, ну ты давай, объясни, что это
за всеми текотениями. А что, посредством чего он осуществляется? Поэтому очень
долгое время в Европе всё-таки рулила теория Декарта. Декартовская концепция
признавалась всеми, ньютоновская далеко не всеми. И критерием истинности закона
тяготения Ньютона, вообще теории механики Ньютона, да, стал вопрос о форме
Земли. Согласно теории Ньютона, она должна была быть сплюснута с плюсов. По
теории Декарта она должна была быть удлинена, ну типа, знаете, мечтан для этого
Гамбола, да? И вот поэтому, чтобы выяснить правоту либо Ньютона, либо Декарта,
была проведена специальная, вот специальная экспедиция в Перу и Лаплангию. Она
подтвердила сплюснутость полюсов и всё, ньютоновская теория победила. Таким
образом, давайте сделаем небольшой вывод и всё уже к научной картине мира,
обобщения. В научных теориях того периода ещё много того, что мы назвали бы
метафизическим или даже герметическим. Наука не рыгалась вот так вот, ба-бах, из
ничего, новенькая такая вот, нулёненькая, сразу же такая научненькая, да? Нет.
Но главное вот что. Да, какие-то элементы герметизма или метафизичности ещё
сохраняются, но существенные изменения проснулись самих подходов к науке, потому
что меняется культурно-исторический контекст. И вот какова же научная картина
мира, новой реальности. Что такое научная картина мира? Вот она достаточно, ну,
лёгкая, но тем не менее, объёмное определение. Пожалуйста, разберитесь с ним,
выясните, как об этом на экзамене рассказывать. Что она включает? Самое главное,
что она включает? Это первая категория причинности, то есть жёсткий детерминизм
или ещё по-другому лапласовский детерминизм. Детерминизм, это означает
причинность. Почему она жёсткая? Потому что вот она, одна причина, одно
следствие, вечная, неизменная вот эта вот связь. Дальше всё это выстраивается в
линейную причинность, во времени разворачивается. и, соответственно, если мы
знаем, как говорил лаплас, положение всех частиц в определённый момент времени,
знаем их массу и знаем ускорение, с которым, ну, там, какое-то действие на них
оказывается, то, вернее, знаем импульс, да, ускорение, то мы в итоге можем
проследить эту причину и следствие как глубоко в прошлое, так и далеко в
будущее. То есть мы знаем всё, мы можем узнать всё. Нам нужен только такой
разум, или, ну, в общем, да, такой разум, который бы мог удерживать в своей, как
говорится, в своей памяти, в своих расчётах положение всех частиц и их массу
импульс. Тогда всё, мы можем, например, описать состояние мира через 200 лет,
ну, в абсолютности, то есть где вот какой конкретный человек, там, где
конкретная сущность. Понятно, да? Что такое жёсткий лапласовский линейный
детерминизм? И поэтому, вот, лаплас предлагает категорию демон, демон, который
может удерживать в памяти всё вот это вот, лапласовский демон, так называемый.
То есть, смотрите, категория случаев полностью отвергается, она становится нужна
просто-напросто, и это очень импонирует новому мировоззрению. Я сейчас
возвращаюсь вне, постараюсь вас вернуть в начало моей лекции, где говорилось о
трудных временах, 30-летней войны и всё такое. Понимаете, действительно, на
фортуну уже больше никто не действует, хочется, а вот, избавиться даже самой
категории случаев фортуны. Всё, новые исследования. В мире, в реальности, случая
нет. Никакая фортуна на вас не действует. Никакая случайность. Надо просто
поднапрячься, надо выяснить все вот эти вот характеристики, все условия, и мы
сможем управлять всем миром. Мы сможем управлять всей реальностью, всё будет под
контролем. Когда мы всё получим под контролем, в смысле, всё получим контроль
над реальностью, но не магически уже, да, а именно вот такой вот контроль,
благодаря нашей способности знать все причины и следствия. то мы в конце конца
будем управлять реальностью, и всё станет у нас хорошо. Понятно, да? Конечно же,
это причина, вне, это понимание, вот это вот линейный детерминизм, он напрямую
связан с категорией деизма. Деизм – это религиозная позиция, противостоящая нет,
где у нас здесь вот деизм, она противостоит принципу теизма, то есть Бог не
личность, а, вернее, Бог личность, который постоянно участвует в реальности,
постоянно её как бы, ну вот, через чудеса, через другие какие-то моменты он
управляет реальностью. А деизм предполагает, что вот всякие разные, как сказать,
внеплановые, внеплановые воздействия на реальность исключены, даже божественно.
всё, ничего уже больше не вот эту вот машину, она заведена один раз, машина
мира, и она фурычит. Нам нужно лишь выяснить, как она устроена, тогда мы сможем
её управлять. Вот, она фурычит, и вот тут, конечно, механицисты разделились,
одни, механицисты разделились, одни говорят, ну, эту машину Бог заводит каждый
раз, еще не каждое мгновение, она, да, она построена им однажды, но он заводит
её каждое мгновение. Например, левниц был таким механицистом, он говорит, нет,
что Бог заводит каждую часть, то есть левниц не был деистом, а вот, например,
тот же самый, кто же, кто же, кто же, кто же, по-моему, он был деистом, потому
что он говорит, что один раз Бог завел эту машину и больше не вмешивался, но
даже не важно, кто был деистом, просто эта позиция разделялась многими. Кант был
очень против деизма, он говорил, что деисты не отвергают веру в Бога, но они
признают лишь первосущность или высшую причину, да, то есть Бог это первая
причина, а дальше он не вмешивается. Но он говорит, что справедливый будет
сказать, что деист верит в Бога, а атеист верит в живого Бога. Ну, сам, конечно,
был не деистом. Ладно, дальше ещё. Деист довольно быстро умер, деизм, как
религия, довольно быстро умер, он был, ну, как бы, не очень так. Я очень строго
рассказывал про религиозную реальность. Важный принцип редукционизма. Вот важный
принцип редукционизма, пожалуйста, вы учтите, у меня времени осталось мало,
поэтому я, скажем так, больше это ставлю на ваше, на ваше усмотрение.
Редукционизм означает, что реальность выстроена, начиная с элементов малых, то
есть от малого к большому, снизу вверх, и узнавая, как, как сказать, узнавая
специфику взаимодействия малого, ну, например, молекул, там, атомы, да, мы в
конце концов узнаем характеристики целого. Редукционизм противостоит холизму, то
есть это противоположная позиция, согласно которому свойства целого определяют
свойства элементов. Это две методологические стратегии, которые борются в
истории науки, истории мировоззрения, ну, то есть они вполне себе такие противо,
ну, как сказать, противоречащие другу, но вот расцвет редукционизма это,
конечно, этап механицистической науки. механицистической научной картины мира. У
меня совсем немного времени осталось, и я просто всё равно обязана сказать,
извините, я вас задержу, о появлении статистики. О чём, почему я это говорю?
Теория вероятности. Невероятности. Женщина, вероятности. Это что со мной было,
когда я это писала? Ой. Ладно. Наука нового времени, значит, она активно изучает
теорию случайных явлений. Ну, вы понимаете, да, раз ты утверждаешь принцип
линейного детерминизма, то тебе всё-таки приходится разобраться с категорией
случаев, например, с игральными этими самыми кубиком, да, игральными костями.
Как же всё-таки так получается, что мы не можем предсказать поведение игральных
кубиков? Этим особенно занимался Лаплас, тот же самый, он и есть на
столкоположной теории вероятностей. Этим активно занимались многие-многие
исследователи. Так рождается наука статистика. Почему статистика? Потому что
совершенно в духе времени, вот в этих случайных процессах основоположники теории
вероятностей нашли законы, то есть из случаев подчиняются законам. Но правда, по
их мнению, это был лишь промежуточный вариант. Теория вероятности не
окончательная. Мы, согласно теории вероятности, можем сказать, сколько раз из,
например, из ста бросков монета упадет на решку или на орла, да, можем сказать.
Но мы не можем, теория вероятности не скажет нам, какой это будет случай-то. И
по, согласно Лапласу и другим исследователям, это лишь, ну, пока тот самый
демон-то не существует еще, да, вот мы еще немножко поднапряжемся и в конце
концов какой-нибудь, какой-нибудь нам там интеллект поможет все-таки исключить
категорию случайности и мы будем знать каждый раз, как монета падает, да. А
другим исследователям крайне понравилось, что теперь вот этот вот хаос
социальный может подчиняться определенным закономерностям, определенным
закономерностям. И вот здесь я фразу Кетли, это страховщик и основатель теории
статистики. Он просто пишет, ну, чуть ли не религиозный текст насчет того,
насколько прекрасно, что теперь вся, он даже дает категорию такую, выводит
категорию социальная физика, что вся, весь вот как бы аспект связанный с волей,
со всякими там разными преступлениями, ну, и вся вот этот социальный хаос, да,
кто-то влюбляется, женится, там, умирает, преступление совершает, в тюрьму
садится, вот этот весь хаос, оказывается, можно посчитать. Смерть, преступление,
брак, рождение детей, это область исследования Кетли, он выводит статистические
закономерности в этой области и делает вывод, нет свободной воли человека ни в
деле брака, ни в преступлениях, есть лишь социальная физика, подобная образец
становится преобладающей, социологические исследования выходит на это, они
вдохновлены идеей того, что вот в этом хаосе социальном принципиально есть
порядок, а я напоминаю, что парадигма порядка, контроля, это и есть ключевая
идеологема этого мира, она воплощена зримо во многих автоматонах, тоже
полюбопытствуйте, и самое последнее, что я хочу сказать, но я не буду это
пояснять, много, я скажу лишь о важности, а вы, пожалуйста, ну, либо
самостоятельно, либо на семинарах проговорите еще этот момент, речь идет об
институциализации науки, появлении науки в социального института, дело в том,
что до вот как бы определенного периода все эти фундаментальные законы природы,
они были делом энтузиастов, но постепенно эти энтузиасты осознали, что требуется
не просто материальное вливание для того, чтобы проводить эксперименты и тому
подобное, а требуется связь с государственной властью, и они смогли убедить
королевскую власть в то, чтобы, ну, вот наука интересовалась, стала областью
интереса государственной власти, а так как к тому времени уже, наверное, вы
обратили внимание, что наука это коллективное дело, тот же Ньютон пишет письма,
да, Лебнеев пишет письма, так называемые невидимые колледжи образуются, когда
вот эти письма зачитывают в публикации, понимаете, то есть уже это дело
коллективное, вплоть до того, что в конце концов именно коллектив исследователей
выбирает, какая теория правильная, ньютоновская все-таки, физика-то правильная,
или декартовская, помните, даже экспедицию собрали, две экспедиции, вот,
соответственно, вот эта вот идея, коллективность науки плюс государственная
власть формирует социальный институт науки, и ее первым взаимовоплощением
становится Академия наук, они конкурировали с университетами, тем более, что
университет это область тогда влияния папы по большей части, а вот, ну, а если
протестантские страны, то, конечно, не папа, но все равно это может быть
религиозная какая-то там мужчина, вот, а Академия это всегда связь с
государством, понятно, да? Пожалуйста, на эту тему обсудите, проговорите, тем
более, что вы будете сдавать экзамен в Академии наук, и будет неплохо, если хоть
немного будете разбираться в этом, ну, в данном случае уже самостоятельно. Все,
на этом у меня все, не смею вас больше задерживать, пока.