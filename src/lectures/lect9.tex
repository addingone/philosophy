% Для некоторых источников, вот таких, которые мы не сильно одобряем, если
% вы по ним будете готовиться к экзаменам, 

% по некоторым источникам наука только в
% это время и начинается, только в 17 веке. 

% Вы должны понимать, что нет. Вас учат
% по такой версии, поэтому можете озвучить ее, но сказать, что... Ну, либо, что
% называется, сражайтесь с нами и доказывайте, что все, что было до этого, не
% наука, вернее, радикальным образом отличается от того, что мы называем наукой.
% Понятно, что... 

% Вот я вам рассказывала в прошлый раз про магию, там еще про что-
% то. Понятно, что это не наука, в современном смысле этого слова. Но тем не
% менее, если мы посмотрим на историю развития науки дальше, вот каких-то таких,
% ну, резкого, знаете, такого, вдруг появилось новое знание. Нет, знание
% трансформировалось из-под того, что было раньше. Оно не появилось вдруг из чего-
% то пустого места какого-то. Вот новая наука взяла и возникла. Но тем не менее,
% эта трансформация очень важна. 

\section{Становление классического типа
рациональности и науки Нового времени} 

\subsection{Социально-исторические и идеологические
условия формирования науки Нового времени}

% Тридцатилетняя война (1618-1648 гг.). 

% Ряд войн общеевропейского масштаба: коалиция французов, англичан, шведов, против династии Габсбургов (Австро-Венгерская империя).
% Династия Габсбургов была католической, однако с ними воевали как протестанты, так и католики. Это война за господством над Европой.



% Используемые наемные армии содержались не только за счет
% государства, но и за счет грабежа захваченных городов и сел. Мирное население серьезно страдало, и часто солдаты истребляли мужское население захваченных городов. Ужасающие события войны:  Варфоломеевская ночь, Магдебургская
% резня и др. 



% На этот период приходится также Малый ледниковый период и эпидемии чумы.

% В Европе в 1650-х годах население сократилось настолько,
% что в Баварии для восстановления населения церковные власти разрешили многоженство. 


% Эти события отразились на мировоззрение людей. Если в эпоху Возрождения двигателем новизны были люди, воодушевленные культом фортуны, то в этот период <<подарки фортуны>> были исключительно негативными.


% Европа стояла на грани совершенно незавидной судьбы быть задворками Османской
% империи: идея фикс Османской империи завоевать Европу и превратить ее
% население в мусульман. Исполнением этих чаяний мешают лишь, знаете, такие дикари (так сами османы назвали жителей России), которые в 17 веке вдруг вмешались в этот расклад. Османскую империю в этот период подтачивают казаки, которых спонсирует Иван Грозный. 

Тридцатилетняя война (1618–1648) --- серия общеевропейских конфликтов, в которых коалиция французов, англичан и шведов сражалась против католической династии Габсбургов (Австро-Венгерская империя). Война велась и протестантами, и католиками, став борьбой за господство в Европе.

Наемные армии содержались не только государством, но и за счет грабежей городов и сел. Мирное население сильно страдало, часто уничтожали мужчин в захваченных населенных пунктах. Ужасающие эпизоды --- Варфоломеевская ночь, Магдебургская резня и другие.

В этот период произошел Малый ледниковый период и эпидемии чумы. К 1650-м население Европы сократилось настолько, что в Баварии церковь разрешила многоженство для восстановления численности.

Эти события изменили мировоззрение: если в эпоху Возрождения люди верили в удачу и новизну, то теперь «подарки фортуны» были лишь негативными.

Вестфальский мир (1648 г.)

После окончания войны заключается Вестфальский мир, устанавливающий Новый европейский порядок. Его основные положения:

\begin{itemize}
    \item Принцип государственного суверенитета --- независимость и верховенство государства в международных отношениях, а также по отношению ко внутренней власти. Включает неделимость, единство территории и принцип невмешательства во внутренние дела страны.
    \item Государство национального типа --- принцип единой границы, очерчивающей единую территорию, в пределах которой устанавливается единая государственная власть.
    \item Принцип международного права как координации отношений между государствами. Это было потребностью не правящих элит, а буржуа --- тех, кто в этот период старается наладить жизнь, чтобы вести торговлю, развивать ремесла, и т.п.
\end{itemize}

% Буржуазные отношения, которые в тот период сложились, возникают из
% бургов --- городов, созданных для реализации профессиональной деятельности, купечества и ремесленничества. 

% Черты представителей буржуазии:

% люди достойной профессии (судьи, адвокаты, прокуроры, нотариусы, врачи и хирурги,
% цирюльники, купцы-негоцианты);

% владение мануфактурами;

% владение пригородными землями;

% жизнь в собственном доме.

% Буржуа --- это новый благородный человек, который при этом не имел юридического статуса благородства как наследные дворяне, аристократы. 



% Революции, возникшие в XVII в., не продиктованы нищетой народа. 
% Народ в это время не духоподъемен, и находится в состоянии выживания. 

% Бертон <<Анатомия меланхолии>>:
% <<Что нигде не слышу новые вести, обычные слухи о войне, о бедствиях, о пожарах, наводнениях, грабежах, убийствах, резне, метеоритах, кометах, привидениях, чудесах,
% призраках, захваченных селениях или осажденных городах, смешение бесчисленных
% клятв, ультиматумов, помилований и указов, прошений и тяжб, ходатайств, законов,
% воззваний, жалоб и обид. Мы слышим это каждый божий день>> 

% Восстания народа поднимает возникший слой крупных буржуа, которые становятся новыми лидерами мнений, а в последствии --- социально-политического порядка.


% Буржуазный тип отношений --- это капитализм. 
% Капитал --- это средства, используемые для получения прибыли и роста
% производства. 

% В рабовладельческом и феодальном обществах производство продукта в единицу времени было преимущественно стабильным. Капитализм принципиально основан
% последовательном увеличении производства товаров в единицу времени. 


% По концепции Макса Вебера, основанием капитализма являлись не экономические потребности, а идейные, культурные, духовные. Так, в XVI-XVII вв. развивался пуританский капитализм, представители которого противостояли роскоши феодалов.


% Пастор Ричард Бакстер считал, что человек должен искать воздаяние в исполнении своих профессиональных занятий, а Господь, который заранее знает, кто будет избран, а кто проклят, указывает нам на избранность. Заработанные деньги должны быть пущены в рост, а оставшиеся после этого --- на благотворительность.




% Техническая основа развития капиталистической формы хозяйствования --- развитие
% машинного производства. Мануфактуры --- довольно крупные предприятия,
% основанные на ручном труде наемных рабочих с применением ремесленной техники. Для мануфактур свойственно разделение труда на отдельные производственные операции. Самая распространенная форма раннего капиталистического производства;
% предшествует машинной индустрии (фабрике).
\paragraph{Буржуа}

Буржуазные отношения того времени возникли из бургов --- городов, созданных для профессиональной деятельности, купечества и ремесленничества. Представители буржуазии --- люди достойных профессий (судьи, адвокаты, прокуроры, нотариусы, врачи, цирюльники, купцы-негоцианты), владеющие мануфактурами, пригородными землями и собственными домами. Буржуа --- новый благородный человек, лишённый юридического статуса дворянства.

Революции XVII века не были вызваны нищетой народа, который в это время находился в состоянии выживания и не был духоподъемен. Как писал Бертон в «Анатомии меланхолии», люди ежедневно слышат одни и те же новости --- войны, бедствия, пожары, грабежи, убийства, резню, природные катастрофы и юридические споры. Восстания поднимает новый слой крупных буржуа, которые становятся лидерами общественного и политического порядка.

Буржуазные отношения --- это капитализм, где капитал --- средства для получения прибыли и увеличения производства. В рабовладельческом и феодальном обществах производство товаров было стабильным в единицу времени, тогда как капитализм основан на постоянном росте производства.
По Максу Веберу, основой капитализма были не экономические нужды, а культурные и духовные идеи. В XVI–XVII веках развивался пуританский капитализм, противостоявший феодальной роскоши. Пастор Ричард Бакстер учил, что человек должен искать воздаяния в профессиональной деятельности, а Господь заранее знает избранных; заработанные деньги следует пускать в рост, а остаток --- на благотворительность.

Технической основой капитализма стало развитие машинного производства. Мануфактуры --- крупные предприятия с ручным трудом наемных рабочих и ремесленной техникой, с разделением труда на операции --- представляли собой раннюю форму капиталистического производства, предшествующую фабрикам.



\textbf{Буржуазные революции}:

\begin{itemize}
    \item Первая буржуазная революция в Нидерландах (сер. XVI – сер. XVII вв.).  В Нидерландах развивается пуританский (аскетический) капитализм. Это двигатель европейской буржуазной трансформации, а также лидер машинизации процессов.
    \item Английская революция (1648-1660 гг.). Имело место знаменитое смещение короля Якова II Стюарта, а также казнь короля Карла I.
    \item Великая французская революция (1789–1799 гг.)
\end{itemize}

Эти революции не были однородными. Они никогда не были процессом, в которых одна лидирующая сила побеждает и переустраивает социальный строй. Их начинают одни люди, которых потом смещают или казнят. Их продолжают другие люди с другими целями; тех людей тоже смещают, и т.д.
Идейным основанием буржуазных революции и буржуазных трансформаций вообще была эпоха Просвещения. 



\subsubsection{Эпоха просвещения (XVIII в.)} 

% Важно, что это новое
% мировоззрение формируется уже не в университетской среде, и уж тем более не в
% университетах монастырей, какого-нибудь там школ соборных. 

% Нет, это люди, которые вот как раз новые интеллектуалы, да, нового типа. 

% Возрождение – это платоновская академия, такие кружки
% по интересам, 

% но они, понимаете, вот как бы все равно в особом месте. Люди
% собраны в особое место, похожие чем-то на академии античные. 

% В Средневековье
% интеллектуальная элита, то есть те, кто формирует вот этот вот фон
% интеллектуальный, это университеты и монастыри. Они между собой как бы даже
% сражаются, да, ну как бы за первенство. Два дискурса – университетский и
% монастырский. Но мы как бы с вами делаем акцент на университетский, потому что
% они действительно были прям во всем победителями, в каждом дискурте нет, такого
% не происходило. 


% В
% античности это академии разного рода, да, это гимназии, там, да, вот это, ну
% такие вот, прото-социальные институты науки. 

% Что же это в новое время, в
% просвещении? Это светские салоны. 

% Но если мы присмотримся к каждому из них по отдельности, это совсем не
% философы, это совсем не лидеры мнений вот в таком смысле, знаете,
% интеллектуального разума. Это публицисты. Это вот скорее современные медийные
% фигуры. 


% Возникли модные люди. Они собирались в светских салонах. Один из
% таких светских салонов, это вот салон Марии Терезы Жоффрен, куда в течение
% целых 25 лет сходились представители новых интеллектуальных сил Парижа.
% Собирались художники, ученые, философы эпохи просвещения, такие как Дидро,
% Вальтер, Даламберг, Гальбах и другие. 


% если мы начинаем смотреть, что эти там ведро и доломберы
% сделали, мы не найдем философских трудов серьезных. Исключением, пожалуй,
% оказывается только Руссо. Вот он писал там прям много основательно. 

% Ну,
% трудно сказать, насколько это, прям философия в чистом виде, потому что систему-
% то он не создавал. Но, тем не менее, хотя бы много написал. Вот. 

% Остальные
% писали работы им публицистического
% характера. 

% производство
% идей стабильное, было связано с формированием энциклопедий. 

% Энциклопедия предлагала набор знаний по всевозможным аспектам: 
% и история, и технологии, и какие-то понятия, но только в новом
% идеологическом ключе. 


% это идеология либерализма. Салон
% вообще, это первое воплощение политической партии. 


% А вот теперь партия это по политическим предпочтениям, по
% тому, что вы считаете предпочтительным в социальном укладе. Так появляются
% политические партии, и вот они возникают именно в таких светских салонах. 

% Что такое идеология? Само слово появилось во времена французской
% революции. Де Трейси, изобрел его для того, чтобы
% формировать науку идей. 

% Ну, даже почитай, разберите слово на составляющую. Идео,
% идея, логия наук, 

% это стало
% формированием программы действия, программы действия для достижения определенных
% целей каких-то субъектов политики. 

% Эти субъекты политики могут называться
% классом, могут называться партией, как сегодня, могут называться общественным
% слоем, но, в общем, в таком целостном, социальном пространстве всегда есть такие
% выделенные субъекты политики, которые очень нечетко фиксируемы, и вот у них
% должна быть какая-то, ну, чувствуется, что должны быть какие-то выразители наших
% идей. 

% Выразители этих идей становятся политиками, несущими определенную диалогу.
% Но надо отличать партию все-таки того периода от партии в современном смысле
% слова. 

% В современном смысле слова мы говорим о массовой политической партии.
% Тогда все-таки партия это вот эти светские салоны, в которых это группа
% единомышленников, которые собираются, обсуждают какие-то процессы политические,
% политические, предлагают решения, а дальше они идут непосредственно уже и
% выступают в каком-нибудь парламенте. 

% Что дает идеология? Почемуона формирует
% основу для легитимности власти. Легитимная власть это та власть, которой доверяют,
% которую признают правомерный, правомочный граждане данного государства. 


% чтобы большинство признавало, власть легитимная, оно должно
% знать, а какие идеи власть-то разделяет. 

% Это нормальный процесс осуществления любого государства. Основа
% для легитимной власти, для новых политических сил, для буржуазии, которые тогда
% формируются. И как раз явились просветители. У всех, у всякой власти есть своя,
% как бы, вот эта основа. А просветители, которые формировали идеологическую
% основу для новой власти, то есть для буржуазной власти, это так называемая
% рациональная форма легитимности. И ее характеристика, ее содержание это
% либерализм. 

% В истории либерализма не было работы, которую можно было бы назвать
% ключевой. 

% Она складывается из многих трудов таких философов как

% Томас Гоббс, Джон Локк, Адам Смит,
% Шарль Луи де Монтескьё, Бенджамин Франклин. 


% что такое
% либерализм? Еще раз, это философско-политическое течение, да? Но это еще и
% политический проект, и политическая культура. А еще должна
% быть успешная идеологема.


% Ключевая идеологема либерализма --- борьба за свободу. По мысли ее
% создателей, свобода, это не то, что определяет человека, это то, что должно, как
% бы, в борьбе осуществляться. 

% Понимаете, как бы, перевес упал не на слово
% свобода, потому что никто не знает что-то такое и не знал, а на слово борьба. 

% По
% мысли философа, свобода --- это то, что определяет человека. И,
% кстати, философы будут тщательно подчеркивать принцип свободы, свободы. 

% А вот в
% политике укрепилась акцент на борьбу. 


% Понятно, что капитализм встал на ноги в ходе
% буржуазных революций, когда буржуа были угнетаемыми. Они кричат, надо бороться с
% угнетением. Боремся с угнетением и политически, достраивая новые законы. И как
% бы, буквально в виде там сражений.

% Но постепенно буржуазия же сама стала
% угнетателями. Убившие дракона становятся драконом. Да? А как бы, понимаете, она
% же не может уже кидать клич борьбу с угнетением. Нет, так уже нельзя, потому что
% тогда бороться будут с тобой. 

% Соответственно, ну, например, ну, хорошо, какое-то
% долгое время еще этот клич работает, но он уже работает в социалистическом
% ключе. Это тоже один из модусов либерализма. Хотя они там были врагами, как
% политические партии, но тем не менее, это модус как бы либерализма в широком
% смысле этого слова. Потому что базовая идеологема тоже борьба с угнетением. Да?
% Но социализм в какой-то там форме случился, и современное общество даже не
% социалистического типа, это уже не борьба с угнетателями. А либерализм-то
% существует, он будет по-прежнему бороться. Да? Он будет по-прежнему бороться за
% что-то. И уже это будет борьба за свободу социальных меньшинств. Да? Вы
% понимаете, о чем я говорю. Это свобода, борьба за вот там еще за чью-то свободу
% и так далее. 

% То есть он уже до бреда доходит, но либерализм, это, как сказать,
% вот это вот, если он прекратит работу этой идеологемы, он остановится. То есть
% эта идеологема, это быть против, бороться за что-то. Это идеологема, которая
% является перпетум мобили губерализма. то есть она уже как бы уходит в массы.

% Политическая культура это то, что разделяет уже не философы, не какие-то даже
% там в отдельные. это вот то, что как бы становится принципом такого нормального
% поведения обычных людей, скажем так, не философов. То есть политическая культура
% быть против становится основой либерализма. То есть смотрите, быть не за, а быть
% против. Ну, соответственно, как бы чем я занимаюсь, будучи там политически
% активным человеком в духе либерализма? Я нахожу, против кого я воюю. Я не
% предлагаю, что сделать, что против, а я не предлагаю, за что я. Я всегда
% озвучиваю против чего. Перманент, вернее, претенденты на роль зла перманентно
% обновляются. Еще раз повторюсь, что в новое время это позволило состояться
% европейскому обществу. Далее это породило кризис того общества, которое,
% собственно, и создан был этим, этой идеологией. А теперь мы как бы к


Новое мировоззрение формируется уже не в университетах или монастырских школах, а среди интеллектуалов нового типа. Возрождение представляло собой платоновские академии и кружки по интересам, собранные в особых местах, похожих на античные академии. В Средневековье интеллектуальную элиту составляли университеты и монастыри, которые соперничали друг с другом, но именно университетский дискурс доминировал. В античности наука развивалась в академиях и гимназиях --- протосоциальных институтах.

В эпоху Просвещения центрами интеллектуальной жизни становятся \textbf{светские салоны}. Их участники скорее публицисты и медийные фигуры, чем классические философы. Например, салон Марии Терезы Жоффрен в Париже на протяжении 25 лет собирал художников, учёных и философов, таких как Дидро, Вольтер, Даламбер и Гольбах. Из них серьёзные философские труды создавал в основном Руссо, хотя системной философии он не разработал, а остальные писали преимущественно публицистические работы. Салоны стали первым воплощением политических партий --- групп единомышленников, обсуждавших политические процессы и выдвигающих решения, которые потом реализовывались в парламенте. Само слово «идеология» возникло во время Французской революции и означало науку идей, программу действий для достижения целей политических субъектов --- классов, партий или общественных слоёв. Выразителями этих идей становились политики.

Партии того времени отличались от современных массовых --- это были узкие группы в салонах, а не массовые организации. Идеология создаёт основу легитимности власти, основанной на доверии и признании её гражданами. Чтобы власть считалась легитимной, большинство должно знать, какие идеи она разделяет. Просветители сформировали рациональную основу легитимности для новой буржуазной власти --- либерализм. Он не имел одного ключевого труда, а складывался из множества работ философов, таких как Гоббс, Локк, Смит, Монтескьё и Франклин. \textbf{Либерализм} --- это философско-политическое течение, политический проект и культура с ключевой идеологемой --- борьбой за свободу. Свобода здесь понимается не как состояние, а как процесс борьбы.

В политике акцент сместился с абстрактной свободы на борьбу. Буржуазия поднималась через революции, борясь с угнетением, но со временем сама стала угнетателем, и клич борьбы стал менее актуален. Он перешёл к социализму --- тоже модусу либерализма в широком смысле, основанному на борьбе с угнетением.
Современное общество, даже не социалистическое, сохраняет борьбу, но теперь --- за свободу социальных меньшинств. Либерализм не может существовать без идеологемы борьбы, она стала перпетуум мобиле его жизнедеятельности.
Политическая культура --- это уже не философы, а повседневное поведение обычных людей, для которых суть либерализма --- быть против чего-то, а не за что-то. Политически активный либерал прежде всего определяет, против кого он борется, а не что предлагает. Претенденты на роль зла постоянно меняются.


\paragraph{Мировоззренческие установки Просвещения.}
\textit{Рационализм} --- приоритет рассудочно-логического и критического мышления. Это культ разума: слово «рационализм» происходит от «рацио», что значит разум.

\textit{Эвдемонизм} --- этическое учение, признающее главным принципом чувство счастья. Это важно, потому что в поздних рассуждениях о правах человека часто упоминается право на счастье вместе со свободой. Философы вроде Вальбаха, Гальбаха и Дитро утверждали, что счастье --- критерий повседневного опыта, достигаемый здесь и сейчас. Никто не давал точного определения счастью или свободе, главное --- бороться за них. Для просветителей счастье --- это благополучие и комфорт, что связано с идеями пикурейства, где счастье --- отсутствие страданий. В условиях постоянной угрозы и смерти это считалось высшим принципом существования.

\textit{Утилитаризм} --- моральное учение, где основой морали является принцип пользы, то есть благо --- это то, что приносит счастье большинству. Польза связана с благосостоянием, понимаемым как безопасность, комфорт и наличие собственности. Принцип собственности ключевой для буржуазного утилитаризма.

% Все эти идеи формируют проект модерна, основанный на бесконечном увеличении счастья, достигаемого через гармоничное общество, где всем хватает ресурсов. Это устремленное в будущее движение, основанное на эсхатологическом библейском принципе --- время не циклично, а линейно, с началом и концом истории, и с новым состоянием реальности, где человек должен приложить усилия, чтобы достичь блаженства.

% Эта идея взята из религии, но переосмыслена: теперь счастье и благосостояние --- дело человеческого прогресса, а не Бога.

С позиции морали важна \textit{пруденция} --- благоразумие, принцип соответствия обстоятельствам. Разница между старой героической моралью (самопожертвование ради общества, воинская доблесть) и новой моралью, основанной на благоразумии, не всегда очевидна. Герои прошлого жертвовали жизнью, ради чего их считали элитой. Сейчас жертвовать жизнью ради других кажется нерациональным, а пруденция позволяет оправдать разные действия с точки зрения обстоятельств.

\textit{Секуляризм} --- естественное состояние современного мировоззрения, где Бог перестает быть активным участником счастья. Благосостояние достигается здесь и сейчас, не после смерти, и религиозность теряет прежние формы. Связан с \textit{антиклирикализмом} --- противостоянием церковному институту. Религиозное содержание переносится на светские объекты --- так возникают секулярные или внецерковные религии. Пример --- культ науки. Просветители видели науку как новую религию Европы --- естественную религию. Говорили о «храме науки», ученых как «первосвященниках». Однако наука стала не религией, а новым типом мышления. Дух просвещения, связанный с масонством, пытался сделать науку продолжением магии и алхимии, но наука стала самостоятельной дисциплиной.

\subsection{Гносеологический поворот в философии Нового времени и её роль в формировании
научной рациональности}

% Почему гносиологический
% поворот? Потому что акцент уже ставится не на антологии учений о бытии, а на
% учений о познании. Гносеология. Но тем не менее, смотрите, общая для всех
% философов и публицистов и не публицистов того периода все-таки категория
% активности. Она ключевая на тот период вообще для всех людей европейских.
% Почему? Потому что она ключевая для буржуа-предпринимателей активность. Для
% научного метода, для философии. Силы и вещества они просто наблюдаются, они
% активно измеряются, активно. Потребность описать, систематизировать все
% природное богатство, вот эти вот категории учета, классификации, порядка. Они в
% науке активно развиваются. Классифицируются и упорядочиваются все. Небесные
% тела, гармонизуются представления астрономического характера, биологическая
% систематика возникает. То есть кладовая природа активно активно человек стоит
% рядом и записывает так, что у меня в распоряжении, у меня в распоряжении то-то и
% то-то. То есть природа противопоставлена субъекту познания. Человек приобретает
% свой статус вот знаете, такого вот распорядителя. Новый статус, новый, высокий
% статус человека, он распоряжается своей кладовой. Природа это лишь то, что ему
% принадлежит и чем он может распоряжаться. это такая вот форма активного начала.


% Чувствуете, человек всегда был весьма значим, но всегда по-разному. Я даже
% подчеркну это. 

% античность, космоцентризм, мировоззрение. И там человек
% это микрокосмос, отражение макрокосмоса. Макрокосмоса. Макрокосмос отражается в
% микрокосмосе. То есть человек значим постольку, поскольку он тоже космос.

% Теоцентризм. Да, весь взгляд, все внимание на Бога, но человек тоже крайне
% важная фигура. Человек, ну я говорила, что раб Божий это никоим образом никогда
% не было уничижительным каким-то, уничижительной характеристикой. Почему? Потому
% что это формула римского права. Когда раба продавали, говорили, это раб, не
% знаю, Цезарь, например, отныне такой это раб Цезаря. А когда христиане
% крестились во имя начислены рабами Божьими, они как будто тем самым говорили, я
% больше не раб никого. Больше я не раб ни Цезаря, ни Помпея, я больше не раб
% никого. Все, теперь я раб только Бога. То есть это юридическая форма. А так
% человек это животное призвано стать Богом. Помните, да, Зианцин выражение? 

% В
% эпоху возрождения человек он творец. Ну, творец, конечно, по преимуществу как
% маг, но тем не менее гомо-виртуозу. То есть он подчиняется реальность. Какими
% формами? Другой разговор, но он подчиняется реальность. Вообще, помните, да,
% первый капитализм, поплыть в море мрака, вообще никуда вернуться с кораблями
% набитым золотом. Вот так вот. Вот это вот. 

% А в эпоху нового времени не
% космоцентризм, не теоцентризм, не антроцентризм, а натурацентризм.
% Натурацентризм, то есть человек обращает, ну, как бы природа становится ключевым
% объектом исследования, а человек это тот, кто ей распоряжается. человек это тот,
% в кладовой природы главное распорядить. 

Акцент смещается с учений о бытии на учения о познании --- гносеологию. Однако для всех философов и публицистов того времени ключевой категорией остается активность. Она важна для буржуазных предпринимателей, научного метода и философии. Силы и вещества не просто наблюдаются, их активно измеряют. Возникает потребность систематизировать природное богатство: классифицировать, учитывать, упорядочивать. В науке развивается астрономия, биологическая систематика --- всё подлежит учёту и гармонизации.

Природа противопоставляется субъекту познания. Человек приобретает статус распорядителя кладовой природы --- природа становится объектом владения и распоряжения. Это новая форма активного начала.
Человек всегда был значим, но по-разному. В античности --- микрокосмос, отражение макрокосмоса, значим в силу своего космического статуса. В теоцентризме внимание сосредоточено на Боге, но человек важен как раб Божий --- термин не уничижительный, а юридический: быть рабом Бога значит быть свободным от других рабств, подчинённым только Богу. Человек --- животное, призванное стать Богом (Зианцин). В эпоху Возрождения человек --- творец, прежде всего как маг, гомо-виртуозо, подчиняющий реальность своим силам. В эпоху нового времени сменяется космо- и теоцентризм, приходит натуроцентризм. Природа становится главным объектом исследования, а человек --- её распорядитель.


\subsubsection{Декарт}
% Обратимся к Декарту, который вводит эту категорию субъекта как того, кто обозначает именно человека. 

% Наверняка вы уже
% понимаете, что раньше, до эпохи нового времени, субъектом называли просто
% подлежащее предложение. Субъектом может быть все что угодно. Например, стол это
% субъект деревянности. То есть это то, к чему
% деревянность относится, как его характеристики. 

% Или, например, субъектом может
% быть космос. Подлежащее предложение такое стоит во всяком случае. 

% А Декарт
% сделал так, в смысле так передумал категорию субъекта, что субъектом становится
% только человек. Он происходил из обнищавшего дворянского рода, недолго учился в
% университете в Пуатье, праздновременную службу, участвовал в битве Хакина. И вот
% на войне формируются его предпосылки механицистического взгляда на человека. 

% Я
% ввожу понятие механицизм, пока его не объясняю. 

% Механицизм уже будет ключевое
% слово, да, для этой эпохи. А почему возникло его понимание человека как
% механизма, как такую, ну как бы куклу, которая, да, оживляет Бог, дух, там
% подобное, но она как бы устроена как механическая кукла. 

% Оттого, что слишком
% много трупов видел, и более того, он эти трупы препарировал, и сам себя он
% называл великолепным мастером препарирования. Он как бы очень, ну как сказать,
% проникся этим делом, он хорошо, будучи, ну, мыслителем, хорошо проник в
% специфику вот этого вот, всего дела, и развивает идеи о том, что человек-то, по
% сути дела, машина, но только машина живая. И он с этими идеями, он пишет там
% небольшую работу, обращается к своим учителям, к иезуитам, но он учился, потому
% что доунин, все это в иезуитской школе. Но они, мало того, что не поддержали его
% начинание, они его как бы сочли опасным, и дальше всю остальную жизнь, он
% опасался их преследованием, переселился в Голландию, как вы понимаете,
% протестантскую страну, где пишет рассуждение о методе, чтобы хорошо направлять
% свой разум и отыскивать истину в науках. Мы его кратко называем рассуждением
% метода. А в приложении к этой книге Декарт опубликовал свои исследования по
% натур философии, по оптике, геометрии и по математике, по алгебре. Ну, то есть,
% все, что мы сегодня связываем в науке с достижениями Декарта, например,
% Декартовую систему координат и еще некоторые моменты геометрической
% геометризации алгебры, мы, оптические исследования, для него это все-таки
% приложение к философии. Его публикации, его идеи получают популярность, вызывают
% восхищение кардинала Ришелье, то есть, никто там его, с закатольки-то его не
% преследовали. Он предлагает издать труды Декарта во Франции, но Декарт уже
% никому там не доверяет, он отправляется в Швецию по приглашению королевы
% шведской, а в дороге он постудился, в итоге заболел и скончался, а многие
% считают, что его отравили. 

% Так вот, Декарт всю
% жизнь искал некой свободы от ограничений, да, свободы для себя, для своей мысли.
% И только свободный, говорил он, является самостоятельным, только когда он может
% сказать «я сам». Самость – это главное для философии Декарта. Я сам. Это
% то, что говорит он, объединяет всех людей. У всех есть эта самость. Это и есть
% основание свободы. Чувствуете, философы-то всё-таки идут по понятиям, да, не
% лезут в понятие свободы, не просто борись за свободу. Борись, а зачем бороться-
% то? Ты объясни, да? Просто найди себе уменьшательный, борись, ну извините, нет,
% так философ не может мыслить. Это такая мысль недоброкачественная. 

% Философия
% всегда обращает внимание на доброкачественные мысли. Вот. 

% Для Декарта основание
% свободы – это самость. Самость можно обнаружить тогда, когда сомневаешься. Вот
% это ключевая позиция. Ты вот эту самость, самостоятельность, самость можешь
% обнаружить тогда, когда сомневаешься. Причем сомневаешься во всем. Смотрите, он
% как говорит, можно сомневаться в том, что это помещение там, такой-то формы.
% Можно вообще сомневаться, что здесь помещение. Можно сомневаться в своем уме.
% Можно сомневаться, а вы существуете или нет вообще? Может, вы мне снитесь? Я
% могу сейчас спать, и мне снится, что я просто веду лекцию. Я могу сойти с ума,
% не будет казаться, что я преподаватель философии, а вы в это время просто эти не
% аспиранты, а эти как санитары, которые наблюдают за моим поведением. Понимаете,
% да? Можешь сомневаться во всем, но единственное, в чем я не могу сомневаться,
% это в том, что я в этот момент сомневаюсь. Вот. Единственное, в чем не получится
% сомниться, в факте самого сомнения. Поэтому это и есть основание самости. Все,
% на что указывает мне естественный свет, естественный свет – это разум, да,
% разум, вот так они тогда называли, естественный свет. Есть сверхъестественный
% свет – это откровение, ну, Библия. А есть естественный свет. Еще раз. Все, на
% что мне указывает естественный свет, никоим образом не может быть сомнительным,
% поскольку из самого факта моего сомнения вытекает, что я существую. Когита
% эргосум. Я существую, потому что сомневаюсь. Мысли, следовательно, существуют.
% Ну, все вот это, то, что вы знаете, известное, оно основано на этом понимании
% самости, как сомнения. А вот то, что уже никоим образом не, как сказать, то есть
% вы во всем сомневаетесь, все, отбрасываете, отбрасываете, отбрасываете, выходит
% на что-то, что несомненно. Принцип непосредственной достоверности. Он ставится,
% Декарта, на первое место в познании. Это принцип интеллектуальной интуиции. А в
% чем мы можем, ну, то есть в чем мы не можем по Декарту сомневаться, это в первую
% очередь в математических истинах. Ну, то есть, например, то, что 2 плюс 2 равно
% 4, мы сомневаться не можем. То, что 2, это вот такое число, которое включает
% именно вот, ну, это количество, и не может быть то 2, например, то 2 яблока, то
% 3 яблока, то 5 яблока. Для нас это настолько очевидная истина, что дальше у нее
% уже никуда нельзя двигаться. Понятно? Так вот, для Декарта крайне важно
% философское исследование вот этого интуитивно-очевидного, несомненного. И при
% этом все-таки откуда мы берем-то вот эти несомненные истины? Вот эти вот идеи
% разума, они так еще называются, обязательные, базовые идеи разума, на которых
% потом выстраивается все остальное, все здание. Мы их берем только из того, что
% Бог не обманывает свое создание в моментах абсолютной достоверности. Вот так
% вот. То есть у Декарта все-таки гарантию человеческого познания обеспечивает
% Бог. Вернее, Бог есть гарантия человеческого познания, причем не просто абы
% какой, а именно добрый Бог. Ну, например, если Бог такой игривый, который любит
% подшутить даже, я же не говорю, обманывать, даже подшутить, например, возьмем,
% то все рассыпается, вся концепция Декарта рассыпается. Тогда Бог скажет, ну
% смотри, интуитивно очевидно и какую-нибудь там ерунду на аккаун. Да? Поняли? Еще
% раз. Мыслью следует, не существую. Интуитивно очевидно это несомненное, а
% несомненное обеспечивается всемогущим, благим Богом. Это называется, вот это
% несомненно, интеллектуальная интуиция. Это прямое непосредственное постижение
% сути дела. Математическое знание. Что за вот эти вот врожденные идеи, которые
% обеспечены нам Богом и которые есть интеллектуальная интуиция? Еще раз.
% Например, идея числа, идея формы, ну фигуры и аксиома равенства. Вот это базовые
% вещи, но там есть еще ряд. Ну их хотя бы назовите, будет нормально. Идея числа,
% идея формы и аксиома равенства. Ну у меня тут достаточно большие цитаты, они
% потом, по-моему, я их даю. Сейчас посмотрю. Да, вот, ну, как бы некоторые
% цитаты, вы, пожалуйста, их просто сами потом самостоятельно прочитайте. Хорошо?
% Извините. Так вот, благодаря свету разума человек может сделать природу своим
% предметом. У него есть эти интуитивные идеи, которые ему позволяют, то есть это
% инструменты, которые позволяют человеку и только человеку быть хозяином
% вкладовой природы. Сделать природу своим предметом. А слово предмет означает то,
% что напротив, воспринимаемое. То есть человек как бы отделился от природы, вот
% есть человек, а есть природа. И вот природа, она несколько так отдельно. Природа
% --- объект, человек --- субъект. И в гносеологии Декарт формирует направление
% рационализма. Рационализма. То есть наше познание основано на идеях разума, вот
% этих вот несомненных идеях разума. На том, в чем сомневаться вообще уже не
% приходится. Дальше. Рационализм. Соответственно, смотрите, вот есть мыслящая
% субстанция, это человек со своими врожденными идеями, со своим рационализмом, со
% своим разумом, да, вот это все. И есть природа протяженная. Имеется в виду, это
% то, что не разум, не человек, ну, то есть все остальное. Философия называется
% дуализм субстанции или субстанциональный дуализм. Не погружаясь в эту тему, она
% довольно интересная, она тоже несет много всяких последствий. Если что, на
% семинарах, пожалуйста, проговорите этот момент. Проговорите и момент его метода.
% Я его только обозначаю здесь, он достаточно простой, понятный на всяком случае
% даже вот в письменном виде, потом я выложу презентацию. Вот. Несколько особняком
% стоит его физика. То есть, да, конечно, физика, это физика механицистическая, а
% мы про механицизм скажем чуть дальше, да, чуть позднее. но тем не менее, как бы,
% она такая, как сказать-то, еще вчерашнего дня. Вот все-таки, как ни крути.
% Вадикавта Вселенной это вихри в вихрях. Ну, догадайтесь, вихри где происходят?
% Конечно же, в пневме, тонкой материи, в, как сказать, в эфире. Вот. Ну,
% знакомая, да, история. Но, все-таки, Декарт рассматривает их механически, эти
% вихри. Он не говорит о пневме, которая проникает там в душу и тому подобное.
% Наоборот, он уделяет, вот есть душа, а есть материя. И путать их больше не
% будем, нельзя. То есть, он противостоит, как бы, вот этому, вот этому магизму,
% как может, разделяет душу и материю на две части. Но, опять же, у этого есть
% последствия. Но, во всяком случае, понятно, почему он это делает. Это, как бы,
% такая идея, но соединение от прежних умозрительных вот этих конструкций. Так
% вот, вот этот эфир, он уже становится просто, как бы, субстанцией, в которой
% заворачиваются разного рода вихри. Эти вихри и есть водоворот такого эфирного
% моря, это и есть движение небесных тел. Они захватывают небесные тела. Поэтому
% небесные тела двигаются. Ось вращения походит через Солнце, поэтому всё-таки всё
% крутится вокруг Солнца. А каждая планета ещё крутится вокруг себя, потому что
% вот эти вихри эфирного моря, её, как бы, всё, обеспечивают всё это движение.
% спутники. Спутники двигаются благодаря меньшим вихриям, и они, то есть, окружают
% тоже каждую планету. А все тела падают на Землю, потому что подталкиваются в
% ходе вот этого движения вихревого мельчайшими невидимыми частицами вихрей,
% флюидами. Ну, то есть, чувствуете, там какая-то смесь аристотельской физики и
% какими-то вот такими новых идей. Вот, как бы, такая философия и физика Декарта.


% Мы остановились на том, что Декарт ставит саму проблему активность познающего
% разума.  активность познающего разума
% сильнее всего. 

Обратимся к Декарту, который вводит категорию субъекта как обозначающего именно человека.

Раньше, до Нового времени, субъектом называли просто подлежащее предложения, это могло быть что угодно: стол --- субъект деревянности, космос --- субъект предложения. Декарт же переосмыслил субъект, сделав его только человеком. Он происходил из обедневшего дворянского рода, недолго учился в университете Пуатье, служил и участвовал в битве при Хакин. На войне у него формируются предпосылки механицистского взгляда на человека.

Понятие механицизма пока без объяснений, это ключевое слово эпохи. Человек как механизм, подобный кукле, оживляемой духом или Богом. Много видевший и препарировавший трупы, Декарт считал себя мастером препарирования, проникся этой спецификой и развил идею, что человек --- живая машина. Он писал работы, обращался к своим учителям --- иезуитам, но они сочли его идеи опасными. Из-за страха преследования он переселился в протестантскую Голландию, где написал «Рассуждение о методе» --- книгу о правильном управлении разумом для поиска истины в науках.

К «Рассуждению» Декарт приложил исследования по натурфилософии, оптике, геометрии и алгебре --- то, что ныне связывают с его именем: декартовы координаты и геометризация алгебры. Его труды вызвали восхищение кардинала Ришелье, который предложил издать их во Франции, но Декарт никому там не доверял и по приглашению шведской королевы переехал в Швецию, где вскоре умер, возможно, отравлен.

Декарт всю жизнь искал свободу мысли, считая, что только свободный человек может быть самостоятельным --- способным сказать «я сам». Самость --- главный философский принцип Декарта, объединяющий всех людей и являющийся основанием свободы. Философия требует не просто борьбы за свободу, а объяснения её смысла --- это доброкачественная мысль.

Основание свободы --- самость, проявляющаяся через сомнение. Сомневаться можно во всём: в окружающем мире, в собственном уме, в существовании вообще. Можно сомневаться, не существует ли всё это во сне или в безумии. Но нельзя сомневаться в факте самого сомнения. Это и есть основание самости. Естественный свет разума, по Декарту, не допускает сомнений, а из факта сомнения следует существование мыслящего --- «Cogito, ergo sum». Мыслящее существо существует, мысли существуют.

Принцип непосредственной достоверности, или интеллектуальная интуиция, ставится Декартом в основу познания. В сомнении невозможны математические истины --- например, 2+2=4. Эти несомненные истины, врождённые идеи разума (число, форма, аксиома равенства и др.), даны Богом, который не обманывает. Бог для Декарта --- гарантия достоверности познания, добрый и всемогущий.

Таким образом, человек благодаря разуму и врождённым идеям становится хозяином природы --- отделяет себя как субъект от объекта природы. В гносеологии Декарт формирует рационализм: познание основано на несомненных идеях разума. Человек --- мыслящая субстанция с врождёнными идеями, природа --- протяжённая субстанция без разума. Это субстанциональный дуализм.

В физике Декарта --- механицистский взгляд: вселенная --- вихри в эфире, которые движут небесные тела. Эфир --- субстанция, в которой вихри создают движение планет и спутников. Тела падают на Землю, подталкиваемые мельчайшими частицами этих вихрей. Здесь смешиваются аристотельские представления и новые идеи.

% \subsubsection{Фрэнсис Бэкон}
% Все-таки принцип полезности науки, он выдвигает на
% первое место. Но он действительно заботится больше всего о государстве, о людях.


% И принцип полезности, да, что все действие наиболее
% полезно, то есть знание наиболее истинное. И вот как раз принцип утилитаризма в
% таком философском воплощении, ну, Баркон Дуз тоже, что называется, очень многое
% сделал для этого. но как бы максимально полезно он предлагает работать в
% качестве, ну, ученому, в качестве экспериментатора. 

% Он говорит, что
% действительно мы будем, то есть как можно быстрее и эффективнее выяснить тайны
% природы можно с помощью пыток. Для него эксперименты это пыточная природа. Опять
% же, ну, лорд-канцлер Бекон понимает, что говорит, да, и вот представьте себе
% человек, ну, конечно, представлять не очень предлагать такое, но тем не менее,
% представьте себе пытки. как можно быстрее выпытать у испытуемого какую-то
% утаиваемую информацию. То есть нужно правильно ставить вопросы, да, и так, чтобы
% ответ был только да или нет. Вы же понимаете, да, так меньше всего возможность
% обмана там или еще чего-то. 

% Поэтому вот он и называет природа есть, эксперимент
% это испанский сапожок, надо, значит, заточить в него природу, чтобы она выдала
% нам все свои тайны. Ну, такая разновидность пыточных. Но надо сказать при этом,
% что вот это распространенное мнение, что именно он предлагает
% экспериментировать, экспериментировать, еще раз экспериментировать, нет, это не
% его позиция. 

% Его позиция не экспериментализм, а эмпиризм. Я объясню сейчас,
% погрузимся чуть полнее. Значит, он создает свой проект, он государственный
% деятель, создает свой проект великого восстановления наук на основе союза опыта
% и рассудка. Это произведение новой Атлантиды. Она издана посмертно. Да, там
% очень много всего того, что вы можете встретить раньше других, более ранних
% этапах развития науки. Там описываются идеальные государства, орден Соломонова
% храма. Да, для него наука все-таки это немножечко религия. А может, и не
% немножечко. Целью храма является овладение силами природы. И у него, конечно,
% есть вот эти мотивы герметических установок. Маг, там, тот, кто подчиняется
% материи. У него магия, кстати, входит в нормальное, ну, как бы, вот это вот, в
% науке, нормальный набор наук. То есть, да, видите, я к чему сейчас сделаю
% ремарку, почему мы не можем взять и сказать, вот, все, в 17 веке начинается
% новый тип науки. А куда Бека наденем? Куда его новую Атлантиду денем? Там магия
% еще выше крыши. Куда мы Декарта денемся его вихрями в эфире, эфирными вихрями и
% так далее. Ну, нельзя такие границы провести. Это, как бы, слишком схематично
% будет. Поэтому мы выстраиваем перед вами историческую картину науки, развития
% науки вот так вот последовательно. А в реальности ничего нет вот такого, знаете,
% ну, схематичного. Да? Поэтому обращаю еще раз внимание на вот такие вот как бы
% двусмысленности в работах, которые, тем не менее, не мешают науке
% последовательно осуществлять новый тип рациональности. Продолжаю. Про Бекона.
% Так вот, в этом ордене храма, в ордене великого храма, Соломонова храма, есть
% такие товарищи, как наблюдатели. Особая группа ученых. Они добывают по всему
% свету тайное знание. Они коммерсанты света, так называемые. Вот. Но эти
% коммерсанты света, им предлагается развивать новый метод. В противовес
% умозрительной философии, говорит он, нужно искать логику изобретений. Нужно
% открывать с помощью индукции. то есть умозаключение от частных единичных случаев
% к общему, от отдельных фактов к обобщениям. Вы понимаете, что это на самом деле
% очень по-новому трактует разного рода даже прежние знания. Попробуйте собрать
% индуктивно теорию пневмы. Попробуйте собрать индуктивно теорию пневмы. не
% соберете. Но нет таких частных фактов, на основе которых вы выстроите всю вот
% эту концепцию. То есть по сути дела вот этим подходом индуктивным он разрушает
% умозрительность прежних концепций. Еще раз подчеркну этот важный момент. Когда
% он предлагает начать собирать факты природы, их анализировать, выстраивать
% таблички, в которых писать их сходства и различия, все стойки, вы становитесь
% эмпириком. Тогда как тот же, например, Аристотель или Фичинос Бруно или даже
% Схоласты, которые, конечно, наук природы не занимались. Я сейчас говорю про те,
% кто природой занимался. Ну, в смысле, не занимались, но меньше степени. Вот. Они
% не начинали от элементов. Хотя, конечно, у Аристотеля в его биологии очень много
% эмпирических факторов. Но он все-таки, его физика, она не эмпирическая.
% Умозрительные, да? Мы создаем умозрительные концепции, которые как бы все
% объясняют. Тогда как Бекон предлагает индуктивный метод, а это означает, обращая
% внимание на частные, конкретные моменты. Мы сравниваем, говорит он, данные о
% схожих предметах, предлагает составлять таблицы. Мы находим общие черты в них, в
% результате получаем обобщение, знание обобщенного характера. Конечно, Бекон
% знает, что у метода индукции есть пределы применимости. Ну, условно говоря, мы
% рассмотрели первого лебедя, второго лебедя, третьего, двадцать первого и так
% далее. Мы сказали, что все лебеди белые. И понятно, что у этого индуктивного
% обобщения есть уязвимое место ровно до тех пор, пока мы не встретим черного
% лебедя. Эта теория будет работой. Но тем не менее, метод индукции все-таки
% приводит к общим выводам, которые могут быть равновероятными. Ну, например,
% лебеди как белые, так и черные. Ну, сейчас пример не очень подходящий, но тем не
% менее, равновероятные выводы. Так вот, чтобы проверить равновероятные выборы, он
% предлагает ставить эксперимент. Понятно, да? Когда у него экспериментализм-то
% включается? Чтобы проверить равновероятные выборы. Так вот, это позволяет по
% Бекону создать знания закона образного характера. 

% И он таки установил некоторые
% законнообразные знания. мы не назовем это закон но тем не менее. Сравниваем
% многие ситуации. Бекон установил, что теплота связана с движением частиц.
% Понимаете? это вот как раз такой эмпирическая закономерность. Он просто наблюдал
% некоторые моменты и выводит эмпирическую закономерность. 

% Социологические опросы,
% например, они тоже, они же не приводят нас к законам природы социальной, законам
% социальной природы. Они говорят нам об эмпирических закономерностях в социуме.
% Ну вот, таким образом. Бекон не особенно участвует в споре о том, как же все-
% таки мы познаем, благодаря чему? Благодаря опыту или знанию. То есть вот в этом
% кнасологическом споре он не участвует, но тем не менее на его работах
% основываются многие исследователи сенсуалисты. 

Бэкон ставит на первое место принцип полезности науки, которая должна служить государству и людям.
Он считает, что истинное знание --- это самое полезное знание. В этом проявляется утилитаризм в философском виде. Бэкон предлагает учёному работать как экспериментатору, чтобы максимально эффективно раскрывать тайны природы.

Он сравнивает эксперимент с пытками --- нужно быстро и точно получить информацию, задавая вопросы, на которые можно ответить только «да» или «нет», чтобы исключить обман. Эксперимент --- это «испанский сапожок», в который природу «загоняют», чтобы она раскрыла свои тайны.
Однако Бэкон не сторонник бесконечного экспериментализма, а эмпиризма. Он, как государственный деятель, создал проект «Великого восстановления наук», основанный на союзе опыта и рассудка. В своей посмертно изданной «Новой Атлантиде» он описывает идеальное государство с орденом Соломонова храма, где наука близка к религии и связана с магией, которая входит в набор нормальных наук того времени.

В ордене Соломонова храма есть «наблюдатели» --- учёные, добывающие тайное знание по всему миру. Им предлагается развивать новый метод: вместо умозрительной философии --- индукция, то есть переход от частных фактов к общим выводам. Этот подход разрушает прежнюю умозрительность и требует собирать факты, анализировать их и составлять таблицы сходств и различий. Бэкон признаёт, что индукция имеет пределы: обобщение, например, о белых лебедях верно до тех пор, пока не найдётся чёрный лебедь.
Чтобы проверить такие вероятные выводы, он предлагает проводить эксперименты. Эксперимент становится инструментом для проверки обобщений и создания закономерных знаний. Бэкон установил эмпирическую закономерность: теплота связана с движением частиц. Это наблюдение, а не закон, но важное эмпирическое знание.


\subsubsection{Дж. Локк}
% Обратимся к сенсуализму. То есть ставится вопрос как человек познает? Что
% первично? Опыт или интеллект? Разум. Сенсуализм чувствуете от слова чувство
% говорит о том, что все-таки первичны именно данные чувств слуха, обоняние и тому
% подобное. Не важно. 

% То есть основание надо искать в чувственном опыте. Сознание,
% говорит Джон Лок, это чистая доска, на которой опыт пишет свои письмена. Ну, то
% есть чистая доска имеется в виду эти восковые вот эти как бы вовращенные
% таблички, на которых сознание оттиски оставляет. Да, дорогие мои, это все еще
% отголоски теории пневмы. Помните, там пневма как бы в оптический центр попадает,
% глаз, да, по оптическому нерву, и оттиски оставляет на внутренней пневме эти
% оттиски фантазмы, которые вот и есть знания. Вот у него примерно что-то похожее,
% только он использует это немножечко уже более метафорически, как бы саму пневму
% уже не упоминает, но тем не менее. И возникает спор с Декартом. Они, кстати, ну,
% как бы спорят, ну, как сказать, спор заочный, да, когда публикуешь, когда ты
% публикуешь свой рот, и говоришь, а вот Декарт не прав, а он говорит, а вот там.
% То есть Лог согласен с Декартом, что самость, субъектность, самостоятельность
% это общее для всех людей, когда они в сознании, подчеркивают. Когда они в
% сознании, у каждого есть самость, самостоятельность, субъектность. Но все эти
% субъекты живут в разных условиях, и будучи одинаковыми, как бы, ну, нулевыми от
% рождения, они наполнены разными влияниями, разными оттисками, да, и,
% соответственно, они становятся разными людьми. По-разному видят мир, по-разному
% познают. И эту, как бы, разность надо ценить, потому что это и есть влияние
% общества. Отсюда все идеи о том, что общество выстраивает личность. Чувствуете,
% как много закладывалось эпоху нового времени? Рационализм Декарта говорит о том,
% что есть некие врожденные идеи, которые позволяют нам самость иметь с самого
% начала. Это, как бы, категория личности, которая с вами, ее никто не
% выстраивает, она своя. А другая позиция, сенсуализм, говорит, нет, от общества
% зависит, что там будет с вашей личностью. Потому что разум чистая доска, нет
% никаких врожденных идей. Понимаете? То есть, это вроде бы спор о том, как
% человек познает, но в итоге этот спор формирует множество вот серьезных как бы
% разветвлений в других областях знания, в других областях мировоззрения. И
% постепенно все большую значимость приобретает принцип различия. Вы наверняка
% слышали, что новое время связано с культом индивидуализма. То есть, вот этот
% индивидуализм. Но это и есть философское обоснование свободы. Нужно иметь
% самость как некую самостоятельность. И различие здесь оказывается ключевым
% моментом. То есть, философ акцентирует не то, в чем мы похожи, а то, в чем мы
% различаемся. Но, конечно, под это должна быть подведена онтологическая база.

Обратимся к сенсуализму. Вопрос в том, как человек познаёт --- что первично: опыт или разум? Сенсуализм, от слова «чувство», утверждает, что основой служат именно данные чувств --- слуха, обоняния и прочего. Основание познания --- чувственный опыт. Джон Локк говорит, что сознание --- чистая доска, на которой опыт оставляет свои письмена. Это метафора, восходящая к теории пневмы, где пневма, проходя через глаз и оптический нерв, оставляет отпечатки-знания. У Локка идея похожая, но пневму он уже не упоминает явно.

Возникает спор с Декартом, который был в основном заочным: каждый доказывал свою позицию через публикации. Локк соглашается с Декартом, что у всех есть самость, субъектность, но эти субъекты рождаются одинаковыми и затем наполняются разными впечатлениями и влияниями, из-за чего становятся разными людьми, по-разному познают мир. Это отражает влияние общества на личность --- именно оно формирует человека. В эпоху нового времени рационализм Декарта утверждал, что существуют врождённые идеи, дающие самость изначально --- это личность как нечто данное. Сенсуализм же говорит: разум --- чистая доска, никаких врождённых идей нет, и личность формируется обществом.

Хотя спор казался о способах познания, он породил важные разветвления в мировоззрении и знаниях. Становится всё важнее принцип различия. Новое время связано с культом индивидуализма --- философское обоснование свободы, когда самость --- это самостоятельность. Ключевой акцент не на том, в чём мы похожи, а в чём различаемся. При этом нужна соответствующая онтологическая база.

\subsubsection{Лейбниц}

% Это мыслитель, которого я, знаете, про него лучше вообще не говорить,
% чем говорить мало. Великолепный, конечно, представитель философии нового
% времени. И вообще, он универсальнейший мыслитель нового времени. Он намного
% опередил свое время во многих областях знания, поэтому он не всеми-то и признан.
% По сферам деятельности он юрист, дипломат, тайный советник юстиции в России,
% кстати, наемный алхимик в области общества разнокрейцеров, придворный историк и
% так далее. Чем только он не позанимался. Список областей в науке, которые мы
% получили в развитии благодаря Лебницу, вообще неохватный. Но я подчеркну одну
% важную идею, связанную с категориями тождества и различия. А именно ноль и
% единица. Метафизические аллюзии чисел нуля и единицы. Надо сказать, что он
% самостоятельно развивал теорию универсального исчисления, согласно которой можно
% было бы построить машину, которая могла бы мыслить. Ну как мыслить? Она могла бы
% помогать людям выбирать правильные идеи. Ну то есть, например, у нас с вами
% спор, мы подошли к этой машине, чик-чик-чик, она нам высчитала, кто прав. И мы
% все согласились. Ну типа, ну это же машина, нам не надо дальше спорить. То есть
% чувствуете, наше различие важно, но надо все-таки находить форму единства, и он
% предлагает создать такую машину. Понимаете, да? Про образ чего это было. Но она
% у него не получилась, она, как бы, осталась незавершенным проектом, но в ходе
% работы над ней он выходит на принцип нуля и единицы, то есть двоичного
% исчисления. Он за него очень радует, говорит, что это очень удобное вычисление,
% предлагает его всем домам, там пишет письма, в смысле королевским домам, пишет
% письма ученым, говорит, давайте пользоваться нулемой единицей. Но он не хочет
% применить ее для этой машины для мысли, то есть пока вот этого соединения не
% произошло, ноль единицы и вот эта машина для мысли. Но это не арифмометр, это
% как бы проект уже именно думающей машины. но тем не менее Лебниц как-то на этом
% нуле и единицы привлекает его очень серьезное внимание. Почему? Потому что в
% этот период, тесно общаясь с азуитами, он обнаруживает в их письмах, там разного
% рода публикациях, он обнаруживает учение и дзин китайское. Но они же в этот
% период активно ездят по миру, активно там участвуют в миссионерской
% деятельности, в образовательной деятельности. И вот доносят до Европы разного
% рода сведения о других культурах. И, кстати, Лебниц очень ценил китайскую
% культуру, считал, что она самая продвинутая. И считал, что Россия, почему он так
% интересовался Россией, участвовал активно в ее жизни, научной жизни, он считал,
% что Россия это будет главный мостик между вот этой великой китайской культурой и
% европейской. Он как бы предлагал такие политические проекты и российским
% правителям, и китайским, и так далее. Вот такая у него была идея интересная. И
% плюс ко всему вот это вот ноль и единица получает у него метафизическое
% значение, потому что Идзин, это книга перемен китайская, она в определенном
% смысле, ну, тоже может быть интерпретирована в этой логике двузначной, да,
% почему? Потому что, ну, если вы когда-нибудь посмотрели, что это такое, там
% такое гадание, которое, ну, типа, король, в смысле, как сказать, орел-решка, ну,
% теснули единица, или, то есть там вот эти парные две, как бы, два элемента, как
% бы, противоположные, они играют ключевую роль. И он им предлагает дальше
% пользоваться этими элементами, тем более, что это очень хорошо сочетается с его
% метафизикой. Он, кстати, мечтал, что вот ему медаль дадут, когда он уйдет, и на
% ней будет надпись «Все может быть выведено из ничего, все, что нужно, это
% единица». Лебедь пришел к выводу, что лучший из возможных миров, наверное, вы
% слышали, да, идеи лучшего из возможных миров, то есть, ну, миров может быть
% очень много, Бог может состоит любой мир, может быть состоит миллиарды миров,
% для него это, что называется, не проблема. Вот. Но мы живем в лучшем из
% возможных. Это мир, в котором огромное разнообразие явлений следует из
% минимально возможного числа предпосылок. А из нуля единицы можно вывести
% бесконечное количество других чисел, умозаключений и так далее. Поэтому его так
% порадовала китайская книга перемен, которая, по сути дела, любое состояние
% опишет вот из этих двух элементов, парный пример. В метафизическом смысле у
% лебницы нуль это и есть тождество. Ну, смотрите, все сливается и превращается в
% ноль. Это полная конкурентность. Мы, условно говоря, тождественные, и это
% делает, ну, как сказать, если мы абсолютно тождественные, то мы неразличимы,
% правильно? То есть мы есть нуль. Но Бог, говорит лебец, когда взывает к бытию,
% когда говорит кому-то, кому-то, будь, то он осуществляет различие. То есть
% чувствуете, он ноль как бы перестает быть нулём, появляется единица. Всё, что
% существует, говорит лебец, существует лишь благодаря различию, неконкурентности.
% Поэтому все люди видят мир как бы из центра мироздания. Ну, наверняка, вы это и
% по себе замечали. Складывается устойчивое ощущение в ходе жизни, что, ну, я в
% центре мира. Не в смысле, я там главный или какая-то ещё, ну, просто я всё вижу
% вот из своей позиции, а все остальные это как бы периферия, да? Ну, как бы это
% вот, они наверняка, ну, ну, как сказать-то, ну, понятно же, что я наблюдаю мир,
% и вот я как бы из него, я в центре этого мироздания. И только, ну, будучи уже
% взрослыми, мы понимаем, что ведь точно так же и другой человек воспринимает, он
% центр мироздания. Ну, условно говоря, давайте современную лексику применю. Я как
% бы реальный игрок, да? А остальные кто, получается, эти самые, да? Списываю
% слово. Подскажите мне, не игровые персонажи, как? NPC. NPC, да. Все остальные
% NPC. Я как бы, ну, я-то реальный, да, остальные NPC. Вот примерно так же, ну, я
% огрубляю очень, преувеличиваю. То есть, каждый так думает. И Ленин в этот момент
% очень уловил и говорит, что душа Исмонада, такая вот единственность, которая
% наблюдает мир, отражает его. и вся совокупность знаний, это, по сути дела,
% синтез знаний разных монад. То есть, все действительно в центре, действительно
% все в центре мира. То есть, вот столько, сколько нас, душ, столько и монад,
% столько, простите, и центров мира. И получается, что мы все смотрим мир, как бы,
% из своей позиции, да? Из своей позиции. Нам видно, несколько иначе мир, чем вот
% другим. И в этом наша ценность говорит о. Монада есть перцепцию, аппетитус и
% виз, то есть, восприятие, стремление и сила. Вот это и есть устройство
% мироздания. А чтобы сочетать все эти взгляды из разного, это означает найти
% возможность из миров, возможность этого сочетания. И вот наш мир, это лучший из
% таких вариантов. Понимаете? Бытие означает различие, но ведь это означает основу
% для противоречия. Мы же по-разному видим мир. Значит, мы в чем-то будем
% противоречить друг другу. Значит, в чем-то будет страдание, потому что
% противоречие могут дорастать до страданий, до зла. И Ленинс говорит о том, что
% надо мужественно принять это обстоятельство, ну то есть то, что в мире есть зло,
% потому что зло – это плата за различие. Это плата за вот это неизбежное смещение
% смысла, которое возникает благодаря тому, что мы каждый в своем центре, мы
% каждый не NPC. А Бог – это трансцендентный субъект, его творение, поэтому
% наилучший из миров. Не потому, что в нем все довольны, ничуть не бывало, не
% потому, что в нем нет зла, он говорит, это нелепая фантазия. А потому, что все
% остальные формы единства различного, единства различающегося, невозможны. Ну,
% или они как бы, они возможны для Бога, но невозможны для того, чтобы это было
% наше, нашим существованием. Бог же сам есть бесконечное различие, бесконечное
% разнообразие, бесконечное богатство. Поэтому никогда не было изначального
% тождества, никогда не было изначального нуля, а бытие и сразу же взрыв
% разнообразия. Ну, по-моему, очень красивая философия, я не говорю, что там одна
% единственная история, но она красива, как мысль. Вот, но скажем честно, все-таки
% тому времени такие мысли были еще пока не созвучны, он как бы вперед смотрел.
% Для научного познания все-таки 17-18 веков, куда было важнее найти сходство, то
% есть тождество явлений в качестве законов природы. Но, смотрите, все равно, я
% хочу подчеркнуть, философское мышление уходило в этот момент в разрыв с научным.
% Как бы наука, философия разделялись, эмансипировались друг от друга. Вы же
% понимаете, что это сыграло науке огромную пользу. Если бы науке до сих пор
% занимались философы, ну, мы не знали бы науку в той форме, которую мы знаем
% сейчас. Хорошо или плохо, мы оценок не ставим, но тем не менее, определенно
% нужно было, чтобы когда-то наука разошлась с философией. И вот в таких вот
% идеях, да, в идеях там Леменца, несмотря на то, что они много сделали для науки,
% но все-таки философия эмансипировалась от естественно-научных исследований. А
% сама философия тем самым продолжает исследовать разум, анализирует, как отражает
% разум реальность. Вот. И постепенно выходит к мысли, которые сегодняшние ученые
% в тот дом разделяют и даже понимают. Речь идет о том, что реальность все-таки
% разумом конструируется, а не просто отражается. Да, не просто пневма, там, глаза
% попадают или еще что-то там, какой-то образ. А все-таки мы конструируем
% реальность. 

Особенно важна его идея, связанная с категориями тождества и различия --- ноль и единица, метафизические аллюзии этих чисел. Он развивал теорию универсального исчисления, предполагая создание машины, которая могла бы помогать выбирать правильные идеи. Например, при споре машина вычисляла бы, кто прав, и все соглашались бы с результатом --- форма единства через различие.
Проект машины мышления остался незавершённым, но в процессе он сформулировал принцип двоичного исчисления --- ноль и единица. Он радовался этому удобному способу вычислений и предлагал его использовать королевским домам и учёным. Однако пока не удалось соединить двоичный код с машиной мышления. Тем не менее, ноль и единица заняли у него важное метафизическое место.


Ноль и единица получили у него метафизическое значение. Ицзин, основанный на двоичных противоположностях, служил моделью для понимания мира. Лейбниц мечтал о медали с надписью: «Все может быть выведено из ничего, всё --- из единицы». Он пришёл к идее «лучшего из возможных миров» --- мира, в котором огромное разнообразие явлений исходит из минимального числа предпосылок. Из нуля и единицы можно вывести бесконечное множество чисел и умозаключений, что очень совпадало с логикой Ицзина.

В метафизике Лейбница нуль --- это тождество, полное слияние и неразличимость, а единица --- различие, которое делает бытие возможным. Бог, создавая мир, вводит различие, благодаря которому всё существует. Мир видится каждому из центра мироздания, каждый воспринимает себя в центре --- это индивидуальная позиция. Лейбниц замечает, что каждая монада --- это уникальный центр восприятия и познания, и вся совокупность знаний --- синтез разных монад. Таким образом, мир состоит из множества центров, каждый видит его по-своему, и это создаёт ценность. Монада --- это восприятие, стремление и сила. Чтобы сочетать разные взгляды, нужен лучший из возможных миров --- наш мир.

Бытие означает различие, которое порождает противоречия, а следовательно, страдания и зло. Лебниц предлагает принять зло как плату за различие --- неизбежное смещение смысла, вызванное тем, что каждый существует в своём центре. Бог --- трансцендентный субъект, творящий бесконечное разнообразие, поэтому изначального абсолютного тождества не было, а был взрыв разнообразия.

\subsubsection{Гоббс и Юм}
% Он утверждает, что разум сам конструирует причинность. мы не можем утверждать,
% говорит философ, что причины и следствия это свойства самой реальности.
% Реальность может быть устроена как угодно. То есть сегодня, например, после
% дождя земля мокрая, завтра, после дождя земля мокрая, и миллионный день
% происходит все точно так же, земля после дождя мокрая. И мы говорим, что дождь
% здесь причина намокания земли. Но на какой-нибудь день может случиться по-
% другому. Но только Юма это не всемогущество Богу отсылает, а просто к
% определенным, ну как типа, давайте, говорит, будем здраво смотреть на свою
% позицию. Мы не знаем реальность во всей ее полноте. Мы же не будем жить, ни один
% из нас не будет жить там 50 миллионов лет. Скажем, не один из нас, ни один из
% нас миллиона жить не будет. Но имеется в виду, что мы же не узнаем то, что
% будет, вот, ни один из нас не узнаем, то, что будет там какую-нибудь такую эпоху
% или было. Поэтому давайте просто примем, что причинность это наша привычка
% разума. Мы ищем вот эти привычки, ориентируемся на них и тем самым конструируем
% знания. 

Гоббс и Юм утверждают, что разум сам создает причинность. Нельзя считать причины и следствия свойствами самой реальности. Реальность может быть устроена иначе: например, сегодня после дождя земля мокрая, завтра --- тоже, и миллионный день подряд так же, поэтому мы говорим, что дождь --- причина намокания земли. Но однажды может случиться иначе. Юм не отсылает это к всемогуществу Бога, а предлагает трезво взглянуть на нашу позицию: мы не знаем реальность во всей полноте и не проживём миллионы лет, чтобы проверить всё. Мы не узнаем, что было или будет в далёкие эпохи. Поэтому разум просто вырабатывает привычку воспринимать причинность, на основе которой строит знания.

\subsubsection{И. Кант}
% Кант
% очень сложный мыслитель, очень такой, как бы это сказать, красивый, красивый,
% да. Почему? Потому что там прям все соединено, все в одну красивую конструкцию,
% все как бы, знаете, красивая архитектура, она выстроена, идеально, не
% подкопаешься, стоять будет вечно. И все-таки позиция Канта стоит вечно. Как его
% не пытались разобучить, как его не пытались оспорить, патиковать, Кант все-таки
% стоит столь же твердо его здание, его философию, как и в самом начале своего
% появления. Так вот, Кант радикализовал идею субъекта в контексте свободы.
% Субъект, говорит он, это тот, кто не предумышленный, не специально, но все-таки
% сам формирует, сам конструирует свой предмет. Иными словами, действительно,
% только постольку, поскольку разум сам конструирует образ реальности, человек
% свободен. Потому что, если не сам, то тогда на него влияют либо там какие-то, в
% общем, какие-то воздействия он испытывает, и, соответственно, ну какая же
% свобода? Где же свобода личности, если он не сам, не самостоятель? Вот
% интересный момент, То есть чувствуете, это все-таки заслоны все от принципа
% магии. Если мы возьмем позицию Канта, то никакая магия вообще невозможна.
% Вообще, принцип. Как бы его не полюбили, товарищи, возрожденцы. Смотрите дальше,
% что он рассуждает. Почему так важен принцип свободы? Наука, а он сам был ученым,
% у него был, так называемый, докритический период, когда он был астроном, ну то
% есть создавал физические, астрономические концепции. И они до сих пор некоторые
% работают. Поинтересуйтесь этим моментом специально, я не буду покружаться, меня
% он интересует как философ в данном случае. Вот. И Кант, будучи ученым, прекрасно
% знал, что наука существенно продвинулась в понимании таких вот механицистических
% процессов. Механика же это представление о линейной причинности, ну условно
% говоря, А толкает Б, Б толкает С, С толкает Д, все подчинено вот этой вот
% линейной причинности. Причем одна и та же причина всегда порождает одно и то же
% следствие. Наука того периода, ну Ньютонская, чуть позднее скажем об этом, она
% ведь как сказала, если у частицы та же самая масса, она толкает, не толкает, она
% движется с тем же самым импульсом, то это в принципе даже неразличимые частицы,
% да ведь? То есть они, все, масса импульсов падает, мы даже не будем говорить там
% о том, что, ну там, какие-то цветы, например, различаются у частицы. Вот.
% Действия и противодействия, вот что мы должны изучать. А Канн задает вопрос, а
% как быть тогда с событиями морального характера? Если мы говорим о выборе и
% ответственности, о свободе, да, потому что свободен только тот человек, кто на
% самом деле может позволить себе, ну, например, даже пожертвовать собой. Это его
% выбор. А если он этого не может делать, почему? Ну, например, потому что боится
% смерти, значит, он не свободен. Или, например, он дал милосты на него, потому
% что, например, у него такая жалостливая натура. Ну вот он очень эмоционален, но
% он плачет от любого, он увидит что-нибудь в несчастье, как он человек, плачет
% он, да, вот такое. И он просто не может быть. У него последняя копеечка, вот она
% последняя, но он настолько жалостлив, настолько у него вот это вот чувство, как
% сказать, сострадания развито, что он это последнюю копеечку отдаст. Так вот, по
% канту, это не свободный поступок. Это поступок под действием причин и следствия.
% Вот эта вот жалостливость стала причиной его действия. Значит, это не свободный
% поступок. Его можно оценивать как угодно хорошо, но он плох тем, что он не
% свободен. Тогда как по-настоящему свободный поступок, это когда я поступаю,
% например, ну, даю милость мне, не из жалости, да, ну, то есть я как бы понимаю,
% что да, действительно, ну, то есть как сказать, меня не толкает собственная
% жалостливость. Это мой выбор свободный. Я могу дать, могу не дать. И понимаете,
% да, я могу, там, ни под каким дурманом я не нахожусь и жертвую собой не потому,
% что у меня какой-то там, не знаю, какое-то воздействие на меня оказали либо
% идеологические, либо там, не знаю, там, магические, как угодно. А я жертвую свою
% жизнь, например, будучи военным, да, потому что я действительно выбираю защитить
% вот тех вот людей, которые за моими спинами. Это мой свободный выбор. Вот
% конкретно очень. И поэтому принцип свободы и принцип вот этого вот
% самостоятельности решения для Канта очень важен. Но как быть, наука говорит, все
% подчиняется причинам и следствиям. Все. Как быть со свободой? Ведь свобода это
% риск принять то иное решение. И разум должен быть сам себе законодателем. Это и
% есть совершеннолетие мюндишкает разума, согласно Канту. Причем свободны все.
% Никто не может восприниматься как средство для достижения цели, ведь каждый сам
% реализует свой выбор. Это вот этика Канта основана на категорическом императиве.
% Категорический императив. Давайте я оставлю эту тему у нас вопрос не по этике,
% но вы, пожалуйста, запомните, категорический императив это крайне важно. Я иду
% дальше. Наука не видит возможности для свободы. Казалось бы, ну, давайте
% откажемся от науки, чтобы сохранить концептуальную возможность свободы. но Кант,
% в чём его действительно открытие, он идёт другим путём, критическим, то есть
% путём внимания к нашему разуму. Он соглашается с тем, что причина это наша
% способность видеть реальность так, а не иначе. Реальность, как она нам является,
% и он вводит понятие феномен. Реальность, как она нам является, феномен. И вот
% феномен это то, что, ну, мир для нас. Но мы должны допустить, что реальность
% сама по себе, для себя, в себе, ноумен, и она может быть совершенно иной. Ну да,
% то есть мы видим, что всё подчиняется причинам и следствиям, всё разворачивается
% во времени, а время и есть категория причин и следствия, если так думаться, что
% вот, ну, вот реальность такая, какой мы и видим, она, это феномен, это просто
% то, как нам является, а реальность сама по себе может быть, например, вне
% категории времени, то есть всё есть сразу, всегда и везде. Она может не
% подчиняться, тогда она и не будет подчиняться категории причин и следствия.
% Вспоминайте этот самый Нолоновский «Интерстелл», да, помните, как там он
% изобразил, ну, они там изобразили, вся команда изобразили, вот это состояние
% пятого измерения, когда всё сразу есть, да, там же, чувствуете, категория причин
% и следствия, всё меняется местами, всё становится, ну, как бы уже не
% выстраиваться в линеечку, одно следует за другие и так далее. Невозможно, что
% времени-то нет, что будет раньше, что дальше. Ведь причина – это то, что раньше
% следствия. А если категория раньше, позже не существует, вот, мы должны
% допустить, что мир, реальность сама по себе не та, которую мы воспринимаем.
% Познать Ноломен нельзя, говорит Кант. Вот такой запрет. Вещь в себе
% непознаваема. Ноломен – это вещь в себе. Вещь в себе непознаваема. Это допущение
% Ноломен, которое позволяет не видеть реальность глазами только науки. Понимаете?
% Сейчас я посмотрю у нас в данном случае этот статус твоего дела. Да. Есть тут
% цитаты кантовские. Достаточно много. Понятие Ноломена взято чисто в
% проблематическом значении, остается не только допустимым, но и необходимым. Наш
% рассудок приобретает… Ну, в этот случай его читать сложно, я даже не буду
% зачитывать. Потом почитайте, ладно? Сами попробуйте разобраться. Если хотите
% понять Канта, ну вот, и не вчитываться прям, это сложно, я понимаю, сейчас не до
% этого. Хотя вы вступление прочитайте. Вот вступление, там очень много лица.
% Критики чистого разума. Вступление. Предисловие критики чистого разума. Много
% даст вам. Уже. Даже это уже будет много. Так вот, смотрите, дальше. А как же
% разум формирует эту картину реальности? Разум формирует картину реальности с
% помощью тех инструментов, которые в него встроены. Понимаете? Я слово
% инструменты беру в кавычки. Инструменты, которые встроены в каждый наш разум.
% Ваш, мой и все такое. Это и есть то, что Канта называет трансцентральный
% субъект. Не буду сейчас покружаться. А что это за инструменты? Это априорные
% способности. Априорные. Априорные, то есть доопытные. Доопытные. Нам не нужно
% какой-то опыт получить, чтобы эти инструменты приобрести. Что это за
% способности? Первое, пространство и время. Вот пространство и время, это не то,
% как мир существует сам по себе. Это то, как разум ощупывает действительность.
% Это тот инструмент, которым разум может эту действительность нам дать для
% осмысления, для понимания. То есть определенная конструкция возникает, вещь для
% нас. Канта называет это априорные формы чувственности. Я понимаю, что сейчас я
% вам это все не объясню и не сделаю так, чтобы все стало понятно. Я просто
% набрасываю пока то, с чем вы будете разбираться. Конечно, спросите еще на
% семинарах. Будет там какой-то другая монада вам объяснит, которая видит мир с
% другой стороны, да, и расскажет вам, как это. По-другому расскажет. Собидишься,
% срезонируйте. Кто-то срезонирует с моими объяснениями, кто-то и так далее.
% Смотрите, вот еще раз. Пространство и время это априорные формы чувственности.
% То, с помощью, те инструменты, с помощью которых мы формируем вот такой, а не
% другой, такой, а не другой вот образ действительности. Помимо форм
% чувственности, большую роль играет и способности рассудка, то есть категории.
% Категория причинности, необходимости, количество, качество отношений и
% модельности. Это априорные формы рассудка. То есть смотрите, когда мы пытаемся о
% чем-то судить, разобраться в чем-то, мы хотим, мы невольно ищем, во-первых, а
% что было причиной, а каковы следствия. Канн говорит, кто вам сказал, что в
% реальности это есть. Тем более, что в реальности у одного события никаким
% образом ни одна причинность, даже если так разобраться. Совокупность немыслимого
% количества обстоятельств может быть интерпретироваться как причина. Но мы же, мы
% так устроены, что мы найдем какую-то одну причину, выделим ее в качестве
% главной, назовем это причиной. Ну, вот это как бы такое грубое объяснение.
% Априорные формы чувственности и рассудка принадлежат субъекту. Они не
% субъективны, в том смысле, что у одних один образ реальности, у другого другой,
% у третьего. Они субъектны. Это не индивидуальные особенности восприятия, это то,
% в чем мы единообразны. Так нас просто разум устроит, человеческий разум. Канн
% говорит о человеке как таковом. Так вот, наука выходит только на мир явлений,
% где можно говорить о времени, о форме пространства, о причинах и следствиях.
% Анализирует мир тел и отношений. То есть она погружается в явления глубоко, но
% все равно будет упираться в самое дно. А это дно есть наш разум. То есть те
% конструкции, которые в него встроены, те инструменты, которые в него встроены.
% Даже само понятие природы возможно только потому, что так, а не иначе устроен
% наш разум. Таким образом, Кант на самом деле додумал до конца концепт
% декартовского субъекта, самости, категорию свободы. Субъект это не тот, кто
% выбирает, кто является, выступает лишь пассивным началом. Ну, то есть просто
% воспринимает. Кант это не субъект, правильно говорит Кант. Вот только если ты
% конструируешь реальность в некотором смысле, набрасываешь на него сеть, свою
% сеть понимания, то это означает, что ты являешься свободным познающим разумом.
% Вообще свободный. Да? Принцип свободного субъекта предполагает, что ты сам
% конструируешь свою реальность. Но при этом чувствуете как до этого понять.
% Хочешь быть свободным субъектом? Будь. Но только тогда прими следствие этого,
% прими неизбежное концептуальное следствие, а именно мир сам по себе, реальность
% сама по себе от тебя сокрыта, она непознаваема. Реальность сама по себе в этом
% случае, если ты самостоятельный свободный субъект, для тебя непознаваема. 

Он радикализировал идею субъекта как свободы. Субъект --- тот, кто самостоятельно формирует свой объект, конструирует образ реальности. Свобода возможна только если разум создает этот образ, иначе личность подчинена внешним воздействиям и не свободна.
Для Канта магия невозможна. Свобода --- ключевой принцип, потому что наука, особенно в классическом механическом понимании, опирается на причинно-следственные связи: одна причина всегда ведет к одному следствию. Механика --- линейная причинность, все подчинено ей.

Но как быть со свободой выбора и моралью? Свободен лишь тот, кто способен выбирать независимо от внешних причин --- например, жертвовать собой не из страха или жалости, а осознанно. Если поступок вызван эмоцией или внешними влияниями, он не свободен, а детерминирован.
Кант, будучи ученым, понимал механистическую науку, но показал, что свобода возможна, если разум сам становится законодателем. Это --- зрелость разума. Каждый человек свободен и не должен восприниматься как средство, а реализует свой выбор, что лежит в основе этики Канта --- категорического императива.
Наука не признает свободу, полагая все детерминированным, но Кант предлагает критический путь: разум формирует реальность через априорные структуры, которые он называет феноменом --- миром, каким он нам является. Но есть ноумен --- вещь в себе, реальность сама по себе, непознаваемая и, возможно, не подчиненная времени и причинности.

Таким образом, мы видим мир через пространство и время --- априорные формы чувственности --- и категории рассудка, такие как причинность и необходимость. Это универсальные инструменты разума, не субъективные, а присущие всем людям.

Наука исследует только феномены --- мир явлений, где действует время, пространство, причина и следствие. Но она упирается в границы разума --- в его конструкты и инструменты, через которые формируется понимание природы.
Кант завершил развитие концепта субъекта как свободного конструктора реальности. Свободный субъект --- не пассивный наблюдатель, а активный творец своего опыта. Однако свобода сопряжена с признанием того, что мир сам по себе, ноумен, для нас непознаваем. Это неизбежный концептуальный вывод: свобода и непознаваемость реальности --- два неразрывных аспекта его философии.

\subsection{Формирование экспериментально-математического естествознания в XVII-XVIII вв.
Основные черты классической научной картины мира} 

% Все,
% что мы связываем с современной наукой, это вот как раз вопрос о формировании
% экспериментально-математического естествознания. 

% Новое
% время --- это эпоха, когда закончился натурфилософский этап развития
% науки, который связан с методом умозрительности,
% метафизические концепции, временами, которые даже уходят в какую-нибудь там
% герметизм. 

% Метод классической науки --- это новый тип рациональности, новый
% тип разумности.

Новое время --- эпоха завершения натурфилософского этапа науки, основанного на умозрительном методе и метафизических, порой герметических концепциях.
Метод классической науки --- новый тип рациональности и разумности.

\subsubsection{Кеплер} 
% Переходный период между наукой прежнего
% типа и наукой нового времени. Трактат «Новая астрономия»: <<Моя цель в том, чтобы
% показать, что небесная машина должна быть похожа не на божественный организм, а
% скорее на часовой механизм. Я показываю, каким образом физическая концепция
% должна быть представлена посредством вычисления и геометрии.>> 

% Он
% усиливает то, что сделал Коперник, взял и математически устройство Вселенной, предложив гелиоцентризм. 
% При этом не сама идея гелиоцентризма была прорывной, а обоснование этого математически. 

% Вот эту линию продолжает Кеплер. И при этом продолжает ее в
% духе механицизма. Механицизм это когда мир
% похож на механизм.  Само по
% себе машина мундий, вот это выражение, это античное выражение. Его придумал еще
% Овидий, по-моему. То есть, его используют в Средневековье, называя мир машиной.
% А вот новая наука начнет это говорить почти
% буквально, что, ну почти буквально, совсем буквально, что мир, скажем так,
% гораздо больше похож на машину, чем на все остальное. 

% Но тем не менее, конечно,
% для самого Кеплера большое значение имело со всеми последствиями теория пневмы,
% то есть, теория эфира. Тело Солнца является источником силы, приводящей в
% обращение все планеты. Почему? Потому что оно испускает эфир, который формирует водоворот, он охватывает всю Вселенную. 

% Все небесные тела у него обладают душой. Они реагируют на определенные гармонические пропорции. Душа является системой резонаторов по
% его концепции, а эти резонаторы можно писать
% математически. 

В трактате «Новая астрономия» Кеплер пишет: «Моя цель --- показать, что небесный механизм скорее похож на часы, чем на божественный организм. Я демонстрирую, как физическую концепцию представить через вычисления и геометрию.»
Он развивает идею Коперника, который математически обосновал гелиоцентризм. Важен не сам гелиоцентризм, а его математическое подтверждение.
Кеплер продолжает эту линию в духе механицизма --- мира как механизма. Выражение «машина мундий» (мир-машина) восходит к Овидию и использовалось в Средневековье. Новая наука говорит об этом почти буквально --- мир действительно похож на механизм.
Однако для Кеплера важна теория пневмы (эфира). Солнце --- источник силы, вращающей планеты, потому что испускает эфир, формирующий вселенский водоворот.
Все небесные тела имеют душу, реагирующую на гармонические пропорции. Душа --- система резонаторов, которую можно описать математически.


\subsubsection{Галилео Галилей}

% принципиальная новизна его подхода заключалась в том, что следствия этих формул
% должны были проверяться специальным экспериментом. 


% он сознательно выбирает новый
% метод и противопоставляет методу умозрения, то есть, аристотливскому методу,
% противопоставляет метод Архимеда. Галилей
% пишет об этом, что надо брать за основу метод Архимеда, потому что Архимед, хотя
% и выстроил свое положение в виде аксиом, но справедливость
% этих аксиом, он обосновывал не логикой или умозрением, а практикой. 

% То есть,
% например, вот есть аксиома, что рычаг работает в зависимости от пропорции длины
% рукавов, ой, простите, плечей, да, каких рукавов, плечей. Вот. Он выстроил такую
% математическую формулу, создал, грубо говоря, какую-то аксиом определенную, а
% затем он строит машину, машину, буквально машину, которая подтвердила бы или
% опровергла это положение теоретическое. Вот, по сути дела, именно это Галилей
% называет механикой, чувствуете, потому что механикой свою работу называют
% Архимед, он механик, да, он не какой-нибудь там арестователь, умозрительные
% конструкции, он механик. И Галилей, поэтому свою работу тоже называет механикой.
% И он призывает механиков-то опираться на конкретные математические расчеты. 

% В
% диалоге о движении Галилей резко раскритиковал метод Аристотеля, ну, конечно же,
% нашим врагов университетских, потому что университет до сих пор тогда еще живет
% по законам, что называется, физики Аристотеля и тому подобное. 

% Вот. Он был
% немножечко бунтарем, читал лекции на разговорном итальянском языке, не на
% латыни, за исключением сложных доказательств. Их он излагал на латинском, чтобы
% математические выводы могли и, ну, как бы все более такие образованные читатели
% осознать. 

% в 1632 году была опубликована книга «Диалог о двух главнейших
% системах» Птоломейской и Коперниковской, в которой Галилей кроме разговора об
% астрономической картине мира много внимания уделяет именно физике. Он
% опровергает аристотельскую концепцию движения в этой работе, формулирует принцип
% относительности, то есть внутри равномерно движущейся системы все физические
% процессы протекают, так же, как и внутри покоящиеся, то есть мы не заметим
% движение, если будем равномерно двигаться. 

% Постольку, поскольку он опирался на
% Коперниковскую книгу, которая находится в индексе, находилась в индексе
% запрещенных книг, туда она попала, почему? Напомню, потому что в этой книге
% святая палата узрела магию. Помните? Джордана Бруно Солнце, такой вот магический
% новый бог, который в центре мира, в центре мироздания. И Копернин говорит, что
% Солнце в центре мироздания, разбираться они не стали. Все, что в центре
% мироздания, не все работы, которые там ставят Солнце в центр мироздания, это еще
% одна очередная магическая работа, поэтому все в интере запрещенных книг. 

% Но
% Галилей в своей работе честно говорит, а он был каноником, он был как бы
% церковным человеком. Он говорит о том, что нет, неправильно, Коперник не говорил
% ничего про то, что вот там Солнце это новый бог. Коперник просто математически
% рассчитал то, что рассчитал, и это очень правильно все это работает. И я
% обосновываю на основе принципа Коперника вот как раз гелиоцентризм. И, конечно,
% его за это не осудили бы, вот вплоть до какой-то смертной казни и тому подобное.
% Ничего подобного и не было. 

% Галилей и инквизиция, тема
% интересная вообще на самом деле. На Галилея наехали на самом деле иезуиты, очень
% серьезно наехали. Вот если бы иезуиты довели до конца своего расследования, то
% вполне возможно Галилей бы поплатился жизнью. Почему? Потому что его-то обвинили
% именно в.
% Дело в том, что Галилей написал книгу о присуществлении, ну то есть это
% превращение хлеба и вина в кровь и тело Христа. Причем там не было никаких
% антирелигиозных, антицерковных идей. Он просто по-новому трактовал. Чисто вот
% религиозно, она даже никак не входит в список научных работ в Галилее. Из-за
% этого он столкнулся с иезуитами очень серьезно. Иезуиты пытались его арестовать,
% вот так вот, посадить его в тюрьму. 

% И за него вступился его один из друзей, Папа
% Римский. Папа Римский был друг Галилея. И чтобы спасти его от этого, что
% называется, суда, его подвели под другой суд. Суд о том, что он использовал в
% своей работе книгу из индекса запрещенных книг. Галилей сказал, все понял,
% осознал, признал, исправлюсь. Вот. То есть, ничего ему не грозило, он понавили
% хозяина тюрьмы, не хозяин, а директор, начальник тюрьмы просидел, положенный
% сок, вышел из этой тюрьмы совершенно здоровым, но полных сил, никаких не было
% над ним пыток и истязаний, ничего это не поддерживает ни один исторический
% документ, это опять пытанка очередная. 

% То есть, я не говорю, что он не подсказал
% от религии, да, потому что иезуиты тоже религиозные народы, но нет, нет
% инквизиции, скажем так, инквизиция в этот случай его как раз спасала. Вот, он
% вышел и после уже выхода из тюрьмы написал свою главную, еще одну главную
% работу, начало и получил признание очень такое, ну, почти, всеевропейское, это
% точно, был очень популярен, и, что называется, не вышел каким-то разбитым
% стариком, это неправда. 

% Галилео Галилей, положил начало процессу инструментализации науки. 
% Двучленное отношение наблюдающий---субъект (наблюдаемый объект), дополнился теперь третьим --- научным прибором. 

% Вообще, в это время к теме прибор, как предмет, с помощью которого мы
% видим мир, получает свое очень такое активное развитие. Долгое время исследующие мир старались не использовать никаких, ни линз,
% никаких очков, очки были изобретены в Средневековье, и мы видим много портретов,
% когда люди в очках, но вот в качестве исследователей очки или линзы старались не
% использовать, потому что считалось, что они вносят искажения (потому что
% очки задерживают пневму, искажают).


% Галилей отбросил концепцию пневмы разного рода,
% отбросив и эти беспокойства. 



% это вечная
% проблема, но на определенный момент Галилей ее решил и создал очень большое
% количество приборов, в том числе механических. 

% все это потребовало, во-первых,
% как сказать, само время. Развитие военного дела поставило задачи расчета
% оптимальной дальности полета снаряда, артиллерии же, да, уже, всякие другие
% динамические проблемы. Кто будет решать? Будет решать ученые, разумеется, все
% эти расчеты производить. 

% Вот. Дальше подзорная труба крайне была важна, да,
% опять же, ну, военное дело создает техническую науку, а военную, вот эту
% подзорную трубу можно же превратить в телескоп, да, соответственно, появляются
% специфично научные инструменты, то есть инструменты как таковые, видите, военные
% не парились, я там вижу, пневма там мне фантазм дает, или я вижу, я просто лучше
% вижу, я вот так вот приглядываюсь, вижу там, кучку людей, а когда я
% приглядываюсь, я в этой кучке могу разглядеть уже где там, понимаете, да, вопрос
% об ней у военных не стоял, и поэтому они сказали, так, все делаем, и не паримся,
% а соответственно, ну, вот это как бы более трезвомыслящее отношение к прибору,
% потом и в научный, как бы, дискурс тоже смещается. 

% Изобретение микроскопа сразу
% же положило начало микроанализу, да, из крупнейшей открытия в самых разных
% областях, итальянский биолог, мальпиги, исследует строение внутренних органов
% животных, вписывает структуру растений. 

% Роберт Гук усовершенствовал микроскоп и пришел к
% выводу о клеточном строении растений, даже вводит понятие клетка, 

% тут же нужно
% упомянуть биолога Левенгука, очень важная, да, фигура, пожалуйста, уточните,
% проанализируйте, поисследуйте эти, о роли этих имен, этих людей.

% Для развития
% физики большое значение имело изобретение телескопа.

% первым
% вариантом был телескоп, более ранний там, его авторы, Литл Сгейм и Литл Сгейм,
% не помню, второй, зрительная труба, а вот следующий шаг, это был Галилеевский  телескоп, и, в общем-то, благодаря нему увидели, ну, Луну в
% совершенно новом облике

% Особую роль играли появление механических часов с более
% точными делениями, да, то есть, когда уже там минутная стрелка, не просто час,
% но минутная стрелка, для физики это крайне важно, чтобы исследовать динамические
% характеристики. 

% Эвангра Агаций, он
% предлагает даже имя собственное для современной науки, вот чтобы, понимаете, не
% путаться с тем, что наука в античности, наука средневековья, наука нового
% времени, он предлагает имя для собственной, ну, то есть, собственное имя для
% современной науки, называет её технонаука, потому что, говорит он, эта наука, и
% приходится с ним согласиться, она уже имеет дело не просто с реальным миром, она
% имеет дело с реальным миром, преломившимся через научные приборы.

% Одной из важнейших его книг является беседой
% доказательства, касающихся двух новых наук. Это была научная сенсация. В ней,
% этой книге, он как раз вводит определение силы, скорости, ускорения,
% равномерного движения, инерции, средней скорости, среднего ускорения, импульс. И
% многие исследователи считают, что теория импульса «Галилея» это переработная
% теория импетуса средневековых схоластов, мертонских схоластов. 

Принципиальная новизна подхода Галилея заключалась в том, что выводы формул должны были проверяться специальными экспериментами.

Он сознательно выбирал новый метод, противопоставляя умозрительный аристотелевский подход методу Архимеда. Галилей утверждал, что основой следует брать метод Архимеда, который, хотя и выстроил систему на аксиомах, обосновывал их не логикой, а практикой. Например, аксиома о работе рычага в зависимости от пропорций плеч была подтверждена созданием и испытанием машины --- то есть экспериментом. Галилей называл такую деятельность механикой, подчеркивая, что и Архимед, и он сам --- механики, которые опираются на конкретные математические расчёты.
В «Диалоге о движении» Галилей резко раскритиковал аристотелевский метод, который продолжали использовать университеты. Он был своего рода бунтарём, читая лекции на разговорном итальянском, а сложные доказательства излагал на латыни, чтобы более образованные читатели могли их понять.

В 1632 году вышла книга «Диалог о двух главнейших системах» --- Птолемеевой и Коперниковской. Помимо астрономии, Галилей уделял большое внимание физике, опровергая аристотелевскую концепцию движения и формулируя принцип относительности: в равномерно движущейся системе физические процессы протекают так же, как и в покое, и движение не ощущается.
Галилей опирался на книгу Коперника, которая находилась в индексе запрещённых книг из-за обвинений в «магии» --- поскольку Солнце в ней поставлено в центр мира, что воспринималось как новый культ. Однако Галилей, будучи церковным человеком и каноником, защищал Коперника, утверждая, что тот лишь математически обосновал гелиоцентризм, без каких-либо религиозных заявлений.

На Галилея были нападки со стороны иезуитов, которыеобвиняли его не в научных, а в религиозных вопросах --- из-за его трактовки таинства превращения хлеба и вина в тело и кровь Христа. Несмотря на отсутствие антицерковных идей, он попал под преследование. Его пытались арестовать, но вмешался Папа Римский, друг Галилея, и вместо религиозного суда Галилей был осуждён лишь за использование книги из индекса запрещённых. Он признал ошибки, исправился и вышел из тюрьмы без пыток и истязаний.

Таким образом, инквизиция в его случае скорее спасла Галилея. После выхода из тюрьмы он написал одну из своих главных работ, получил широкое признание и не стал разбитым стариком --- это распространённый миф.

Галилей положил начало процессу инструментализации науки, добавив к дуализму «наблюдающий --- наблюдаемый объект» третий элемент --- научный прибор.
В то время использование приборов для исследования мира быстро развивалось. Долгое время исследователи избегали очков и линз, опасаясь искажений из-за пневмы, которая якобы влияла на зрение. Галилей отбросил эти представления и создал множество приборов, включая механические.
Развитие военного дела требовало точных расчетов оптимальной дальности артиллерийских снарядов и других динамических задач, что способствовало привлечению учёных. Подзорная труба, разработанная для военных целей, была преобразована в телескоп и стала важным научным инструментом.

Изобретение микроскопа открыло микроанализ. Итальянский биолог Мальпиги изучал строение органов животных и растений. Роберт Гук усовершенствовал микроскоп и ввёл понятие клетки. Важную роль в развитии биологии сыграл также Левенгук.
Для физики большое значение имело изобретение телескопа. Первые зрительные трубы создали Литтл Сгейм и его коллеги, а следующий шаг сделал Галилей, позволивший увидеть Луну в новом свете.
Появление механических часов с точной минутной стрелкой стало критически важным для изучения динамики движения.
Эвангра Агаций предложил термин «технонаука» для обозначения современной науки, подчеркивая, что она оперирует не просто с реальностью, а с реальностью, преломлённой через научные приборы.

Одна из важнейших работ Галилея --- «Беседы и математические доказательства, касающиеся двух новых наук», которая стала научной сенсацией. В ней он вводит понятия силы, скорости, ускорения, равномерного движения, инерции, средней скорости и ускорения, а также импульса. Многие исследователи считают теорию импульса Галилея переработкой средневековой теории импетуса мертонских схоластов.

\subsubsection{Исаак Ньютон} 

% Он родился в год смерти Галилея, в
% своём труде «Математические начала натуральной философии» обобщил открытие
% Галилея и добавил к ним третий закон, закон всемирного тяготения. 


% Он активно занимался алхимией. То есть Ньютон как раз это такой
% учёный, который не перешёл ещё из прежней науки в тип новой науки. Да-да-да, вот
% как ни странно. Вроде, эта наука новая, типа называется ньютонянская, но сам-то
% Ньютон был алхимиком по преимуществу. Более того, когда ему прислали письмо,
% Роберт Гук прислал ему письмо, предложил Ньютону решить ряд проблем вычисления
% теории планетных движений. А в ответ Ньютон написал, что в его возрасте
% затруднительно заниматься уже всякой ерундой. Его куда больше интересует
% алхимия. И вот он просто, в силу того, что он был талантливым человеком,
% мыслительно талантливым, он предложил Гуку поставить эксперимент по проверке
% теории Коперника. 


% он начал решать задачу, которую ему Бог послал. Вот. И создал вот
% новую механику, Мирненскую механику. Было отчётливо сформулировано сам подход к
% природе, да, то есть к тому, что нужно, понимаете, какой был подход. Помните,
% знаменитая Ньютоновская, я не изобретаю гипотез, я гипотез не измышляю. То есть,
% давайте будем анализировать только то, что можно эмпирически наблюдать и
% описывать это. 

% Ньютон в предисловии к первому изданию говорит, новейшие авторы,
% подобно древним, стараются подчинить явление природы законам математики, но в
% данном случае он говорит законам таким умозрительным, а я же намерен заниматься
% тем, что относится к силам, притягательным и напирающим. 

% То есть, естествознание
% позиционируется как исследование сил, математика в нём не цель, а средство,
% метод. Галилей проводит серию опытов с маятником, ну, чтобы проверить выводы
% Галилея, убеждается в этих, ну, в том, что Галилей прав, как сказать, обобщает
% эти выводы, выводит категории массы как единственной причиной гравитационного
% взаимодействия, масса --- это нужные свойства вещества, вес --- это сила тяжести,
% действующая на тело, то есть динамика Ньютон, да, она позволила решать любые
% задачи о положении движущегося тела. Вот. В этом важно. Положение движущегося
% тела. И реальность стала рассматриваться как совокупность движущихся тел. при
% этом пространство и время абсолютно. Это вместилище этих тел. Сон, что абсолютно
% это означает, что они никак не влияют на сами движения, да, этих тел. Они
% безотносительно, так называемые, к движущемуся телу. Все движения
% рассматриваются механическими, то есть они могут быть подвергнуты
% количественному анализу. 

% При этом, когда его спрашивали, а его очнити на
% спрашивали, ну, а вот это всемирное тяготение, это, благодаря которому яблоко
% падает на землю, и луна падает, бесконечно падает на землю, и другие небесные
% тела находятся в состоянии вот этого всемирного тяготения к другим телам, как
% его понимать? Но вот тут Ньютон ускользал от ответа и придумал эту концепцию. Я
% не измышляю гипотез. Потому что он был бы вынужден сказать, что вот это
% всемирное тяготение им понимается самим, всё-таки магическим. 

% Как вот эта вот всеобщая связь у Джордана Бруно. Да? 

% Для него, вот как бы, он же не случайно был
% химиком, это была сфера его интереса, главная сфера его интереса, по большому-то
% счёту. Вот. Для него абсолютное пространство, вот это вот, в котором
% осуществляется тяготение, это, по его выражению, «sensorum dei», то есть
% чувствилище Бога. Это некое, скажем так, ну вот, тело Бога. То есть, если мы с
% религиозной позиции отнесёмся к Ньютону, ну, не обратимся к Ньютону, то станет
% понятно, что, конечно, он еретик. Ну, то есть, он разделяет позиции пантеизма,
% пространство --- это божественная самореальность, некатегория времени, отсюда и
% дальнодействие, принцип дальнодействия. 

% То есть, например, вот, упавшее яблоко
% на землю действительно перераспределяет весь вот этот вот расклад сил во
% Вселенной, причём мгновенно, мгновенно, сразу же, не там, не на волнах эфир, а
% вот мгновенно и сразу же, просто Вселенная знает. Почему знает? Потому что это и
% есть тело Бога, чувствовище Бога. 

% в основном
% прикали за принцип дальнодействия и за то, что он, ну ты давай, объясни, что это
% за всеми текотениями. А что, посредством чего он осуществляется? Поэтому очень
% долгое время в Европе всё-таки рулила теория Декарта. Декартовская концепция
% признавалась всеми, ньютоновская далеко не всеми. И критерием истинности закона
% тяготения Ньютона, вообще теории механики Ньютона, да, стал вопрос о форме
% Земли. Согласно теории Ньютона, она должна была быть сплюснута с плюсов. По
% теории Декарта она должна была быть вытянута.

% И вот поэтому, чтобы выяснить правоту либо Ньютона, либо Декарта,
% была проведена специальная, вот специальная экспедиция в Перу и Лаплангию. Она
% подтвердила сплюснутость полюсов и всё, ньютоновская теория победила. 

Он родился в год смерти Галилея и в своём труде «Математические начала натуральной философии» обобщил открытия Галилея, добавив третий закон --- закон всемирного тяготения.

Ньютон активно занимался алхимией и оставался учёным старого типа, не полностью перейдя к новой науке. Хотя ньютоновская наука считалась новой, сам Ньютон по сути был алхимиком. Когда Роберт Гук предложил ему заняться проблемами теории планетных движений, Ньютон ответил, что в его возрасте это сложно, и его больше интересует алхимия. Тем не менее, он предложил Гуку эксперимент для проверки теории Коперника.

Он решил задачу, которую считал посланной Богом, и создал новую механику --- ньютоновскую механику. В ней был чётко сформулирован подход к природе: не измышлять гипотезы, а анализировать только то, что можно эмпирически наблюдать и описывать. В предисловии к первому изданию Ньютон указывал, что в отличие от умозрительных законов других авторов, он занимается силами, которые реально воздействуют --- притягательными и напирающими.

Естествознание рассматривалось как изучение сил, где математика --- лишь инструмент. Галилей проводил опыты с маятником, а Ньютон подтвердил и обобщил его выводы, выделив массу как единственную причину гравитации. Масса --- свойство вещества, вес --- сила тяжести, действующая на тело. Ньютоновская динамика позволила решать любые задачи о движении тела, рассматривая реальность как совокупность движущихся тел в абсолютном пространстве и времени, которые не влияют на движение и существуют независимо.

Когда его спрашивали о природе всемирного тяготения, Ньютон избегал прямого ответа, утверждая, что не измышляет гипотез, так как вселенская гравитация была для него чем-то магическим. Подобно Джордано Бруно, Ньютон воспринимал абсолютное пространство как «sensorum dei» --- чувствилище Бога, тело Бога. С религиозной точки зрения, он был пантеистом: пространство --- божественная самореальность, откуда следует принцип дальнодействия.

Например, упавшее яблоко мгновенно изменяет силы во Вселенной, не посредством эфирных волн, а потому что Вселенная --- это тело Бога, которое всё чувствует сразу. За принцип дальнодействия его критиковали, требовали объяснений, как именно происходит взаимодействие на расстоянии. Поэтому долгое время доминировала декартовская теория, признанная большинством, в то время как ньютоновская принималась не всеми.

Критерием истинности ньютоновской теории стал вопрос формы Земли: по Ньютону она должна быть сплюснута у полюсов, по Декарту --- вытянута. Для проверки была проведена экспедиция в Перу и Лапландию, которая подтвердила сплюснутость полюсов, и ньютоновская теория победила.

%  В научных теориях того периода ещё много того, что мы назвали бы
% метафизическим или даже герметическим. 


% , но существенные изменения коснулись самих подходов к науке, потому
% что меняется культурно-исторический контекст. И вот какова же научная картина
% мира, новой реальности. 

% Что такое научная картина мира? 

% Что она включает? 

% категория причинности, то есть жёсткий детерминизм
% или лапласовский детерминизм. Детерминизм, это означает
% причинность. Почему она жёсткая? Потому что вот она, одна причина, одно
% следствие, вечная, неизменная вот эта вот связь. Дальше всё это выстраивается в
% линейную причинность, во времени разворачивается. и, соответственно, если мы
% знаем, как говорил лаплас, положение всех частиц в определённый момент времени,
% знаем их массу и знаем ускорение, с которым, ну, там, какое-то действие на них
% оказывается, то, вернее, знаем импульс, да, ускорение, то мы в итоге можем
% проследить эту причину и следствие как глубоко в прошлое, так и далеко в
% будущее. То есть мы знаем всё, мы можем узнать всё. Нам нужен только такой
% разум, или, ну, в общем, да, такой разум, который бы мог удерживать в своей, как
% говорится, в своей памяти, в своих расчётах положение всех частиц и их массу
% импульс. Тогда всё, мы можем, например, описать состояние мира через 200 лет,
% ну, в абсолютности, то есть где вот какой конкретный человек, там, где
% конкретная сущность. 


% линейный детерминизм, он напрямую
% связан с категорией деизма. Деизм – это религиозная позиция, противостоящая  принципу теизма: Бог личность, который постоянно участвует в реальности,
% постоянно её как бы, ну вот, через чудеса, через другие какие-то моменты он
% управляет реальностью. 
% А деизм предполагает, что вот всякие разные, как сказать,
% внеплановые, внеплановые воздействия на реальность исключены, даже божественно.
% всё, ничего уже больше не вот эту вот машину, она заведена один раз, машина
% мира, и она фурычит. Нам нужно лишь выяснить, как она устроена, тогда мы сможем
% её управлять. 

% Вот, она фурычит, и вот тут, конечно, механицисты разделились,
% одни, механицисты разделились, одни говорят, ну, эту машину Бог заводит каждый
% раз, еще не каждое мгновение, она, да, она построена им однажды, но он заводит
% её каждое мгновение. Например, Лейбниц был таким механицистом, он говорит, нет,
% что Бог заводит каждую часть, то есть Лейбниц не был деистом, 

% Он (?) был деистом, потому
% что он говорит, что один раз Бог завел эту машину и больше не вмешивался, но
% даже не важно, кто был деистом, просто эта позиция разделялась многими. 

% Кант был
% очень против деизма, он говорил, что деисты не отвергают веру в Бога, но они
% признают лишь первосущность или высшую причину, да, то есть Бог это первая
% причина, а дальше он не вмешивается. Но он говорит, что справедливый будет
% сказать, что деист верит в Бога, а атеист верит в живого Бога. Ну, сам, конечно,
% был не деистом. 

% Деизм, как
% религия, довольно быстро умер, не очень строго
% рассказывал про религиозную реальность. 

% Принцип редукционизма. 
% Редукционизм означает, что реальность выстроена, начиная с элементов малых, то
% есть от малого к большому, снизу вверх, и узнавая, как, как сказать, узнавая
% специфику взаимодействия малого, ну, например, молекул, там, атомы, да, мы в
% конце концов узнаем характеристики целого. 

% Редукционизм противостоит холизму, то
% есть это противоположная позиция, согласно которому свойства целого определяют
% свойства элементов. 

% Это две методологические стратегии, которые борются в
% истории науки, истории мировоззрения, ну, то есть они вполне себе такие противо,
% ну, как сказать, противоречащие другу, но вот расцвет редукционизма это,
% конечно, этап механицистической науки, картины мира. 

% О появлении статистики. Наука нового времени активно изучает
% теорию случайных явлений. 

% Ну, вы понимаете, да, раз ты утверждаешь принцип
% линейного детерминизма, то тебе всё-таки приходится разобраться с категорией
% случаев, например, с игральными этими самыми кубиком, да, игральными костями.
% Как же всё-таки так получается, что мы не можем предсказать поведение игральных
% кубиков? Этим особенно занимался Лаплас, тот же самый, он и есть на
% столкоположной теории вероятностей. Этим активно занимались многие-многие
% исследователи. Так рождается наука статистика. 

% Почему статистика? Потому что
% совершенно в духе времени, вот в этих случайных процессах основоположники теории
% вероятностей нашли законы, то есть из случаев подчиняются законам. Но правда, по
% их мнению, это был лишь промежуточный вариант. Теория вероятности не
% окончательная. Мы, согласно теории вероятности, можем сказать, сколько раз из,
% например, из ста бросков монета упадет на решку или на орла, да, можем сказать.
% Но мы не можем, теория вероятности не скажет нам, какой это будет случай-то. 

% И
% по, согласно Лапласу и другим исследователям, это лишь, ну, пока тот самый
% демон-то не существует еще, да, вот мы еще немножко поднапряжемся и в конце
% концов какой-нибудь, какой-нибудь нам там интеллект поможет все-таки исключить
% категорию случайности и мы будем знать каждый раз, как монета падает, да. 

% А
% другим исследователям крайне понравилось, что теперь вот этот вот хаос
% социальный может подчиняться определенным закономерностям, определенным
% закономерностям. 

% И вот здесь я фразу Кетли, это страховщик и основатель теории
% статистики. Он просто пишет, ну, чуть ли не религиозный текст насчет того,
% насколько прекрасно, что теперь вся, он даже дает категорию такую, выводит
% категорию социальная физика, что вся, весь вот как бы аспект связанный с волей,
% со всякими там разными преступлениями, ну, и вся вот этот социальный хаос, да,
% кто-то влюбляется, женится, там, умирает, преступление совершает, в тюрьму
% садится, вот этот весь хаос, оказывается, можно посчитать. 

% Смерть, преступление,
% брак, рождение детей, это область исследования Кетли, он выводит статистические
% закономерности в этой области и делает вывод, нет свободной воли человека ни в
% деле брака, ни в преступлениях, есть лишь социальная физика, подобная образец
% становится преобладающей, социологические исследования выходит на это, они
% вдохновлены идеей того, что вот в этом хаосе социальном принципиально есть
% порядок,
% а я напоминаю, что парадигма порядка, контроля, это и есть ключевая
% идеологема этого мира, она воплощена зримо во многих автоматонах


% дело в том,
% что до вот как бы определенного периода все эти фундаментальные законы природы,
% они были делом энтузиастов, но постепенно эти энтузиасты осознали, что требуется
% не просто материальное вливание для того, чтобы проводить эксперименты и тому
% подобное, а требуется связь с государственной властью, и они смогли убедить
% королевскую власть в то, чтобы, ну, вот наука интересовалась, стала областью
% интереса государственной власти, а так как к тому времени уже, наверное, вы
% обратили внимание, что наука это коллективное дело, тот же Ньютон пишет письма,
% да, Лебнеев пишет письма, так называемые невидимые колледжи образуются, когда
% вот эти письма зачитывают в публикации, понимаете, то есть уже это дело
% коллективное, вплоть до того, что в конце концов именно коллектив исследователей
% выбирает, какая теория правильная, ньютоновская все-таки, физика-то правильная,
% или декартовская, помните, даже экспедицию собрали, две экспедиции, вот,
% соответственно, вот эта вот идея, коллективность науки плюс государственная
% власть формирует социальный институт науки, и ее первым взаимовоплощением
% становится Академия наук, они конкурировали с университетами, тем более, что
% университет это область тогда влияния папы по большей части, а вот, ну, а если
% протестантские страны, то, конечно, не папа, но все равно это может быть
% религиозная какая-то там мужчина, вот, а Академия это всегда связь с
% государством.

В научных теориях того времени сохранялось много метафизики и герметизма, но существенно изменились подходы к науке из-за смены культурно-исторического контекста. Научная картина мира включала категорию причинности --- \textbf{жёсткий, или лапласовский, детерминизм}. Это означает, что каждая причина неизменно ведёт к одному следствию, и эта связь разворачивается во времени. Если знать точное положение, массу и импульс всех частиц в данный момент, можно предсказать состояние мира и в прошлом, и в будущем, при условии существования разума, способного удерживать такую информацию.

Линейный детерминизм тесно связан с \textbf{деизмом} --- религиозной позицией, которая отвергает постоянное вмешательство Бога в мир. По деизму, Бог создал мир один раз, запустил его и больше не вмешивается. Машина мира работает самостоятельно, и задача человека --- понять её устройство. Однако среди механицистов были разные мнения: например, Лейбниц считал, что Бог поддерживает движение машины каждое мгновение, то есть не был деистом. Кант критиковал деизм, считая, что деисты признают лишь первопричину, но не живого Бога, и сам был противником деизма. В итоге деизм как религия быстро угас, не давая строгих объяснений религиозной реальности.

Принцип \textbf{редукционизма} предполагал, что реальность объясняется через свойства её элементарных частей, начиная с малого и поднимаясь к целому. Это противоположно холизму, который считает, что свойства целого определяют свойства частей. Эти две методологии противостояли друг другу, но расцвет редукционизма связан с механистической наукой.

Одновременно возникла \textbf{статистика} --- наука о случайных явлениях. Лаплас и другие учёные пытались понять, как случайные события, вроде броска кубика, подчиняются законам. Теория вероятностей стала инструментом, позволяющим предсказывать частоту событий, но не конкретные исходы. По мнению Лапласа, в будущем может появиться разум, способный исключить случайность и предсказать всё.