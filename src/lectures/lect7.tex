\section{Социально-исторические условия формирования европейского Средневековья}

% 5 15 века как раз рамки этих внутри этих рамок в общем-то достаточно стабильная эпоха  

% тем не менее есть свои границы 
% раннее 
% зрелое/высокое средневековье

% да и вот ранние средневековья еще одна еще один миф по поводу того
% что здесь значит все было так темные века все было в состоянии какого-то краха и
% так далее вот этот вопрос я специально подниму обязательно почему потому что ну
% вот от того поймете вы это или не поймете как это поймете зависит ваше
% восприятие средневековья вообще значит о чем мы говорим мы будем говорить сейчас
% о том что социальная стратификация речь идет о том какие группы социальные
% группы социально психологические группы составляют эпоху какого-то перехода
% кризисную эпоху а мы сейчас имеем дело мы сейчас говорим 5 век 4 3 4 5 век это
% по хоро кризисная глобально кризисная эпоха который определенными настроениями
% какие настроения будут скажем так решающими для эпохи средневековья значит
% смотрите во все вот эти вот трудные времена включая сегодняшний вот можете
% кстати экстраполировать вполне даже подходит 

% Существует как минимум три группы настроений, которые связаны социально-психологическими особенностями 
% и три группы которые можно
% разделить декаданс лоббильность и ригидность но 

Средневековье охватывает период с V по XV века. Это была достаточно стабильная эпоха.

Выделяют два периода:
\begin{itemize}
    \item Раннее Средневековье (V-XI вв.)
    \item Высокое/Зрелое Средневековье (XI-XIV вв.)
\end{itemize}

Можно выделить три ключевых группы настроений того времени:
\begin{enumerate}
    \item Стоицизм Марка Аврелия хорошо отражает настроение \textit{декаданса}, когда всё кажется потерянным, и остаётся только мудрость и крепость духа, чтобы выдержать неприятности. Это настроение ориентировано на рефлексию и внутреннее состояние, а не на будущее или прошлое. Люди, переживающие эту эпоху, фокусируются на своих чувствах и переживаниях, а не на создании чего-то нового.
    \item Состояние \textit{ригидности} --- это уверенность, что всё останется, как раньше, несмотря на потрясения. Синдром Сидония, названный в честь Аполлинария Сидония, который игнорировал крах, даже когда Рим был завоёван вандалами, выражает слепую уверенность: <<ничего не происходит>>. Сидоний продолжал вести философские беседы, не замечая изменений вокруг.
    \item \textit{Лабильность} --- это уверенность в действии и активности. В этот период германцы, как наёмные воины, занимают важное место в армии. Многие из них уже христиане. Марк Аврелий и Седоний, не замечая изменений, продолжают считать эти силы маргинальными, не видя в них будущего. Однако именно такие <<неучтённые>> силы, как германцы и первые христиане, станут лидерами будущей эпохи. Они не писали тексты, а занимались делом, создавая новые структуры.
\end{enumerate}

 
% значит давайте с ним очень и когда это ситуация вот к нему здесь у нас портрет Марка Аврелия стоицизм лучше всего подходит для вот этого настроения декаданса которая говорит о том что все все пропало ничего уже не будет прежним как бы все у нас летит в
% тартарары и соответственно единственно что нам остается это мудрость
% определенная крепость духа и просто выдерживать все неприятности которые на нас
% сваливаются почему это эпоха простите почему это настроение настроение скажем
% так не будущего не того что вообще составляет будущую средневековую попу потому
% что это во первых такой энтропийный подход энтроп поворот на себя энтроп да
% поворот внутрь то есть это настроение людей которые пишут тексты которые
% рефлексируют и говорят о том по сути дела о своем состоянии их вообще не
% интересует ни будущее ни прошлых интересует только свое вот это вот состояние
% это такие люди как правило ну знаете вот они внутренне ориентированные вот так
% вот на себя они не видят вокруг потому что их интересует первую очередь
% внутреннее состояние конечно многие философы немного психологи в современности
% философы но вот из этого да из этого региона социально культурного такого слоя
% частная судьба как правило их интересует до свои чувства здесь сейчас вполне
% закономерно что этим настроением вообще-то новое не создается 

% вот состояние  ригидности это уверенность что ничего не изменится все
% будет так как раньше вот сейчас чуть-чуть это самое прошло называется потрясет и
% все вернется аппалинарии седоний вот есть такой синдром седония синдром
% аппалинарии седония он означает способность не видеть краха не видеть крах то
% есть абсолютно уверенность что ничего не происходит даже если вы находитесь
% жирли вулкана уверенность конечно это уверенность но это уверенность слепоты
% почему седоний попал вот так скажем в название потому что это был интересный
% человек который писал письма в этот период и этих письмах он рассказывал
% прогулках там же в галла завоевали рим уже вандалы там разрушали рима он
% рассказывал там что встать ничего не случилось все нормально у нас вот тут вот мы тут с друзьями посидели так хорошо поговорили на философские темы в общем ничего не происходит и 


% Лабильность то есть не
% для подвижность и здесь тоже некая форма уверенности но это уверенность действия
% уверенность активности настоящий и в тот период вот эта группа лабильности это
% в основном германцы я хочу подчеркнуть и может быть надеюсь вы это отметить у
% себя почему германцы к тому времени они уже занимают почти лидирующие силы в
% армии а почему потому что они наемные воины они этим занимаются зарабатывать
% этим себе на жизнь очень многие дальше как правило многие из них уже христиане
% потому что это люди из других регионов там христианство приняли чуть раньше так
% вот люди типа марка аврелия и уж тем более типа сидония называют эти силы по-
% прежнему какими-то такими знаете маргинальными ну то есть не германцы и не
% христиане их не интересует они не видят за этими силами какого-то будущего но в
% этом есть своя как дается свое основание на тот период христиан всего три
% процента от общего объема до вероисповедания римской империи но они не я имею
% ввиду марка в реле тот же самый и Апполинарий Сидоний это для нас сейчас будут
% нарицательны такие вот эти люди и люди типа этого они не замечают что
% вдохновлены то энергия то как раз на стороне тех кто вот для них ванна варвары
% какие-то вот маргинальная маргинальная секты и так далее то есть вот эти так
% называемые не учитываемые не учитываемые обычно и становится лидирующим началом
% будущей эпохи вот серьезно у нас сейчас с вами глобальный кризис нужно смотреть
% на не учитываемых и конечно воодушевленных только они и будут выстраивать новую
% эпоху вот точно так же происходило в раннем средневековье почему мы не знаем
% позицию этих вот лабильных вот этих вот германцев тому подобное этих первых
% христиан которые по сути дела и составили будущую эпоху который мы сейчас будем
% говорить уже как бы полностью да почему потому что они не пишут текстов вот как
% правило мы в ту эпоху как минимум они не пишут это люди дела это воюющие люди
% люди создающие новые структуры там например ну то есть это люди лабильного сектора не пишут делом занимаются вот
% этот пожалуйста расклад учитывайте когда будете следующий раз читать о том что
% эпоха средневековья особого раннего особенно раннего средневековья это эпоха
% тоски упадка чего-то такого совершенно ну как скажем ну в общем из чего ничего
% нового не строится так вот эпоха который о котором говорим ранее средневековья
% это резкое изменение расклада культурных сил афины уже не в моде но мы это на
% семинаре говорили вы на лекции это говорили афины уходят это становится глубокой
% провинции она афинский университет не интересны афинской школы интересны в
% ресурсе в таком в моде если даже угодно сказать александрия египетская 


% конечно нас к сожалению будет
% интересовать только европейский мир несмотря на то что вокруг происходит тоже
% очень много интересного ну ладно сегодня охватишь так вот 

% средневековье это
% римская империя вот это я хочу вас запомнить нельзя говорить о том что римская
% империя закончилась она продолжается очень долгое время условно говоря вообще-то
% люди не знали что уже средневековье началось вы же понимаете да что это назван
% нас гораздо позднее дальше что означает император раз империя то император нужно
% понимать что император это всегда статус статус связанных духовным лидерством
% потому что по своей должности император был принцип сам то есть император
% римской империи по должности принцип а вот император по статусу это тот кто
% чувствует что его государству то есть его людям и землям которые под его властью
% объединены под силам быть защитником и проводником какого-то духовного принципа
% я сейчас буду неважно империи было много империи разные но их объединяет некий
% духовный принцип и соответственно империи называют ту область где этот высший
% духовный принцип организует жизнь всех и каждого в той или иной мере вот это
% очень важная на мой взгляд момент понимание империи потому что традиционно мы
% думаем что империя это какое-то единое государственное образование но надо
% сказать что государства типа вот наших современных государств образовались
% только в 17 веке вот до этого их не было не было четких границ не было четко их
% вот определение где чья власть а уж чем более для такой большущей империи
% поэтому она довольно рано начинает разделяться на дверь и империи ну как она от
% империи одна но чувствуете империи это не государственное образование это скорее
% вот земля и народы объединенные одним принципом духовным что был за принцип у
% императоров римской империи это цивилизация права хорошей дороги и цивилизации
% вот этот принцип который нанесли римляне в мир скажем так христианская империя
% это уже вот империи саду и 

% к концу четвертого века происходит окончательный
% раздел на западную восточную римской империи восточная римская империя она
% организуется вокруг города византий отсюда позднее историки придумали название
% византия понятно византии не называли себя средневековые люди западно римской
% перия так и называлась римская империя и восточная римская империя называлась
% римская империя на долгое время скажем так власть закрепилась за Восточной
% Римской империи, то есть за Византией современной. Она мыслила себя наследницей
% Рима и не предполагала вносить каких-то кардинальных изменений, кроме
% религиозных. Не в законы, не в особые юридические принципы. 
\paragraph{Римская империя}

К концу IV века Римская империя окончательно разделилась на западную и восточную части. Восточная империя, ставшая известной как Византия, воспринимала себя как продолжательницу Рима. Византийцы не стремились к кардинальным изменениям в законах или юридических принципах, изменяя лишь религиозные аспекты.

% Западная Римская империя, так мы будем ее называть, все-таки мы будем следовать
% традиции, которые еще в школе у нас заложены. И Западная Римская империя есть в
% Византии. Так вот, сейчас в Западной обращаемся. В Западной Римской империи в
% ней большое значение сыграл синтез античных, то есть греческих и, главное,
% римских, то есть романских. Они назывались романские народы. Романских народов,
% синтез вот этих вот греческих и романских народов и принципиально новых
% этнических социальных, культурных влияний. Что это за влияние? О них я уже
% говорила. Это германцы. Вот здесь на слайде немножечко другая информация, не та,
% что я говорю, но вы это просто понимаете, презентация это скорее вам на будущее,
% когда вы будете готовиться. Так вот, германцы. Это одна из ветвей единой
% индоевропейской культуры, единой языковой семьи. Средневеков в истории
% псевдомаврики так описывают германскую среду. Он называл германцев белокурными
% народами. Белокуры народы весьма ценят свою свободу, они презирают всякого, кто
% струсит хотя бы немного, отступит в бою. Смерть они тоже презирают, они свирепы
% на поле боя, верхом на коне, в пешем строю. Они очень много пишут про них, да.
% Они ничуть не беспокоятся о том, чтобы принять меры безопасности, выслать
% разведку. В общем, это то, что относится к военным, да, к таким традиционным
% военным принципам, ценностям, ну, то есть такие аксиологические характеристики
% именно людей военных. Но надо понимать, да, среди германцев были военные отряды,
% и они как раз и нанимаются, охотно нанимаются ко всем императорам, ко всем
% князьям. Соответственно, вы наверняка слышали про призвание там князей на
% царство, призвание князей туда, призвание князей сюда. Да, это все понятно. Но в
% Римской империи они были настолько активны, что, надо сказать, к третьему веку,
% вот боевая-то часть римской армии, это в основном германцы. А за гражданами
% Рима, в общем-то, за ними одна функция закреплена. Это лимос, то есть, ну,
% охрана границы. Надо понимать, что германцы --- это не немцы. Вот я так на всякий
% случай говорю, хотя прекрасно понимаю, что люди, знающие историю, им не нужно
% это уточнение, но я на всякий случай уточню. Германцы --- это не немцы. Немцы ---
% это этнос, который потом сложится на основе германцев, причем, ну, не самые его
% лидирующие части, алиманы будут, да, там в основе. Это этническое образование,
% которое в той или иной мере повлияет потом и на становление французов, и на
% становление немцев, британцев, и на становление русского народа. То есть, вот
% как бы это прото-этнос. Германизация римской армии, еще раз, вот подчеркну, это
% важно. Почему это важно? Не только потому, что они полностью изменили способы
% войны. Почему? Потому что римляне, как правило, воевали другим оружием. Они не
% воевали на лошадях, они лошадей только для обоза использовали. Они другие
% тактики выбирали. То есть, германцы оказались более умелыми воинами на тот
% период, и они как бы стали ядром римской армии. Но важно, да вот, что какой-то
% период времени германцы даже поставляли императоров. Это была эпоха так
% называемых солдатских императоров, почти весь Третий век. Эпоха солдатских
% императоров. То есть, понимаете, да, степень влияния этой среды. И, как я уже
% говорила, очень многие среди них были христианами. Поэтому вот примерно весь,
% почти все раннее Средневековье, ну вот как раз вот этот Второй, Третий век,
% Четвертый даже отчасти, но это уже в меньшей степени, христианство в Риме
% называли солдатской религией. Вот такие вот, ну опять же, нарицательно, марки
% Аврелии и Аполлинарии Сидонии довольно презрительно говорили о ней как о том,
% что скоро закончится. Ну это же солдатская религия, Господи. Ну солдат поубивают
% вот там в боях, вот и религия закончится. То есть чувствуете, да, связь? Вот это
% вот напора, умение воевать, и, соответственно, некого духовного принципа,
% который, в принципе, пришел с этими людьми. Да, в это время в Риме уже были
% христиане и не воины, не германцы даже, а римляне, то есть граждане Рима. Они
% тоже становились христианами, потому что христианство распространялось. Но это
% же называется несколько другая история. Это уже деятельность скорее апостолов,
% да, и это уже катакомбное христианство. Если получится, мы о нем поговорим чуть
% больше. Ну, можете просто узнать самостоятельно или как-то там поинтересоваться.
% Рекомендую вам книгу Доминика Бартоломи. Это одна из новых интерпретаций того,
% что понимать под рыцарством, под германством и так далее. Но к тому, насколько
% важны и насколько единообразны образы древнего вот этого этноса, я вам
% фотографию Рихстага ставлю. Вот посмотрите, эти товарищи, которые у нас, скажем
% так, простите, протоэтнос, да. 

Западная Римская империя, как и Византия, стала важной вехой в синтезе античной культуры, объединяя греческие и римские элементы с влиянием германцев. Германцы были частью единой индоевропейской культуры, часто описываемой как <<белокурые народы>>, ценившие свободу и военную доблесть. К третьему веку германцы составляют основную боевую силу римской армии, при этом многие из них становятся христианами. Христианство в Риме в этот период считалось <<солдатской религией>>, в том числе благодаря германцам, которые активно распространяли её. Германцы, среди которых было много воинов, меняли облик римской армии и влияли на развитие будущих народов.

% Значит, условно, средние века делятся на
% романский запад, да, протофеодальный мир. И здесь можно выделить 5-8 век, это
% эпоха мировингов. Именно в этот период происходят все
% пертурбации германцев в Европе. Очень бурная жизнь. И надо сказать, что эта
% эпоха, которую мы называем великое переселение народов 4-7 век, это очень такая
% важная эпоха. И она связана с тем, что не только наемники в римской армии
% становятся как бы активны, ну германцы наемники, но и поселение, вернее,
% население, которое связано с обычной областью хозяйствования, с обычными там, ну
% земледельцев среди них пока еще маловато, потому что они когда сядут, вот они
% станут земледельцами. Ну, то есть это вот здесь перечислены самые разные, как их
% называют, отряды, нет, не отряды, а это было слово. Подскажите, если вспомните.
% Англы, саксы, фризы, готы. Готы, вот это, пожалуй, самая такая интересная...
% Племена? Племена, спасибо. Готы, вес, гот, рост, готы, наиболее интересные
% племена. Почему? Потому что вот императоры очень предпочитали все-таки на службу
% нанимать готов. Ну, видимо, такая наиболее военизированная часть была. В V веке,
% когда особенно активно развернулось одно из негерманских племен гунов, вот
% негерманские племена гуны тоже в тот период активничали на территории
% Причерноморья, и вот они выступили таким бичом для германских народов, погнав их
% в границы Римской империи. Поэтому Атиллый назван бичом Божиим, бич, то есть
% подгоняющий, да? Это его такая, ну, средневековая кличка. И вот эти германские
% народы, их погнали в границы Римской империи. Они достаточно быстро преодолели
% границу. Некоторые императоры их, ну, под этим напором просто пускали, селили на
% какое-то время даже в то, что мы сегодня называем гетто. Ну, знаете, лагеря для
% беженцев, да, вот такое вот. Но, как обычно, лагеря для беженцев в такие большие
% объемы народа долго не выдерживают. И в определенный период там случился бунт в
% одном из лагерей, и войско двинулось, вот, то есть там и войско, и просто, так
% скажем, невоенные люди, они двинулись на Рим, захватили его. Но 476 год,
% знаменитая дата захвата Рима, но самое главное, Рим, то что называется, и до
% этого захватывали, и разграбляли, и так далее. Грабили, простите. Вот, но 476
% год знаменит тем, что последний император Западно-Римской империи, Рому-Аугусту,
% отказался от императорского сана, передал мантию германскому вождю Ада-Акру, и
% дальше всю жизнь прожил на Вилле, получая от него пенсию. Ну, то есть вот власть
% как бы символически, условно, она перешла к германцу. 476 год. Вот она, 5 век,
% это граница между античностью и средневековью.

\paragraph{Романский запад. Протофеодальный мир}

В период с V по VIII век, когда происходит эпоха Меровингов, наблюдается великое переселение народов (IV-VII века). В это время германские племена, такие как англы, саксы, фризы и готы, активно вторгаются в Европу. Готы, в частности, были предпочтительными наемниками в римской армии. В V веке гунны, под предводительством Аттилы, стали угрозой для германских народов, заставив их переселиться на территории Римской империи. В 476 году последний император Западной Римской империи, Ромул Август, передал власть германскому вождю Одоакру и отошел от власти, что знаменует конец Римской империи и начало Средневековья.

\paragraph{Каролингское возрождение}

% Вот 8-10 век, это франки. Роль
% франков повышается. Не надо сказать, что это совсем иная среда, нежели готы. Вот
% готы-воины, да, а франки-торговцы, по большей части. Отсюда и слово франк, как
% денежная единица. Франк переводится в свободный. И чувствуете, здесь такая как
% бы, ну, французская несколько ноточка. Ну, как мы сегодня бы это называли
% французской, тогда, конечно, еще никаких французов нет, естественно. И вот 8-10
% век, это эпоха каролингов. Чем она для нас важна? Это тем, что в этот период вот
% эта вот германская власть получила серьезную спайку с уже тогда существующим
% христианством в виде папской власти. Папа уже венчает на царство. Папа венчает
% на императорский сан, дает статус, простите, не сан, конечно, статус. И еще эта
% эпоха важна тем, что Карл Великий, знаменитый Карл Великий, от кого, собственно,
% пошло даже слово Карл, он сделал очень много для того, ну, не то, что бесплатно,
% специально, но так получилось, чтобы западные и восточные римские империи
% окончательно были разорваны, чтобы христианство было разорвано, случился раскол
% и так далее. Ну, то есть, вот скажем так, почему западная Европа так ценит и
% любит Карла Великого, не потому, что он там слишком много сделал, многие, как
% говорится, императоры, многие короли много чего сделали, но он есть основа
% какого-то такого вот самодостаточности западного мира, самодостаточности Европы.
% Он, по сути дела, разорвал связи с Ближним Востоком, с Византией, и
% единственное, с кем он там поддерживал активно связь, это с арабами, но арабы-то
% три, мусульмане, поэтому тесные спайки, конечно же, произойти не могут. Ну, вот
% венчание Карла императорской короны произошло в 800 году, и это очень важная с
% точки зрения вот самосознания народов Европы дата. Плюс ко всему эпоха Карла
% Великого существенно повлияла на социальный статус, на социальный статус
% рыцарства. 
В VIII-X веках усиливается роль франков. Это была не совсем иная среда по сравнению с готами --- если готы были воинами, то франки в большей степени торговцы. Именно отсюда происходит слово <<франк>> как денежная единица, что связано с понятием свободы. В этот период, эпоха Каролингов, германская власть объединяется с христианством через папскую власть. Папа начинает венчать королей, придавая им статус императоров.

Значение Карла Великого, правителя этой эпохи, заключается в том, что он сыграл важную роль в разрыве связей между Западной и Восточной Римскими империями, а также между христианством на Западе и Византией. Карл Великий создал основу самодостаточности западноевропейских народов, одновременно прервав контакты с Ближним Востоком и Византией. Он поддерживал связи с арабами, но из-за религиозных различий тесных союзов не возникло. Важным событием было венчание Карла в 800 году, что имело огромное значение для самосознания Европы. Также эта эпоха повлияла на социальный статус рыцарства.

\paragraph{Становление рыцарства}

% Вот давайте немножко поговорим про рыцарство. Я сейчас посмотрю
% следующий. Королевское возрождение, можете вот про него тоже поинтересоваться. Я
% сейчас, скажем так, упускаю. Вот некоторые вещи, про которые я не говорю, это
% вам на самостоятельное изучение, либо, ну, поинтересуйтесь на семинарах. А мы
% сейчас перейдем дальше. Становление рыцарства. Почему рыцарство так важно?
% Потому что это военное сословие является высшей аристократической силой эпохи
% Средневековья. Ну, то есть это элита Средневековья. Когда мы будем у вас
% спрашивать, ну, так скажем, кто правил эпоху Средневековья, вы скажете короли,
% естественно. Мы скажем, ну, короли, короли, а кто они были? Короли-то это кто?
% Вот тут обычно заминка происходит, потому что король это рыцарь, причем первый
% среди равных. Тогда королевская власть, это не то, что, ну, вот мы немножечко
% сейчас воспринимаем, такая наследственная, да, вот передача. Очень долгое время
% наследственной передачи не было власти. Была власть сильных. Поэтому королем
% становился только тот, кого выберет дружина. Дружина выбрала, ты первый среди
% равных, ты король. Очень долгое время параллельно существовали две важнейшие
% силы, ну, как бы силы власти. Это король и герцог. Потому что герцог это военный
% вождь. Ну, вот он выбирается на конкретную военную кампанию. На какую-то долгую
% военную кампанию выбирается герцог. Это вот среди древних германцев даже такое
% было. И, соответственно, герцоги, ну, по статусу, они иногда были даже выше
% короля. То есть король, например, так, ну, как невеликий харизматик, и он, так
% скажем, ну, вот как-то хозяйственными делами занимается, какие-то там проводит
% встречи, а правит какой-нибудь герцог в каком-нибудь княжестве. Почему? Ну,
% потому что военный вождь. И поэтому Карл Великий первым делом, по сути дела,
% отменил герцогство как особый статус и ввел статус графов. А графы, они по, как
% бы, уровню тоже высшее знать, но это знать, которое назначается уже королем. Ну,
% то есть вот он присуждает статус графа. Ну, как ты скажешь, и запретил выбирать
% военных вождей. Чувствуете, королевская власть усиливается, усиливается, и
% уменьшается та прежняя социальная, ну, как бы уменьшается статус той прежней
% социальной структуры, которая существовала еще от древних германцев. Когда
% действительно вот это военное сословие, оно самообновлялось по принципу силы и
% авторитета. Ну, то есть если ты заслужил авторитет в боях, то тебя выбирают в
% качестве лидера. А в эпохе Карла Великого уже укрепляется, до этого местами
% возникающая на разных территориях, традиция передачи власти и статуса по
% наследству. Ну, то есть вот если ты герцог, если ты граф, если ты еще какой-
% нибудь там маркиз, то твой статус передается по наследству к сыну. Ну, там
% разные бывают системы, к младшему, к старшему, и в основном чаще всего к
% старшему. А вы понимаете, по наследству далеко не всегда передается харизма,
% сила и что-то еще. Соответственно, вот этот вот наследственный принцип передачи
% эристократической власти, он становится началом конца Средневеков. Началом
% конца. И, конечно, это происходит намного заранее, чем этот конец станет
% очевидным. Кризисы закладываются в самом расцвете. Так вот, один из таких вот
% важных принципов, принцип наследования власти, он становится и началом конца.
% Да. Но смотрите, что такое рыцарство еще, военные сословия, как они живут. Воюют
% много, но надо понимать, при этом Средневековье не является эпохой постоянного
% какого-то вот такого участия в войне. Почему? Вот вам, пожалуйста, Доминик
% Бартоломий, и это подтверждают все другие исследователи Средневековья и
% рыцарства конкретно Средневековья. Он пишет вот что. Как, во-первых, происходила
% война? Вначале выстраивается, в кавычках, это, ну как бы цитата, мир, основанный
% на клятве. Вот если вы смотрели игру сезонов, то там даже отчасти это проявлено.
% Где-то, где-то очень, конечно, так неявно, но тем не менее. Вначале мир,
% основанный на клятве, потом обязательно кто-нибудь да нарушает клятву, ну вы
% понимаете, все же люди, да. И, соответственно, дальше война во имя мира. Вот так
% вот. То есть это не какая-то постоянная агрессия, да. Это вполне себе
% структурированная система, которая вот и позволяла рыцарству существовать.
% Потому что рыцари это всегда воюющая, как бы, часть элиты. Так вот, Бартоломи
% пишет, феодальная война была, по сути, сезонной, потому что ее содержание
% сводилось к набегу на земли противника, к осаде одного из замков. Эти операции,
% направленные против конкретного противника, перемежались общим
% разглагольствованием, когда велись переговоры, интриги плелись и так далее. То
% есть вот что такое война. То есть рыцари не только постоянно махали мечом, да,
% вот война. То есть набег на замок, там, значит, победили или там проиграли, и
% пошли проводить переговоры и вести интриги. В целом, они хотят победы правой
% стороны, но что бы ни обошлось без демонстрации военной доблести. Хотят красивых
% сражений, но без кровопролития. Феодальные войны, следуя очень старым нормам
% посткаролинских вассалитетов и христианства, щадили жизнь и потомство не только
% знати, но и христианского класса, а также очень ценили богатство страны.
% Следствием этого был рост сельского населения, и благодаря этому начался рост
% городского населения вместе с бургами. То есть вот стремление вырезать население
% врагам, ну, скажем так, всех крестьян, там, этого врага и тому подобное, никоим
% образом не касалось Средневековья. Крестьян весьма оценили, понимаете, этот
% ресурс, это ресурс, который вот обеспечивает еду. Нельзя было, как говорится,
% всех порубить, потому что, а кто будет кормить твою конницу, кто будет
% заготавливать сено, кто будет заготавливать мясо людям и так далее. Ну ладно,
% рыцари сами хорошо себе заготавливают мясо, потому что еще одна забота рыцарей –
% это охота. Но, тем не менее, косить траву-то ведь они не будут, да, хлеб убирать
% они тоже же ведь не будут. Соответственно, для этого нужно крестьянство. И
% крестьян действительно щадили, и основные социологические исследования
% показывают, что в эпоху Средневековья крестьянство-то разрослось, весьма
% разрослось. Поэтому, в принципе, жилось-то хорошо, и не было каких-то бунтов,
% таких вот крестьянских серьезных, каких-то там выступлений. Это все позднее, это
% ближе к возрождению, и там у этого есть свои причины. Вот как бы это такая
% социальная договоренность интересная, когда рыцари одновременно и защищают, и в
% то же время постоянно воюют, да, вот это, но подержалось очень долгое время. Это
% социальная система одна из самых долгоживущих, как выяснилось на сегодня. Ну,
% класс еще был странствующих рыцарей, это рыцарские ордена. Большая роль их, как
% выяснилось, становление наследственной аристократии. Например, орден храма, это
% 12 век уже. Вот он объединял две группы воинов, рыцари и сержанты, белые плащи и
% коричневые. Чтобы надеть белый плащ, соискатель должен быть сыном рыцаря,
% происходить от рыцаря по отцовской линии. Чувствуете, да? То есть у
% госпитальеров такое же правило было. Вот, то есть мы можем говорить о ситуации,
% когда не только личные заслуги, но и происхождение начинает влиять на социальный
% статус. Ну, вот это то, чего не было у германцев. 

Рыцарство в Средневековье --- это военное сословие, которое становилось высшей аристократической силой эпохи. Оно было элитой своего времени, а короли, как правило, были рыцарями, то есть первыми среди равных. Ранее, однако, власть не передавалась по наследству --- короля выбирала дружина, и он становился первым среди равных, что создавало сильную военную элиту.

Параллельно существовали две силы: король и герцог, последний, как военный вождь, часто имел больший статус. Карл Великий отменил герцогство как отдельный статус и ввел статус графов, что укрепило королевскую власть и уменьшило влияние прежней военной элиты. Принцип наследования власти и статуса, введенный Карлом Великим, стал основой конца Средневековья, так как не всегда передавалась харизма и сила.

Средневековые войны не были постоянными агрессиями. Феодальная война носила сезонный характер: вначале создавался «мир, основанный на клятве», который часто нарушался, после чего начинались военные действия. Эти действия обычно заключались в набегах и осадах замков, а затем --- в переговорах и интригах. Война в Средневековье имела ограниченный характер и щадила жизни крестьян, так как они были важным ресурсом для поддержания сельского хозяйства и снабжения армии.

Рыцарство также оказывало влияние на становление наследственной аристократии, как, например, в рыцарских орденах. Орден храма, например, требовал, чтобы его члены были сыновьями рыцарей, а это означало, что происхождение стало важным фактором социального статуса, чего не было в германских племенах.

\paragraph{Феодализм}

% Таким образом, мы говорим о
% феодализме. Феодализм. От слова феодум, это средневековый латинский термин
% феодум, владение. Это специфическая система экономических, социальных и
% политико-правовых отношений. В марксистском анализе это общественная информация.
% Чем она характеризуется? Вот список у вас здесь есть, и я думаю, вы его просто
% должны помнить еще со школы, но обновлю, да? Это, во-первых, натуральное
% хозяйство, то есть отсутствие капитала. Что того капитал? Имущество,
% используемое для получения прибыли. Ну, в первую очередь. Это я так очень-очень
% общо говорю. Дальше. Условный характер земельной собственности. Что имеется в
% виду? Земля не принадлежит феодалам. Земля не принадлежит даже королю. Почему?
% Потом позднее поговорим, но принцип был все-таки такого, духовного свойства.
% Земля принадлежит Богу. Никому не принадлежит земля. Соответственно, земля...
% Все только наделяются землей. Король наделяется землей, потому что его армия
% завоевала эту землю. Он как бы получил ее в распоряжении. И он, соответственно,
% потом наделяет кусками этой земли своих верных соратников, то есть рыцарей, да?
% А они, в свою очередь, делят эту землю на какие-то наделы вокруг замка и
% наделяют этой землей тех, кто ее будет обрабатывать, то есть крестьянство. Это
% называется... А, и все вот эти вот люди приносят клятву верности друг другу.
% Христиане клятву верности приносят рыцарям, рыцарь приносят клятву верности
% другим рыцарям более высокого статуса, те, соответственно, королю. Вот это вот
% аммаж, церемония присяги, вассального договора. На ней основана вся система
% экономических, социальных и правовых отношений. Это специфично средневековая
% форма. То есть там нет... Так скажем, вот специальные историки отмечают, что
% крайне редко существовала в средневековье крепостная зависимость. Это очень
% редко, и мы потом с вами выясним, почему, если бы крепостная зависимость была,
% средневековье так долго бы не продержалось. Ну вот, таким образом, государство в
% средневековье, и это слово надо всегда иметь, вот в кавычках держать его, это
% система личной зависимости, которая образуется как политический и экономический
% факт в виде присяги. Клятва верности одного человека по отношению к другому.
% Когда будете рассказывать про феодализм, обязательно этот момент поменять.
% Значит, таким образом, социальная иерархия феодального общества это высший слой
% рыцарства, далее идет слой свободных общинников, и есть еще один слой, это
% клирики, имеется в виду люди, принадлежащие к церкви. Об этом поговорим чуть
% позже. Но опять же, все практически социальное, вот это вот устройство, оно
% зиждилось на идеи, что воины защищают, клирики молятся, ну там все и монахи, и
% как бы священнослужители, христиане трудятся. И по сути дела, ну, всех ведь это
% устраивало, тем более, что какое-то достаточно долгое время можно было
% переходить вот из этих сословий, совершенно спокойно переходили. Ну, не хочешь
% ты быть клириком, ты идешь, хочешь рыцарем становишься, идешь, покупаешь меч,
% учишься воевать, идешь, нанимаешься в какую-то армию, приносишь аммаш, и вперед.

Это система экономических, социальных и политических и правовых отношений, основанная на натуральном хозяйстве, отсутствии капитала и условной собственности на землю. Земля не принадлежала ни феодалам, ни королю; она считалась собственностью Бога. Земля передавалась людям, которые наделялись ею за службу. Король, завоевав землю, наделял ею рыцарей, которые в свою очередь делили землю между крестьянами, что обеспечивало её обработку.

Основой феодальных отношений была клятва верности --- оммаж, или вассальный договор. Эта система зависела от личной верности между людьми, а не от крепостной зависимости, которая была редкостью. Социальная иерархия феодального общества включала три основных слоя: рыцарство, свободных общинников и клириков (церковных людей). Все сословия выполняли свои обязанности: воины защищали, клирики молились, крестьяне трудились. Изначально переходить из одного сословия в другое было возможно, например, стать рыцарем, купив меч и вступив в армию, принося при этом оммаж.

\paragraph{Бурги} 

% в XI-XIII век, Европа становится
% городской цивилизацией. Некоторые историки вот здесь
% ставят высшую точку в истории западного средневековья, но, конечно, но это
% делают именно буржуазной историографией, потому что, собственно, здесь и
% началось начало буржуазной эпохи. Слово буржуазный происходит, слово бук, это
% тип города, который появился именно для профессиональной деятельности, то есть
% исключительно для того, чтобы люди, живущие в нем, работали, не жили, поэтому
% там, бурги, они очень неудобные для жизни. Это просто скопление каких-то домов,
% мастерских и тому подобное, причем потому что, ну, как бы это такое почти
% временное состояние, которое только постепенно потом становилось, ну, уже таким
% серьезными городскими образованиями. Изначально бург, это так по-быстренькому
% сошлись, чтобы, значит, образовать какую-то мастерскую, какую-то там, я эту
% торговую площадь, и вокруг нее вот какие-то мастерские и тому подобное. Поэтому
% все, судя и дело, историки, кроме тех, кто почему-то до сих пор придерживается
% буржуазной историографии, однозначно говорят, что это начало тоже конца
% Средневековья. Так вот, в бургах формируются такие профессиональные сообщества,
% которых человек мыслится в первую очередь человеком определенной профессии. Вот
% сегодня у нас у всех буржуазное сознание, потому что, если вас, ну, как бы, вы
% знакомитесь с человеком, он спрашивает, ну, вот, а ты кто? то вы отвечаете, кто
% вы по профессии, правильно? Вы не ответите, что я там хозяин трех кошек,
% условно, ну, или что-нибудь такое, как вы себя-то идентифицируете, что у вас-то
% там в сознании? Нет, потому что социальная, ну, как бы, структура такова, даже
% не структура, а социальный дух таков, что вы есть тот, кто вы есть по профессии.
% Я вот работаю, так и так-то говорим, говорим мы все, да? Так вот, это типичное
% мировоззрение бурга, да? Совсем иначе было бы в других территориях средних
% веков, средних веков, ну, там, где-нибудь в замках, где-нибудь в каких-нибудь
% там, ну, в общем, в других небугах, да, местах. То есть эти в бургах создавались
% цеха и гильдии, это корпорации, объединения по профессиональному принципу.
% Гильдии, это в основном торговые, цеха, это мастерские. Они имели не только
% экономическую, такую, территорию, коммерческую функцию, но также управляли
% жизнью города. То есть город, ну, если человек рождался, его цех или гильдия как
% бы называла определенным именем, да, то есть совместно вот так вот. Когда он
% умирал, гильдия оплачивала похороны. Свадьба тоже, если, ну, вы значимый человек
% в гильдии, свадьба тоже, это было дело гильдии или там цеха. Можно привести
% пример одного, одно из таких, ну, важнейших средневековых образований бургов,
% которые жили вот исключительно за счет гильдии, это Венеция. Девиз Венеции был
% даже таков, вначале мы венецианцы, то есть, по сути дела, торговцы, да, а потом
% христиане. Но на самом деле любая торговая гильдия могла бы, наверное, взять
% такой девиз себе. Вот так формируется новый
% социальный слой буржуа. И какое-то время, 13-14, даже отчасти 15 век, недалеко
% не все города чувствуют свою силу, они пока еще вот такие вот маргинальные
% образования на территории средневековья. Но постепенно, вот спустя 2-3 века
% после своего появления, именно буржуа будет новой элитой, общества, которая
% будет противостоять старой элите общества, буржуа против рыцарей. Это такой вот
% будет фундаментальный раскол общества и очередная трансформация, очередной
% кризис. 

С XI по XIII век Европа становится городской цивилизацией. Этот период некоторые историки рассматривают как вершину Средневековья, особенно в буржуазной историографии, так как здесь начинается эпоха буржуазии. Слово <<буржуазный>> происходит от <<бург>>, поселение типа города, созданного для профессиональной деятельности, а не для жизни. Бурги представляли собой временные поселения с мастерскими и торговыми площадями, которые позже становились настоящими городами.

В бургах формировались профессиональные сообщества --- цехи и гильдии. Люди начали идентифицировать себя через свою профессию, что стало характерным для буржуазного сознания. Цехи и гильдии не только решали экономические вопросы, но и управляли социальной жизнью города: определяли имя человека, организовывали похороны и свадьбы.

Примером бурга, процветающего за счет гильдий, является Венеция, чей девиз звучал как: <<сначала мы венецианцы, потом христиане>>. Сначала города были маргинальными образованиями, но с течением времени буржуа стали новой элитой, противостоящей старой аристократии, и этот процесс стал основой для фундаментального раскола общества.

% и вот именно в поздний период, когда рыцарство уже живет в настроении
% медикаданса, тогда мы с вами встретим тех рыцарей, которые вот нам кино
% показывают, описывают в самом разном образом, но не как бы исторические
% свидетельства, а такие вот околохудожественные. Это вот весь закованный блатной с
% вот такой вот рыцарь, да, но надо сказать, что рыцари, они были все-таки вот
% такими, потому что воевать в этом крайне трудно, вот они, да, то есть максимум
% на те кольчуга, а это вот уже для турниров, вот это турниры. А турниры это
% далеко не все в Средневековье, это самое-самое позднее. Вот этот культ
% прекрасной дамы, турниры средневековые, это уже такие вот причуды, которые
% характеризуют декаданс, то есть упадок рыцарства, когда рыцарство уже не имеет
% значения серьезного, ни как военная сила, не как социально-политическая сила,
% уже игрища, игрульки. 

% Так, международный контекст Средневековья, очень важный
% контекст, чего тут у меня открылось, извиняюсь. Так вот, конечно же, самое
% главное перипетии в международных отношениях это в Средневековую эпоху, это
% контакты разного рода контакты с мусульманской империей. Немножечко о
% мусульманстве. Мусульманство, ислам, более точное название, одна из трех мировых
% религий. Магомед или Мухаммед мыслил своей миссией восстановления подлинной
% веры, то есть, как все мусульмане прекрасно знают, это вера, которая описана в
% Ветхом Завете, ну или просто в Библии, как мы можем сказать, это вера Ноя,
% Авраама и других пророков, которая, эта вера, по мнению Магомеда, была искажена
% иудеями и христианами. Ну то есть, как всегда, когда новая религия
% образовывается, она находит какой-то старый источник и, соответственно, обвиняет
% в каких-то искажениях уже существующие на основе этого источника другие
% религиозные традиции. Так частенько бывает. Так вот, Магомед излагал свои учения
% в кратких изречениях, которые, будучи собраны, составили священный текст ислама,
% книгу Алкарам. Магомед еще, как говорится, только он вот вошел, я всю историю
% христиан, простите, мусульманство рассказывать не буду, тоже, надеюсь, знаете
% или поинтересуетесь. В свое время Магомед отправил послов в Константинополь с
% предложением покориться исламу. Ну вот как бы, типа, ну вот она, истинная
% религия Ноя и Авраама, поэтому, ну давайте все-таки ее разделять. Но, конечно
% же, император Иракли отверг это предложение, и Магомед начал военное действие.
% После его смерти в 632 году войну продолжил его преемник Абукерк, затем Амар, и
% вот наместники пророка, это калифы, с 7 века завоевали очень большую территорию,
% просто огромную. И мы подчинились, ну, то есть, наместникам, калифатам,
% подчинились, разным калифатам, подчинились Сирия, Египет, Палестина, Испания,
% другие страны. Для европейской истории огромное значение и в то же время для
% развития науки имеет вот эта вот часть, то есть, испанские земли, захваченные

% мусульманами. Это мусульманская Испания или так называемая Аль-Андалуз.
% Культурная политика первых калифов была, по сути дела, основана на уничтожении.
% Ну, то есть, вначале идет волна такого разрушения. Движущие силы, потому что
% были достаточно такие необразованные бедуины, но ситуация изменилась, когда
% мусульманскую культуру влились в интеллектуальные силы Сирии. Да, я понимаю,
% сегодня Сирия на слуху, но мы будем говорить о той средневековой Сирии со
% столицей в Антиохии. Это регион, являющимся важнейшим для становления Лижнего
% Востока, для становления мусульманства и христианства. В IV-V веке именно
% сирийские христиане составляли большинство на Древнем, на Востоке, просто на
% Древнем, просто на Востоке. В сирийском христианстве имелась древняя традиция
% учености. Вот понимаете, Сирия находилась на том месте, где пересекались все
% культурные и торговые пути, и огромным значением пользовалась Эдесская школа.
% Вот Эдесс, да, это то, что от нее осталось средневековый Эдесс. Эдесская школа
% формировала своего рода мост между греческим и восточным миром. Почему? Потому
% что там хранились и активно изучались, переводились тексты античных философов,
% активно развивалась математика, гуманитарные дисциплины. Именно сирийцы перевели
% Платона, Аристотеля и так далее на все восточные языки. И поэтому Восток узнал о
% Платоне, Аристотеле и других философах античных. А так как арабский язык основан
% на основе сирского, имеется в виду алфавит, то есть письменный арабский язык
% сформировался на основе сирского алфавита, сирийского по-другому. Классический
% арабский язык вот это он. Соответственно, арабская культура впитала все вот это
% античное знание через Эдесскую школу, по сути дела, через Сирию. И когда уже
% западная культура, будучи европейская, встала на ноги там после долгих периодов,
% так скажем, небольшого интереса к философии, так вот, после вот этого периода,
% когда она встала на ноги, там возникли еще некоторые, потом мы о них поговорим,
% сюжеты, но тем не менее она получила знание об античности именно из рук арабов,
% вот из той самой мусульманской Испании. Вот здесь было скопление, большое
% скопление ученых и арабов, и евреев, и они имели с собой тексты Платона,
% Аристотли и так далее. Единственное, что еще нужно сказать о межкультурном таком
% взаимодействии в эпоху Средневековья, наверное, стоит говорить о крестовых
% походах, но на самом деле я не хочу говорить о них много, потому что эта тема
% огромная, большая, и очень противоречивая, потому что это был феномен, где люди
% проявляли как свои лучшие, так и свои худшие качества, где западная цивилизация
% проявила как свои лучшие, так и свои худшие качества, ну то есть, скажем так,
% либо начинать про них говорить, либо не начинать, но все-таки когда вы будете
% интересоваться походами крестовыми, я вам, конечно, рекомендую поинтересоваться
% четвертым крестовым походом, это, по сути дела, разграбление Византии, и
% некоторыми такими, знаете, ну очень для ними специфичными походами, многимя
% крестовым походам детей, да, действительно, в 13 веке крестовый поход детей.
% Заканчиваю обзор истории средневековой Европы, в результате вот этих вот
% крестовых походов Византия сильно ослабела, в результате в первую очередь
% четвертого похода, который был не столько против мусульман, сколько против
% Византии. Кстати, его оплачивала Венеция полностью, поэтому до сих пор на
% венецианских площадях красуются украденные в Византии, имеется в виду в
% Константинополе, статуи и другие драгоценности. Я бы сказала, что главный враг
% Константинополя, главный враг Византии, это, конечно, Венеция, несмотря на то,
% что Венеция была тогда православной. То есть вот это буржуазный подход, он
% изначально разрушал все, что казалось на века, в Византии это казалось на века.
% Ну, в общем, в 1453 году на эту ослабленную уже в Византии нападает Османская
% империя. Началась осада Константинополя. Папа призвал к помощи Константинополю,
% то есть Самбона. Вот он в очередной раз просил западных рыцарей собрать войско и
% пойти на защиту Константинополя, то есть христианской тоже святыни и так далее,
% но европейскими уже не откликнулся. Было столько этих походов, что все по
% призыву папы, что уже никого и не появилось. Зато помощь от Запада пришла
% султану, тому самому, кто нападал на Константинополь, султану Мехмеду II.
% Оружейник по имени Урбан создал для Мехмеда огромную бомбарду, пушку под
% названием базилика. Осада продолжалась два месяца. И вот тут, конечно, все
% объединились. Византия это вечная внутренняя грязня, а тут все объединились. Но,
% в общем, спасти око вселенной, как назывался Константинополь, не удалось.
% Император Константин XI предпочел умереть, бросившись в эпицентр боя. И
% последняя битва велась в краме Святая София, где в этот момент не
% останавливалась литургия. И существует легенда, что тогда, когда турки уже
% ворвались в собор, священники растворились в стенах. То есть, может быть, там
% был какой-то тайный ход, но, во всяком случае, легенда так вот красиво закончила
% историю, по сути дела, Средневековья. Потому что XV век это и конец
% Средневековья. Не то, чтобы Западная империя пала только из-за того, что
% Константинополь был захвачен, но, тем не менее, это все-таки уже был конец той
% эпохи, которую мы называем Средневековья. Хотя, еще раз подчеркну, Западный мир
% был разрушен не внешним нашествием, а внутренними проблемами, а именно
% проблемами в рыцарстве и вот этой новой социальной силой, которая появилась в
% Бургах. 

\paragraph{Международный контекст Средневековья}

Международный контекст Средневековья был важен, особенно отношения с мусульманской империей. Ислам, одна из трех мировых религий, был основан Мухаммедом, который считал свою миссию восстановлением истинной веры, и обвинял иудеев и христиан в искажении оригинальных учений. Мухаммед отправил послов в Константинополь с предложением принять ислам, но император Ираклий отверг это предложение. После смерти Мухаммеда, начиная с 632 года, калифы продолжили завоевания, включая Сирию, Египет, Палестину и Испанию.

Мусульманская Испания (Аль-Андалус) стала важным центром для науки. Влияние сирийских ученых (Эдесской школы) было значительным, так как именно через них арабская культура усвоила античное знание, в том числе работы Платона и Аристотеля. Арабский язык был основан на сирийском алфавите, и это способствовало распространению античных знаний.

Крестовые походы, хотя и противоречивы, играли важную роль в международных отношениях. Четвертый крестовый поход в 1204 году привел к разграблению Константинополя, ослабив Византию. Венеция, финансировавшая поход, унесла драгоценности из Константинополя, что стало причиной длительного конфликта с Византией.

В 1453 году, после ослабления Византии, Константинополь был захвачен Османской империей. Несмотря на призыв папы помочь, западные рыцари не откликнулись. Османский султан Мехмед II, при поддержке оружейника Урбана, осадил город. После двух месяцев осады, император Константина XI предпочел умереть в бою. Константинополь пал, что стало концом Средневековья, хотя проблемы Запада были внутренними, связанными с рыцарским сословием и восходящей буржуазией.

\section{Христианство как основа мировоззрения европейского средневековья}

% Основные онтологические, гносеологические, антропологические принципы средневековой
% картины мира. Сразу же подчеркну, что мы не будем с вами говорить о каких-то
% глубинных принципах христианства. Для этого наша лекция, ну то есть можно было
% бы, но наша лекция для этого недостаточно. И главное, мы ведь не будем
% спрашивать на экзамене у вас каких-то тонких моментах христианства. Наша задача
% показать, как христианство повлияло на становление науки. Оно очень серьезно
% повлияло. Но все-таки общие какие-то положения христианства, самые общие, вы
% должны знать. Это общекультурные позиции. Так же, как и о мусульманстве. Вот я
% вам рассказала, тоже нужно знать. Это как бы общие такие базисные знания. Любой
% образованный человек, не верующий, не занимающийся гуманитарными науками, не
% особенно любящий историю, просто должен знать. Так вот, что нужно говорить о
% христианстве, когда мы говорим в качестве общих понятий. 

\textbf{Христианство}~---~это мировая монотеистическая теистическая авраамическая религия.

\begin{itemize}
    \item мировая, метанациональная, т.е. не связанная с одним этносом или национальностью (как, например, связан иудаизм);
    \item монотеистическая, т.е. исповедуется единобожие;
    \item теистическая, т.е. Бог понимается как личность, в отличие от пантеизма, где  Бог присутствует везде --- в природе, людях и вещах;
    \item авраамическая: Авраам --- ключевая фигура в некоторых религиях (христианстве, исламе и иудаизме); в Библии он описан в книге Бытие как отец всех верующих, считается основоположником единой духовной традиции;
    \item религия, по Лактанцию, объясняется через слово <<легерэ>>, что означает связь, а <<религерэ>> --- это восстановленная связь с Богом.
\end{itemize}

% теистическая с этим сложнее. Если
% будут сложности, поизучайте или обратитесь на семинар преподавателям.
% Теистическая противопоставлена пантеистическому. То есть, смотрите, бог-
% личность, некая собранность в одно начало. Пантеизм это растворенность везде.
% Бог везде. В природе, во мне, в тебе, вот это вот в вещах, это пантеизм. Есть
% еще панентеизм, это немножечко другое, это уже такие детали пантеистического,
% пантеистических религий. А вот теистическое бог понимается как личность. Надо
% сказать, что даже слово личность появилось в лексиконе только тогда, когда
% обсуждалась христианская догматика. до этого было слово персона, до этого было
% слово фигура. А слово личность это новое изобретение именно христианского
% богословия. Но это русское понимание, не русское его транскрипт, а перевод
% русский, а греческое слово ипостась.  

% Это авраамическая религия. Авраамических религий у нас несколько. Авраамическая религия, кроме христианства, еще мусульманство и иудаизм. То есть Авраам один из главных персонажей Библии, он в первой книге описан, его как бы жизнь и события описаны в первой книге Библии Бытие. Так вот, Авраам отец всех верующих, отец множества
% слова Авраам. Вот тут написано изначально Авраам, потому что в Библии основным
% лицам священной истории дается как бы дополнительное имя, оно дается Богу. То
% есть маму, папу назвали Авраама Авраамом, а Бог дал ему имя Авраам. Ну и так
% дальше он еще потом Авраама тоже переименовал, дает им другие имена. По как бы
% своему, ну как сказать, по статусу, наверное, это библейский патриарх,
% основоположник единой духовной традиции, но не то, чтобы в Библии не было
% указано каких-то других источников монотеистической духовной традиции. И есть
% довольно загадочная фигура Мельхиседек, цар Мельхиседек, который благословляет
% Авраама. По сути дела он говорит ему ты верной дорогой идешь. То есть по сути
% дела как бы были еще, видимо, фигуры, но тем не менее центральной выбран именно
% Авраама. 

% Ну и слово религия тоже стоит разобрать.
% Религия вот у Цицерона, он говорит так, религия происходит от понятия эльгерэ,
% то есть собирать. И вот его цитата, те же, кто добросовестно, многократно
% переработал и словно заново собрали все, что связано с почитанием богов,
% именуется религиози или религерэ. то есть у Цицерона религия связана с
% собиранием каких-то данных, то есть как почитать богов. Но вы чувствуете, что
% это относится к языческой религии, то есть это римский подход, когда понтифики
% могли знать, какому богу что нужно принести, в какой храм и так далее. А вот
% монотеистическая религия, ее как бы понимание, основываясь также на римском
% слове, объясняет лактанцией. Религия происходит, это слово легерэ, связь,
% легерэ, связь, то есть а религерэ это возвращенная связь, то есть слово религия
% означает возвращенная связь с богом. 
\paragraph{Библия}

% В христианстве священным текстом является библия. Как переводится
% библия? Книги. Не книга, да, книги. Поэтому, ну вот у вас на экзамене будут
% такие вещи спрашивать, какие-то понятия, что называются. Библия христианская
% делится на две основные книги Ветхий и Новый Завет. Ветхий Завет, то есть часть
% еврейской Библии, это пророчество о пришествии Мессии. То есть там много книг и
% их суть это пророчество о пришествии Мессии. Это общая тема для иудаизма и для
% христианства. Но принципиальная разница заключается в том, что Мессия по
% иудейским канонам придет еще когда-нибудь и его функция будет совсем иная, а
% Мессия по иудаизму просто восстановит царство евреев и они будут владычествовать
% на земле. Мессия. Это самое грубое объяснение. Мессия в христианском смысле
% означает спасение. Спасение мистического свойства. То есть спасение, связанное с
% личной жизнью и приобщение к Богу не в знании, не в понимании, а
% непосредственно. Так вот, самое слово Мессия это на греческий язык переводится
% как Христос. Поэтому Иисус Христос это не, как вы понимаете, не имя, фамилия,
% это имя и признание, что именно этот человек тот Мессия, который предчувствован
% и предсказан Ветхим Заветом. 

В христианстве священным текстом является Библия, что в переводе означает <<книги>>. Библия делится на Ветхий и Новый Завет. Ветхий Завет, часть еврейской Библии, включает пророчества о пришествии Мессии. Для иудаизма Мессия --- это будущий спаситель, который восстановит царство евреев. В христианстве же Мессия --- это Иисус, который приносит спасение, связанное с личной жизнью и приобщением к Богу. Слово <<Мессия>> на греческом переводится как <<Христос>>, поэтому Иисус Христос означает признание его как Мессии, предсказанного Ветхим Заветом.

% А вот Новый Завет это уже свидетельство о том, что
% именно Иисус является тем Мессией и Богом. Соответственно, это называется
% синоптический Евангелие от слова синоптик, свидетель. То есть свидетельства. Они
% были записаны не одновременно с жизнью Иисуса, они были записаны позднее в
% общинах, которые образовались вот для того, чтобы помнить Иисуса, помнить Его
% учения и так далее. И были записаны четыре основных Евангелия, остальные там
% есть еще Евангелия, но они апокрифические, то есть они не взяты в канон. Поэтому
% по поводу у них существуют серьезные такие вопрос признания их всегда открыт.
% Ведь для церкви он закрыт, а для некоторых людей он по-прежнему открыт. Дальше
% книги Евангелия включают еще несколько других книг. Одна из самых известных из
% них это так называемый апокалипсис, которая переводится просто книга откровения.
% Слово апокалипсис откровения. Надо сказать, что ее долгое время, очень долго не
% включали в канон. Она вызывала серьезные сомнения у многих отцов церкви, но все-
% таки ее включили в канон. А не включали это по какой причине? По причине
% невозможности толкования. И на самом деле она до сих пор не толкована. Нет
% однозначного толкования книги апокалипсиса. Есть лишь версии. Даже у самой
% церкви есть только версии. Поэтому сказать точно, что кто-то понял, что же было
% предсказано Иоанном богословом в отношении последних дней человеческой
% реальности, ну, это, скажем, будет преувеличением. Книга еще не исполкована.

Новый Завет --- это свидетельство того, что Иисус является Мессией и Богом. Его основа --- синоптические Евангелия, написанные не в период жизни Иисуса, а позже, в общинах, которые его помнили. Включает четыре канонических Евангелия и несколько апокрифических, не признанных каноном. Одним из известных текстов является Апокалипсис (Книга откровений), которая долго не входила в канон из-за трудностей с толкованием. До сих пор нет однозначного понимания этого текста, и у церкви есть лишь разные версии трактовок.

Христианская община называется экклессия, что на русский переводится как церковь. Таким образом, церковь в своём основном отношении --- это община.

\paragraph{Раскол церкви (Великая схизма)}

% Очень важная дата, раскол, вы простите,
% раскол христианской церкви на два лагеря, на две конфессии. Православная и
% римско-католическая церковь. Конечно, если вы будете слово переводить
% католическое, то вы переведете его как ортодоксальное, правильная вера. Но и
% православная церковь называется тоже ортодоксальной. Ну, тут вот есть, да, такие
% небольшие сложности, но закрепилось понимание, что есть римско-католическая
% церковь и есть православная. Почему она произошла? Почему произошел этот раскол?
% Тема, конечно, очень сложная. Напряжение довольно формировалось долго, не в XI
% веке, но в XI веке католическая церковь внесла в символ веры два слова. Символ
% веры это и есть догматика христианства. Символ веры это догматика. Это несколько
% фраз. Поэтому, когда вы говорите согласно догматики христианства земля там
% вращается вокруг, не, солнце вращается вокруг земли, то это полный бред.
% Догматика христианства не говорит ни о земле, ни о солнце, ни о физике, ни о
% политике, ни о том, как там креститься или там не креститься, носить крестик.
% Это все не христианская догматика. Для этого есть другое понятие, в частности,
% понятие керидма. А христианская догматика это несколько фраз, которые объясняют
% то, как объясняют, просто обозначают то, как понимается Христов, как понимать
% Христа, богом или не богом, вот это вот. Так вот, когда была внесена добавка в
% символ веры, где было сказано, что Святой Дух исходит не только от Отца, но и
% Сына, Филиокве, вот эта вот фраза, Филиокве, фраза, слово, вот это слово
% послужило таким толчком для окончательного раскола. Раскол означает то, что
% Константинопольский патриарх не упоминает на во время, службы, во время
% Евхорестия не упоминает уже в качестве благословения не упоминает патриарха
% римского, а римский патриарх, иметь святую Папу римский, во время Святого не
% упоминает Константинопольского патриарха, ну и все, соответственно, кто-то
% присягает, другие патриархаты присягают верности римскому патриарху, а кто-то
% остается в пределах верности, патриарху Константинопольскому. Ну, как мы
% понимаем, потом появились и другие патриархаты, в частности, патриархат русской
% церкви, но это уже другая история. Кто же внёс этот филёк в это слово, кто внёс
% в символ веры? Вообще-то тот самый Карл Великий, да-да-да. Казалось бы, какой он
% к этому имеет отношение? Он, будучи очень обиженным на восточных правителей, на
% императрицу Ирину, которая отказала ему в сватовстве, он сватался от ней, он,
% будучи очень обиженным на византийских императоров, там ещё были какие-то
% проблемы, о них долго говорить. В общем, обида была серьёзная, и он, будучи
% умным человеком, он понял, что бить надо прямо, так вот скажем, в самое сердце.
% Самое сердце – это религия. И он своим францином-богословом поручил написать
% новый символ веры, в который вот это включил, просто какую-то вот такую деталь.
% Они написали, будучи верными своему королём, и предложили папе Римскому в
% качестве, ну вот давайте такое изменение сделаем. Папа Римский пришёл в ужас. И
% надо сказать, что в Ватикане до сих пор выбита стелла, где символ веры, вот он,
% тот, который православие использует, то есть старый, без филёква. Но так как
% Римская католическая церковь всегда была очень политически ориентирована, то
% недолго она сопротивлялась и приняла филёква. 
% Раскол очень серьёзно повлиял на все культурные традиции. 

Христианская церковь разделилась на две конфессии: православную и римско-католическую. Если переводить слово <<католическая>>, то оно означает <<ортодоксальная>> или <<правильная вера>>, но и православная церковь тоже называется ортодоксальной. Раскол произошел в XI веке, когда католическая церковь добавила в символ веры фразу <<и Сына>> (Филиокве), утверждая, что Святой Дух исходит не только от Отца, но и от Сына. Эта фраза стала толчком для окончательного раскола. Константинопольский патриарх стал не упоминать Римского Папу, а Римский Папа --- Константинопольского патриарха. Это привело к присяге верности римскому патриарху одними, а верности Константинопольскому --- другими патриархами.

Фраза «Филиокве» была добавлена по инициативе Карла Великого, который был обижен на византийских правителей, особенно на императрицу Ирину. Он поручил своим богословам внести эту фразу в новый символ веры, предложив Папе Римскому. Папа пришел в ужас, но из-за политического давления Римская католическая церковь приняла это изменение. Раскол оказал глубокое влияние на культурные традиции.

\paragraph{История развития христианства}
% истории развития
% христианства. Но вот здесь вот основные имена. Не будем мы вас жёстко
% спрашивать, что там тертулиан такой. Но знать эти имена, да, знать нужно. Вот
% Патристика – это греческие всё имена. Афанасий Великий, Ефрем Сирин. Чувствуете,
% да, из Сирии человек. Василий Великий. Вот он как раз, имеется в виду, Василий
% Ефрем Сирин – это как раз один из таких великих переводчиков и одновременно
% отец, учитель церкви. Вот. Римские, собственно говоря, уже, ну, латинские такие
% отцы церкви – это вот Августин Аврелий, он один из главных у них, то есть такой
% вот. Ну, он принят в качестве отца церкви и даже в православии. Что о них нужно
% сказать? Мы не будем их учения изучать, тем более, что их учения, они чисто
% богословские. И надо сказать, что эти люди, которые просто были философами,
% пришли, скажем так, на помощь церкви, когда понадобилось разрулить очень многие
% проблемы на вот как бы возникшие, на непонимании. Да как Христа-то понимать? Он
% человек, который стал Богом? Или он Бог, который притворился человеком? Как это
% понимать? И вот это вот непонимание послужило очень серьёзными проблемами уже
% социального плана. То есть люди начали воевать, и император Константин Великий,
% он, по сути дела, сделал то, что единственное мог сделать император. Он созвал
% собор. Говорит, давайте все, кто имеет такую вот духовную, я не скажу власть,
% духовную харизму, духовный статус христианина, настоящего христианина, те
% соберутся и будут обсуждать эти богословские темы и принимать решения. На первый
% собор были приглашены, это был, конечно, наверняка очень зрелищный собор, потому
% что были приглашены мученики. Первый собор, четвёртый век. Тогда ещё очень
% многие христиане, ну, это выжившие мученики, отсидевшие в тюрьме, там, спасшиеся
% из этих рвов со львами и так далее. То есть это были покалеченные люди
% физически, но не сломленные духом, как вы понимаете. По-другому в этих тюрьмах-
% то и не выживают. То есть люди с жуткими там, какими-то такими, знаете, ну,
% шрамами. И вот эти вот мученики и другие, заслужившие, вот, скажем так,
% высочайшее уважение именно как христиане, они составили первый Вселенский собор,
% который решал очень многие проблемы. 

Основные имена в истории христианства: Иустин Философ, Тертуллиан, Ориген, Афанасий Великий, Ефрем Сирин, Василий Великий и Августин Аврелий. Эти люди были философами, которые помогали решать богословские проблемы, например, как понимать Христа --- как человека, ставшего Богом, или как Бога, ставшего человеком.
Император Константин Великий созвал первый Вселенский собор в IV веке, чтобы обсудить эти вопросы. На соборе присутствовали мученики, выжившие в тюрьмах и сражениях, и они помогли решить многие теологические проблемы церкви.

Император Константин Великий не был христианином, он крестился на смертном одре, но его мать, Елена, была христианкой. Константин близко общался с христианами, поскольку большая часть его армии, включая личную охрану, состояла из готов, которые были христианами. Чтобы прекратить гонения на христиан, он предоставил христианству законный статус в Римской империи. Он не сделал христианство главной религией, но разрешил его исповедовать. 

Главной и государственной религией христианство стало только при императоре Юстиниане в VI веке.

\paragraph{Влияние на онтологические основания}

Мировоззренческий принцип, который связан с эпохой Средневековья~---~это теоцентризм. (Это \textbf{НЕ} власть церкви, которая дескать заставила людей писать огромное количество текстов, посвящённых богословской проблематике, вообще быть верующими, исповедовать христианство.) 

% Когда, вот, например, князь какой-нибудь принимал христианство, и его
% дружина принимала христианство, то его земледельцы тоже довольно быстро
% принимали, хотя в душе оставались совершенно не язычниками. Их называли даже
% погани. Погани, то есть, ну, поганусы. Древнее слово, которое означало, потом,
% вернее, стало означать земледельца или просто сельского жителя. А в армии
% римской оно означало человека, склонного к дезертирству. Поэтому вот эти вот,
% как бы, скажем, земледельческое христианство, которое принято, ну, только
% потому, что властитель принял, ну, как вы понимаете, да, достаточно безопасно
% тоже перестать бегать и поклоняться идолам. Ну, как-то, сами понимаете, да, не
% очень. Вот. И, соответственно, вот это вот христианство, оно, конечно,
% заполонило после IV века, когда многие стали свободно исповедовать христианство,
% власть имущие вот эти вот, ну, высокие эшелоны населения. Вот они-то как бы
% исповедовали искренне, причём настолько искренне, что многие свидетельства
% заставляют даже умиляться. Как вот эта знаменитая сцена, которая описывает
% летопись одного из этих вот мировинских князей, где им вот рассказывают, но они
% же не все умеют читать. И вот им приходит там один из лидеров, духовных лидеров,
% монах, будучи потом, вдруг там стал епископом, и вот он рассказывает им о
% распятии Христа, рассказывает о том, вот, ну, всю вот эту историю, о том, что мы
% что-то поймали, там, все остались. И вот момент, когда, ну, он рассказывает о
% распятии, вся дружина соскакивает, хватается за мечи и кричит о том, что вот нас
% там не было, нас мы бы отбили, мы бы, ну, понимаете, да? И тот пытается
% объяснить, что, ну, простите, ну, так нужно было, это же замысел Божий, так
% было, иначе бы спасли. Ну, то есть это прям, ну, ну, народ просто, да как, ну,
% ну, за жены. Вот, понимаете, да? То есть вот это воодушевление, это никакой...
% Что-то... Их всех заставил? Давайте вот их заставить, да? Вы можете себе это
% вообще представить? Я нет. И поэтому давайте вы тоже не будете бить всякую путь.

% Так вот, власть, конечно, есть духовного свойства. Имеется в виду, что вот это
% теоцентризм. Люди сами тянутся. Тянутся. Либо тянутся к власти мужчин,
% чувствуете, да? Которые в это время тестили. Дальше. 

Онтологические основания:
\begin{itemize}
% Во-
% первых, библейский креационизм. Что это такое? Ну, я думаю, вы слышали о
% креационизме как о позиции, которая говорит о том, что мир створён Богом. Но это
% не самое главное. Все языческие религии говорят о том, что мир сотворён так или
% иначе богами. Но именно христианский креационизм говорит о том, что, ну, в
% смысле, библейский, не только христианский, но библейский креационизм говорит о
% том, что Бог сотворил мир из ничего. То есть основание вещей творческая воля.
% Григорий Нисский говорит о том, что если бы мы познали вещь в её в самом внутри,
% в самом сущностном свойстве, мы были бы ослеплены той силой, которая эту вещь
% создала. Мы были бы ослеплены самим актом существования. Вот этим актом
% существования наделяет весь мир именно Бог. Соответственно, смотрите, ну,
% понятно, что Бог будет первичным интересом, а материя и всё материальное будет
% несколько вторичным. 
\item Библейский креационизм утверждает, что мир был создан Богом из ничего актом творческой воли. Григорий Нисский отмечает, что познание сущности вещей ослепляет, так как их создателем является Бог. В этом контексте Бог является первичным, а материя --- вторичной.

Христианство отвергает магию, поскольку она предполагает контакт с духами природы, что противоречит вере в одного Бога. Маги, по мнению христиан, могут контактировать только с падшими ангелами, что считается греховным. Несмотря на попытки магов утверждать, что они взаимодействуют со <<светлыми>> ангелами, это остаётся спорным вопросом.

\item Провиденциализм: линейная концепция истории и невозможность ограничения воли Бога любого рода необходимостью.
\end{itemize}
% При этом никоим образом не проводите вот эту вот позицию,
% будто бы богословы когда-то говорили, что материальный мир презренен в отличие
% от мира духовного. Такого ни один нормальный философ вообще не скажет, а
% богословы тем более, тем более христианские. Ведь Христос воплотился, Он принял
% эту материю. Как же может христианский богослов сказать, что материя --- это зло?
% Ну, это что ж получается? На Христа наехать? Ну, как-то глупо. И вам,
% соответственно, не стоит повторять. 

% То есть, смотрите, Бог в христианском
% миропонимании, Он выше философской оппозиции бытие-небытие. Он не есть бытие, Он
% не есть небытие, Он выше вообще вот этих категорий. А стало быть, эти категории
% опускаются на несколько более низкий уровень интереса, внимания. Он тот, кто
% дарует бытие, творит из ничего. 

% Дальше. Отсюда же неприятие магии в христианском
% мировоззрении, которое, тем не менее, вот долго вообще пыталось определиться. Мы
% вообще как бы с магией дружим или не дружим? Очень трудно. Мы про это поговорим
% чуть позднее, когда про университеты будем говорить. Но тем не менее, смотрите,
% любой маг по мысли древних людей контактирует с какими-то силами, которые там
% вот в природе, с какими-то душами. Душа растения, душа камня, душа звезды и так
% далее. То есть маг умеет находить с ними общий язык и как бы заклинать их.
% Заклинаю тебя и вот что там происходит. То есть это не... Магия, это надо ее
% понимать как бы исторически более верно, как... Вот не как Гальпотт, да? То есть
% это не просто палочка что-то там делает и плюс какое-то слово. А это контакт с
% какими-то духами. Но раз Бог один, то стало быть, ну а с какими вы там
% собираетесь еще контактировать духами? Только получается с ангелами можно
% контактировать, которые имеют определенную властность материи. Но ангелы верные,
% их невозможно заставить, они служат только Богу. А остальные ангелы, они,
% пардон, падшие, и это враги Бога. Стало быть, магия это контакт с падшими
% ангелами, что не камильфор. Так вы понимаете совершенно. Вот. Другое дело, что
% сами маги очень долгое время пытались доказать, имеется в виду магии. Те, кто
% себя считали магами, я сейчас вот так вот покажу. Они очень долго обосновывали,
% что они контактируют со светлыми ангелами, никоим образом не спадшими. Но это
% вопрос, что называется, открытый оставим его. Но во всяком случае, вы должны
% понимать, что понимание материи как того, чего можно просто изучать, просто
% изучать, не контактировать с ее какими-то божественными там, сущностями, которые
% везде там распределены по всем. У реки есть своя там сущность божественная, ну,
% всего. А можно, то есть Бог-то один, а все остальное просто обычный мир, который
% можно изучать.

% теоцентризм, он заставил усомниться в некоторых
% очень важных позициях, например, в отношении небесных тел, небесных тел. А
% именно, в античности небесные тела, солнце, планеты, это божественные сущности,
% умы, души. Вот если вспомните, там неоплотанизм, там еще, или, например,
% аристоидовскую антогольд, вспомните, это все эфир, и это определенная такая
% тонко божественная субстанция, которая вот там вот она, сгущаясь, образует
% разного рода планеты, которые, ну, или другие небесные тела, которые
% одновременно божественные сущности. Но, благодаря утверждению библейского
% криоцинизма, однозначно отвергается мысль об их одушевленности. Небесные тела –
% это просто небесные тела, говорят, богословы христианские. То есть, это уже не
% какие-то боги, просто небесные тела. Император Юстинян, вот тот, по которому мы
% с вами уже упоминали, вот этот Юстинян, да, он вводит в качестве обязательного,
% например, положения. 543 год. В 543 году он вводит обязательно положение. Если
% кто-то говорит, что небо, солнце, луна, звезды, под небесной воду суть
% одушевленной и разумной силы, да будет анафема, то есть осуждения. Небесные
% тела, только небесные тела, не боги и не властители человеческих судеб. Таким
% образом, появились предпосылки для естественно-научного, а не религиозного или
% мифологического понимания космоса. космоса. То есть, теоцентризм средневековья,
% и это я подчеркиваю особенно, добавляет новое требование к исследованию природы.
% Божественное всемогущество превосходит всякий природный порядок. Следовательно,
% любая система представлений принципиально может рассматриваться только как одна
% из возможных. 

% Именно христианское мировоззрение заставляет говорить, что наука должна меняться, потому что и это тоже доктринально закреплено. 13 век епископ Тимпье прямо на площади произнес о том, что может существовать несколько миров, небо может двигаться прямолинейно
% поступательным движением, пустота возможна и так далее. То есть вот эта
% аристотельская физика не единственная, говорит он, ну и не только он, это было
% доктринальное христианское положение. Может быть любая другая физика. Тогда как
% сами ученые умы хватали за голову, они говорили, нет, только аристотельская
% физика, она правильная, только она, вы чего кричите, вы говорите. но на что
% богословы отвечали, если существует только одна физика и вообще возможно только
% одно понимание физики, значит Бог не всемогущий. А это уже, извините меня,
% недопустимо. соответственно, доктринально еще раз подчеркиваю, возглашено, что
% наука должна и обязана и имеет право искать какие-то другие варианты, кроме
% аристотельской физики. Вот такой вот ход, неожиданный, правда? Так вот, эта
% установка, то есть смотрите, никакая физическая концепция, никакая
% натурфилософская концепция вообще не может получить канонического статуса. Да,
% исторически было так, что какие-то статусы получали, но это уже политические
% вещи. А с точки зрения именно богословской позиции ни одна научная концепция не
% может получить окончательного статуса, имеется в виду единственного верного,
% статуса единственной верной концепции, потому что реальность вообще
% контингентна. Бог захотел и физика изменилась. 

\paragraph{Влияние на науку}

Теоцентризм отвергает идею одушевленных небесных тел, утверждая, что Солнце, Луна и звезды --- это просто небесные тела, а не боги. Император Юстиниан в 543 году установил, что любые утверждения о том, что небесные тела --- разумные силы, подлежат осуждению. Это открыло путь к естественно-научному, а не мифологическому пониманию Космоса. В XIII веке христианские богословы утверждали, что наука должна искать новые теории, а не следовать только аристотельской физике, поскольку если бы существовала только одна правильная теория, это бы ставило под сомнение всемогущество Бога.

\paragraph{Влияние на гносеологические установки} 

% Конечно, здесь ключевым моментом является статус веры, но вера, в первую очередь, не как религиозная вера, вера именно Христа, вера там в какие-то позиции библейские, а вера как гносеологический феномен. Все философы средних веков активно занимались темой веры и знания, как они соотносятся, но мы немножечко позже об этом про
% веру и знания поговорим, потому что там, где про схоластику у нас будет, вот
% здесь немножечко просто один момент ключевой, я скажу, что, смотрите, почему
% вера так ценилась, не потому, что она позволяет быть религиозным человеком, если
% вы подумайте хотя бы так, вот так, ну, вы прекрасно поймете, что религиозным
% человеком можно быть и вообще-то на основе знания, да, 

% вера это особый такой
% гносеологический инструмент, особый когнитивный инструмент, инструмент познания,
% инструмент мысли, который вообще-то гарантирует нам свободу, вот за это очень
% любили средневековые люди веру, почему, потому что знание это фактор несвободы,
% ну, то есть, смотрите, если мы что-то знаем, мы уже не можем перестать это
% знать, ну, можно, конечно, получить амнезию, но это уже такие моменты, а если мы
% в нормальном состоянии что-то знаем, не знать-то уже невозможно, а вот вера, как
% раз, она и основана на том, что ты постоянно как бы, ну, выбираешь, да, вера –
% это постоянный свободный выбор, и, соответственно, риск, риск веры – это очень
% важное понятие, которое постоянно анализируется, особенно вот в более поздний
% период, в экзистциализме и так далее, то есть веру ценят как, не просто как веру
% в какого-то там бога, да, в определенную какую-то религиозную истину, а веру как
% инструмент очень ценит средневековое познание, вот, дальше, здесь еще принцип
% откровения, о котором я скажу только совсем чуть-чуть, имеется в виду, что в
% средневековье принцип откровения делился на три части, естественное,
% сверхъестественное, совершенное, это понималось таким образом, то есть человек
% видит бога по мере того, как сам бог открывается, и в первую очередь бог
% открывается ну, как бы естественным образом, то есть естественное откровение, а
% именно в природе, поэтому познать бога можно познавая природу, это будет
% естественное откровение, дальше, сверхъестественное, это уже познание через
% Библию, то есть познать бога можно читая священные тексты, где будет говориться,
% ну, то есть уже немножечко другое, да, ну, не другое, в смысле, другой уровень,
% другой ход, другие позиции, а вот совершенное откровение по мысли христианских
% богословов, это сам Иисус Христос, ну, то есть вот бог открылся, вот,
% совершенное откровение, и приобщение к христу, причем приобщение именно
% практическое приобщение, а это уже соответственно определенные ритуалы, ну, это
% даже не ритуалы, это определенные, это вот так называемые таинства, да, то есть
% это мистические действия, которые совершаются непосредственно в ходе
% христианского богослужения, которые долгое время были тайнами, и соответственно,
% плодили самые какие-то там очень, ну, пока христианство было тайным в
% катакомбном периоде, да, естественно, никто никому не рассказывает, что такое
% евхаристия, но, что удивительно, например, во времена Макса, тут же Макс с
% интересом, вот эти вот байки рассказывает про питье крови младенцев там и все
% остальное, хотя это очень забавно, ну, понятно, что он, скорее всего, ерничает,
% потому что, ну, на тот период все прекрасно знали, как происходит христианское
% богослужение, никакой тайны уже не было, ну, и он, соответственно, ну, я так
% думаю, знаете, так как Маркс был тем еще троллем, он был вообще таким панком 19
% века, поэтому, я думаю, он что-то троллил хорошо своих читателей, ну, и вы,
% пожалуйста, тушите с этим, внимательнее будьте, вот, ну, то есть, вот это как бы
% мистическая, таинственная часть этого как раз бога приобщения, и отсюда же
% антропологический немножечко, антропологический аспект, как сейчас понимается,
% человек в средневековье, в богословии развивается учение о теозисе, вот здесь
% вот есть это слово теозис, наверняка, да, нет, нет его, ну, значит, оно для нас
% так, но тем не менее, я все равно скажу, теозис, то есть приобщение,
% практическое приобщение обожение, святой Василий Великий, представитель,
% патриистики говорил, что человек – это животное, призванное стать богом, что ты
% тут, животное призванное, призванное стать богом, то есть сотворенное по образу
% подобию, он потерял образ и должен приобщиться к подобию, вот, и святой Максим
% Исповедник тоже говорил, стал богом через обожение, то есть теозис, человек мог
% бы самим богом созерцать дела божьи, ну, то есть вот, чтобы быть самим богом,
% понимаете, это как бы довольно древняя традиция, но ты не можешь, не будучи
% сожженным этой великой силой оказаться рядом с богом, тебе требуется что-то
% изменить в своей природе весьма существенное, да, то есть по сути дела стать
% богом, чтобы быть близким с богом, то есть стать рядом, условно говоря, не
% будучи сожженным, и вот вся как бы все знание, весь интерес к человеку
% сосредоточился на этом, что нужно сделать, чтобы суметь стать рядом с богом и не
% быть уничтоженным этой великой силой, этой великой, ну как войти в огонь и не
% сгореть,

В Средневековье вера рассматривалась как особый когнитивный инструмент, который предоставлял человеку свободу. В отличие от знания, которое ограничивает (когда мы что-то знаем, мы уже не можем это <<не знать>>), вера была чем-то, что можно выбирать и менять, что давало чувство свободы и открытого выбора. Вера считалась чем-то динамичным, связанным с риском, потому что она не была фиксированной истиной, а скорее процессом непрерывного выбора и принятия.

Принцип откровения в Средневековье делился на три части. Первое --- естественное откровение, которое подразумевало познание Бога через природу, через наблюдение за миром вокруг. Человек мог познать Бога через изучение природы и ее законов. Второе --- сверхъестественное откровение, которое происходило через священные тексты, такие как Библия, где человек встречал более глубокие истины о Боге. Третье --- совершенное откровение, которое воплотилось в личности Иисуса Христа, как окончательное проявление Божества на Земле.

Кроме этого, в христианском богословии существовала концепция теозиса (обожения), которая утверждала, что человек, сотворенный по образу и подобию Божьему, должен стать Богом. Эта идея выражалась через учение, что человек, несмотря на то что он был создан как животное, был призван достичь богоподобия. Святые отцы, такие как Василий Великий, учили, что человек должен стать Богом, чтобы быть в близости с Богом, при этом преобразившись через духовное обожение.

\section{Средневековые университеты}

% Первые университеты, они возникли еще в Византии, в Константинополе в 9
% веке, там император основал его, и, кстати, в этом университете учились
% знаменитые лев-математик, да, а еще более нам, во всяком случае, известный,
% блестящий филолог, лингвист и богослов Константин Философ, и вот он со своим
% братом Мефодием возглавил славянскую миссию, и, в частности, создали славянскую
% азбуку, затем, после того, как он поменял в монашестве свое имя, мы знаем
% Кирилла, да, и Мефодия, я думаю, эти имена вы знаете. Ну вот,
% Константиномбольский университет имел два факультета, только юридический и
% философский, что должно вас натолкнуть на мысль? Первыми людьми, которые нужны
% были в качестве выпускников университета, были, конечно же, не богословы. Ну,
% опять же, поймите, никому не нужно, вот если вы, представьте себя, вы начальник
% церкви, вам нужно много богослов, чтобы все богословствовали, что-то там меняли,
% изменяли, чтобы как можно больше были дискуссий всяких разного рода, чтобы снова
% вернулась эпоха, когда нужно вселенские соборы собирать, а они и так-то были с 4
% по 8 век на секундочку, только что называется, ну, как бы подуспокоилась, все,
% неужели нужно? Никому этого не нужно. 

% Поэтому богословский факультет, он что
% называется, исключительно в силу необходимости самих студентов. Я пойду, вот это
% как бы высший факультет, и туда поступали уже вот просто по воле, собственно,
% никуда больше, только туда. А самой, ну, как бы элите, самой власти нужен в
% первую очередь юридический факультет. Почему? Потому что нужны грамотные люди,
% грамотные чиновники, грамотные управленцы. Во всю эпоху, ну, нужно же управлять
% во всей эпохе огромные империи, огромные территории. Нужно как-то это все
% организовывать, в одно правовое пространство, одни законы и так далее.
\paragraph{История основания первых университетов}
Первый университет возник в Византии, в Константинополе, еще в IX веке. Основал его император Варда, и среди студентов были знаменитые личности, такие как Константин Философ (Кирилл) и его брат Мефодий. 

Университет имел два факультета: юридический и философский, что отражало потребности власти. Юридический факультет был важен для подготовки грамотных чиновников и управленцев, необходимых для управления огромной империей, а богословский факультет был ориентирован на потребности студентов, желающих изучать теологию.

% Университеты Запада уникальны тем, что это в первую очередь
% профессиональная гильдия, новая профессиональная гильдия. Он был возможен,
% конечно же, в среде, где был товар, да, и был спрос на него, спрос тогда
% огромный, потому что вот эта вся эпоха, когда князья менялись там, когда друг
% друга там германцы, вот этот, утрясался вот этот вот этнически сложный
% социально-культурный как бы, ну вот этот слой, да, и, соответственно, история,
% связанная с бурлением в этом слое. В эту эпоху были разрушены все римские
% институты, которые там рулили, и на долгое время именно было поручено церкви вот
% эту вот юридическую функцию, то есть чиновниками становились сами, ну то есть
% просто образованные люди, а кто был тогда образован, образованы были
% священнослужители, потому что вот именно при монастырях были в основном школы, в
% городах было школ очень мало, соответственно, в каждом монастыре обязательно
% нужно было учить грамоту, любой монах был грамотный, и поэтому, собственно
% говоря, на церковь возлагалась вот эта вот задача, быть, ну, властью, быть
% властью, как-то контролировать вообще вот эту вот жизнь. Постепенно эта функция,
% ну, как бы, становится уже, она отягощает религию. Не скажу, что Папа Римский
% отказался от идеи власти, да, ну, не сам лично, а вот как институт, но тем не
% менее, это уже скорее такая власть, ну, она где-то вот так за ширмой, она не
% занимается напрямую какими-то там институтами, чиновничеством и тому подобное.
% Но нужно было из светских людей власть. И эту функцию возложили на
% университетах. 

Западные университеты уникальны тем, что стали профессиональными гильдиями. Они появились в эпоху, когда рушились римские институты, а церковь взяла на себя роль власти, обеспечивая образование. В монастырях были школы, а священнослужители становились образованными людьми и чиновниками. Постепенно эта функция власти, хоть и была связана с религией, перешла в светскую сферу, и университеты стали центрами, где готовили светских управленцев.

% Откуда брать грамотных людей, да, кто будет преподавать в первых
% университетах? Это вот очень интересный выход. Первая профессура была набрана,
% по сути дела, вот тут я пропущу несколько слабых, если получится, я к ним
% вернусь, из тех, кто развлекал, ходил по дворам, королевским и другим замкам,
% жонглеры-голиарды. Жонглеры – это просто люди, которые умеют развлечь публику.
% Можно песню спеть, можно стихи рассказать, можно глубокую мысль развить, ну,
% можно, ну, то есть, самыми разными способами. Голиарды – это люди, ну, которые
% когда-то имели какое-то монашеское там прошлое, потом они, значит, это, как
% правило, ну, пошли так вот, жить такой вот жизнью светской, но в то же время они
% уже образованы, они могли именно, как сказать, там, поддержать интересную
% беседу. И именно из этого слоя, из жонглеров и бильярдов, появляется первая
% профессура. И вот Пьера Абеляр, он как раз один из этих. 

Первая профессура университетов была набрана из жонглеров и голиардов --- людей, развлекших публику в королевских дворах. Жонглеры могли петь, читать стихи или делиться философскими размышлениями, а голиарды, когда-то бывшие монахами, вели светскую жизнь, но сохраняли образование и умение вести умные беседы. Одним из таких был Пьер Абеляр.

% Можете почитать его
% знаменитую исповедь, где он описывает «Перепетит свои жизни». Знаете, я еще,
% Абеляр, фильм еще не сняли по этой жизни, потому что это было бы просто,
% знаете, ну, это, если бы хорошо сняли, это можно было бы даже подумать, что это
% выдуманное, а он мне там пишет по своей жизни. 

% В общем, так или иначе, желающие
% учить и учиться нашли друг друга, а нашли они пока на полянке. В начале первого
% университета это просто люди собирались на полянке. Недалеко от Парижа, в
% хорошем месте, красивом, при теплой погоде собирались люди и просто платили
% этому учителю денежку за какую-то лекцию, за какой-то там семинар. Постепенно
% этих людей, кто учат и кто учится, организовывают в некоторые общины, такие вот,
% ну, как сказать, в некоторые организации, которые арендуют здания в каком-нибудь
% городе или город им предоставляет на безвозмездное пользование какие-то здания.
% причем эти университеты очень часто переезжали из города в город, потому что
% университет, это посудило люди, не само здание, изначально люди, а не здание.
% Здания потом появились, все эти вот, инфраструктура. 

В начале университеты представляли собой собрания учителей и студентов, которые собирались на открытых площадках недалеко от Парижа, где учителя проводили лекции за плату. Постепенно эти группы организовались в общины, арендовавшие здания или получавшие их в пользование от городов. Университеты часто переезжали, так как важными были не здания, а люди.

% Но самое главное, этому,
% этим организациям, этому единству нужно придать статус. Правильно? А кто в тот
% период может давать статус? То есть, кто может лицензировать это учреждение,
% как, условно говоря, как такое учреждение, выпускники которого могут идти
% преподавать, которые могут учить. То есть, надо же получить лицензию
% образовательную, как вы понимаете, ну, говоря современным языком. То есть, нужна
% власть. Королевская власть тогда этим совершенно не заинтересована. Королевская
% власть, это, помните, рыцари, да, это рыцарство. Это первые среди равных.
% Воевать, да, это его стихия, а вот это образование, это вот, это не его стихия.
% Соответственно, только папы, только, ну, либо императоры, как в Византии, либо
% папа римский. Соответственно, вот именно право промоции, то есть, возможность
% после соответствующего испытания давать лицензии на право преподавания, то есть,
% выдавать дипломы, по-нашему. Вот это право промоции каждому университету выдает
% папа римский. Поэтому, с точки зрения университета, как социального института, а
% мы всегда, когда говорим о социальном институте, мы говорим еще и о власти,
% которая этот институт как бы держит, да, то здесь, в данном случае, будет не
% государственная власть, государственная еще и нет, а будет власть именно
% церковная. Как папа римский контролировал, он назначал ректоров. Ну, собственно,
% на этом весь его, он бы и хотел еще раз говорил, но это очень сложно
% контролировать вот эту вот всю стихию. И не случайно средневековые университеты
% это пример такой вольницы, вольницы  легенд и тому
% подобное, когда эти студенты, ну, в общем, если университет поселялся в каком-то
% городе, жители этого города делали так, нет, потому что это постоянное пьянство,
% это гулянки, это совращение жен и так далее. То есть, конечно, университет был,
% ну, он, собственно, таким и остается до сегодня, особенно западный тип
% университета, это такая вольница. Ну, то есть, я буду думать, что хочу, делать,
% что хочу, и никто мне не указан.

Для того чтобы университеты получили официальный статус и возможность выдавать дипломы, им нужно было право \textit{промоции}. Королевская власть была к этому не заинтересована, так как она больше фокусировалась на военных делах. Лицензию на преподавание могли давать только Папа Римский (или Император, как в Византии). Таким образом, Папа Римский контролировал университеты, назначал ректоров и выдавал дипломы. 

Университеты стали местами вольности, где студенты часто устраивали беспорядки, что вызывало недовольство местных жителей.

\paragraph{Обучение в университете}
% Так, теперь, собственно, об
% обучении в университетах. Ну, я, конечно, не все рассказала, просто можно очень
% много говорить. Университет, первый факультет, базовый, то есть, его заканчивают
% все, это то, что сегодня называется бакалавр, да? Этот факультет состоит, это
% факультет свободных искусств, он так называется в тот период, это еще античное
% название, и еще из античной же программы берется тривиум-квадривиум. Тривиум-три
% пути, квадривиум-четыре пути

Обучение в университетах начиналось с факультета свободных искусств, который был базовым для всех студентов, аналогичный современному бакалавриату. Этот факультет основывался на античной программе, включавшей тривиум (грамматика, риторика, диалектика) и квадривиум (арифметика, геометрия, музыка, астрономия).

% тривиум-это в основном как бы гуманитаристика, как
% мы сегодня бы сказали, вот, пожалуйста, грамматика, диалектика, риторика,
% выясните самостоятельно, чем они занимаются, это несложно. 

% А квадривиум-это вот
% эти четыре, естественно, научные, как мы бы сказали, сегодня дисциплины, хотя
% вас, наверное, смутит музыка, но тем не менее, дело в том, что арифметика и
% музыка это все о математике, музыка тогда это о математике, вот пение это о
% музыке, в современном смысле этого слова, а музыка в средневековом понимании, да
% и в античном тоже, это всегда математика. Дальше, геометрия и астрономия, это
% по, как сказать, это по пространству, геометрия, это по пространству земли, да,
% вот как бы, как тут земля, и так далее, а астрономия по пространству не было.
% Это не все факультеты, были еще так называемые факультативные, да, какие-то
% дисциплины, если у нас останется время, о них поговорим. Значит, теперь сами,
% это первый факультет, сейчас я как бы обращусь к структуре факультетов, первый
% факультет, дальше, 

% высший факультет, то есть вы, если сдаете, вы пошли выше учиться, вы
% либо идете на юридический, это универсально, да, либо вы идете на медицинский,
% не во всех университетах, но был, потому что, как вы понимаете, юристы и медики
% крайне нужны. 

Высший факультет --- это юридический или, реже, медицинский, в зависимости от университета. Юристы и медики были особенно востребованы. А также богословский.

% И вот самый высший, если уж вы, как говорится, все освоили, если
% вы выдержали соответствующие диспуты, если у вас точно хватает сил, времени, ну
% и денег тоже, вы идете на богословский. Богословский факультет, он чем
% занимался? По сути дела, когда мы читаем про схоластов, мы читаем именно про
% богословский факультет, хотя, еще раз подчеркну, далеко не все схоласты
% занимались богословием. 

\paragraph{Схоластика}

% И вот теперь давайте поговорим о схоластике. Значит,
% схоластика это соединение, в самом общем смысле этого слова, соединение такой
% вот христианской тематики и формально-логического подхода. То есть соединение
% логики и богословия. С какой целью? С целью внести в эту богословскую тематику
% какую-то рациональную, ну, рациональный такой вот рациональный элемент, логику.
% И, соответственно, откуда же взять логику? Кто у нас в античности главный логик?
% И вообще, какая логика только известна? Известна только аристотельская логика.
% Поэтому, конечно же, Аристотель в университете это главная книга. Все труды
% Аристотеля и комментарии на труды Аристотеля. Вот это главная книга. Библия не
% главная книга. Я постараюсь объяснить, почему. Библия, она главная для человека
% книга. Для кого-то в большей степени, для кого-то в меньшей степени, они там все
% христиане. Вот. А, соответственно, для ученого схоласта главная книга – это
% Аристотель. Любые его труды. Схоластик – это очень сложная мысль, оперирующая
% логическими схемами. И контрпозиции цитат из Аристотеля же, конечно.
% Схоластические диспуты – это всегда упражнения в логическом рассуждении. Причем
% специально бралась какая-нибудь тема, какая-нибудь, ну, вообще чокнутая тема.
% Ну, там, сколько чертей может уместиться на острие иглы. Ну, то есть, заведомо…
% Это как бы тренировочные темы. Схоласты учились диспутировать. Схоласты учились
% добывать логические аргументы для каких-нибудь совершенно, ну, немыслимых
% композиций ментальных. Понимаете, что такое схоластика? Но, тем не менее, почему
% здесь существовала проблема, именно проблема с религиозным пониманием? Против
% схоластики выступает, например, Петр Домианин. Это монах. То есть, он не
% принадлежит к университетской среде, он не схоласт, но он как бы значимая
% фигура. И вот в споре с Абеляром он задаёт такой вопрос. Может ли Бог сделать
% Рим несуществовавшим? Ну, то есть, Рим как бы существовал, но можно ли сделать
% так, чтобы он был несуществовавшим? То есть, здесь логическая, как бы, какой
% парадокс? И возникла вот такая вот дилемма, конфликт, как называем это, веры и
% разума, как он называется, а именно, схоласт настаивал на том, и вообще
% схоластика настаивает на том, что Бог и логика, это как бы одно и то же. Ну, то
% есть, мы и узнали логику только потому, что это как бы то, как мыслит Бог.
% Правильно. Тогда как представители как бы антисхоластической направленности, ну,
% вот Пётр Домиани, например, они настаивали на том, что Бог выше всякой логики,
% Бог выше всех вот этих вот строгих, рациональных, контр каких-то, не обязательно
% контр, имеется в виду каких-то вот выстроенных суждений. И поэтому Бог, да,
% может сделать Рим небывшим. Это нелогично, это развивает всякую логику, но для
% Бога нет такого ограничения, как логика. Тогда как схоласты, для которых логика,
% это был их хлеб насущный, это был их инструмент, это их всё, настаивали на том,
% что вот как раз божественный разум и человеческий разум едины в логике. 
Схоласт~---~это представитель средневековой университетской науки.

Схоластика --- это соединение христианской тематики и формальной логики, цель которой --- внести рациональный элемент в богословие. 

Логика, в первую очередь аристотельская, становится основой для ученых. В университете главными источниками оказываются труды Аристотеля, а не Библия. Схоласты учат логическому рассуждению, используя контрпозиции и диспуты по сложным, порой абсурдным темам. 

Однако против схоластики выступали такие фигуры, как Петр Домианин, который утверждал, что Бог выше логики, а схоласты верили, что божественный и человеческий разум едины в логике.

% Метод, схоластический
% метод – это комментаторская деятельность. Комментаторская деятельность. То есть
% всегда комментируются другие тексты. Текст на текст, текст на текст.
% Схоластическое открытие – это открытие в пределах какого-то текста. Например, а
% посмотрите, какая интересная мысль здесь. А посмотрите, как её по-новому нужно
% повернуть. А посмотрите ещё. То есть это всегда работа с текстом. Схоласты, они
% заслужили своё такое, ну, немножечко презрительное название, тем, что они на
% мир-то не смотрят, они смотрят только на тексты. И ещё раз повторяю, ключевыми
% из них являются аристотельские тексты. И его метафизика, и его физика. 

% Схоласты внесли структуру не только в мысль, но и в тексты. Вот
% всё, что мы с вами сегодня знаем в качестве абзацев, параграфов, глав, это всё
% изобретение схоластов. Собственно говоря, раннее средневековое письмо, это даже
% точек не... Ну, то есть даже знаков припинания не было. Не говоря уж про абзацы
% и тому подобное.

% Дальше, сама структура научного текста, это тоже
% схоластическое. Введение, основная мысль заключения. Единственное, что отличает
% современный научный текст от схоластического, это то, что схоласты в начале, ну,
% этот метод такой был, в начале аргументы против себя выдвигали. Очень, кстати,
% действенные. А потом их опровергали.

Схоластический метод --- это комментаторская деятельность, основанная на анализе и интерпретации текстов. Схоласты не смотрели на мир, а работали только с текстами. 

Они внесли структуру в научные работы, изобретя абзацы, параграфы и главы, а также структуру текста (введение, основная мысль, заключение). В отличие от современных научных текстов, схоласты начинали с аргументов против своих позиций, а затем их опровергали.

% Вот основные схоласты, которых мы так или иначе
% упомянем, но если не упомянем, просто знайте, что они такие и есть. 

\paragraph{Феномен арабизма}
% С
% комментаторской сущностью, вот с комментаторством связан феномен арабизма. Вот
% как раз то самое, о чем мы часто слышим, что античные мыслители пришли от
% арабов. Но мы уже об этом говорили, не от арабов, а в первую очередь от
% сирийцев, которых завоевали арабы. Дальше, они эти тексты уже переведены на
% арабский язык, на еврейский язык, ну, не на еврейский, естественно, там нет
% такого еврейского языка, да, на еврейский язык, ну, вот на эти языки в
% мусульманскую Испанию эти тексты попадали. А откуда же европейцы взяли? Почему
% вдруг из мусульманских рук-то они берут тексты? Почему не из рук самих же
% греков, не из Византии? Я думаю, вы уже понимаете, с Византии раскол. Византия
% это среда, в которую она вражеская. Я думаю, немножечко похоже на наше время
% сегодняшнее, но как бы общение прекратилось, у ученых тоже прекратилось общение.
% Если раньше совершенно спокойно греческому языку ездили учиться в Византию, ну,
% то есть в восточные земли, да, к грекам, как они говорили, то теперь грекам
% нельзя. Чтобы выучить греческий язык, они идут в мусульманскую, они ездят в
% мусульманскую Испанию, и там учатся и еврейскому, и греческому языку, и берут
% там тексты для перевода. Так вот, таким образом в университеты попали большое
% количество, ну, как бы не в университеты, а именно к людям, которые
% представляются в университет, попали тексты Аристотеля. Их перевели с арабского,
% перевели на латинский, как вы понимаете, это язык общего общения всех народов,
% будучи там немцы, французы, но в средние века во время учебы все говорят на
% латинском. Вот. И возникла проблема, потому что тот Аристотель, который открылся
% в этих текстах, он очень платонический, он очень, как сказать, видоизменен
% неоплатоническим учением. И наиболее грамотные схоласты, то есть те, которые
% познакомились из других источников с Аристотелем, этим крайне недовольны. И
% недовольны еще и потому, что они не только образованы и знают Аристотеля, а еще
% и потому, что они христиане. Ведь согласно, смотрите, тут какая позиция,
% дилемма. 

Античные тексты, переведенные арабами (в основном через сирийцев), попали в мусульманскую Испанию, а оттуда в Европу. Европейцы начали изучать их, так как общение с Византией прекратилось из-за раскола. 

Тексты Аристотеля, переведенные с арабского на латинский, оказались в университетах, но они были изменены под влиянием неоплатонизма. Некоторые схоласты, знакомые с идеями Аристотеля из других источников, были недовольны этим и конфликтовали с платоническими интерпретациями, особенно в контексте христианской веры.

\paragraph{Проблема универсалий}

Универсалии --- это общие понятия. 
Например, есть конкретные люди (партикулярии), а есть универсалия --- человек. 

% Видит ли Бог универсальны, в смысле партикулярии, Бог-то
% воспринимает меня, как там Светлану, или для него только человечество имеет
% значение? Это, конечно, очень личный вопрос, и его требовалось решить, и его
% решали несколько веков. Это вот и есть проблема универсалии. Вопрос был
% поставлен еще в античности Парфирием, это один из учеников Аристотеля, который
% вот систематизировал его учения. Он спросил, как бы на основе аристотельского
% учения, а как же быть с общими понятиями, с категориями? Ведь, смотрите, даже
% время, движение, это же тоже некие сущности, но они не предметы справедливости,
% милосердия, они же должны как-то существовать. А если мы говорим, что существуют
% только по-аристотельски, только правильные, в смысле, только отдельные предметы,
% люди, существа, ну, отдельные, да, сущности, то, соответственно, а как быть
% существованием вот этих вещей? Он задал такой вопрос, на него не ответил, пишет
% о том, что я прочитал работу категории, не нашел у Аристотеля ответа. И вот все
% средневековье пытается решить этот вопрос. Один из первых вопросов, вовремя

\textbf{Реализм} --- позиция, согласно которой универсалии существуют подлинно, первичнее вещей. 
Эту позицию поддерживали Блаженный Августин, Ансельм Кентерберийский и Гильом из Шампо и др. 
Крайний реализм говорит о том, что универсалии существует только в божественном
уме. 
Отличия партикулярий --- это случайные свойства, и они не принципиальны. 
Бог не относится к нашему миру, он абсолютно трансцендентен. 

Некая средняя позиция, она была, ну, она заняла лидирующие позиции. Это
в основном позиция Шартра (\textbf{концептуализм}). Вот этот
умеренный реализм говорит о том, что, ну, как правило, вот эти общие понятия
существуют до самих вещей, например, как мысли Бога. Они существуют в вещах,
потому что Бог сотворил, когда все предметы сотворил, Он дал каждому предмету
его сущность, а сущность это и есть понятие, да. А теперь общие понятия
существуют не только в божественном уме говорит о нем, но и в самих вещах, им
нравится. а потом Бог сотворил человека, дал ему разум, дал ему способность эти
предметы как-то называть, и человек, познав эту сущность, которую Бог вложил в
предметы, да, ну, например, все кошки, а сущность в них одна, кошковость, вот, и
познав эту кошковость, человек дает ему имя, звучание в виде колебаний воздуха и
буквок на бумаге. это уже универсально существующие после вещей. Таким образом,
универсально существуют до вещей, как у мысли Бога, в вещах, как и сущность,
родовое свойство предмета и после вещей. 

Уильям Оккам, последовательный \textbf{номиналист}, выступал с критикой схоластической философии. Он, в отличие от Иоанна Дунса Скота, был разочарован продолжительной метафизической полемикой, считая её избыточной и бесплодной. Иоанн Дунс Скот, обсуждая идеи Фомы Аквинского, неверно приписывает ему утверждение о том, что Бог знает только универсалии, но настаивает: Бог обладает знанием каждой индивидуальной сущности --- так называемой <<стоимости>> и <<этогости>>, что представляет собой характерно схоластические категории.

Оккам, напротив, отвергает претензии философии на участие в богословии. По его мнению, философия не должна служить религии; она лишь усложняет понимание веры. Он настаивает на том, что философское познание должно ограничиваться эмпирической реальностью и конкретными вещами. Абстрактные категории, такие как «справедливость» как таковая, считаются им ненадёжной почвой для познания. Он предлагает сосредоточиться на конкретных проявлениях --- «справедливости этой войны» или «справедливости этого закона».

Оккам утверждает, что вера и философия несовместимы в своих основаниях. Философия должна ограничиваться рассуждениями о частных вещах, не вторгаясь в сферу теологических построений. Он подчеркивает необходимость воздержания от излишнего введения сущностей --- принцип, известный в интерпретации как <<бритва Оккама>>, хотя в данной формулировке он сам его не высказывал. В конечном счёте, философия, по Оккаму, должна быть самостоятельной, но ограниченной областью, не вмешивающейся в вопросы веры.

% А вот Уильям Оккам про него не забыли, он был
% явным номиналистом. И, ну, например, Иоанн Дун Скотт начинает с того же, с чего
% весь спор и начался. Он говорит, почему Фома утверждает, ну, тут в данном случае
% он немножечко врет, потому что Фома этого не утверждает, но неважно, почему Фома
% утверждает, что Бог знает, только общие понятия. Бог не знает меня лично, разве
% Бог не знает меня лично? Он знает только человека вообще? Нет. Даже с этим самым
% подчеркнутое это слово. В Боге есть знание о каждом предмете, о том, что
% представляет собой каждый предмет. В Боге не существует сущности, а существует
% штоиности и этогости. Ну, это сложно, это специфично схоластическое понятие. вот
% это вот стоимость и этогость. Дальше. Уильям Оком крайне разочарован
% многовековой полемикой. Он разочарован вообще философией. Он говорит, философия
% только умножение проблем говорения. Хватит. Философия не может быть служанкой
% богословия. Философия только все запутала. невозможно примирить, говорит он,
% философию и религиозную веру. Вера не может считаться с какими-то положениями
% философии. Вера верой, а философия вот она там своим словоблудием занимается.
% Философия, познающая мир, должна оперировать понятиями, но пусть она их относит
% только к предметам реальности, говорит Уильям Оком. Не нужно, говорит он, какие-
% то вот эти сложности распространять на религиозные позиции. Например, стол. Вот
% это хороший предмет для познания. Стол. А вот справедливость это уже предмет, в
% котором познание скорее всего запутается. Это уже блуда. Поэтому давайте поближе
% к конкретным вещам, конкретным партикуляриям и обсуждаем конкретную, например,
% справедливость. Справедливость этого человека, справедливость этой войны,
% справедливость этого закона. То есть давайте заниматься конкретным, предлагает
% Уильям Оком. Вы понимаете, как бы тут речь идет о том, что он, защищая
% религиозную веру, оттаскивает от нее философию и говорит философия занимайся
% миром просто. мы можем, говорит он, использовать наш рассудок, чтобы думать об
% отдельных вещах и поэтому не существует никаких оснований для теологических
% спекуляций. Не нужно, говорит он, увеличивать сущности без необходимости. Но он
% этого не говорил, но общая мысль в его текстах встречается. Что это означает? Не
% нужно философствовать там, где философствовать не нужно. Вот такая его позиция.
% Я думаю, что многие из вас бы ее разделили, но понимаете, вы не разделили бы
% общей интенции Оккома, потому что для вас философия это просто блуда, которое
% много слов и ничего, а для него это защита религии в первую очередь. То есть не
% надо лезть в своей философии к нам в религию, говорит Уильям Окком. 

% А чтобы вы лучше поняли философию специфичного
% Средневековья, схоластическую философию, я вам предлагаю чуть-чуть ознакомиться
% с философией святого Фомы Аквинского. В данном случае святой, когда я говорю
% святой, это его официальный статус. То есть не то, что там у меня какое-то
% особое отношение. Нет, это его официальный статус. Фома Оквинский аристократ из
% семьи очень значимой. Это вассал германского императора граф Оквинский. Она была
% связана родственными узами с императорской семьей. А Фома был, конечно же,
% предназначен к очень хорошей карьере. Ему уже было как бы и вакансия подыскана.
% Нужно было только пойти в Бенедиктинский монастырь. Оттуда, как правило,
% выходили все высшие иерархи церкви. Но Фома Оквинский оказался таким хипарем, и
% он пошел в совершенно не тот монастырь, который нужно было. Он пошел в
% нищенствующий орден именно францисканцам. Ой, простите, доминиканцам. Он
% доминиканцам был. Но семья Фомы очень возмутилась. В общем, всю жизнь он боролся
% с какими-то препятствиями, потому что такой был свободолюбивый человек. Очень
% много путешествовал, учился во многих местах. Главным образом это Неапольский
% университет. Но центр его образования это Кёльн. Главный учитель Альберт
% Великий. Как раз Альберт Великий работал над объединением Аристотеля и
% христианства. То есть очищая Аристотеля от арабского влияния, он объединял
% Аристотеля и христианства. Позиции христианства, позиции Аристотеля. И Фома
% конечно же тоже продолжит эту линию. Он не был, иметь в виду Фома, не был
% привязан к какому-то определенному университету, но во многих преподавал. И умер
% незадолго до 53-я, простудившись, потому что постоянно путешествовал от одного
% города к другому на своем мостике. Он исколесил всю Европу. Там поучится, там
% преподает, там что-то повыясняет, там подискутирует. Ну вот слишком рано умер.
% На самом деле достаточно долгую жизнь мы смотрим по времени жизни тех, кого мы
% знаем. 70-80 это нормально было для жизни, для людей, которые не военным делом
% занимались, как вы понимаете, для мыслителей. Но все-таки ранняя смерть, как вы
% понимаете, тогда гораздо чаще сегодня. Так вот, философия Фомы это философия
% бытия. Очень интересная философия, имеется в виду, смотрите, он говорил таким
% образом, философия постоянно интересуется тем, что есть какая-то вещь, что есть
% этот стол, что есть эта книга, что есть Бог, то есть нас интересует сущность,
% эссенция, сущность, и он называет эту философию эссенциализма. Он говорит, но
% гораздо важнее существует ли эта вещь, то есть не что она существует, а именно
% тот факт, что существует ли она, понимаете. то есть для Фомы Аквинского сущность
% самой совершенной вещи, самой прекрасной вещи, но если она выдуманная, не
% существующая, так вот ее сущность в разы менее значима, чем сущность какой-
% нибудь мушечки, главное, чтобы она существовала, главное, чтобы она была
% реальна. существование, говорит он, это самое важное. Чувствуете, не что это
% мушка есть, а вот именно существование. То есть если Бог выдуман, говорит он, то
% он совершенно не имеет значения. Имеет значение только сам факт его
% существования, то есть то, что он есть. Этим, говорит, и будем заниматься, этим
% я и хочу заниматься. То есть проблема существования главная и для философии,
% говорит он, и для богословия, и для, лично для него. Конечно, главный вопрос
% существования Бога. В некотором смысле вся философия Фомы, это, знаете, такое
% вот очень харизматичное, очень яркое выступление и такое направленное к Богу
% клич «будь». Мне даже не важно, какой ты, главное, просто «будь». Вот это
% философия Фомы. «будь». А всё остальное не так важно. Это я бы назвала клич
% любви. Ифома в рассуждениях о существовании Бога, ещё раз напоминаю, для него
% главный тем, он вырабатывает определённый метод. Говорит он, я не буду опираться
% на откровение, имеется в виду, на Библию. Вот на Библию говорит, опираться, но
% это религиозный подход, религиозная позиция, самая, конечно, верующая. Но
% философия, если она не сможет открыть своими силами, без обращения к Библии,
% если она не сможет открыть существование Бога, значит, скорее всего, это
% существование надо, либо полностью отказываться от философии, просто уходить в
% богословие, просто верить. Но философия не должна существовать, а он любит не
% только Бога, но и философию. Ему важно и статус философии поддержать, и под
% философию в данном случае вообще всё познание, он понимает, познание
% человеческое, рациональное познание. Либо просто смещаться в сторону велики. И
% для этого он вырабатывает особые так называемые пути доказательства бытия Бога.
% То есть чисто интеллектуальными усилиями, без обращения к Библии. Это вот та
% самая проблема веры и знания, в которой Фома Аквинский их разделяет, говорит,
% вера верой, и она очень важна. Вера для таких вопросов, которые вот они просто
% даны именно как вопрос веры. Ну, например, там Троица. Понять Троицу вот это
% единство до конца невозможно. В этом можно просто верить, просто верить, что Бог
% есть Троица. Это все оставляем, это для веры. Есть еще несколько вопросов таких
% важных, их тоже для веры, но есть вопросы, которые исключительно для философии,
% туда не надо вмешивать веру, разделяет веру и знание. Он повторяет, что
% авторитет священного писания, работа святых отцов не могут быть авторитетами для
% философии. Для философии важно лишь интеллектуальное постижение. Помог Минский
% сторонник того, что знание не только равноценно вере, но и в некотором смысле
% превосходит ее. Это очень верующий человек. То есть, тем не менее, он говорит,
% знание выше веры, несмотря на то, что многие как раз в тот период в его ставили
% выше знания. А почему? Откуда такая позиция? Почему он преувеличивает статус
% знания по сравнению с верой? Почему он делает такую градацию? Потому что он
% говорит, его цитата, «Вера, привитая на не вполне развитую человеческую природу,
% вы имеете в виду разную, всегда будет несколько ущербна. Она будет развиваться
% способом негармоничным, неверным». То есть он ратует за то, что вера, особенно
% такая сложная, как христианская, религиозная вера, она должна ложиться на уже
% готовый рационально организованный разум. Только тогда она будет нормально
% функционировать, а иначе беда. Ну, вообще-то в этом есть что-то, согласитесь.
% Дальше, он утверждает, есть истинные боги, которые должны быть открыты человеку,
% чисто вот откровения, да, вот я говорю, там, троица и другие вещи. Есть вещи,
% которые должны быть чисто философские. Это я обобщаю эту мысль. Конечно же, в
% данном случае он очень высоко ценит разум человека. Он даже спрашивает, может ли
% натуралис рацио, то есть вот эта природная способность человека дойти до
% познания Бога, то есть как бы высшую ступень. Как может, предлагает своих пять
% путей доказательства бытия Бога, как бы к ним не относиться, вот во всяком
% случае они были даны Фомой Аквинским. Но наиболее важной все-таки для Фомы
% является позиция, согласно которой он говорит, наконец-то мы нашли имя Бога.
% Ведь в Библии Бог, когда его спрашивают, кто ты, Моисей его спрашивает, ну, Бог
% обращается над Моисею из неопалимой купины, да, ну, из вот этого горящего куста,
% Моисей спрашивает, кто ты, и Бог говорит, он говорит, аз есмь, знаменитая фраза,
% аз есмь, или яхва, да, а как это перевести, это тот, кто есть, вот так это, ну,
% то есть, Моисей спрашивает, ты кто, он говорит, тот, кто есть, то есть, и обычно
% это воспринимается как то, что Бог утаил свое имя, ну, то есть, человеку нельзя
% знать имя Бога, и откуда всякие разные поиски имени Бога, чтобы им, этим именем,
% с помощью этого имени творить всякие магические штуки, а Фома Ахвинский говорит,
% так это и есть имя Бога, потому что Бог есть само существование, я есть тот, кто
% есть, то есть, я есть само существование, и, говорит Фома Ахвинский, Бог
% наделяет собой, наделяет существованием все, что существует, все, всякая
% существующая вещь, все от Бога, поскольку вот в этом вот внутри, в ядре, вот это
% вот божественная воля, будь, будь, и вы смотрите, в ответ одна из существующих
% таких вещей, как Фома Ахвинский, говорит Богу, будь, пожалуйста, будь, очень
% красивая философия, знаете, особенно в таких вот, ну, закольцованных таких
% смыслах очень интересных. Ну, и еще Фома Ахвинский достаточно много поговорил о
% социальном, антропологическом, других аспектах. Он, ну, практически по всем
% вопросам высказался, вот как Аристотель для античности, Фома Ахвинский для
% Средневековья. Например, вся система существующего, каким образом? Понятно, что
% Бог, но Бог, он даже выше существования, он наделяет существование. Дальше, мир
% нетелесный, это ангелы. Дальше, мир телесный, и мир телесный состоит из пяти
% классов. мельчайшие невидимые частицы, он называет их синолы, ну, мы бы сегодня
% назвали их атомами, ну, что-то в этом роде, или частицами, как субатомными
% частицами. То есть, это глубинная структура всех телесных естеств. Это не основа
% существования, это просто глубинная структура всех существ, всего существующего.
% Дальше, тела созданы из множества синол, ну, или частиц, назовите это так. И это
% царство неживого. Дальше, третье, растительные организмы, животные организмы и
% люди. Ну, и кроме того, он делил всё существующее на искусственное и
% естественное. Душа, это не что-то, вот, вроде, знаете, какого-то, какого-то
% элемента, помещенного в футляр тела. Это психофизическое единство. Телесность
% крайне важна для христианского богословия. никогда не слушайте вот эти вот
% позиции, будто бы христианство разделяет душу и тело. Да, есть такие позиции,
% есть ещё много всякого бреда, но надо знать, откуда вы это берёте. Сейчас я
% немножечко отвлекусь от, собственно, фомы. Почему? Потому что есть проблемы, я
% чувствую, что у вас даже, может быть, диссонанс некоторый возникает. большинство
% позиций по отношению христианского или богословского, богословских идей
% возникает в нашем мировоззрении из так называемых сборников примеров, экземплы,
% знаменитые экземплы. Их огромное количество известно. Это сборники примеров для
% странствующих этих самых проповедников. И там написана вся та чушь, которая
% сейчас наполнена головами, как будто они до сих пор ходят. Они до сих пор ходят
% где-то по нашим степям и весям и рассказывают нам вот эти примеры, где там
% рассказано про то, что резкий дуализм души и тела, например, что Бог душу любит,
% тело не любит. Там еще какой-нибудь такой бред. Где рассказано про ведьм, где
% рассказано про то, что ад со сковородками, вот это там, да, прям там все
% завалено этими. Тогда, как пример, богословие говорит, ад это место без Бога. И
% причем ад создан Богом для того, именно из любви к человеку. Ну, то есть,
% например, вы атеист, вы никогда Бога не верили. А когда вы умираете, где ваше
% место? Это просто место в аду, не потому, что там сковородки и вы наказаны, а
% потому, что вам нужно какое-то место, то есть, там не обязательно мучения,
% просто без Бога. Там нет сковородок. А для тех, кто хотел быть с Богом, для него
% вот это место с Богом, и это называется рай. То есть, видите, а вот это со
% сковородками, а рай это сады и под каждым кустом можно поспать, это совсем из
% других источников. Это не из богословских источников, это из доников, примеров,
% из других народных таких книжечек, вот сейчас такие книжечки продаются в
% подземных переходах. Тогда такие книжечки тоже были и активны очень, и как
% всегда они более распространены. Ну вот, так вот, Фома Аклинский это не это
% книжечки, это не это богословие. И, соответственно, вам надо знать именно вот
% это, вы все-таки другая среда, да? Еще раз, антропологические тела крайне важная
% сущность человека, потому что именно психофизическое единство и есть душа. По
% Фоме Аклинскому. Он, кстати, это выводит из Аристотеля. Дальше. Подобно
% Аристотелю, Фома говорит о том, что человек это социальное существо,
% обязательно. Для людей жизнь в обществе чем ценна, чем значима? Поскольку
% позволяет раскрыть их индивидуальные особенности, способности. Мы все
% различаемся, а стало быть, различаются наши способности. И они зависят и от
% свойств тела, и от свойств души, и от свойств разума. Следовательно, политика,
% говорит Фома Аклинского, служит таким занятием, которое, ну, то есть, политика
% такая хороша, которая позволяет людям раскрывать вот эти собственные способности
% к реализации. И политика плохая тогда, когда эти способности, ну, им трудно
% раскрыться. ну, нет возможности им раскрыться. Дальше. Политику не обязательно
% быть христианином. Главное, чтобы он был хорошим политиком, чтобы он вот именно
% эту сущность политики раскрывал. А хорошим человеком может быть и не христианин.
% Это он однозначно говорит. То есть, политика в значительной степени не зависит
% от откровения, не зависит от религиозной даже позиции человека. Фома Аклески
% считает, что нехристиане вполне могут вести добродетельную жизнь в обществе и
% приобретать какое-то важное знание, делиться им и так далее. И вот как раз
% задача политики сохраняется, засечается в этом. Просто дайте человеку хорошую
% жизнь для реализации его способностей. Тогда остается вопрос, а зачем
% христианство-то? Зачем вот это вот Бог и все остальное? нет, не туда. Вопрос,
% наверное, у вас возникнет, естественно. Так вот, он и говорит о том, что для
% хорошей, правильной, просто хорошей жизни христианство не нужно. Христианство
% нужно для того, кому мало просто хорошей жизни. Ну, просто мало и все. Ну, как
% есть один такой, совсем даже не христианским, тем не менее, певец, который очень
% ярко поет Браухеммер, он не по христианству говорит, но не важно. мне нужно
% больше. Вот для них, говорит он, истинное откровение, Библия и все остальное. И
% вера, конечно же, это инструмент особый. Вера не нужна для того, чтобы наладить
% хорошую жизнь. Вера нужна для большего. И далеко не всем это большее нужно,
% поэтому давайте не наказывать тех, кому это большее не нужно, кому достаточно
% просто хорошей жизни. 

\paragraph{Фома Аквинский.} Чтобы лучше понять схоластическую философию Средневековья, стоит немного ознакомиться с философией Фомы Аквинского. Его называют святым в официальном церковном смысле, не из-за личного отношения. Он происходил из знатного рода --- его отец, граф Аквинский, был вассалом германского императора и имел родственные связи с императорской семьей. Фоме прочили блестящую церковную карьеру, готовили место в бенедиктинском монастыре --- оттуда выходили высшие церковные иерархи. Однако он нарушил ожидания семьи и вступил в нищенствующий орден доминиканцев. Родственники были этим крайне недовольны.

Фома был свободолюбив и много странствовал, учился в разных местах, особенно в Неапольском университете. Главным центром его образования стал Кёльн, где он учился у Альберта Великого, который пытался согласовать учение Аристотеля с христианством, очищая его от арабских интерпретаций. Фома продолжил эту линию.

Он преподавал в разных университетах, не был закреплён за каким-то одним. Умер до 53 лет от простуды --- постоянные переезды подорвали здоровье. Хотя по средневековым меркам жизнь до 70–80 лет была возможна для тех, кто не участвовал в войнах, его смерть всё же была ранней.

Философия Фомы --- это философия бытия. Он интересовался не только сущностью вещей (что это такое), но прежде всего фактом их существования. Он считал, что даже самая совершенная и прекрасная сущность, если она не существует, менее значима, чем самая простая, но реальная. Существование важнее сущности. Поэтому если Бог не существует, то его «сущность» теряет смысл. Его философия --- это призыв к Богу: «будь», просто «будь», независимо от деталей.

Главный для него вопрос --- существование Бога. Он разрабатывает философский метод, отказываясь опираться на библейское откровение. Если философия не может доказать бытие Бога своими силами, она теряет право на существование как рациональное знание. Поэтому он создаёт пять путей доказательства бытия Бога --- сугубо разумом, без обращения к Священному Писанию. Он разделяет веру и знание: вера касается вопросов, которые не могут быть постигнуты разумом (например, Троица), а философия занимается тем, что можно понять без откровения.

Для философии авторитет Библии и святых отцов не обязателен. Важна только интеллектуальная аргументация. Фома считает, что знание не только не уступает вере, но и в чём-то превосходит её. Хотя он был глубоко верующим, он утверждает, что вера, насаждённая на неразвитый разум, будет ущербной. Вера должна опираться на уже сформированный разум.

Он различает истины, которые открываются через откровение (например, Троица), и те, что доступны философии. Фома высоко ценит человеческий разум и задаётся вопросом: может ли естественный разум дойти до познания Бога? Его «пять путей» --- это как раз такая попытка.

Для Фомы важно, что в Библии Бог на вопрос Моисея «кто ты?» отвечает: «Я есмь», то есть «Я --- Сущий». Это и есть, по Фоме, имя Бога --- Он и есть само существование. Бог не утаивает имя, наоборот --- Он сообщает, что Он и есть бытие, дающее существование всему. Всё, что существует, существует благодаря Богу. И человек, такой как Фома, в ответ тоже говорит Богу: «будь». Это философия взаимного бытия.

Фома писал и о социальных, антропологических, политических вопросах. Как Аристотель --- для античности, так Фома --- для Средневековья. В его космологии Бог стоит выше существования, затем --- мир духов (ангелы), затем телесный мир, который делится на пять классов: мельчайшие невидимые частицы (синолы, аналоги атомов), неживая материя, растения, животные и человек. Также он делил всё сущее на естественное и искусственное.

Душа у Фомы --- не отдельная субстанция, вложенная в тело, а психофизическое единство. Тело важно для богословия. Идеи резкого дуализма души и тела --- позднейшие и часто искажённые интерпретации, распространённые через популярные сборники примеров (экземплы) для проповедников. Именно оттуда --- образы ада со сковородками, райских садов и ведьм. В подлинном богословии ад --- это просто место без Бога, а не место мучений. Рай --- это место с Богом. Таковы официальные взгляды, отличающиеся от народных представлений.

Фома следует Аристотелю в утверждении, что человек --- существо социальное. Общество позволяет человеку раскрыть свои индивидуальные способности, которые зависят от души, тела и разума. Хорошая политика помогает раскрывать эти способности, плохая --- мешает. Политик может быть нехристианином --- главное, чтобы он способствовал раскрытию человеческих потенциалов. Добродетельная жизнь возможна и без христианства.

Фома подчёркивает: христианство не нужно для просто хорошей жизни. Оно нужно для тех, кому этой жизни мало. Вера --- не для того, чтобы жить хорошо, а для того, чтобы стремиться к большему. Не всем это нужно, и потому не стоит осуждать тех, кому достаточно просто хорошей жизни.

% Эпистемологические сообщества. Так вот, одни из
% них были, вот я даже сейчас вам найду этот слайд, чтобы называть их, вот,
% эпистемологические сообщества средневековья. Была схоластика, был герметизм,
% сейчас поговорим об этом немножечко, такие главные представители это алхимики,
% да, и были природознацы. Это вообще люди, которые в университетах не учились,
% это те, кто занимался практикой и познанием природы непосредственно в
% практической какой-то деятельности. Ювелиры, моряки, металлургии разного рода,
% понимаете, да, они вот составляли свой сектор мысли в натурфилософской. Давайте
% вернемся вначале к университетам, вначале туда, к схоластам, как же они
% занимались в натурфилософии и что новенького такого дали, а дали они не больше,
% не меньше, как основания для современной физики. Я ничуть не преувеличиваю. Так
% вот, 

% схоласты, которые заинтересовались физикой Аристотеля, сосредоточились,
% конечно же, на исследовании движений. Почему? Потому что сам Аристотель физики
% говорит о том, что движение, ну то есть как предмет исследования, есть самый
% главный предмет для физики, для науки физики. Но средневековые исследователи
% обратили внимание на одну проблему аристотельской физики, что движение
% подброшено вверх камня, да, вот оно подброшено, согласно физике Аристотеля,
% камень летит, потому что воздушная среда очень плотная, как и любая другая
% среда, это очень плотная среда, природа не любит пустоты. Камень летит, и вот он
% пробивает, вот представьте себе, что это камень, он пробивает как бы плотную
% стену воздуха в каждом своем акте движения. И воздух, огибая камень,
% подталкивает его вперед. Когда это движение заканчивается, то камень просто
% падает к своему естественному месту. Поэтому, надеюсь, я знаю, что вам про это
% говорили, и надеюсь, вы помните вот эту физику Аристотеля. 

% Так вот, некоторые
% средневековые мыслители говорят, что-то здесь не то, вот что-то не то, и все
% тут. Ну как-то вот, ну какого-то элемента не хватает. И Жан Буридан, это 14 век
% уже, профессор в Парижском университетах, в вопросах к физике Аристотеля говорит
% вот о чем. Он говорит о том, воздух действует как сопротивление, но не
% двигатель. Ну он будет останавливать камень, он не как, то есть, благодаря чему,
% собственно говоря, камень-то пробьется сквозь эту плотную структуру. Нужен же
% какой-то толчок, то есть, оказывается, нужен еще один в физике элемент, который
% он называет импетус. Импетус. Импетус – это сила, напор, стремление к движению.
% И это представление уже встречалось раньше у Филопонта, Янхилопонт, это 6 век, и
% Филопонт это назвал запечатленная способность, то есть, рука, бросающая камень,
% мы сегодня говорим, импульс, да, придает, а в 6 веке это запечатленная
% способность, а в 14-м это импетус. Причем, импетус, говорит, Буридан,
% пропорционален скорости и массе. Ничего знакомого вы здесь не видите? Я думаю,
% видите. Чем больше материи, тем больше импетус может приобрести тело, и тем
% больше интенсивности оно будет. Понятно, что это как раз и есть исток теории
% импульса. Вот теория Жанна Буридана, теория импетуса, попала во все
% схоластические курсы, в учебники, так назовем это, хотя тогда учебника в
% столовом смысле не было, но в лекционные курсы попала во многие, и, как говорят
% современные исследователи, Галилей, конечно же, учащийся в университете, про это
% слышал и знал, про теорию импетуса. Ну, нельзя было не знать, но только надо
% было очень плохо учиться, чтобы не знать про Буридановскую теорию импетуса.
% Соответственно, вот ее как бы он узнал, и вот вам и теория импульса. 

% Дальше, еще
% какие изобретения? Ну, вот очень интересные мертонские схоласты предложили
% теорему о скорости. Здесь бы надо, конечно, уловить эту хитрость, потрясающую
% хитрость. Смотрите, скорость, опять же, комментируя Аристотеля, исключительно
% комментируя Аристотеля, Аристотель определял скорость через понятие времени и
% пространства, как и мы сегодня определяем через понятие времени и пространства.
% Скорость мыслилась как характеристика интенсивности движения, чувствуете,
% интенсивность движения, а время как количество. Ну, то есть, как быстро,
% медленно, количество. И схоластовое обратили внимание на то, что, смотрите,
% скорость-то это качество, но пространство и время это количество. пространства
% много-мало, а скорость быстро-коротко, быстро-скоро, быстро-медленно, слово с
% головы. Это количество. И он говорит, а как мы, то есть, качество определяем
% через количественные характеристики, но это же неправильно, это какой-то
% мавитон, Аристотель такой логичный, и здесь упустил вот этот вот пробел
% логический. Нужно, говорит, они сделать все либо количеством, либо все
% качеством. Как это сделать? И они предложили это самое пространство и время
% изображать. понимаете, изображать. А как изображать? Широтой интенсивности,
% широтой пути. Тут самое главное, даже не пытайтесь вникнуть вот в эти вот
% скорости, количество. То есть, они просто изобрели систему координат. не
% изобрели систему координат, потому что изобрели возможность скорости быть
% изображенной. А дальше пошел теорема о квадрате скорости. Имеется в виду, как
% площадь прямоугольника. Весь, по сути дела, Декарт математический, физическая
% вот эта вот, физико-математическая вот эта вот, вот часть, связанная с
% декартовской философией, она заложена здесь. То есть, изображение, попытка
% сделать Аристотеля более логичным, привело схоластов к потребности изобразить
% скорость, время, пространство. И, соответственно, они изобрели систему координат
% и поправили Аристотеля. Другое дело, что они не нанесли риски, да, они нанесли
% вот эти количественные, им не нужно это было, не их задача, что называется. Но
% когда их нанесли уже на эту систему координат, мы получили просто рывок в
% математической физике. Ну вот, исследователи средневековой науки сразу же
% обратили внимание на вот эти вот параллели между физическими учениями поздних
% схоластов, да, и механикой 17 века. И вот действительно высказывались
% предположения о совершенно однозначном как бы предшествии средневековых
% университетах среди схоластов вот этого будущего расцвета механики в 17 веке.
% Надеюсь, это вы как-то так. Даже не столько бы там непременно нужно запоминать
% сами эти теоремы, но запомните, что не сидели, что называется там схоластов,
% уткнувшись в Библию, они решали огромное количество проблем. 

\paragraph{Мертонские схоласты, импетус.}

Схоласты, заинтересовавшиеся физикой Аристотеля, естественно сосредоточились на исследовании движения. Это объясняется тем, что сам Аристотель считал движение основным предметом физики.

Средневековые мыслители обратили внимание на одну проблему в аристотелевской теории: согласно ей, подброшенный вверх камень движется потому, что воздух, будучи плотной средой (а природа, как утверждал Аристотель, не терпит пустоты), подталкивает камень, огибая его. Однако при завершении этого движения камень падает в своё <<естественное>> место.

Некоторые схоласты усомнились в этом объяснении: чего-то в нём явно не хватало. Жан Буридан, профессор Парижского университета в XIV веке, в своих комментариях к Аристотелю утверждал, что воздух, скорее, сопротивляется движению, чем способствует ему. Он предложил новое понятие --- импетус --- некую силу, сообщаемую телу при броске, которая затем и поддерживает его движение. Это идея развивалась ещё Филопоном в VI веке, называвшим её <<запечатлённой способностью>>. У Буридана импетус пропорционален массе и скорости тела. Здесь уже намечается теория импульса. Идеи Буридана стали частью университетских курсов, и вполне вероятно, что Галилей был с ними знаком.

Другим важным вкладом стали исследования мертонских схоластов, которые, комментируя Аристотеля, предложили <<теорему о скорости>>. Аристотель определял скорость через время и пространство, где скорость --- это интенсивность движения (качество), а время и пространство --- количественные параметры. Схоласты заметили логическое противоречие: нельзя определять качество через количество. Они решили это путём графического представления движения --- ввели идею изображения скорости, времени и пространства через «широту интенсивности» и «широту пути». По сути, это предвосхищение координатной системы.

Далее была сформулирована <<теорема о квадрате скорости>>, аналогичная площади прямоугольника. Эти идеи заложили основы для физико-математической модели, которая позже разовьётся у Декарта. Хотя сами схоласты не применяли численные методы, их подход подготовил почву для математизации физики.

Таким образом, позднесредневековая схоластика сыграла важную роль в формировании будущей механики XVII века. Исследователи подчёркивают: схоласты не просто толковали Библию, а активно занимались решением физических задач.

\paragraph{Герметизм}

Другое направление средневековой науки в большей степени было ориентировано на практическое знание и применение теорий. В первую очередь это касается герметической традиции --- алхимии.

Герметизм связан с понятием «матезис» --- особой формой знания, не тождественной арифметике, геометрии или астрономии. Матезис был связан с мантикой --- искусством предсказания и магии, которое долгое время преподавалось в университетах, несмотря на сомнения в его соответствии христианскому мировоззрению и научности. Алхимию, как древнюю науку, не решались исключить из университетской программы, хотя отношение к ней оставалось неоднозначным. Один ректор мог позволить её преподавание, другой --- запретить. Таким образом, она не была тайной, но и не получала полной поддержки. Алхимик воспринимался как маг: дело было не только в смешивании веществ, но и в произнесении заклинаний. 

Алхимия была настолько популярна, что её символика проникала даже в архитектуру. Существует мнение, что собор Нотр-Дам-де-Пари посвящён не Богоматери, а именно алхимии: вся его символика трактуется как отображение стадий алхимического процесса.

\paragraph{Природознатцы} --- учёные, которые не только практиковали, но и писали труды: бестиарии, энциклопедии о камнях, болезнях, растениях, животных. Они дополняли античные тексты и включали в них описания вымышленных существ. Почему же средневековые авторы так охотно описывали фантастических животных, вроде драконов и виверн? Один из авторов прямо писал: пусть никто не видел такой остров или зверя, но если он может появиться --- вы должны быть готовы. Его задача --- предупредить читателя, даже если тот потом сочтёт автора наивным. Главное, чтобы читатель знал, как себя вести при встрече с драконом: чем он питается, как от него скрыться и т.д. Именно поэтому средневековые бестиарии и энциклопедии содержали множество вымышленных описаний --- это отражение заботы и научного подхода своего времени.

% Другое направление
% средневековой науки как раз в большей степени ориентируется на практическое
% знание, на практическое применение своих теорий. и в первую очередь надо
% говорить о герметической традиции. Про
% алхимию. Да, мы должны будем говорить про алхимию. Это герметизм. Откуда берется
% слово герметизм? Единственное, скажу, что матезис, матезис,
% это вот математика, которая не та, не арифметика и не геометрия, да, и не
% астрономия. Матезис, это была особая наука, которая была связана с мантикой.
% Мантика, это сфера гадания, сфера магии, и она тоже преподавалась в
% университетах. Долгое время не могли определиться, потому что насколько это, ну,
% как-то так, соотносится с христианским мировоззрением, насколько это научно. Ну,
% то есть, вы понимаете, магия, или, например, алхимия, это древняя наука, и, ну,
% как-то ее просто взять и выкинуть из университета, ну, никого, рукав не
% поднялась. Но, с другой стороны, какое-то сомнение по поводу нее всегда
% существовало у университетских ученых, поэтому, ну, вот один ректор, например,
% разрешает преподавать, а другой не разрешает. Вот такое промежуточное состояние.
% То есть, она не была тайной, не была какой-то закрытой, но в то же время как-то
% вот не особо поощрялась. Алхимик, потому что, ну, маг. Мало смешать разные
% составы, нужно же запленание, еще сказать, правильно, да? Ну, здесь вот просто,
% когда вы получите презентацию, вы обратите внимание, например, на вот эту вот
% картинку. Это все эти самые рельефы из собора Парижской нашей матери. Никто не
% называл собор Парижской богоматери, но там Нотр-Дам-де-Пари. Нотр-Дам-де-Пари
% переводится как собор нашей матери. И существует устойчивое и, по-моему,
% абсолютно доказанное мнение, что это все-таки собор, посвященный алхимии, не
% богоматери, а алхимии. Вот это вот алхимия, а не богоматерь. Мы прекрасно
% понимаем, что здесь вся символика алхимическая. Вот здесь вот эти вот, это все
% стадии алхимического делания изображены. Ну, в общем, вот такая вот интересная
% позиция, насколько алхимия была популярна, что он ведет в свой собор, если
% честно. Ну и совсем уж прям заключением немножко о тех самых природознатцах,
% которые вот здесь вот. Они не только занимались практическими мероприятиями, но
% и писали книги, писали бестиарии, писали связанные, ну, то есть любые
% энциклопедии связаны с какой-то частью действительности. Например, посвященные
% камням, посвященным болезням, посвященным растениям, посвященным, ну, знаменитый
% физиолог. Вот такие вот, как бы, композиции они составляли, огромные, пополняли
% античные энциклопедии и активно в них изображали разного рода выдуманных
% существ. И мы спросим, почему, в чем же специфика такого средневекового внимания
% к разного рода драконам, правда, здесь вот драконы, вот это вот только разные
% виды драконы, виверны и крылатые драконы, там есть разные, вот это виверн,
% например, дракон, дракон на четырех ногах, это другой же тип дракона, но, в
% общем, тут вот есть своя специфика, все это рассказывается. Это крокодил, это
% летучая мышь, но я просто не могу мем пройти, это же шикарные совершенно вещи, я
% всегда их всем показываю, могу показывать их бесконечно, сказать спасибо, что к
% этому слайду. Вот. Так вот, почему средневековые люди так любили всякое
% выдуманное и совершенно легко к этому относились? Так пишет один представитель
% вот такого вот направления в науке, направления природознатцев, он пишет о том,
% да, я описываю выдуманный, ну, в смысле, не выдуманный, а остров, один из
% странных островов в море, его описываю, как он там, ну, в смысле, он появляется
% и исчезает от остров, но я его описываю, почему, потому что он же может
% появиться, вы должны быть готовы, я, и, соответственно, смотрите, они описывают
% драконов, потому что, ну, хорошо, их никто не видел, хорошо, есть сомнения,
% существуют ли они, но вдруг вы их встретите, то есть это забота, в первую
% очередь, и читателя, то есть вы думали, не выдумали, вы меня, как автора, можете
% потом дураком назвать, мне все равно, главное, чтобы вы были предупреждены по
% поводу драконов, вы знаете, как там с ним, что он кушает, как от него там
% спрятаться, ну, вот это вы все знаете, я вам рассказала, поэтому вот эти вот
% бестиарии средневековые и другие энциклопедии, посвященные не только животным,
% бестиарии, это животное посвященное, они, как правило, много выдумок содержат,
% но вот такая позиция средневековых ученых.