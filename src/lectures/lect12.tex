Однако прежде чем перейти непосредственно к
антропологическому и лингвистическому поворотам современной мысли, нам нужно
понять, во-первых, почему именно эти вопросы о природе человека и его языковой
способности вышли на первый план, а во-вторых, что, собственно, нового было
сказано. Все времена давались ответы на вопрос «Что такое человек?» Ну и
проблемы языка в его связи с сознанием, познанием и так далее осмыслялись, ну,
прямо в рамках платоновской концепции эйдоса, в средневекового спора об
универсалиях, декартовских правил руководства разума и так далее. Нам это нужно,
ребят, поскольку сегодняшняя тема важна для каждого не столько как для ученого,
сколько как для человека, чтобы понимать, почему все вокруг именно так,
разрозненно, раздроблено, многомерно. Почему ведущие умы сегодня задаются именно
такими вопросами об искусственном интеллекте, о языке в его связи с сознанием и
действительностью, о духовности, о поисках устойчивых оснований, в том числе для
социально-политического миропорядка. Давайте ради этой задачи исторически
откатим немного назад по сравнению с прошлой темой о постной классике,
восстановим контекст, осмыслим, на какой почве и какие семена были посажены или,
не знаю, разбросаны неклассическими мыслителями. Об этом вы уже в курсе, так что
полезно нам начать с закрепления и краткой систематизации уже знакомого
материала. Несмотря на то, что будет много имен, большинство из которых на самом
деле вам уже знакомо, давайте настроимся прежде всего на восприятие основных
идей. А их не так уж много, просто они с разных ракурсов будут рассматриваться.
И это полезно, потому что позволит нам многогранно охватить предметы нашего
сегодняшнего исследования человека и его язык. Хорошо? Готовы? Ну тогда погнали.
Принято считать, что поворотным временем, повернувшим к собственной
современности, стал XIX век, преимущественно его вторая половина. Мы это время
называем не классическим этапом, а в зарубежной литературе модерн. Вам это слово
знакомо в английском modern, новый, современный. Латинский корень в разных
европейских языках повторяется и имеет сходное значение. В русском это
заимствованное слово, от которого, например, модернизировать, в смысле
осовременивать, и в нем нам слышится также улучшать, совершенствовать,
оптимизировать. Название модерн в отечественной традиции это скорее обозначение
периода в развитии искусства, чем культуры в целом и философии в частности. Но
поскольку хронологические рамки совпадают, вторая половина XIX-XX столетий, и
базовые идеи имеются в виду одни, как для искусства, так и для философии, науки,
мы будем отождествлять понятия неклассической философии и философии модерна,
переосмысления которой затем родился постмодерн, в качестве состояния или
самоощущения современного человека уже конца XX-XXI веков. Отметьте себе сразу,
что специфика модерна по сравнению с предыдущими эпохами заключается в
низвержении абсолюта или отказе от него. Имеется в виду, что современность со
всей трезвостью осознает, что первое, что нам открывается, когда мы ищем
устойчивые основания, это то, что их нет. Мы можем на ровном месте взять и
решить, что есть единое основание первопричины и первоосновы всего. Например,
это античный порядок Логос, или средневековый Бог-Творец, или новоевропейский
мыслящий разум. Но ничто не гарантирует нам, что мы правы. Во-первых, вот
история, в которой меняется это представление о самых фундаментальных основаниях
всего. А во-вторых, в окружающей доступной нам действительности мы реально
наблюдаем изменчивую природу, текучую стихию жизни, трансформирующиеся
социальные взаимоотношения, в которых нет ничего нерушимого, вечного и
неизменного. Тогда, чтобы не впадать в непродуктивные иллюзии, а ведь чем они
лучше других объяснений мира, от культа тотемного крокодила до теории всемирного
заговора, начнем с признания того факта, что по-честному мы не знаем, есть ли
эти безусловные неизменные основания. И если есть, то какие они? Эта
мыслительная ситуация потерянности, растерянности перед открывшейся
безусловностью получила в европейской традиции название преодоление метафизики.
Но давайте по порядку. Для того, чтобы хорошенько понять суть этой идеи,
вспомним с шестой нашей темы, античность, вторая часть, что такое метафизика.
Метафизика со времен Аристотеля, первая философия. Дисциплина, занимающаяся
вечными, неизменными, сверхприродными, первоначалами и первопричинами всего. Она
подвергается критике в рамках неклассической философии, поскольку бытие начинает
представляться как текучее, изменчивое, множественное. То есть сама возможность
говорить об антологических основаниях как единых, вечных и неизменных ставится
под вопрос мыслителями 19-го столетия. Прежде всего, Георгом Вильгельмом
Фридрихом Гегелем, Сёрином Киркегором, Агюстом Контом, Карлом Марксом, Зигмундом
Фрейдом и Фридрихом Ницше. Вроде бы все должны быть знакомые вам имена. Видя
несостоятельность метафизики в том, что она предлагает фиксированную картину
мира, которая по определению не может в себя вместить всю полноту бытия,
философы неклассического толка замечают, что в истории имеет место смена этих
метафизических представлений и в конечном счёте ни одно из них не может быть
принято за абсолютно истинное. Представления об этих вечных и неизменных началах
и причинах всего трансформируются в истории европейской цивилизации. Логос, Бог,
разум. Да и последнее основание. Автономный мыслящий субъект оказывается
вторичным. Вспоминайте десятую тему. По отношению к социально-экономической
реальности по Марксу, к бессознательному по Фрейду, к иррациональности самой
жизни по Ницше. Несмотря на рационалистический пафос предшествовавшей эпохи
просвещения, мыслители XIX века понимают, что в реальном человеке зачастую
главенствуют ни разу. Мы во многом зависим от социально-политической среды,
великолепной страстями и естественными инстинктами. И с другой стороны, мир
вокруг постоянно изменяется. Нет ничего абсолютно устойчивого и нерушимого. В
природе имеет место эволюция, а в культуре исторически одна картина мира сменяла
другого, третью, так что приходящие и наши представления и верования. Где
истина? На что опереться? Неклассическая философия имеет мужество посмотреть
реальности в лицо и заявить, опереться не на что. Все наши картины мира всего
лишь нами созданные конструкты, ограниченные и полностью действительности не
соответствуют. В связи с этим и возникает идея о том, что от метафизики как
фиксированной картины мира так или иначе следует уходить. Однако хотелось бы
подчеркнуть один важный момент. Такая позиция вовсе не предполагает, что надо в
таком нигилистическом, то есть отрицающем жесте отринуть всю предшествующую
философскую традицию. Каждый настоящий крупный мыслитель от Парменида и
Гераклита до Декарта и Ганта понимает подвешенность над бездной любой
метафизики, условность любых наших умственных построений. Настоящая мысль всегда
антология, которая, во-первых, начинает с вопроса, а не с каких-то догматических
утверждений. И во-вторых, старается иметь дело с бытием, как оно есть во всей
полноте и парадоксальности, выдерживая это напряжение и не спеша давать бытию
имена. Поэтому было бы слишком грубо и утрированно говорить, что современность
преодолевает Плутона или Декарта. Нам бы хоть на небольшую долю по-настоящему их
прочитать и понять размах их мысли. Речь о преодолении наших непродуктивных
ходов мысли, которые от трудности иметь дело с парадоксом подсовывают нам наше
сознание, желающее успокоиться на удобных схемах и стереотипах. Для обыденного
сознания, Маркс безусловно прав, нужна идеология, какое-то простенькое
объяснение мира. Но нам с вами, как исследователям, занимающимся изучением
реальности, как она есть, такой путь иллюзии не подойдет. Так что последуем
мужественно за теми мыслителями, которые как раз стараются разобрать костные,
неработающие представления о действительности, чтобы расчистить наше видение.
Формируются в русле преодоления метафизики две тенденции. Одни мыслители
предлагают буквально отказаться от нее, перестать задаваться метафизическими
вопросами, поскольку у нас о них не может быть опыта. Дескать, так логично
поступить рационально. Если мы принципиально не можем знать, например, есть ли у
мира начало или он существовал вечно, существует ли Бог, уничтожается ли душа со
смертью тела и так далее, то и не стоит растрачивать себя, пытаясь постигнуть
непостижимое. Следует сосредоточиться на том знании, которое можно добыть с
помощью опыта. А с другой стороны, ряд мыслителей обращает наше внимание на то,
что несмотря на невозможность дать окончательные ответы на метафизические
вопросы, человеческое существо не перестанет ими задаваться. Об этом, кстати,
предупреждал еще и Мануил Кант в предисловии к своей критике чистого разума. Да,
это вроде бы нерационально, однако в нас эта необходимость осмысления даже таких
превосходящих нашей способности вещей не нами самими вложено. Поэтому
продуктивнее такое наше парадоксальное положение в бытии принять и учитывать
иррациональные стороны человеческого существа, также множественность возможных
ответов на метафизические вопросы, но все же при этом не оставляя попыток
философски осмыслять бытие в его полноте изменчивости и парадоксальности.
Множественность возможных онтологических оснований и текучесть бытия отражаются
в многообразии новых философских подходов. Возникают как рационалистически
ориентированные направления философии, так и учитывающие фундаментальную
иррациональность бытия в философских течениях. Для примера перечислим некоторые
неклассические подходы именно в этом ключе. Линию позитивизма, постпозитивизма с
современной его версией в виде аналитической философии неокантианства,
неорационализм, марксизм и его развития в диалектическом материализме и
неомарксизме. Самое современное уже 21 века направление спекулятивный реализм
вместе с его основанием так называемый объект ориентированный или бессубъектный
антологий мы отнесем к рационалистически ориентированным течениям. И с другой
стороны философский психоанализ, экзистенциализм, герминевтику, структурализм и
постструктурализм с более широкой его версией постмодернизма мы представим как
подходы, учитывающие помимо рациональных, также иррациональные аспекты мира и
человеческого бытия. Это условная съемка, просто чтобы вы ориентировались в
названиях. Самое главное, общее, объединяющее эти подходы здесь следующее. В
ситуации преодоления метафизики задача неклассической философии становится не
дать знания, представления о человеке и мире, не нарисовать определенную картину
мира, как это делалось в рамках метафизики, но дать человеку орудие понимания,
вдохновить его на то, чтобы он сам выходил к осмыслению бытия и собственных
оснований, потому что неклассическая философия нам показала, единых
универсальных оснований нет, и на что-то субстанциальное мы не можем с
надежностью опереться. Таким образом, ни один из этих подходов уже не может
претендовать на единственность и абсолютную истинность, поэтому полезно понимать
их взаимную дополнительность. Исходя из разных оснований, различные направления
помогают нам увидеть мир и положение человека с разных ракурсов, обратить
внимание на различные совокупности проблем. Вспоминайте позднюю античность,
похожую на сегодняшний наш контекст, и очень ситуация похожа на переходную эпоху
Возрождения. Вот и особенностью современной философии по сравнению с эпохами
стабильных метафизических оснований является множественность или плюрализм
всевозможных концепций, подходов, направлений. Они всё равно едины в своём
истоке, поскольку отталкиваются от одной и той же ситуации открытия
безусловности, но различные буквально как разные точки зрения на одно и то же,
как относительные позиции в поле, которое создано принципиально нерешённостью
вопроса об абсолюте. И ещё раз, это не значит, что абсолютно нет, равно как и не
значит, что он есть. Всё напряжение, всё буйство, многообразие современной мысли
держится тем, что этот вопрос принципиально открыт. Так вот, за что, как вы
думаете, естественно, схватиться в первую очередь в ситуации, открывшейся под
ногами бездны безусловности? Ничего нового по сравнению с античностью и
Возрождением. За себя. А кто такой я? Кто вообще действует или что? Как понять
окружающую стихийность и непредсказуемость? Мною управляют мои бессознательные
желания и страхи. Мною манипулируют какие-то социально-политические силы. Как?
Через язык в том числе. Но как понять бессознательное, если очевидно, что у
сознания для этого нет инструментов, да и разум только что низвергли всеми
возможными способами с места главного метафтического основания? Как обрести
свободу? Как увидеть истину, если я понимаю реальность только через язык? А
порядки говорения мне навязывают. Моя нация, моя социокультурная ситуация,
политические настроения вокруг. Вот такой круг вопросов возникает. Первый из
которых, что же такое человек? Кто такие мы? Второй же связан с языком.
Поскольку я хочу понять, а не выросив в языке, не смогу. 


\section{Антропологический поворот неклассической философии} 

\subsection{Определение}

Реально большинство этих направлений разрабатывались практически
одновременно, параллельно в диалоге друг с другом. Но я бы пыталась
хронологически расположить в порядке появления. Также необходимо показать вам
как рационалистически ориентированные подходы, так и те, что иррациональные
аспекты учитывают. 
Самый главный вопрос, как нам быть в этой вот такой странной плюралистической
ситуации. Мне, например, пришлось заняться философией ради выяснения этого. И
пройдя сквозь огонь, воду и немного медной трубы, могу ответственно заявить. Я
нашла, что все небезнадежно. И хочу с вами поделиться этим. Давайте соберем
самые продуктивные идеи. Начнем с определения. Антропологический поворот от
греческого антропос-человек заключается в переориентации философского внимания
мыслителей на проблематику человеческого бытия с вопросов об общей методологии
познания и о метафизических первопричинах всего существующего в мире.
Справедливости ради стоит отметить, что начался он, по сути, с нового времени,
особенно благодаря Рене Дикарту и Эммануилу Канту, поскольку уже в рамках
классической рациональности эти мыслители утверждают приоритет субъекта, его
активность и ключевую значимость в картине мира, хотя внимание и было смещено к
его познавательным особенностям. Поэтому там мы говорили о гоносиологическом
повороте. Далее, как вы помните, не классические мыслители вроде Фридриха Ницше
доводят Декартовский принцип сомнения до всевозможных пределов. Если Декарт как
бы гипотетически прикидывает в голове, что, дескать, можно сомневаться в любых
наших привычных знаниях, то Ницше начинает разоблачать всё и вся, вскрывая
несостоятельность принятых ценностей, любой метафизики в целом и обоснование
человека на мыслительной способности в частности. Хотя как бы он это сделал,
если бы не мыслил сам? Поэтому мы говорим, что Ницше попадает в парадокс лжеца.
Однако благодаря этому снова обнажается непонятность и безосновность нашего
собственного человеческого бытия, настроение, которое передалось в наше сегодня.
В условиях потери надёжных устойчивых ориентиров, что мы обозначили как
преодоление метафизики, естественно, искать опору в себе. За что ещё хвататься,
если всё в мире поплыло, рассыпалось, стало сомнительным, недостоверным,
неочевидным? Поэтому мы и говорим, что при попытке вынуть себя из этого болота
за волосы, как в Бронненхаузе, встаёт не меньший вопрос, чем по поводу оснований
мира. Что такое человек? Эта проблематичность была замечена первыми
неклассическими мыслителями XIX века. Однако если человек не рациональный
мыслящий субъект, то и ни в каком другом определении схватить существу человека
не получается. Как быть? Что делать с этой растерянностью перед невозможностью
дать ответы на подобные метафизические вопросы? В конце концов, что делать?
Делать-то что-то надо, а то некомфортно. 

\subsection{Позитивный проект О. Конта}

Первым эти настроения улавливают и на
открывшиеся проблемы отвечает французский мыслитель Акиуст Конт. Рассматривая со
своей позиции историю европейской цивилизации, Конт в курсе позитивной философии
выделяет смену трёх стадий поступательного хода человеческого разума, то есть
различных способов осмысления мира в рамках различных эпох. Мыслитель
провозглашает эту концепцию как открытие закона трёх стадий, который объясняет
не только историческое развитие общества, но и смену взглядов на мир в течение
жизни отдельного человека. То есть история общества повторяет стадии взросления.
Конт пишет, я вам буду цитировать и отмечать тезисом. Значит, этот закон
заключается в том, что каждая из наших главных концепций, каждая отрасль наших
знаний последовательно проходит три различных теоретических состояния. Состояние
теологическое или фиктивное, состояние метафизическое или отвлечённое, состояние
научное или позитивное. На первой стадии человек выявляет за всеми фактами
абсолют, видит устроенность всего при помощи божественного проведения и сам
стремится к абсолютному знанию, пытаясь постичь первые и конечные причины всех
поражающих его явлений. Цитирую далее. В метафизическом состоянии
сверхъестественные факторы заменены отвлечёнными силами, которым приписывается
способность самостоятельно порождать все наблюдаемые явления. Объяснение явлений
сводится к определению соответствующей ему сущности. То есть человек осмысляет
всё в терминах некоторых постоянных, абстрактных внутренних оснований,
сущностей. Наконец, как пишет Конт, в позитивном состоянии человеческий разум,
признав невозможность достигнуть абсолютных знаний, отказывается от исследования
первых причин, начал и назначения всего, поскольку мы, как конечные внутримирные
существа, не можем постичь ни замысла Творца, ни внутреннюю сущность вещей в
себе. Зато мы можем познавать взаимосвязи явлений, объяснять последовательность
фактов, с которыми имеем дело, и обобщать эти закономерности, используя их для
целей ориентирования в мире и поддержания собственной жизни, достойных условий
человеческого существования. В этом и заключается прогресс науки. Программа
вполне декартовская. Позитивное знание, как его определяет Конт, это знание,
которое можно продуктивно использовать для жизни, которое ясно объясняет
эмпирические факты и основывается на опыте, а не содержит суждения, которые
нельзя ни доказать, ни опровергнуть, например, существование сказочных фей, или
которые скорее запутывают наше сознание, чем проясняют реальное положение вещей.
К таким суждениям он относит большинство метафизических высказываний о мире, о
Боге, о душе. Прежде всего, позитивным является научное знание, соответствующее
критериям ясности, истинности и объективности. Надо записать значение этого
слова. Позитивное. Во-первых, реальное, не химерическое. Во-вторых, полезное, не
негодное. В-третьих, достоверное, надежное, не сомнительное. В-четвертых,
точное, не туманное. В-пятых, положительное, продуктивное, в смысле
организующего, а не разрушающего. И, наконец, относительное, не абсолютное.
Абсолютное знание, которое, например, можно предположить, есть у Бога, не
зависит от фактов и не привязано к ним. Относительно всё связано с вещами и с
другим знанием, хотя и уточняется, но в ходе роста. Тогда как абсолютное расти
не может. Конт обращает внимание на то, что благодаря развитию науки удается
достичь реальных ощутимых результатов, в связи с чем наука провозглашается
высшей формой знания и впервые мыслится не только как самостоятельная от
философии область деятельности, но и как превосходящая её по своему значению для
общества. Вспоминайте с темы 4. Мы с вами уже с этим понятием имели дело. Это
мировоззренческая установка, называется сцентизм. Однако это не означает, что от
других феноменов культуры следует отказаться. Сам Конт прежде всего философ. Он
просто переосмысляет задачи философии. Теперь она не должна быть теологией и не
должна строить метафизику, иллюзорные и эндогматичные конструкты, но должна
стать, выражаясь словами неопозитивиста Морица Шлика, деятельностью по
прояснению смысла, в том числе смысл научных суждений, в силу какой логики и
методологии они формулируются, на чём основываются и для чего служат. Несмотря
на возникающие затруднения, позитивистская концепция оказалась чрезвычайно
популярной в эпоху бурного развития естественных и формирования социально-
гуманитарных наук. Благодаря позитивистским воззрениям, прежде всего самого
Конта, в первой половине уже XX века точные математические методы познания по
образцу естественных наук будут вводиться в исследования по истории, психологии,
лингвистике, социологии, политическим наукам и так далее. И значительно потеснят
в этих областях традиционные описательные и интуитивно-аналитические методы. Что
же касается основной задачи позитивной философии по Конту, она должна прежде
всего исследовать методологические проблемы науки и отвечать на следующие
вопросы, программные для дальнейшего развития всей позитивистской традиции.
Каковы критерии определения научного знания? Этот вопрос носит название проблемы
демаркации или развлечения научного и ненаучного знания, чем прежде всего сам
Конт занимался. Далее, каким образом из опыта добывается научное знание? Это
проблема понимания категории опыта и проблема поиска так называемых атомарных
фактов, то есть на что в опыте можно с достоверностью опереться. И, наконец, как
соотносится выраженное в языке научное знание с эмпирической действительностью?
Этот вопрос касается проблем языка науки и интерпретации опыта. Развитие
классического позитивизма продолжается в эмпириокритицизме. Давайте пару слов
отметим о замеченной Эрнодском Махом черте нашего познания. На опыте мы видим не
голые факты, а уже смотрим на все как бы через призму той или иной
интерпретации. На чувственный опыт в ходе осуществления научного познания
накладываются уже имеющиеся у нас начальные знания и представления. То есть
ощущения на самом деле не являются непосредственной достоверностью. Их никак не
отделить, не очистить от примесей уже имеющихся у нас представлений. И возникают
вопросы. Существует ли реальный мир, если между мной и миром всегда ощущения?
Каков тогда статус научных законов, если мы не можем познать закон ощущений? Ну
что Мах дает вполне позитивный, позитивистский ответ. Научный закон может не
совпадать с реальностью, однако при этом он не теряет своей значимости,
поскольку дает положительный эффект в форме упорядочивания нашего опыта. Если он
работает для нас, что-то помогает понять и как-то применить для собственной
пользы, то почему бы им не пользоваться? Даже если в мире самом по себе нет
этого закона, если никто в мир его не вложил, мы ведь можем себе его придумать и
как инструмент продуктивно использовать. Далее мы срезали этого направления,
которые концентрируются на логическом анализе языка. Их называют
неопозитивистами или логическими позитивистами, среди которых, помимо
упомянутого Морица Шлика, давайте отметим Рудольфа Карнапа, которому принадлежит
идея преодоления метафизики логическим анализом языка, а с другими выдающимися
представителями этого этапа, в частности, особенно с Людвигом Витгенштейном, мы
познакомимся на второй паре в рамках следующего нашего экзаменационного вопроса
о лингвистическом повороте. С такими постпозитивистами, как Карл Поппер, Томас
Кун, Майкл Палани, Пол Фераппент, мы с вами уже знакомы по первым темам нашего
курса. И вы помните, наверное, что эти мыслители занимались основными вопросами
развития науки, а также вне рациональными компонентами познания, творческими,
интуитивными и так далее. Но этот материал у нас уже в достаточном объеме был,
поэтому здесь не будем про них повторять. Самое главное, давайте отметим такой
момент, почему у нас вообще позитивизм вдруг попал в вопрос про
антропологический поворот, когда речь вроде бы о познании и этапах его развития.
Мне лично симпатичен классический позитивизм Конта прежде всего целостностью его
идей и социальным проектом. Конт занимается наукой не в первую очередь. Наука у
него инструмент создания добротного общества. В XIX веке именно Агюст Конт ввел
термин социологии. Он сам, правда, был не очень им доволен, пользовался часто
другим физико-социума, который более наглядно отражал вхождение строгого
научного метода в пространство социума. С мистификациями пора покончить. Социум,
если он здоровый, не муравейник, а сообщество, содружество разумных существ,
подчиняется законам таким же точным, как звездное небо. Порядок и прогресс, то
есть живое равновесие функций и рост. Вот двигатель истории. Социум не может не
упорядочиваться и не расти, если убраны помехи для этого. Так что надо браться
за трудную, практическую, но радостную работу. И в законе поступательного
развития общества, с которого мы начали выше, давайте заменим следующее. В трёх
стадиях, которые проходят у Конта всякое знание, богословской, теологической,
метафизической и, наконец, позитивной. Первая богословская вовсе не ошибка.
Вообще позитивная философия никогда ничего не отвергает, как ничего в истории
человечества не было зря. Ломать приходится только то, что начинает уже мешать
прогрессу и само готово уйти, как скорлупу ореха, когда он готов уже прорастать.
Сначала из первобытного состояния, пусть в воображении, пока в мечте, но
стремление к познанию уже позвало из хаоса природного окружения, из лабиринта
страстей в мир света, смысла и надежды. Так ребёнок мечтает воображать себя
птицей, но это не заблуждение и не ошибка, хотя построить летательный аппарат
может только взрослый человек. Конечно, при этом придётся расстаться с
мечтательным охватом всех мыслимых миров и верховного бытия, но зато становится
знание, на которое можно надёжно опереться сейчас и в предвидении. Особенно
верное предвидение — надёжная проверка позитивного знания. В астрономии, которой
исследуют самые простые и общие явления, предвидение стало возможным давно.
Задача физики сложнее, её выход на позитивный уровень совершился позже. В химии,
тем более в биологии. Позднее всех позитивным знанием, допускающим предвидение,
становится по конту физико-социума, социология. Мне нравится контовский посыл.
Пусть мы не знаем, есть Бог или нет. Пусть не можем постичь первопричины и
первоосновы мира. Давайте оставим эти вопросы открытыми. Но здесь и теперь мы
родились, и надо что-то делать. Давайте позитивное, продуктивное делать.
Посмотрим, что мы можем. Познавать, выявляя конкретные эмпирические
закономерности. Это само по себе интересно и увлекательно. Можем изобретать
полезные техники, чтобы друг другу помогать, лечить болезни, облегчать тяготы
нашей жизни. Можем творить, выражать, осмыслять, организовывать, оптимизировать.
Так давайте этим и заниматься, а не деструктивным или туманно-мистическим. Я
надеюсь, выбирая науку сегодня, вы делали тот же этический выбор, желая
заниматься по крайней мере неразрушительной деятельностью, а собственно
человеческой, созидательной и приносить пользу. Пусть маленькую, но конкретную,
работающую. И в этом ваша бесконечная ценность. Поскольку, как говорит Конт, нет
другого спасения, исключая химерическое загробное, кроме как в родном роде, где
продолжатся все человеческие достижения и где продолжусь я в моих потомках и в
моих изобретениях. То есть позитивное, позитивистское мышление с необходимостью
социально. И Конт предполагал, что человек всё теснее будет солидаризироваться с
другими в обществе, находя личное счастье во взаимной поддержке, в
самореализации в качестве разумного, познающего и приносящего пользу индивида,
которому социум и даёт возможности пространства для творческой, созидательной,
организующей работы, в которой способности каждого пригодятся. Это и должно,
согласно Конту, стать советской моралью добротного общества. Как этого достичь?
Как победить деструктивные и мистические настроения, иллюзии и разрушительные?
Всеобщим, доступным для всех научным образованием, говорит Конт, в котором
верховное место и должна занять такая позитивная философия, которая по замыслу
своему социологии. Красивое решение. И ему современность во многом последовала.
Я думаю, что не безуспешно. Хотя, конечно, как мы видим, не все хотят
образовываться и учиться свободно мыслить. Что ж, возможно, социальные
потрясения последних лет побудут молодёжь больше выбирать исследовательскую
деятельность и разработку новых технологий в качестве продуктивной жизненной
стези. Ну, а наше дело не гасить этот огонь в собственных сердцах и продолжать
нести науку в массу. 

\subsection{Проект <<универсального>> человека}

У Карла Маркса и его друга и коллеги, соавтора многих
трудов Фредриха Энгельса, нам в данном контексте интересна концепция
универсальной сущности человека. Хотя, понятно, этой идеей их творчество не
исчерпывается. Началось всё с того, что эти мыслители были возмущены
эксплуатацией одних людей другими. Прежде всего, в XIX веке это эксплуатация
труда рабочих капиталистами, владельцами производственных предприятий. Но и в
предыдущие эпохи обнаруживалось, что правящие свои общества так или иначе
подчиняли себе остальных, концентрируя собственность в своих руках. То есть, по
мысли Маркса и Энгельса, каждый человек в таком неравноправном обществе вынужден
часть своих возможностей негласно отдавать на реализацию другим членам общества,
теряя при этом свою изначальную универсальность. Также в рамках
капиталистического уклада он, вложив в произведенные товары свой труд, отдает
их, они не являются его собственностью. Этот феномен авторы называют
отчуждением, имея в виду отдавание от себя частью своих естественных
возможностей, способностей и результатов труда, как бы в пользу других. А это
означает автоматически отстранение от себя неполноту собственного бытия. Маркс
критикует разделение труда из-за того, что за каждым человеком закрепляется
определенная работа, которая принуждает его отказаться от осуществления других
видов деятельности, что в итоге негативно сказывается на каждом отдельном
человеке и на обществе в целом. В результате получается, что одни люди
эксплуатируют других за счет разделения труда и института частной собственности.
Это неправильное устройство общества, в котором, как полагают Маркс и Энгельс,
человек не может быть свободен и счастлив. Мыслители критикуют предшествующие
эпохи за навязывание искаженного мировоззрения высшими классами низшим. Это не
просто объяснение устройства мира и места человека в нем, но буквально ложные
идеологии, призванные прикрывать классовые интересы, обосновывая необходимость
подчинения этих низших социальных слоев той или иной власти, выжитям, главам,
родов в первобытном обществе, аристократической власти в античности, церковной в
средние века и так далее. Но на самом деле, как замечают Маркс и Энгельс,
изначально в своем свободном состоянии человек – универсальное существо. Он
способен заниматься совершенно разными видами деятельности. Каждый может и
управлять, и изготавливать какие-то предметы, и обучать других, или, скажем,
исполнять какое-то художественное произведение, роль, песню, танец, и никакие
факторы, ни социальный слой, ни пол, ни раса, ни то, чем занимаются родители, не
могут определять, указывать человеку, чем именно, и что чем-то только одним он
должен заниматься. Поэтому коммунистический строй мыслился как цель, которой
капиталистическое общество должно стремиться, поскольку только в рамках
коммунизма, предполагающего отсутствие частной собственности разделения труда,
никто не будет заинтересован в искажении картины мира. И человек может стать
счастливым, получив возможность воплощать все свои способности, заложенные в нем
как в универсальном существе. Например, с утра пойти порыбачить, днем заняться
производством обуви, а вечером пописать стихи. Затем, в отсутствие денежной
системы, обменять сделанные товары на необходимые вещи, произведенные другими.
Но мы говорим, что это утопия. Почему ее не удается воплотить? С одной стороны,
если всматриваться, тут действительно много подводных камней. Например, что
делать, если я буду производить продукты, которые никому не нужны, со мной никто
не будет обмениваться, значит, это все равно принуждение, и придется не своими
хотелками заниматься, а учитывать спрос со стороны других. И что делать с теми,
кто захочет отдыхать, а не работать? Маркс Энкельс просто искренне верит, что
свободный человек обязательно сам захочет трудиться. С другой стороны, говоря об
универсальности человека, на мой взгляд, они видели слишком абстрактно. Забывая
об одном из самых главных качеств, которые делают человека собственным человеком
— осознание собственной конечности. Абстрактно, да. Вроде бы каждый способен
абсолютно к любому виду деятельности, от пения до решения дифуров, от
приготовления пищи до написания философских работ. Но как только мы вспомнили о
своей конечности, все перемагничивается в этом поле. Оказывается, что я не
хотела бы тратить часть драгоценного времени в своей жизни на виды деятельности,
которым не чувствую интересов и способностей. И если не получу в этой
деятельности особого смысла, что называется, душа к этому не лежит. И, может
быть, мне бы даже хотелось выучить много языков, владеть мечом или в
совершенстве кататься на коньках, подставьте сюда свой личный опыт. Но чувствую,
что важнее мне потратить время своей жизни на другое. А все возможности я не
успею перебрать и абсолютно во всем реализоваться. Гипотетически, да, любого
можно обучить тому, что способен делать кто-либо из людей. Но как быть, если мне
интересно другое? И я хочу успеть осуществить то, что кажется лично для меня
более смыслосодержательным. Поэтому другая положительная сторона конечности в
пространстве. Хорошо, что есть другие, что мне не все подряд придется
самостоятельно делать. И отдав часть своих якобы нереализованных способностей
другим, я только стану счастливее, получая, простите за оксимурон,
действительную возможность отдаться тому, к чему как раз имею наибольшие
способности и что мне наиболее интересно. Пусть другие выращивают, охотятся,
готовят пищу, шьют одежду, производят электричество, управляют государством. А я
буду исследовать и преподавать. Тоже полезно что-то для этих других делать. И
возможно это даже лучше для целостности общества, когда понимаешь, что мы
взаимозависимы и при этом взаимосвободны, когда каждый делает свое. Так что мне
плутоновская утопия идеального государства как-то ближе по духу. Однако и
марксистское видение человека тоже во многом продуктивно. Понимание
универсальной сущности человека, если не буквально его применять, обязывая всех
в рамках коммунизма заниматься по чуть-чуть всеми видами деятельности, то оно
как бы обнуляет настройки и возвращает к тому же истоку, из которого в том числе
мысль Плутона. Никакие обстоятельства не могут детерминировать самореализацию
человека. Пусть будет счастлив в любом занятии. То есть речь здесь о том, что
человек – существо открытое, и следовательно, хорош тот социально-политический
порядок, который это учитывает. Два каждому человеку равные возможности для
самореализации. А это означает прежде всего обязательное поголовное, добротное,
разностороннее образование и воспитание, плюс, соответственно, равный доступ к
равнообразным профессиям, рабочим местам и трудовым отношениям. Вот это как раз,
на мой взгляд, самая правильная идея, которая реально была осуществлена в рамках
отечественной социалистической организации общественного бытия. Понятно, было
много минусов и сдержек системного характера, да и чрезвычайно тяжело обеспечить
единство целей и ценностей на таких огромных территориях. Но идея, которая, к
сожалению, постепенно забылась и выветрилась в этом строе, что все это ради
человека на самом деле. Не на содержаниях надо зацикливаться, вообще не надо
зацикливаться, а обеспечивать открытость и равенство возможностей. Ну, это так,
мое мнение. Просто хотелось бы, чтобы вы, самое главное, ловили эту
амбивалентность марксистской идеи. Еще раз. С одной стороны, особенно в
современных условиях текучести и непредсказуемости полезно быть разносторонним
человеком. Наверное, совсем универсальным все-таки в реальности невозможно
стать. Но широкое образование и получение навыков в различных занятиях и
ремеслах это тот неотчуждаемый ваш собственный капитал, который поддержит вас в
жизненных перипетиях. Просто, чтобы вы не боялись. Если даже по совершенно
случайному стечению обстоятельств потеряете место работы, там, не знаю,
предприятие закроют, или ваша узкая специальность окажется невостребованной, или
жизненная ситуация заставит иначе зарабатывать на жизнь, вы не пропадете, если
умеете делать разные. А главное, готовы учиться новому, осваивать новый вид
деятельности, быть может, не менее интересный, если присмотреться, чем текущее
занятие. С другой же стороны, важно не впадать в дурную бесконечность поисков
себя путем перебора возможностей и попыток реализоваться во всем подряд. От
ведения канала о фитнесе и, там, не знаю, блога о рукоделках, до занятий нукой и
игрой на гитаре. Тут большая опасность распылиться и как раз потерять себя,
поскольку в условиях ограниченности времени и физических сил просто невозможно
одновременно всему отдаваться с равным энтузиазмом. И главное во всем
одновременно достигать успехов, получать удовлетворение. Короче, продуктивнее
нащупывать золотую середину между одним видом деятельности и многообразием
возможностей. 

\subsection{Психоанализ}
Для Зиммунда Фрейда. Человеческое сознание оказывается
обусловленным не только извне, как у Маркса за счет определенной социально-
экономической среды, в которой мы рождаемся, но и изнутри. А тут уже совсем
сложно становится понять природу человека, если открывается в качестве
фундамента ее внутренняя иррациональность. Выдающийся психоаналитик показывает
обусловленность, подавляющая части нашей деятельности, наших ценностей и
стремлений, иррациональным, бессознательным, которым мы принципиально не можем
управлять, и силы которого полностью перевести на осознаваемый уровень не
удастся. То есть мы не только зависим от меняющихся, неконтролируемых отдельными
индивидами внешних условий, но и от неизвестности внутренней. По Фрейду сознание
вторично по отношению к бессознательному и является лишь рационализацией наших
животных в лечении и инстинктов. Поскольку мы не можем реализовать в обществе
все наши стремления к удовольствию, наталкиваясь на моральные запреты, мы
вынужденно придумываем, рационализируем свои желания, чтобы для других выдать их
за нечто правомерное с точки зрения общественного мнения, а всевозможные
старания человека, психоанализ сводит к компенсации различных комплексов, с
детства свернувшихся в нашим бессознателем. Значит, смотрите, к чему приходит
Фрейд. Сознание — это всего лишь порождение бессознательного, которое
придумывает рационализация, чтобы для окружающих людей наши инстинктивные,
бойные и аморальные желания выставить в качестве как бы разумных, правильных,
завуалированных и поэтому, по видимости, безопасных для остальных. Но на деле
всегда ведет либидо. Как все объясняет Фрейд. Бессознательно погружена не только
та часть нашей психики, которая называется «Оно», ответственное за животные
инстинкты, страсти и желания, но и неосознаваемые нами до конца части «Я» и
«Сверх Я». Последнее представляет собой такую контролирующую инстанцию, которая
выбирает в себя функцию совести нашего внутреннего надзирателя и судьи, а также
содержит идеальный образ «Я», к которому заставляет нас стремиться, наказывая в
случае неподчинения или если мы дали волю желаниям «Оно» в чистом виде. Конечно,
пытаясь все-таки как-то понять бессознательно, Фрейд, как и многие другие что-то
утверждающие о новых основаниях негативических мыслей, не попадает в парадокс
лжеца. То есть мы его можем спросить, в чем тогда истинность его подхода, если
это тоже может оказаться всего лишь рационализацией глубинных желаний и
завуалированных комплексов, ложной по определению. Но идея Фрейда оказалась
продуктивной для понимания природы человека и сложной структуры личности. Фрейд
обращает внимание на то, что человек рождается аморальным. Ребенок не знает
сначала о совести, не умеет контролировать свои естественные влечения и тупо
стремится к удовольствию. В то время как родители или воспитатели на первых
этапах играют роль как раз такой внешне контролирующей и наказывающей инстанции.
Затем ребенок интериализирует, то есть принимает вовнутрь принципы общественной
морали, правильного поведения, самоконтроля и так далее. Но часть этого
формируемого сверх-я уходит бессознательно и дает впоследствии различные
психологические травмы и комплексы. Так что нормальных по Фрейду нет, все более
или менее проблемные и несчастные. Хотя у намеченного Фрейдом направления было
много и до сих пор немало последователей, мне кажется, самая знаковая фигура
Жака Лакана, французского психоаналитика и философа. В 20-30-е годы XX века
Лакан, пытаясь разобраться в действительных основаниях бессознательного и его
проявления, концентрирует свое внимание на языковом измерении человеческого
бытия, стараясь разобрать, как среда языка влияет на человека, формирование его
личности и выстраивание взаимоотношений с другими. Так в его философии возникает
особое понятие дискурса, которое дословно переводится как «речь». Однако
подразумевается не обязательно конкретное озвучиваемое кем-то высказывание, но в
широком смысле область того, о чем в определенном порядке может быть нечто
высказанное. То есть язык понимается как среда, которая организована
неоднородно. Существуют различные области бытия, которые описываются языком
различными соответствующими способами. В пределе у каждого человека определенный
способ говорения. Лакан использует популярный в начале XX века структуралистский
подход к исследованию языка. Об этом у нас в следующем вопросе будет подробнее,
так что пока не заморачивайтесь. Но поэтому его концепцию называют структурным
психоанализом, то есть исходящим из распознания языковых структур, которые с
различным наполнением воспроизводятся в человеческом обществе и управляют нашим
сознанием. Для Лакана важно такое понимание языка, как разделенного на различные
способы говорения, чтобы с одной стороны попытаться по специфике речи пациента
понять природу его психических нарушений, а с другой, чтобы показать действия
бессознательного во взаимоотношениях человека с другими, вскрыв разрывности, как
бы языковые пропасти между «я» и другим. Мастер-те говорит, что поскольку
психоанализ должен строиться как наука о бессознательном, исходить следует из
того, что бессознательное имеет языковую структуру. Из этого тезиса выведена
топология, призванная дать отчёт в том, как складывается субъект. То есть
обратите внимание, субъект в данной концепции уже, конечно же, тоже далёк от
новоевропейского первичного автономного мыслящего сознания. Он складывается
динамически из действия бессознательного и отражается в языке. Лакан предлагает
буквально прочитать бессознательное как своеобразный текст. Методологически это
выражается в работе аналитика с речью или записями пациента, в которых он нечто
высказывает. Например, рассказывает о своих сновидениях. Соединяя фредистский
подход в психоанализе с структуралистскими практиками расшифровки языка
мыслитель приходит к пониманию неразрывной связи трёх измерений человеческого
бытия реального, воображаемого и символического. Эти три составляющие в своей
динамике изменчивости по мнению исследователя отражают структуру человеческой
личности. Лакан изображает их геометрически в виде колец Баромео. Если рассыпать
одно из них, то вся система развалится. То есть все три эти момента обязательно
должны присутствовать, чтобы мы могли говорить о том, что перед нами
человеческое существо. Человеческое существование по лакану проходит в динамике
трансформации этих трёх взаимно переплетённых измерений. По сути дела, от
реального как такового мы фундаментальным образом отделены посредством
интенсивно используемой именно человеческим существом прослойки воображаемого,
которые, например, животные слабо осваивают и даже пользуясь воображаемыми
структурами не отдают себе в этом отчёт и не развивают способность к
воображению. Приведу пример, но не лакановский, а свой, поскольку я этим много
занималась, изучая развитие интеллекта и феномен игры, когда животные играют. А
у них много игр. В охоту, состязания, защиту, заботу и так далее. Они как бы
представляют себе то, чего на самом деле нет. Скажем, кот хочет реализовать свои
охотничьи инстинкты, но рядом нет ни мышки, ни птички, поэтому он способен
увидеть жертву в какой-нибудь смятой бумажке или в мячике и может за этими
предметами гнаться, сам запускать лапки, останавливать, выслеживать из-за угла.
Человек уже может играть во всё, представлять себя с самого глубокого детства
кем и чем угодно и менять правила игры. То есть, например, кошка вряд ли станет
воображать себя сама жертвой, испугавшейся мышкой или летящей птицей. Её игры
отвечают лишь природой, при природой заложенном стремлении. А вот человек
способен на себя примерить любую роль, придумать и поменять правила любой игры.
Это свидетельствует о каких-то особых отношениях человеческих существ с
пространством воображаемого. И здесь для человека основополагающий статус имеет,
во-первых, язык, а во-вторых, другой. Данное открытие как раз принадлежит
Лакану, который выявляет в развитии человека так называемую стадию зеркала. На
этой стадии все дети в период 6-18 месяцев начинают постепенно, во-первых,
различать пространство слов, то есть языка речи, от вещей и событий мира. Они
приобщаются к тому, что посредством слова можно управлять событиями. Вы же
знаете феномен детских приказов, когда малыш года от рода усвоив всего несколько
слов взрослого языка, говорит «дай, пойдём, на ручки». Он буквально управляет
миром как царь, вызывая по своему желанию те или иные события. Во-вторых, в этот
период дети начинают отделять себя от окружающих других людей и как следствие
узнавать себя в зеркале. Лепетная речь постепенно заменяется способностью к
выражению на естественном языке, если ребёнок погружен в среду общения с
другими. Причём стимулируется это развитие не только когда родители или
воспитатели общаются между собой и с ребёнком, учат произносить слова и
обозначать им предметами, но и в ответ на речь, в том числе лепетную других
детей. Человек вообще от рождения отвечающее существо, всё в мире замечающее и
готовое к деятельному участию, освоению всего, что к нему приходит. Если же на
данном этапе малыш не находится в среде общающихся других, как это бывает в
случае детей маукли, которых воспитывают животные, такой человек никогда не
станет уже собственно человеком с человеческой системой восприятия. В более
позднем возрасте его невозможно будет научить языку и пониманию смысла,
поскольку способность к неограниченному доступу в измерение воображаемого не
была в нужный момент стимулирована и развернута в его сознании. Когда же ребёнка
учат говорить, он начинает понимать, что слова и вещи не совпадают однозначным
образом, то есть символическое измерение, язык в широком смысле, не
накладывается абсолютно дождественным образом на реальное, то есть на окружающее
положение вещей. другой со своей позиции видит меня и то, что в данный момент
общения с ним не видно мне. Поэтому между нами возникает такое топологическое
пространственное различие. Мы представляем буквально разные точки зрения. И это
различие включает в осознание человека понимание некой разницы потенциалов,
пролегающих между различными акторами, между словами и вещами, в пределе между
собой и миром, вследствие чего и возникает представление о себе как отдельной от
мира и других самостоятельной единицы. Происходит осознание границ своей воли. Я
не могу управлять миром полностью, поскольку он не мною строит. Я не могу
управлять другими, полностью вжившись в них. Они сами такие же, как я, отдельные
неприступные миры. То есть взрослея, младенец начинает обращать внимание с одной
стороны на различие собственных состояний в связи с изменениями в окружающем
мире и в собственном теле, а с другой на различие себя и мира, вещей в мире,
других слов, вещей и так далее. Так уже к 2-3 годам формируется самосознание и
активно начинается использование воображаемого как инструмента пространственно-
временного ориентирования в нём. В какой-то момент в возрасте от 2-3 лет
происходит первый элементарный акт рефлексии. По латинским частям слово
«догадывайтесь» это складывание, удвоение, когда мы впервые от сплошного
переживания смены аспекта, то есть того, как мир по-разному поворачивается к
нам, обращаем своё внимание на то, что дано — это всё мне. То есть открываем
тождество. Вот я. И одновременно, значит, различие. На этом построена,
собственно, человеческая система восприятия на элементарном акте развлечения
одного и того же самого, в данном случае, самого ближайшего себя. Проговорим
чётко, важный здесь момент о связи воображаемого границы различия и другого с
большой буквы или категории иного как такового. В мире самом по себе нет границ.
Они возникают только в наших представлениях о мире, в нашей голове. Это понятно,
да? Или вы сможете глазами различить на квантовом уровне и показать мне точно,
например, где реально заканчивается стена и начинается воздух. А цветущее
дерево, аромат которого далеко растягивается, плоды которого съели птицы и
животные, а живое тело, обновляющееся, видящее, дышащее, злющие звуки, слышащее
и так далее, где реально его границы? С другой стороны, эмпирически нам каждый
раз встречаются уникальные во времени и пространстве люди, деревья, стены и так
далее. Но мы можем их отождествить с соответствующей идеей дерева вообще,
человека вообще, стены вообще, благодаря тому, что в воображаемом пространстве
нам доступны такие абстрактные образы. А в реальном этих образов самих по себе
нет. Нет березы вообще в реальном лесу и идя по улице, вы никогда не встретите,
например, женщину вообще, только конкретных представителей из прекрасного пола.
У животных другая система восприятия. Для них мир остается окружающим миром,
только тем, с чем они имели за свою жизнь чувственный контакт, только положением
вещей и спектром собственных состояний, хотя они, безусловно, тоже запоминают
что-то, смутно пользуясь некоторыми структурами воображаемого. Но человек может
представить себе те места и исторические эпохи, в которых он никогда не был,
может представить, что в каких-то условиях чувствует другой человек, хотя сам не
испытывает непосредственно того же самого. Короче говоря, человек может выходить
за свои собственные границы, но возможным это становится только потому, что он
вообще понимает идею границы или различия. По крайней мере, почему-то люди
чрезвычайно активно осваиваются с границей различия, встроено этой категорией
буквально все свое бытие и мощнейшим образом развивают воображающую способность.
Так что не будем категоричны, не будем игнорировать факты и забывать, что у
многих лучших млекопитающих воображение тоже есть. Например, собака может
посчитать щенков, слон вспомнить в засуху из детства, путь к удаленному
источнику воды, орангутан, исхитриться использовать подручные средства, чтобы
добраться за сладкими фруктами и так далее. Короче, главное, мы, ребята, не
рождаемся людьми, а только становимся собственно людьми в меру развития
способности к неограниченному воображению и освоению идеи чистого различия,
иного как такового. Но это означает разрыв и потерю внутреннего единства.
Поэтому психоанализ и концентрируется на, так сказать, фундаментальной травме
человека, которая отличает его способ бытия от животного. Вот это открытие
лакана о природе человека. Таким образом, по мысли лакана, для человека реальное
становится принципиально недоступным своей первичной сущности, поскольку мы
работаем со всеми вещами и феноменами нашей действительности через воображаемое,
выражая его для других через символическое. Однако реальное, конечно же, никуда
не девается, но, как говорит живой классик, словенский комментатор и
интерпретатор лакана Словой Жижик, начинает для нас функционировать в модусе
нехватки, как отсутствующее недоступное, пустое место. Отсутствие прямого
доступа к реальному и подсознательному учительному для человека, поэтому он
пытается прикрыть это пустое место. Бессознательное как раз выступает стихия,
тянущая нас к этому недоступному реальному. Так как мы постоянно ощущаем
нехватку чего-то в своей жизни и обычно не можем понять, чего именно не хватает,
наше бессознательное подкидывает нам объекты желания, завладеть которыми мы
начинаем стремиться. В том числе мы сами можем потенциально стать объектом
желания другого. Нами могут захотеть завладеть в различном смысле от подчинения
в любовных отношениях до управления нашим сознанием в рамках политической или
экономической сферы. Поэтому человек также стремится выстроить защиту от
потенциально опасного другого, формируя для него образ себя, который полностью с
ним самим не совпадает. Принято называть этот несовпадающий образ симулякром.
Согласитесь, мы по-разному ведем себя с различными другими, показывая себя с
разных сторон разным людям. Мы одни для друзей, другие для родителей, третьи для
коллег и так далее. Мы как бы постоянно играем разные роли в разных образах,
защищая себя от проникновения других в наше настоящее я. До конца другой не
может поглотить наше я, потому что он принципиально занять наше место и быть
нашим способом не может. Однако иногда в обход выстраиваемый нами внешней защиты
кому-то все-таки удается начать управлять нашими желаниями, частично нами
завладевая изнутри. Что касается внутренней составляющей наших желаний,
достижения вот этого недостающего, если даже достижение каждого конкретного
желаемого происходит, например, захотели новое платье, пошли купили, то,
естественно, полное удовлетворение никогда не наступает, мы принципиально не
способны как конечные существа заиметь все желаемое реальное, а не достает нам
всего в целом. Да еще и имение чего-либо не означает того, что этим самым мы
получили к желаемому прямой доступ во всей полноте. Имение это один из способов
взаимоотношения с чем-то, который, тем не менее, существует лишь в нашем
воображаемом и не ведет к поглощению этого чего-то в качестве реального для
себя. В качестве заплатки на месте не хватающего реального воображаемом
выстраиваются картинки и образы, которые рисуют для нас уже достигнутое
желаемое, принося нам этим некоторое наслаждение. Эти сценарии реализации
желаний получили название фантазмов. С одной стороны, действительную осмысляющую
активность в нас могут замещать эти фантазмы. Они всегда приходят извне, из
ситуации, взаимодействия с другими, поэтому на самом деле они нашими-то не
являются. Фантазмы начинают действовать вместо нас настоящих. Этим сейчас
чудовищным образом пользуются рекламный дискурс и идеологическая пропаганда.
Через рекламу, изображающую красивый образ жизни, нам навязывается желание что-
то конкретное купить, в том числе и для того, чтобы якобы символически
обеспечить свой образ для другого. Самый безбидный пример. Нам навязывают что
круто ехать, отдыхать в теплые страны, райское наслаждение и многие мечтают о
каких-то супер красивых местах, абсолютном комфорте, строят планы, а в итоге
разочаровываются от антисанитарии в некоторых отелях, им не везет с погодой,
где-то местные могут обмануть или обокрасть приезжих, кожа на солнце обгорит,
подцепим какую-нибудь заморскую заразу и так далее. То есть картинка красивой
жизни завладевает нами, мы начинаем верить в то, что если копим это
предлагаемое, то достигнем удовлетворения. Ясно, что такого рода попытки
удовлетворить свои желания приносят лишь одну боль и мы ведем несобственное
существование, отдавая сюда во власть фантазмы и этим закрывая свое видение
реального. Значит, на деле, наоборот, отдаляясь от реального. Это касается не
только рекламы, но и пропаганды в СМИ и соцсетях, которые заставляют людей
верить в угодную властям картинку реальности. С другой стороны, мы видим, что
манипулирование бессознательным людских масс формирует косвенным образом и саму
реальность. Например, создав образ врага в сознании людей, можно заставить их
поверить в то, что соседний народ плохой, несправедливый, поэтому с ним надо
воевать. Хотя на самом деле соседний народ такие же люди, просто говорящие на
своем языке, с нескольким вкладом жизни, несколько другими культурой и
привычками, не хуже и не лучше, а просто другие. То же самое и с экономической
реальностью. Жижек в 1989 году в своей книге «Возвышенные объекты диалоги»
приводит забавный пример, которым он предсказал реально случившееся в 2020 году
в начале пандемии. Возможно, вы помните этот мент. В магазинах полным-полно
туалетной бумаги, но вдруг появляется слух о том, что туалетная бумага это
дефицит. Каждый думает, очевидно же, что бумаги много на полках магазина, однако
наверняка найдутся простаки, которые в это поверят и начнут скупать бумагу как
сумасшедшие. Пойду-ка я куплю паразапас на всякий случай. То есть не обязательно
реально должны существовать люди, верящие этим слухам. Срабатывает механизм в
нашем воображаемом. Хотя вроде бы мы все рационально понимаем, что изначально
этого дефицита нет. Результат будет один. Дефицит туалетной бумаги. Это
прекрасная иллюстрация того, как наше воображаемое формирует реальность.
Конечно, надо понимать, что возникновение фантазмов в нашем сознании неизбежно.
К тому же приятно для человека, однако с ними необходимо бороться в себе,
преодолевать их, поскольку действия фантазмов разрушительны. На самом деле мы,
например, не просто хотим купить платье, а для чего-то, поскольку представляем,
как можем предстать в красивом виде для желаемого другого или как сходим в
ресторан в таком образе и к нашей персоне будут прикованы взгляды. Эти мечты,
когда мы их воображаем, приносят сами по себе удовольствие. Мы стремимся
воплотить фантазма в действительность, однако при этом напрочь забываем, что-то
лишь сценарий в нашем голове и реальная полнота настоящего никогда не совпадет с
тем, чего мы хотим, потому что мы, во-первых, не способны учесть абсолютно все
факторы реальной ситуации, а во-вторых, не можем управлять сознанием,
восприятием и поступками других людей. Нам этого хочется, но мы только в своей
голове вообще-то эти неправомерные вещи представляем. Поэтому больно, когда
обреально эти наши фантазмы рушатся. На мой субъективный взгляд, похлебнувший на
опыте, к чему приводит богатое воображение, чтобы этой боли избежать, самым
действенным средством является настроить себя на понимание того, что на самом
деле никакой нехватки нет. Это паническая реакция нашего бессознательного на
факт нашей конечности. Продуктивнее понять в нашей ситуации и так уже есть все
необходимое. Вопрос в том, видим ли мы это или наш взор захламлен иллюзиями,
фантазмами. Видим ли мы реальность? Например, если нас желаемый человек и так
любит, не обязательно тратить деньги на супершикарное одеяние, достаточно просто
выглядеть, как обычно, и от этого вас не перестанут любить. Если чувство
настоящее, то вполне хватит элементарно следить за собой и умеренно, адекватно
одеваться, что называется, быть собой. Зачем из кожи вон лезть, изображая того,
кем вы не являетесь? Этим симуляром можно только других обманывать и формировать
к себе ненастоящие отношения. А напротив, если человеку вы не нужны, то как не
разоденьтесь, чем не пожертвовать ради него, чем не попытайтесь удивить,
добротных отношений не получится. Да, больно разрушать в себе подобные мечты, но
зачем растрачивать себя на ненастоящее? Ради людей, которые вас не оценят, что
бы вы ни сделали, и остаться с пустотой? Продуктивность сосредоточиться на том,
что на самом деле есть, и осмыслять действительность в чистоте. Тогда, даже
понимая нечто грустное, к примеру, осознав, что не нравитесь вы человеку, в
которого вы влюбились, можно все же испытывать единственную, высшую, собственно,
человеческую радость понимания. В конце концов, ребят, минутка мудрости. Извне.
Нас никто и ничто до конца не сможет сделать счастливыми, потому что, если мы
распространяемся вовне, то теряем единство с собой, становимся несвободны,
потому что зависим от того, что вне нас, а в целом можно быть лишь в одиночку. И
парадокс в том, что как только мы это мужественно принимаем, занимаясь во всем,
что бы мы ни делали, универсальным, собственно, человеческим занятием
осмыслением, и не желаем большего, тогда только нам даруются все это
необходимое, желанное. Нас начинают другие по-настоящему любить, ценить,
уважать, и жизнь наполняется действительным смыслом, когда самое главное вы
находите во всем вокруг что-то интересное. Но это уже решение скорее не с
позиции психоанализа, а экзистенциализма. 

\subsection{Экзистенциализм}

Принято считать, что началось это
направление с идеи датского населителя Сорина Киркегора, бундуя против
гигельянской попытки все объяснить с помощью тотального рационализма абсолютного
духа, еще в 18 веке, ой, простите, в 19 веке Киркегор в своем творчестве
исследует иррациональную сторону человеческого существования в категориях
переживания, страха смерти, отчаяния, свободы, веры, любви. Он обращает внимание
на то, что живой человек не какая-то абстрактная сущность, он всегда конкретный,
и в своих уникальных условиях зачастую не разумом руководствуется, особенно в
пограничных ситуациях предельных переживаний. Иначе невозможен был бы героизм,
невозможна была бы живая искренняя вера, настоящая любовь. Это такие
смысложизненные вещи, без которых даже со всей своей рассудительностью человек
не был бы человеком. Поэтому, говоря о человеке, нельзя игнорировать
уникальность и единичность его переживаний, которые никогда не уложатся в
универсальную схему. Михаил Михайлович Бахтин в 20-е годы 20-го века,
предвосхищая идеи немецких и французских экзистенциалистов, тоже концентрируется
на уникальности каждого реального человека. Единственность и незаместимость
положения каждого в бытии говорит не просто об индивидуальности, но означает и
то, что о поступках, захватывающих нас делах и происшествиях, любви, дружбе,
вере, осмыслению и так далее не может быть теоретического знания. Ведь оно
обобщение, которое как раз абстрагируется от уникальности и конкретики реальных
обстоятельств. Как пишет Бахтин, философия, пытающаяся вскрыть бытие события, не
может строить общих понятий положений и законов об этом, но может быть только
описание. Феноменология этого мира события может быть только участно описано. То
есть понимание события мысли возможно, если только пропускать все через себя,
непосредственно участвуя в проживании, переживании всего, что происходит. В
связи с этим, если целью исследования становится осмысление глубинной сущности
феноменов человеческого бытия, то действовать абстрагирующим и обобщающим
теоретическим путем становится непродуктивно. Истина чистого события в своей
полноте будет ускользать из сетей универсальной схемы. Например, любви можно
дать какое угодно научное определение, только это не поможет каждому из нас в
переживании и понимании реальной собственной любви. Чтобы понять во всей
полноте, что такое любовь, ее нужно пережить участным образом, так, чтобы она
захватывала все наше существо целиком. Другого пути нет. Не пережив ни разу
состояние любви, ничего истинного, никакой любви мы помыслить не сможем. Какими
бы теоретическими представлениями, чужими готовыми схемами много не попытались
воспользоваться. Смотрели фильм «Умница» Вилл Хантинг? Там как раз психолог,
которого играет Робин Уильямс, практикует такой подход по отношению к главному
герою. Его молодой Медеймон играет. Этот психолог говорит, мол, парниша, ты
много книг прочел, много знаешь, думаешь, что таким теоретическим образом ты
понимаешь, что такое по-настоящему любить? Нет. Это только из опыта у частного
переживания берется. Только когда ты всего всегда ставишь на карту, всем
рискуешь, страдаешь, все это испытываешь на своей шкуре, вот тогда что-то и
можно понять. понять об этом конкретном возлюбленном человеке, о своих чувствах
и о настоящем подлинном истинном чем-то, что через тебя в такие моменты течет.
Понять себя и понять, что такое жить по-настоящему, это не из книг и не из
теории берется, а должно уникальным образом случиться с нами, да еще и чтобы мы
приложили усилия к осмыслению этих своих переживаний, не просто отдались им, не
придумали, что такое с нами происходит. Тогда, с другой стороны, никакого
понимания без внутренней работы не случится, никакой опыт не усвоится. Почему-то
в массовом сознании экзистенциализм связывают с чем-то негативным и есть немало
мемов про это. Сейчас будет минутка отдыха, пожалуйста, на экзамене не
рассказывайте про мемы. Обычно у тех, кто сам не читал первоисточники,
складывается расхожее представление, что речь о тоске, отчаянии, обреченности
человека. Вот, мол, и Сартер писал драматические произведения, например,
«Тошнота». Такое себе название не очень воодушевляющее. Мои друзья, когда я из
химии ушла в философию, очень хотели меня поддержать и, зная, что мне близок
этот подход, начали выискивать всякие экзистенциальные мемы и мне их присылать.
Ну, много всего присылали и до сих пор иногда меня потролливают. Это я вам самые
безобидные показываю. Просто, чтобы вы не грустили и понимали. Потому что не на
этом акцент в рамках данного подхода, а на специфике такого уникального
существа, человека, которое прежде всего в его творческой и осмысляющей энергии
заключено. Но, чтобы ее реализовать и преодолеть нашу природную ограниченность,
конечно, надо над собой усилия совершать, работать над собой. А для большинства
это печально, потому что лень или страшно. Давайте разберем, каков
экзистенциализм на самом деле. В своих основных тезисах он очень похож на
стоицизм, с одним только отличием, что сформированы эти положения теперь в свете
не классической интуиции без основности, а античные стойки все же исходили из
того, что есть некие устойчивые основания всего, пусть и до конца непознаваемые.
Название направления экзистенциализм происходит до латинского экзистенция
существования, но для человека оно особенное. Мартин Хайдегер осмысляет
специфичность положения человека в бытии. Оно экзистенциально по своей сути, то
есть, как объясняет мыслитель, это особый тип существования, в котором возможно
понимание, выход к смысловому уровню, выделяя особость человеческого положения
бытия среди других вещей, мыслитель подчеркивают, что такое положение отнюдь не
дает права превосходства этому сущему способу понимания бытия. Напротив, наше
понимание действительности должно осуществляться в предельной ответственности,
поскольку оно задает способ нашего поведения в мире и отношения к остальному
сущему. Критикуя идею господства человека над природой и нигилистическое
отрицание ценностей, доведённое до предела философии Ницше, Хадегер показывает
опасности такого миропонимания. Это становится разрушительно как для самого
человека, оказавшегося в ситуации множественности ценностных ориентиров и
невозможности выбрать среди них одно устойчивое основание. Это для мира, в
котором всё с чудовищными темпами перерабатывается силами безликого
производства, не оставляя времени приспособиться к новым условиям.
Безответственность по отношению к своему существованию и к существованию всего
основного мира начинается с нежелания пробираться к настоящему смыслу
действительного положения вещей. Кроме того, в век потребления мы не привыкли
замечать действительную хрупкость всего вокруг и, занимаясь лишь паническим
обеспечением себя, обустройством своего положения, слишком мало заботимся о
человеческой стороне в нас, об осторожном, вдумчивом, понимающем существовании.
Хадегер называет такое стремление, направленное лишь на манипулирование сущим,
на обустройство в мире и нежелание при этом осмыслять настоящее несобственное
существование. Жан-Пульсартер в своем программном выступлении «Экзистенциализм
это гуманизм» более четко выделяет основную идею этого направления, отразившуюся
в названии. Для человека не сущность определяет существование, как, например,
для предметов форма и материал ножа определяют его назначение, но существование
определяет сущность. То есть, во-первых, нам никто и ничто не задает наше
предназначение, мы сами должны это решить, и во-вторых, поэтому мы своей жизнью,
своим существованием создаем свою сущность, а не так, что заранее есть какая-то
предрешенная программа, которую мы реализуем, как по рельсам и не можем ничего
иного. То есть, о том, каков человек можно судить по тому, как и чем он
наполняет свое существование. Оно как бы открыто и заранее ничто не
детерминирует, не принуждает нас ни к определенным занятиям, ни к этическому
выбору тех или иных поступков, ни к отношениям с теми, а не с другими людьми и
так далее. Именно свободой осмысленного и каждый раз уникального выбора и
специфичен человек. Так, ребята, насчет времени. У нас как бы подходит конец
пары, но все-таки давайте добьем до конца этот вопрос немножко подольше. Может,
еще минуток 10-15 у нас будет. Хорошо? Склассно? Ну да, как-то прерываться
будет. да, давайте тогда до конца этот вопросик добьем и потом уже пойдем на
перерыв, потом второй вопрос. Значит, смотрите, экзистенциальная традиция в
центре внимания, которой окажется человек во всей своей полноте и
парадоксальности, во всей конкретности своих жизненных обстоятельств выделяет
следующие черты специфического положения человека в бытии, которые принято
называть экзистенциалами. Во-первых, человек знает о своей конечности и это
побуждает его осмыслять. В том числе стремится выходить за границы простого
наличного положения вещей и заменять что-то в себе и в окружающем мире. Такие
уже знакомые нам мыслители, как Мартин Хайдегер, Пауль Тилех, Жан-Пуль Сартер
разбирают феномены страха, тревоги, ужаса, которые охватывают нас при
переживании опыта небытия. вспоминайте нашу четвертую тему про основание типов
культур эпох. Он может возникать не обязательно в форме страха смерти, но
проявляться и при осознании безосновности, окружающего хаоса, необеспеченности
своего бытия или когда мы в какие-то моменты ощущаем бессмысленность,
растерянность. Однако этим деструктивным страхом необходимо мужественно
противостоять и опорой может стать открытие в себе уже данного своего
уникального способа бытия, исполнение, в котором только и может дать нам полноту
жизни наполнить ее смыслом. Так что базовый экзистенциал осмысляющего
существование человека отталкивается от знания о своей конечности. Поэтому мы
задаемся вопросами, хотим что-то успеть, успеть осуществиться,
самореализоваться, а для этого не все подряд полезно. Надо различать что хорошо,
а что плохо. Значит, прежде всего думать. с этим связан второй момент. Только у
человека поэтому формируется совесть. Само это слово буквально означает
совместное ведание, а ведание это устаревшее слово для более привычного нам
знания. То есть у меня есть совместное знание. Что это такое? Это эффект
возникновения у нас в детстве самосознания. Мы начинаем понимать, что вот я и
вот мир. И я о нем столько всего знаю, столько всего мне в этом мире
открывается. И вот что мне с этим знанием делать, как его использовать, как в
его свете поступать. Я сам с собой начинаю вести внутренний диалог и совершать
каждый раз какой-то определенный выбор. И вот это все только дело меня самим
собой. Я сам с собой совместно ведаю согласованием морали и собственных желаний,
сопереживанием к другим или безразличиям, правдой, ложей и так далее. И поэтому
Сартер говорит, что человек открытое существо и наполнение жизни зависит от
свободного выбора самого человека, о котором он как бы сам договаривается,
только с самим собой. В этой связи Сартер называет человека строящим себя
проектом. Поскольку мы не представляем собой предзаданных сущностей, а самим
своим существованием характеризуем каждый себя, то своими выборами, решениями
ответственны как за свое индивидуальное бытие, так и за всю человечность или,
так сказать, идею человека. Насколько в своих поступках мы осуществляемся как
достойные существа, мужественные, осмысляющие, честные. Сартер говорит, что мы
обречены на свободу, не можем не выбирать, потому что даже если бездействуем,
ничего не выбираем, то это уже тоже выбор. Еще интересно, что бывает в некоторых
сложных ситуациях, пока мы не решились и не выбрали какую-то из возможностей, мы
не сможем понять, правильно ли мы поступили, потому что мы заранее никогда не
знаем, во-первых, с какими другими неучтенными обстоятельствами сцепятся наши
действия, а во-вторых, что мы сами будем переживать при этом, ведь это всегда
новый опыт. Как тогда можно говорить об ответственности субъекта? Я думаю, что
она обеспечивается не абсолютным предвидением расчетом, на который мы в реальных
условиях, условных ситуациях вряд ли способны, но готовностью извлекать для себя
опыт из любых событий, в которые мы попали, осмыслять, перерабатывать этот новый
опыт и исправлять свои ошибки в случае их совершения. человек не общее
универсальное понятие. Нет ведь человека вообще. Мы всегда, например, либо
женщина, либо мужчина, всегда конкретный человек в определенных обстоятельствах,
не идеальный, обладающий каким-то абсолютным знанием или агрективным видением.
Поэтому, в-третьих, подчеркивается фактичность нашего положения. Здесь имеется в
виду, что нельзя человека судить по каким-либо общим меркам. Чтобы действительно
понять, надо рассмотреть всю конкретику факта, его жизненные обстоятельства,
уникальные цели и ценности переживания в момент совершения поступка. То есть, те
самые факты, они всегда уникальны и единичны для определенного отдельного
реального человека в конкретный момент времени, в конкретном месте, здесь и
сейчас. Также, поскольку у нас нет предзаданной сущности и нам никто извне не
скажет о нашем предназначении, о смысле нашей жизни, это означает нашу
оставленность. Меня в свое время поразил в хорошем смысле взгляд Бибихина на
этот экзистенциал. Он говорит, что не обязательно вслед за Сартром видеть в этом
пессимизм брошенности, обреченности, отчаянного нашего положения. А что, если
наоборот, несмотря на всю нашу неприемлемость, нас оставили быть, причем
оставили быть свободными? Тогда круто, нам дали шанс, и от нас зависит не
упустить его, что-то понять и стать хоть шуточку лучше. Альбер Камю, французский
философ и писатель, также осмысляет обреченность человека и трудность, даже
абсурдность его положения в бытии. Он говорит, что мы вынуждены постоянно
совершать сизифов труд, если хотим быть по-настоящему людьми. Как сизиф
поднимает на вершину горы тяжелый камень, после чего тот снова скатывается к
подножию и нужно опять спускаться за ним и тащить его наверх. Также, сколько бы
мы не разгребали завалы своих печалей, страхов, непродуктивных желаний, сколько
бы побед над собой не одержали, эти состояния будут вновь возникать. Но мы не
должны сдаваться, даже если кажется, что совершать эти усилия абсурдно. Выход в
том, чтобы принять такое наше положение в бытии. Не значит опустить руки, но
напротив сосредоточиться на том, что мы действительно можем и взять свою
максимальную амплитуду в данных условиях. Про сизиф тоже много мемов есть.
Надеюсь, они помогут вам отнестись к сложностям человеческого бытия с юмором. И
на самом деле юмор это одно из замечательных оружий против страха, поскольку он
позволяет дистанцироваться от объекта страха. Давайте продолжим. По этому поводу
Рабка Смешно-Мурдашвили говорит такую потрясающую высказку, потрясающую идею.
Человек это прежде всего постоянное усилие стать человеком. Это не естественное
состояние, а состояние, которое творится. Напоминает возрожденческое определение
человека, которое Франческо-Петрашко дает в своих письмах. Человек рожден для
усилия, как птица для полета, рыба для плавания. В Амурдашвили замечает, что
добро, например, само по себе не делается. Честь и совесть нам не обеспечены
автоматически, нужно в поступках стараться их проявлять, в том числе борясь с
эгоистическими стремлениями и своими негативными качествами. Естественным
образом существуют только хаос и распад. Если мы не сопротивляемся в себе и
вовне, но нужно вопреки этим рассеивающим, расстраивающим тенденциям собирать
себя непродуктивно, когда мне плохо, когда меня что-то терзает, когда я страдаю.
Мне хорошо, только когда я целиком собираюсь на чём-то, на продумывание какой-то
мысли, в том числе о том, что беспокоит, или на полной отдаче себя какому-то
делу. Кроме того, плохо, если мы откладываем в будущее что-то по типу потом я
достигну какого-то полного знания, понимания, вот-вот всё налажу, расставлю по
полочкам, ножки свешу и тогда-то заживу. Этого ведь всего нету в реальности,
только в нашей голове. А что здесь и сейчас, что мешает мне пребывать во всей
полноте здесь и сейчас? Если моя жизнь состоит из череды моментов теперь, то
если в каждой я не вкладываюсь во всей полноте, откладываю каждый раз на потом,
то ничего настоящего не останется потом. Такое парадоксальное существо спасти
может только творчество. Что в нас собственно человеческого? Творчество смысла
во всём, с чем бы мы ни встречали, что бы мы ни делали, от изготовления
табуретки до создания стихотворений. Любая деятельность может принести счастье,
если я осмысленно этим занимаюсь, ещё что-то новое обогачаюсь знаниями, опытом,
если это не проходит мимо меня, как серая обыденность. Я участвую во всём
встречаемом своим чётким вниманием. Смысла нет без нашего усилия его
производства. С одной стороны, это тяжело, потому что надо постоянно расчищать
свой горизонт от схем и стереотипов, бороться с собой, со страхами и желаниями,
да и понимание не гарантировано, даже если мы этого очень хотим. Но ведь без
настройки своего состояния понимание даром нам не дастся, поэтому не бесполезно
готовить себя к моментам озарения. И с другой стороны, это ведь свобода и высшее
удовольствие думать самостоятельно, самим стараться разбираться. Никто за
каждого из нас так не сможет. Мы свободны в том, думать или нет, а также
ответственны перед самими собой, стремимся ли к полноте, к целостности, к
счастью.

\subsection{Итоги}

Что такое человек? Это уникальное существо, которое проходит
в своем развитии определенные стадии, на каждой из которых что-то теряет, но и
что-то приобретает. Так с открытием самосознания к трамгодам мы начинаем терять
изначальную свою гармоничную и естественную слитость с миром и единство с самим
собой. Но единство всегда желанно и всю жизнь мы не перестаем стремиться
воссоединиться с собой и миром. К пяти годам мы развиваем внутренний диалог
совесть, осмысляющий диалог с самим собой, который и должен помогать нам на
более сложном уровне, чем у всех остальных животных, достигать единства. Это
удивительный приобретение возможность познавать и обрабатывая опыт поступать,
свободно выбирать и стараться своими силами улучшать жизнь. В чем смысл жизни? В
том, чтобы быть счастливыми. Хотя и счастье у каждого уникально, для него не
существует инструкций, но мера одна на все времена. И в этом плане ничего по
сравнению с предшествующими эпохами не поменялось. Человек счастлив, если един с
самим собой, если осмысляет, стремясь к этому единству, то есть, если
обогащается опытом, смыслом и этим как бы расширяет свою душу, а не обедняет ее,
не сжимает. Самое важное, что нам сейчас открывается, возможность увидеть, как и
в античности, как и в возрождении, это то, что мы уникальные существа, потому
что имеем дело с парадоксальностью всего, и вокруг нас, и в нас сами. В нас
вдвинуты одновременные противоположности, способность и к добру, и к злу, и
логичная рассудительность, и иррациональная спонтанность, и материальная
составляющая, и духовная, встреча с многим на опыте, и умение умор выделять
единые законы, взаимосвязи, явлений. С одной стороны, это дарит нам открытость и
свободу, а с другой, накладывает огромную ответственность за осуществление
собственной жизни и за отношение к окружающему миру. Но Мордышвили абсолютно
прав. Человек — это усилие быть человеком, это неестественное состояние.
Благодаря лакану мы яснее видим нашу специфику в развитии воображения.
Подчеркнем, что не воображение вообще, оно может и непродуктивно
функционировать, если мы начинаем жить в модусе нехватки, фантазировать, служа
его идеальным образом, который всегда иллюзией, поскольку в реальном мире
идеального нет. Но если мы научаемся использовать воображение как инструмент, не
подчиняясь ему, можно продуктивно по назначению применять его для творчества.
Если подытожить, это все словами, сказанными уже в 21 веке нашим современникам,
итальянским философам Пауло Верно, специфика человека в том, что он не
специфичен. Возможно, мы появились случайно, но каким-то чудом увидели
ограниченность и хрупкость собственного бытия, встревожились и начали
придумывать, как выжить, как себя спасти. Собственно, человеческими
способностями, универсальными, обратите внимание на еще один смысл этого
маркетинского термина, являются наши коммуникативные и когнитивные способности.
Никто, кроме человека, не исполняется в общении с другими индивидами и не
обучается непрерывно всю жизнь, находясь именно в этом, в человеческих
взаимоотношениях, в понимании себя, смысл жизни. Не в каких-то конкретных
занятиях, ребят, мы по-настоящему исполняемся, а в поступках по отношению к
другим и в творчестве нового. Заботясь о выживании таких непредспособных
существ, мы наизобретали кучу полезных техник, открыли столько всего нового и
интересного, но если теперь мир превратился в глобальный конвейер, то это
очередной шанс задуматься, как быть. Не мы, не наше с вами поколение довели
природу до экологических и энергетических проблем, не мы выбрали родиться в
условиях, с одной стороны, скучной рутины, однообразного, до ужаса,
алгоритмизированного производства и товаров, и услуг, и даже научных статей для
массового потребления. С другой стороны, в условиях глобальной социально-
политической напряженности и нестабильности. Но наша задача в том, чтобы
мужественно увидеть это наше дано, воспринять условия, задачи и своими
поступками решить, сделать выбор, терять себя на этом конвейере, в том числе
идеологическом, или вопреки всему совершать творческий акт, прорыв к по-
настоящему новому. Оно возможно только в индивидуальных актах осмысления и
только в поступках, в поддержке окружающих таких же потерянных других. Сетева
натягивается и от нас с вами зависит, насколько мощно и насколько в правильном
направлении будет выпущена стрела и будет ли, не сорвется ли. Сможем ли мы своим
творчеством преодолеть конфликты, болезни, все низменное, оправданы ли наши
жизни, стремимся ли мы к настоящему или вязнем во внимостях, впишем ли в
культуру строчку вдохновения. От ответов на эти вопросы, которые каждый из нас
себе задает и от реальности наших поступков в свете такого осмысления зависит,
что же такое человек и насколько мы достойные существа. Благодарю за внимание,
дорогие друзья. Счастливо буду ответить на ваши вопросы в особенности по такой
животрепещущей теме. Сейчас я обратно перекручусь на Google Meet. А ребята?
Получается, в итоге вы собрали итоги каждого из вариантов, каждого из
направлений? Я постаралась. Вон кто-то руку поднимал. Да, я ребята за
презентацией, просто она у меня на весь экран открывается, и я не всегда вижу,
если у вас там вы с текстом что-то пишете или там вот эту руку поднимаете, то вы
лучше словами говорите. Ну, или сейчас текстом напишите. Вот, если у вас ребят
пока и потом что-то созреет, вы всегда можете написать мне на мейл, я вам
отвечу. Вот. Я предлагаю пойти на перерыв. Да, листается. Все, отлично, тогда
давайте теперь с вами ко второму вопросу перейдем и заглянем в самую сущность
языка в его связи со смыслом и пониманием. 

\section{Лингвистический поворот в неклассической философии}

Мы будем двигаться как обычно по плану.
Предлагаю такую логику. Сначала попробуем сами задаться вопросами, что такое
язык и что такое смысл. Затем познакомимся последовательно с самыми выдающимися
мыслителями и направлениями философии уже больше 20-го столетия. По неслучайному
совпадению все они занимались с учением именно на феномена языка. Уникальные
презрения Людвига Витгенштейна, фундаментальные исследования Мартина Хайдегера и
продолжение экзистенциальной аналитики языка в герменевтике его учеником Гансом
Георгом Гадемером. Не оставим без внимания и величайшего русского мыслителя
Михаила Михайловича Бахтина, который на несколько десятилетий раньше французских
философов разработал концепцию диалогичности и труды которого переведенные через
полвека будут вдохновлять постструктуралистов на их лингвистический проект. Мы
должны будем оговориться об особенностях человеческого языка упирающегося в
тайну нашего происхождения и наконец познакомимся с важными идеями
структуралистов прошлого столетия. Поскольку ни в одном учебнике по философии
науки да и наверное по философии вообще не найти этого материала в таком объеме
такой глубине и широте на котором мы сейчас с вами замахнемся будьте предельно
внимательны но надеюсь этот вопрос будет удивительным для вас во многом новым а
значит расширяющим горизонт и вдохновляющим.

\subsection{Проблема определения языка и смысла}

\paragraph{Язык}
постараемся воссоздать ту логику, в которой на язык вдруг обратили внимание самые крупные мыслители 20 века, осуществив лингвистический поворот философии.
Оговоримся только: поворот от чего к чему? Речь о смещении основного фокуса исследований на проблематику языка
небосущности структуры функций и механизмов буквально управления нами нашим
мышлением философии господствовавшей до этого проблематики познания вспоминать
гносиологический поворот нового времени и растерявшийся перед открывшейся
безосновностью первой философии в начале этой темы мы сегодня рассмотрели
преодоление метафизики в том числе в антологическом смысле несмотря на то что
общение между людьми существует по сути столько же сколько само человечество
философия пристального внимания на понятие языка и его роль в человеческом
обществе обращается лишь в 20 столетии в эпоху стремительных трансформаций
действительности глобальной взаимосвязи и взаимозависимости активизация
международного общения интерес возникает к глубинной сущности самого языка
поразительным образом участвующего в конструировании нашей реальности и так
привычнее всего нам думать что язык это средство общения или или что-то еще или
совсем не средство а что-то другое насколько мы понимаем что такое язык когда мы
сталкиваемся с вещами настолько близкими и самопонятными в смысле какие могут
быть вопросы это же самое ближайшее и понятнейшее для человеческого существа о
чем тут говорить полезно останавливаться и удерживать себя от того чтобы не
проскочить мимо самого главного самые казалось бы привычные вещи на самом деле
таят в себе такие огромные пространства необычного захватывающего и опасно
спешить поскорее мимо них не задумываясь отмахиваясь почему опасно потому что
так мы рискуем упустить настоящее а упускать настоящее означает жить не по-
настоящему то есть в иллюзорном мире в мире готовых решений а значит в мире
скучном и обыденном в мире в котором все и так понятно все уже изучено и нет
места новому и вот если места новому нет если мы не умеем открывать новое в
обыденном то мы обречены на совершенно нетворческое существование на конвейерное
однообразие естественно мы как ученые ядром деятельности которых является
творчество нового научного знания изобретения инновации мы не хотим остаться без
настоящего подлинного не хотим оказаться на периферии на обочине вдали от
творчества так что должны изо всех сил тренировать умение видеть и видеть
действительно суть окружающей нас реальности давайте потренируемся это делать
например языка как увидеть язык по новому по настоящему возможно ли это что
такое язык вот в ответ на этот вопрос сразу же попадается готовая схема
сформулированная общественным мнением язык это средство общения или коммуникации
то есть такой инструментальный смысл такой казалось бы служебное у него значение
при поверхностном взгляде и правда кажется что мы для того чтобы что-то сообщить
передать другому берем слова нашего естественного языка родного или иностранного
ставим их в определенной последовательности как принято строить предложение и
произносим или записываем другой получает сообщение выделяет значение слов и в
зависимости от их порядка понимает то что мы ему передали но как тогда объяснить
феномен непонимания вы ведь знаете на своем опыте что даже когда мы произносим
абсолютно известные всем слова в совершенно грамотном с точки зрения синтаксиса
порядке нас сплошь рядом не понимают или понимают неправильно искажая смысл того
что мы хотели донести кому-то может наша интонация не понравится хотя мы ничего
такого не имели в виду а у кого-то просто смысл не сложится еще мы можем думать
одно а говорить другое врать по сути к примеру вскрывать свои истинные чувства
ревнуем например но вместо того чтобы сказать об этом говорим противоположное
мол рад за тебя что у тебя с этим человеком складываются хорошие отношения тогда
ребят если смотреть по-честному мы должны будем дополнить наше первоначальное
определение язык это не только средство общения но и средство разобщения не
только инструмент понимания но и инструмент непонимания с другой стороны бывает
что мы не очень хорошо подобрали слова и выражения может быть даже с речевыми и
синтаксическими ошибками что-то сказали но другой каким-то чудом понял нас хотя
можно было бы сказать точнее и грамотнее или мы использовали в своей речи слова
незнакомые другому но человек догадался по контексту о чем идет речь и даже не
зная точного значения все-таки нас понял вообще точное понимание большая
редкость это буквально чудо почему так происходит потому что слова сами по себе
не содержат смысл без нашего участия без усилия осмысления и очень часто фразы
обращенные к нам реплики и даже целые книги остаются для нас просто набором слов
если мы не потрудились их понять это происходит поскольку смысл не совпадает со
значением а понимаем мы смысл а не значение а значения слов вроде бы всегда даны
это их определение через другие слова которые изложены в толковых словарях
например и более или менее определенные даже если у слова несколько непохожих
значений например ключ как источник дверная отмычка и музыкальный символ море
как вид водоема и в значении много и кстати язык как орган вкусовых ощущений и
как знаковая система так вот мы все употребляем одни и те же слова например
любовь дружба семья совершенно обычные повседневные слова но понимание их для
каждого человека сопряжено с его собственным видением с его собственной
жизненной историей для кого-то семья высшее благо самые родные люди поэтому надо
стремиться к созданию семьи ее сохранению и заботе о близких для другого
человека с кем к примеру в детстве плохо обращались слово семья будет
наполняться чуть ли не противоположным смыслом чего-то тягостного неприятного
сдерживающего обязывающего и такой человек может не стремиться к созданию
собственной семьи заботе о родственниках заведению детей значение одно семья это
совокупность особей связанных родственными связями а смысл для разных людей в
нем сворачиваться может совершенно различный а могут быть вообще неосмысленные
слова мы часто механически употребляем какие-то привычные выражения клише
языковые штампы и даже не вдумываемся в них слова остаются просто словами пустым
звуком как говорится и вот на экзамене когда аспиранты приходят отвечать и
произносят списанные пустые для них фразы то сразу чувствуется когда
экзаменуемые не понимают о чем он говорит просто произнося типа умные слова это
не значит что на экзамене следует избегать каких-то конкретных формулировок нет
но вы должны показать понимание отдавая себе отчет в каждом слове которое
произносите иначе язык сыграет с вами на экзамене дурную шутку став средством
разобщения с преподавателем лучше сказать простыми и понятными вам самим словами
и пусть даже немного коряло но если вы вкладываете смысл в то что говорите
настроенный на понимание собеседник вас поймет и если нужно подправить подсказав
более точные слова для выражения вашего замысла герцогиня дает кэрловской лисе
потрясающий совет думая о смысле а слова придут сами в ответственные моменты на
выступлениях на экзамене ребят очень губительно думать о том как вы выглядите со
стороны достаточно ли сложные термины используете и так далее когда требуется
лишь одно сконцентрировать все свое существо в момент говорения на смысле на
сути того что вы хотите сказать если вы действительно подберутся сами еще
моменты непонимания могут быть связаны со всевозможными метафорами и переносными
значениями вы же знаете что фразеализм нельзя понимать буквально надо учиться
чувствовать когда собеседник имеет ввиду прямые значения слов когда он украшает
свою речь сравнениями аллегориями метафорами здесь же раз уж речь зашла об
экзамене к примечанию расскажу вам о показательном случае имевшем место в рамках
преподавания нашей дисциплины несколько лет назад Светлана Викторовна читает
лекцию о средневековых науке и философии и делает знание очень харизматично там
есть вопрос про возникновение университетов и увлекшись воссознанием в красках
той интеллектуальной атмосферы которая царила в Европе 12 века Анна лекция
произнесла метафору средневековый университет это Пьер Абеляр сидящий на лавочке
и рассуждающий естественно сказано это было именно в переносном значении потом
на семинаре я спрашиваю своих ребят как они поняли что представляется по
средневековому университет и кто-то мне на полном серьезе воспроизводит эту
фразу как определение мы конечно на душе посмеялись я объяснила тем кто не понял
что это метафора но самое печальное что то был не единичный случай и потом на
экзамене мы слышали несколько таких же ответов из других групп когда люди
бездумно воспроизводили эти слова еще и когда мы пытались поправить возмущались
а нам так сказали на лекции и трясли своими конспектами так вот пожалуйста
рефлексируйте ребята оценивайте что из сказанного преподавателями является там
метафора или их личным мнением и сказано в контексте определенного разговора а
что какое-то объективное определение к сожалению на экзамене нередко приходится
слышать подобное бездумное воспроизведение вырванных из контекста фраз поэтому
ребята вы будущее отечественной науки единственная думающая элита страны
пожалуйста осмысляйте если у нас будут недумающие ученые то что тогда удивляться
поводу остального населения какая у нас тогда будет страна привносите смысл в
свои слова не говорите пустое чтобы самим не превратиться в пустышке 

\paragraph{Смысл}
что же
такое смысл и как так получается что он оказывается не совпадает с языком
автоматически не присутствует в словах слова максимум указывают на смысл смысл
это выражаемое предложение как говорит французский философ постмодернист
Жильдалес смысл тончайшая и неуловимая ускользающая штука существующая только в
наших уникальных и единичных актах понимания осмысления уникальных и единичных
потому что за каждого из нас в нашей голове никто другой подумать не может смысл
существует только в момент осмысления только в настоящем времени здесь и сейчас
и то только если мы совершаем это усилие по осмыслению его не приберечь на
будущее никак не сохранить и не обеспечить ни себе возможность понимания ни
другому понимания меня если мы понимаем что-то сейчас это вовсе не гарантирует
что мы будем способны понимать в следующий момент или через какое-то
продолжительное время нам каждый раз придется начинать заново прокладывать
дорожки к смыслу но если мы поняли что-то это обогатило нас каким-то странным
образом как бы расширило наш горизонт расчистило наш взор это пожалуй самое
приятное человеческое чувство поэтому раз что-то поняв мы стремимся вновь и
вновь оказаться в состоянии понимания с временем если мы не будем практиковать
осмысление тренировать свою способность к пониманию буквально как мы тренируем
мышцы тела физическими упражнениями чтобы поддерживать его форме этот эффект
чего-то понятого пройдет и без нашего усилия заново творения смысла не вернется

в связи с этим есть такой лайфхак если вы хотите что-то выучить то зубрить и
запоминать не особо понимая это совсем не продуктивный путь надо понять и
разобраться почему так зачем это нужно как это применяется и тогда от чуда
ничего учить не придется вы просто поняв сможете каждый раз заново с нуля
воссоздавать нужные знания или способы решения у вас будут в голове возникать
нужные ассоциации и все разархивируется так сказать если вы натренировали
нейроны в своем мозге выстраивать эти дорожки а это очень хорошо делается когда
вы сами что-то ищите в разных источниках рассматриваете с разных сторон и
стараетесь разобраться в причинах и здесь можно заметить что 

\subsection{Философия языка Л. Витгенштейна}
с помощью языка
формируя суждение человек как бы примеривается к миру мы все завешиваем на да
нет истинно ложно существует не существует хорошо плохо и так далее говоря ту
или иную фразу мы набрасываем рисунок того явления события вещи о которых речь
естественно этот рисунок не является прямой копией того уникального феномена с
которым мы здесь и сейчас имеем дело тем более не есть он сам тем же рисунком в
других обстоятельствах можно будет обозначить что-то иное например круг может
обозначать в одном контексте колесо в другом мир например я на семинаре по
античности кружками изображаю микрокосм космос и ничто не мешает нам в какой-то
момент увидеть в нем обозначение скажем города Москва любые наши сколь угодно
сложные фразы похожи на схемы в которых мы пытаемся схватить смысл схема от
греческого схейн схватывать отсюда несколько следствий во-первых слова нашего
языка скорее что-то вроде палитры красок у художника а не готовые картины
соответствующие предметам действительности все зависит от нашего выбора как что
скомбинировать чтобы показать то есть это просто набор возможностей но
определяет смысл высказывания то как мы ими пользуемся умеем ли подручными
средствами схватить главное во-вторых одно и то же можно выразить разными
словами и лежащий в основании смысл как цель направленность высказывания от
этого не пострадает или же наоборот в разных контекстах или даже сказанное с
разной интонацией определенная фраза в одних и тех же обстоятельствах может
указывать на разное положение дел например есть разница если я вам скажу идите
отдыхайте или идите отдыхайте и в третьих данная особенность языка означает что
за словами не закреплено соответствие с предметами реальности они призваны не
жестко обозначать вещи и явления как мы привыкли полагать нет это сочетание слов
их взаимная организация в фразе соответствует связям и корреляциям предметов в
мире а не сами слова предметом не то что маленькие АБЦ слова соответствуют
предметам реальности большим буквам АБЦ нет мы передаем порядок связей предметов
и явлений и слова могут быть заменены другими например W X Y но отношения будут
теми же например можно сказать у человека есть априорные формы чувственности или
до опыта в нашем сознании уже содержатся структуры в которой потом опыт будет
укладываться я выразила сейчас двумя разными фразами один и тот же смысл и
кстати умение об одном и том же сказать по разному а также показать на примере
достаточно надежные критерии понимания смысла мы на экзамене так и проверяем
просим пояснить что стоит за какими-нибудь умными словами которые вы говорите и
привести примеры если человек может иначе сказать и проиллюстрировать свое
высказывание то скорее всего понимает смысл своих предначальных формулировок и
это показать на экзамен самое главное осмысленность понимания поэтому истинные
или ложные не слова которые мы подобрали для высказывания и тем более не сама
реальность но истинными или ложными могут быть выстроенные нами связи насколько
они схватили действительное положение вещей или не схватили ну и точно так же мы
имитируем понимание говоря умные слова или по-настоящему поняли и свободно
выражать смысл различными формулировками на подобные моменты обращать внимание в
своих трудах выдающийся австрийский философ Людвиг Вильгинштейн в окопах Первой
мировой он пишет свое известнейшее потрясшее весь интеллектуальный мир того
времени произведение логикой философских трактат это просто кладезь
концентрированных формул для разворачивания смыслов и расширения своего видения
того как все есть на самом деле в первую очередь с нашим языком и отношением к
миру через язык говоря о фразе как рисунки передающем то или иное положение дел
Вильгинштейн приходит к тому что язык является по сути органом нашей души точно
так же как мы контактируем с физическим миром посредством органов чувств получая
данные об окружающем мире язык представляет собой бестелесный орган понимания по
измериваться фразами ко всему миру в целом далеко за пределами наших органов
чувств измерять надо и нет проверять предполагать конструировать свой мир
который несоизмеримо больше того что мы за всю свою жизнь можем сами познать
эмпирически и еще один момент который я хотела бы обсудить с вами из
витгенштейновских известных формул границы моего языка означают границы моего
мира давайте вдумаемся в эту фразу у нее несколько взаимно дополняющих смыслов
самый прямой мы можем понять мир настолько насколько у нас хватает средств языка
и способности выражать свое понимание мира то о чем мы не можем ясно высказаться
и то о чем мы никогда не слышали по сути для нас в нашем мире не существует
поскольку до смысла мы в таких случаях не доходим поэтому в частности необходимо
тщательным образом работать над своими языковыми способностями в том числе
расширять свои познания в рамках других языков не случайно говоря сколькими
языками ты владеешь сколько раз ты человек однако прежде чем бросаться изучать
иностранные языки следует конечно же максимально оттачивать навыки выражения
своих мыслей на родном о самом глубоком и самом значимом в какой среде мы бы не
находились мы все равно думаем на родном языке но есть в этой формуле еще как
минимум один важный для нас сейчас смысл мы как бы просвечиваем все в мире
границами или формами выделяем единичные вещи выявляем их отдельные
характеристики разграничиваем все и вся иначе не то что науки не было бы вообще
человеческого способа мышления не было бы в предыдущем вопросе сегодня мы уже
затронули то что формы и границы любого рода различия привносятся в картину мира
нашим воображением их самих по себе вне нашего мышления нет например кошка
вообще это форма чистая идея благодаря которой мы различаем в мире конкретных
кошек но абстрактные кошки вообще в мире на опыте мы никогда не встретим умеем и
имеем дело соответственно каждый раз только с отдельными уникальными кошками эта
форма нам дана в языке как слово с помощью которого можно упорядочивать
действительность ориентироваться в ней и строя суждения примериваясь к миру
этими чистыми формами понимать так вот открывается гораздо более глубокий смысл
того что границы тут котик всем передает привет кот не кричи извиняюсь ребят
значит смотрите более глубокий смысл того что границы моего языка означают то
есть не только являются как но и обозначают как процесс действия границы моего
мира буквально языковая способность это способность прочерчивать и перечерчивать
границы привнося их в мир способность различать и прямое следствие чем более
развита наша различающая способность тем более ярок сочин многообразен и глубок
мир для нас поэтому то насколько каждый уделяет внимание своему языку или своему
языковому развитию настолько все руки границы нашего мира настолько мы способны
быть собственно людьми быть способом понимания быть по-настоящему счастливы но

\subsection{Язык~---~среда бытия: М. Хайдеггер, Г.-Г. Гадамер}
тогда язык буквально среда нашего бытия среда в которой существует человек
который нас и разъединяет единичными актами творчества смысла и соединяет
возможности понимания мира и друг друга Мартин Хайдегер в своей философии
опирается на специфичность человеческого положения в бытии по сравнению с
другими вещами и существами как мы с вами уже в прошлом вопросе ответили в
предыдущем человеческое бытие отличается наполненностью смысла данное особое
положение человека в бытии принято называть как мы тоже уже отметили
экзистенциальным от экзистенции особое существо наполненное смыслом то есть
можно говорить что человека от всего остального отличает способность к пониманию
смысла в пределе смысла вещей смысла мира и смысла своего собственного бытия
согласитесь человек не может полноценно быть человеком если не понимает хоть
каким-то образом смысл всего мира и своего положения в мире в бессмысленности мы
не можем существовать даже если все остальные потребности удовлетворены например
есть что кушать какую одежду носить с кем жить и так далее наше существо всегда
ищет смысл стремиться понимать а поскольку самая фундаментальная единая основа
всего бытия то благодаря чему все есть то для того чтобы все вокруг понимать
человек должен хотя бы интуитивно то есть не обязательно проговаривая это для
себя понимать смысл бытия Хайдегер говорил что наше понимание устроено
интересным образом бытие отражается в любом человеческом языке в каждом из
которых есть слово быть или есть в любом суждении любого языка даже если эта
связка опускается как например в русском присутствует хотя бы в качестве
подразумеваемого или сворачивается в глаголе слово есть мы строим наше суждение
в целом в рамках конструкции с есть п то есть субъект подлежащая есть предикат
сказуемое например по-английски I'm Catherine присутствует связка M по-
французски Je suis Catherine связка Sui по-испански Yo soy Catarina связка Soy я
не знаю по-японски Watasiva Catarina Dasu хотя порядок слов обратный чем в
европейских языках та же связка Dasu запишите подобные примеры своим именем
значит в каждом высказывании нашего языка содержится или как в русском бывает
предполагается отсылка к тому благодаря чему есть то о чем в данном высказывании
говорится отсылка к смыслу бытия в этом ключе Хайдегер называет язык домом бытия
то есть такой средой через которую понимание смысла становится возможным в
детстве мы научаемся пользоваться языком и многие фундаментальные вещи
сворачиваются для нас в языке как сами собой понятные однако для того чтобы
понимать все в целостности необходимо стремиться разворачивать смысл незаметно
свернувшийся в самых привычных словах поэтому мыслитель призывает к особой
чуткости и внимательности к языку и действительно всматривание в язык дает очень
много для углубления нашего понимания о мире и о себе продолжая такую традицию
понимания языка своего учителя Ганс Георг Гадамер разрабатывает особую
методологию осмысления философского герменевтику от древнегреческого эрминэотике
толковательно интерпретирующий это предполагает работу со словом его этимогогии
то есть происхождение его оттенками и значениями в других языках Гадамер
называет язык всеобъемлющий предвосхищающий истолкованностью мира это означает
что естественный язык который мы пользуемся всегда в свернутом виде содержит
особо интуитивное предпонимание бытия которое в историческом контексте меняется
то есть одни и те же понятия в силу различного мировоззрения основанного на
различном понимании бытия по-разному нагружены в смысловом плане в различные
эпохи вспоминайте хотя бы традиционный в этом отношении пример различном
понимании прекрасного в человеческой истории в различных социокультурных
локальностях например каноны женской красоты совершенно различных в различных
обществах толкование зависит от контекста той ситуации в которой мы себя
обнаруживаем от круга вопросов которые кажутся в ней наиболее важными от угла
зрения на эти вопросы на прояснение этих моментов и направлены прежде всего
герменофтические техники интерпретации таким образом герменофтическое понимание
обращает наше внимание на историческую и культурную обусловленность выражения
человеческой мысли действительно понять говорящего можно лишь погрузившись в тот
контекст в котором сформировались у этого человека представления о мире понятия
которыми он мыслит и способ выражения этих представлений в языке то есть
недостаточно говорить с человеком на одном языке например по-английски чтобы по-
настоящему глубоко понимать позицию другого нужно понимать еще и основания
культуры в рамках которой эта позиция сформировалась вспоминайте нашу четвертую
тему об антологических основаниях мировоззрения тогда мы действительно будем во
всех смыслах слово говорить на одном языке и именно для вот этих вот целей мы с
вами в нашем курсе чтобы понимать исследователей других эпох и погружались в тот
самый культурно-исторический контекст и старались выйти к наиболее
фундаментальным антологическим основаниям каждого периода в любом случае полезно
совершать герметические практики обращая внимание на происхождение и разные
оттенки не только новых но и знакомых привычных слов и словосочетания это
помогает лучше видеть все как оно есть как Хайдеггер говорит бытие высказывается
в каждом слове и именно таким образом замалчивает свое существо наш язык
одновременно и говорит и умалчивает например слово медведь говорит о том что он
знает ведает где мед медведь то есть это любитель меда но вместе с этом
умалчивается о том что это мощный дикий зверь с которым страшно безоружным
столкнуться возможно вам тяжело поэтому спешу успокоить без сомнения не все люди
обязаны глубочайшим образом всматриваться в существо языка это обязанность
прежде всего философов и специалистов вопросов языка тех чьим призванием
является непосредственно исследование речи коммуникации и так далее однако
заслуга Хайдегера в том что он во-первых обращает наше внимание на связь языка с
тем благодаря чему мы и все остальное есть с бытием а во-вторых призывает нас не
быть слишком небрежными с языком мы все как не специалисты погружены в язык и
работать с языком периодически нам всем необходимо пусть и не включая
предельного усилия выхода к смыслу самого бытия всматривание в язык и развитие
навыков пользования языком обогащает нас понимание у Хайдегера конечно навека
вперед открытие для осмысления но еще один примечательный момент из его книги
«Бытие и время» давайте здесь тоже отметим почти 100 лет назад вышло в свет это
произведение а там наследитель описывает буквально то что происходит с
современным человеком который беспрестанно залипает за всевозможные соцсети
мессенджеры и навыков ленты Хайдегер показывает как люди вовлечены в
пространство болтовни сплетен передачи друг другу каких-то мелочных сведений в
русском переводе в «Бытие и времени» это слово не болтовня а толки то есть люди
толкуют о том и сём в чём обычно немного смысла или он не особо ценен всю минуту
но это болтовня захватывает мы очень легко заслушиваемся чем-то совершенно
повседневным разговором читаем передаём друг другу и обсуждаем не особо ценную
информацию ну например мне пишут друзья что вот вышла такая-то озвучка нового
сериала а ты уже там начала смотреть или а ты видела новость про там допустим
анонс ремейка нашей любимой старой игры или там какая-нибудь знаменитость
высказалась в поддержку чего-то а ты что думаешь и так далее так существует язык
он живёт актами говорения и в форме болтовни воспроизводит себя через нас хотя
бы вот на таком элементарном уровне как бы предоставляя возможность потенциал
для осмысления и понимая такие сообщения мы используем их как материал для
тренировки сознания чтобы хоть в чём-то думать постоянно поддерживать в тонусе
свои интеллектуальные способности во-первых а во-вторых обитать в одной и той же
информационной среде что и те люди с которыми нам хочется общаться с другой
стороны этот момент помог мне глубже понять словосочетание естественный язык
сколько в языке воспроизводится генерируется фраз информации которой мы
подключаемся и которая чуть ли не питается наше сознание то есть там такая
природная естественная автоматичность царит что надо иметь сильную волю и
соответственно понимать свою цель чтобы периодически выходить для себя из этой
чистой информационной среды на более продуктивный уровень конечно надо
подключаться болтовни но чтобы она не рассеивала и не растрачивала наше существо
на такой низший уровень воспроизводства просто информации необходимо
использовать это естество языка для действительного осмысления например
непродуктивно залипать за интернадцать летний картинки знаменитостей там не знаю
моделей политиков если вы подписаны скажем на художников или дизайнеров на
научные паблики или на каких-нибудь рукастых ютуберов которые копаются в
электронике и сами что-то создают и черпаете для себя вдохновение знакомясь с
такими произведениями или читая практически совет то это достойный продуктивный
уровень на котором вы не просто потребляете контент который на самом деле
никогда не насытится но соучаствуете в рождении смысла обогащаетесь новым опытом
подытожим экзистенциальная аналитика в лице Харри Гера и геременефтика в лице
Гадемера приводит нас к тому что язык есть особая среда дающая увидеть смысл
прийти к нему буквально среда обитания человека но это все также означает что
другой думает сам в своей голове или не думает мы не знаем какие у него
ассоциации возникают как он воспринимает нас и наши сообщения в пределе
абсолютно понять друг друга мы не можем даже если давно знаем человека но
принципиально физически не способны переживать то же самое что он переживает
видеть вещи в том же свете в котором он видит однако трагедии в этом нет просто
смысл живет таким вот мерцающим образом передается только изменяясь то есть
когда каждый индивид каждый раз самостоятельно интерпретирует какие-то свои
оттенки добавляя свой путь проходя понимая другого в данном ключе к нему
невозможно относиться как к объекту с этой инаковой позиции мы выстраиваем
отношения как с другим субъектом имеющий собственный отличный от нашего способ
бытия но тем не менее также способный открыть для нас целый мир 

\subsection{<<Философия диалога>> М.М. Бахтина}
по этому поводу
потрясающий отечественные мыслители Михаил Михайлович Бахтин философ и
литературовед еще в начале 20-х годов 20-го века пишет следующее зачитаю вам
объемную стату ее конспектировать не нужно просто послушайте и основу идеи для
себя отметить когда я созерцаю цельного человека находящегося в ней против меня
наши конкретные действительно переживаемые кругозоры не совпадают ведь в каждый
данный момент в каком бы положении и как бы близко ко мне не находился этот
другой созерцаемый мною человек я всегда буду видеть и знать нечто чего сам он
со своего места в ней против меня видеть не может части тела недоступны его
собственному взору голова лицо его выражение мир за его спиной целый ряд
предметов и отношений которые при том или ином взаимоотношении нашим доступны
мне и недоступны ему когда мы глядим друг на друга два разных мира отражаются в
зрачках наших глаз этот всегда наличный по отношению ко всякому другому человеку
избыток моего видения знания обладания обусловлен единственностью и
незаместимостью моего места в мире ведь на этом месте в это время в данной
совокупности обстоятельств я единственная нахожусь все другие люди вне меня то
есть смотрите каждый из нас уникален благодаря тому что занимает в бытии
единственное и незаместимое место ведь мое место физически никто не может встать
и психологически переживать мои состояния тоже никто со мной не может поэтому
ценен каждый другой и изначально от рождения наделенные каждый своим местом мы
все в этом фундаментальном смысле равны друг с другом и поэтому уже взаимно
дополнительно Бахтин говорит о специфике человеческого положения в быти а именно
о парадоксальности нашего сосуществования друг с другом мы не можем ни слиться с
другим ни избавиться от него совершенно у каждого из нас своя неповторимая
уникальная позиция по отношению к другим что с одной стороны дарит нам
целостность и единство собственного существования но с другой стороны
отграничивает нас от других невозможностью переживать опыт другого за него в его
точке тем не менее выстраивая свои дорожки к смыслу одного и того же мы
обогащаемся от общения с другим взаимно благодаря друг другу расширяя свои
кругозоры по отношению к я этим другим может оказаться не только конкретный
человек другое я но и группа людей народ иная культура природа или даже например
бог с такими другими необходимо выстраивать диалог необходимо это делать не
просто из повседневной бытовой нужды общения с другими людьми или из
профессиональной любознательности к природе но и в целях саморазвития и
самообогащения смыслом думаю каждому из нас знакома непродуктивность ситуации
что называется вариться в собственном соку так даже на психологическом уровне
специалисты советуют делиться друг с другом своими переживаниями и мысли
опрощаться к текстам культуры или к общению с природой в ситуации потерянности
или глубокого негативного переживания тем более это нужно когда мы не можем
разобраться в той или иной проблеме во-первых проговаривание то есть
формулирование этой проблемы вслух нам помогает лучше прояснить ситуацию для
самих себя а во-вторых другой ценен тем что представляет собой отличную отношу
буквально точку зрения то есть он смотрит на то же самое с другой позицией пусть
и близко к нашей но другой поэтому дает нам ценнейший взгляд со стороны
обогащающий наше понимание да и обращающий внимание на то что нам может быть не
видно со своего как бы места в принципе тогда общение можно понимать предельно
широко например обмен подарками представляет собой непростое перемещение вещей в
пространстве но обладает глубоким смыслом для людей и зачастую подарок в который
мы вкладываем частичку себя может сказать другому больше чем мы способны
выразить словами исследуя в данном контексте природу коммуникации Бахтент
подчеркивает что на самом деле единицы языка не отдельные слова или предложения
как это полагается в науке лингвистики но целостные завершенные высказывания
границы которых определяются сменой речевого субъекта то есть в широко понятом
глобальном диалоге настоящие единицы языка фразы индивидов разграничиваются
целостностью замысла локальностью каждого говорящего эти единицы должны быть
анализируемы непосредством расщепления на отдельные элементы вплоть до звуков и
букв а именно исходя из смысловой завершенности высказывания из подвора
говорящим средств формы для конструирования своего выражения независимость от
того это высказывание допустим представляет собой междуметие эх или целую книгу
и даже молчаливый ответ или пауза тоже своего рода высказывания имеющие замысел
смысл цель определенного воздействия и так далее 

\subsection{Тайна происхождения языка: молчание и бессознательное}
и тут мы подошли к важнейшим
двум поразительным отличиям человеческого языка от языка животных в целом языка
природы во-первых животные не выходят на уровень смысла на уровень понимания или
непонимания символы которые они друг для друга создают оставляя метки железами
например запахом задирая кору на деревьях передавая голосом или движениями тела
сообщения друг к другу распознаются однозначно тигр почуяв определенный запах
оставленный сородичем однозначно расшифровывает послание это территория другой
особи пчелы летят в нужном направлении увидев определенный танец пчелоразвечек
показывающий в коей стороне поляновскими видами цветов языки животных птиц на
самом деле очень богатые многообразны у одного вида проживающего на удаленных
друг от друга территориях есть свои диалекты например воробьи из Москвы чирикуют
по-своему не так как в Екатеринбурге но только для человека возможно различные
интерпретации одного и того же сообщения без изменения самого сообщения или
вообще его непонимания во-вторых только человек способен молчать в смысле не
сказать замолчать промолчать умолкнуть птичка не может не заливаться трелями в
мае человек способный встретиться с целым миром охватить мысленным взором все
почувствовать чудо мудрого мироустройства поразительную сложность и
продуманность всего вокруг умолкать перед этой тайной способной выражать мы
сталкиваемся с тем что все не выразить не потому что языка не хватит а потому
что мы конечно во времени и в пространстве то есть смертны и не можем управлять
другими мы ограничены и в какой-то момент в глубоком детстве мы это впервые
осознаем проходя своеобразное второе рождение становясь тогда по-настоящему
людьми вспоминайте из предыдущего сегодняшнего вопроса человеческая система
восприятия специфичной тем что только она построена на идее границы или чистого
различия или еще точнее на идее иного как такового воображение мы оперируем
формами абстрактными границами которых самих по себе в мире нет и не просто к
ним обращаемся но рефлексируем оттачиваем развиваем эту свою теоретическую
способность и опираемся именно на нее во всем жизни для ориентирования в мире
для поддержания своей жизни для построения отношений с другими но мы такими не
рождаемся мы рождаемся подобными животными но только в среде человеческого
общения и пройдя лакановскую стадию зеркала мы затем накапливаем опыт
несовпадения слов и вещей опыт смены аспекта результирующийся к 2-3 годам во
внезапном появлении самосознания и в этом смысле можно предложить еще одно
определение языка выражаемое воображаемое вспоминайте все что мы выше тоже
говорили о смысле и далее ввиду открытия отдельности себя и мира возможности
посмотреть на себя самого со стороны мы активно осознанно начинаем использовать
воображаемое как третье измерение нашего существования наряду с тем что животные
практически полностью погружены в символическое и реальное в их достаточно
однозначной корреляции и если воображаемым и пользоваться то достаточно
автоматически не замечая это за собой например когда играют но также природы у
нас нет специальных органов речи мы пользуемся как умеем ртом языком как органом
дыхания учимся создавать нужные звуковые эффекты мы не рождаемся уже говорящими
на человеческом языке мы способны лишь к так называемой лепетной речи слышали
выражение детский лепет представляете себе по сути своей это слоговая проторечь
лежащая в основе всех человеческих языков мира кстати согласно исследованиям мы
все с полугода примерно начиная активно лепетать сначала делаем это все по всему
миру дети абсолютно одинаково потом к году наша лепетная речь начинает
приобретать фонетическую специфику той культуры в которой воспитывается человек
а к полутора года мы уже начинаем пользоваться словами соответствующего
национального языка в среде которого с нами говорят при этом все дети по всему
миру изначально лепечат на одном протоязыке правда расшифровать его не удается
он слишком иррационален у нас у всех есть способность ко всем языкам и они как
бы свернуты в детском лепете это доказывает случай например шизофреника одна из
личностей которого в совершенстве владела сербским языком хотя человек реально в
своей жизни этот язык не изучал или например русской девушке которая лепетала
как потом оказалась на японском родители записали на пленку ее детские монологи
а во взрослом возрасте она выучилась на востоковеда имея природную тягу к этой
культуре и найдя в какой-то момент аудиозаписи поразилось что это был чисто
японский еще интереснее то что ребенок использует лепетную речь не для общения
со взрослыми по крайней мере не в первую очередь как именно средства
коммуникации и не пытается поначалу имитировать речь взрослых переход к
естественному национальному языку дело постоянной языковой практики когда мы 10
миллиардов раз с детьми повторяем нужные слоговые комбинации и поощряем
правильное произношение ребенок лепетчатый один в пустой комнате монологами с
определенными тональностями по-разному радостно или жалобно серьезно или игриво
медленно или быстро и так далее и это для него данная от природы тяга выражать
собственные состояния переживания ощущения озвучивать то с чем он имеет дело так
сказывается призвание человека в мире дать миру слово причем здесь молчание
примерно к 3 годам в детстве мы в той или иной ситуации вдруг понимаем
интуитивно конечно не концептуализируя это что мы во всех смыслах ограничены для
нас впервые открывается хронологическое время с прошлым и будущим и настоящим
как чистым различием скользающей границей теперь между уже отсутствующим прошлым
и еще не наступившим будущим мы конечно в пространстве максимум чем мы можем
научиться управлять и по-настоящему распоряжаться это на какую-то малую долю
только собственное тело и своя жизнь и мир устроен так не нами до нас так что
поменять мы тут ничего не можем мы впервые по-настоящему умолкаем потому что не
знаем что за этими границами ведь у нас есть только эмпирические свои
собственные переживания и состояния и только опыт жизни нет опыта о том что
после но вернемся к природе человеческого языка помимо знакомого окружающего
мира оказывается есть и совершенно незнакомое причем в этом же окружающем мире и
в нас же самих открывается что-то бездонное чуждое непонятное неподконтрольное
какие-то законы которыми если не наша воля то все управляется да и я оказывается
сам на себя могу смотреть со стороны почему зачем что с этим делать и это не
выразить мы умолкаем перед этой тайной но при этом конечно непродуктивно убегать
от нее от непостижимого открывшегося в себе же самом уже некуда деться остается
только мужественно иметь с этим всем дело всматриваться осмыслять взвешивая
каждое слово свое на весах молчания и чутко соблюдать баланс между потоком
говорения и глухим молчанием потому что и молчание тоже может быть неверно
истолковано кстати интересно мы обычно думаем что человеческая речь создается
говорением способностью говорить но нет скорее готовностью слушать молчаливым
вниманием другого готовой но только человек способен и я сейчас вам что-то
говорю не потому что умеем говорить а потому что вы умеете слушать и готовы меня
слушать спасибо вам за это Как, откуда произошёл человеческий язык, мы не знаем.
Но это удивительнейшее изобретение. Сейчас, чтобы вы немного передохнули после
такого тяжёлого блока для осмысления, я вам расскажу пару баек про язык о
молчании. Пожалуйста, вот это не надо на экзамене рассказывать, это будет в
рубрике «Шутки, юмора». Но вы также должны понимать, зачем это делается. В
каждой шутке есть доля шутки, а значит, доля правды. Существует такая байка, что
язык придумали женщины. Кстати, косвенно это подтверждает тем, что девочки
раньше начинают говорить, чем мальчики. И антропологи говорят, что первым
человеком была в особи женского пола. Но это никакой не феминизм с моей стороны,
я его предыдущая, честно говоря. Сейчас вы поймёте, почему. Другая байка.
Женщины не способны молчать. В Древней Греции, 6 век до нашей эры, пифагорейская
община. Особенностью школы пифагора было то, что ученики жили все вместе, и
наравне с мужчинами учились также женщины. Чудрико ходить, жена пифагора, Чиана,
первая женщина-философ и математик. Другая особенность. Пифагорейцы обязывались
не разглашать учения основателя, только можно было пользоваться для своих целей,
для своей жизни, но всем подряд нельзя было рассказывать. Чтобы стать
преподавателем, после определенного количества лет обучения, человек, который
тоже хочет стать учителем в этой школе, давал обет молчания. То ли на 3 года, то
ли на 5 лет. Так вот, у пифагорейцев не было женщин-учителей, хотя многие из них
были выдающимися исследовательницами. Потому что ни одна женщина не выдерживала
этот обет молчания. Шутки шутками. А я по себе знаю, как трудно сдерживать эту
естественную тягу к выражению, как трудно бывает промолчать, когда даже понимая
разумом, что лучше не говорить. Постоянно приходится бороться с собой, и не
всегда это успешно получается. Итак, как и откуда взялся? Кем и когда был
разработан человеческий язык, мы не знаем. Скорее всего, достоверно никогда не
узнаем. Возможно, первоначально эти логические звуковые сочетания мыслились как
похожие на те или иные феномены действительности, на которые не указывали.
Например, стол, отствовать, стоять, ветер, веять, завывать и так далее. Но язык
постоянно изменяется, он буквально живет и развивается, трансформируется,
появляются новые слова, устаревшие, уходят в историю. Менять же значение и
акценты, правила написания, произношения и составления предложений, а язык, как
собственный язык, остается. Единственное, что здесь можно сказать, язык
иррационален. Он скорее похож на сон или на бессознательное, чем настроен на
логичную, упрядоченную систему. Но все попытки науки и лингвистики
сконструировать непротиворечивое знание языке разбиваются, а его
парадоксальность. Нет ни одного правила, ни в одном языке мира, у которого бы не
было исключения. И если даже в одной своей области, например, в плане
синтаксиса, язык строк, например, английский с его жестким порядком слов-
продолжений, то значит, в какой-то другой своей области у этого языка будет
чудовищный хаос. В английском это произношение. Там исключений больше, чем
правил, и проще запомнить, как произносится каждое новое слово, чем знать
типичные правила произнесения буквосочетания и расстановки ударений. Русский
язык учителя иностранцев тоже чрезвычайно трудно. Большой вопрос. Насколько мы
знаем тот или иной язык? Мы им научаемся пользоваться? Успешно или не очень, но
буквально так же, как мы в детстве постепенно научились пользоваться своим
телом. Насколько мы знаем тело, и можем ли всё рационально о нём знать? Такой же
вопрос. Но тут интересно обратить внимание на то, что в обыденном смысле знание
о теле или языке – это отдельная пришлёпка. Насколько необходимы. Чтобы поднять
руку, мне не обязательно иметь представление о том, какие стоят за этим явлением
химические и биологические процессы. Чтобы обозначить нечто в мире с помощью
языка, мне не обязательно знать лингвистические правила. Например, ребёнок до
школы не знает о них. При этом прекрасно говорит, и у него к языку гораздо более
сильное чутьё, чем у образованных взрослых. Примерьте к себе. Когда вы хорошо
овладели иностранным языком, вы строите в нём фразы не потому, что знаете,
каждый раз вспоминаете правила, а потому, что научились его средства
использовать для передачи смысла. Вы уже не задумываетесь, в каком здесь времени
глагол употребить, поставить ли артикль, какое из множества синонимов, оттенков
взять. Даже если с ошибками вы уже говорите на этом языке, видя его средствами
смысл. И вот у меня часто бывает, что я не помню, на каком языке я прочитала
какую-то идею, потому что, хорошо зная английский и русский, я запоминаю смысл
прочитанного, а не слова, в которых он был выражен. Но в этом есть и негативный
момент. Найти потом нужную ночную статью трудно, потому что читаю многое, но
дома на другом языках. Почему же я сказала, что язык похож скорее на сон? В нём
есть настолько уверенный замах на классифицирование всего и вся, но который
постоянно срывается. Как и во сне сплошные непредвиденные трансформации, которые
не по законам логики, вообще не по законам происходят. Вы же прекрасно знаете об
этой проблеме классифицирования. Хотя бы по себе, когда сохраняете что-то на
компьютере, пытаетесь распределить по папкам. Статья, которую вы хотите
сохранить, это в рубрику по авторам или в папку методы, или в раздел литература
по темам. Ну, а уточняя и ветвя каталоги, мы им попадаем в дурную бесконечность,
в которую для каждого отдельного файла будет своя папка, и каждый файл будет
дублироваться в нескольких категориях. В общем, оказывается, посмотреть на
естественный язык как на систему рациональным образом до конца никогда не
получится. И в этом его богатство, полнота, открытость новому и настоящесть. Он
отражает полностью природу человека, одновременно рациональную и иррациональную.

\subsection{Структурализм: язык управляет нами?}
Наконец, философия структурализма. Также концентрируется на вопросах языка, хотя
в достаточно своеобразном ключе. Это направление развивает понимание
коммуникации и функционирования языка на основе лингвистической концепции,
предложенной швейцарским исследователем Фердинандом де Сассюром. Идея
симеологии, как науки, изучающей по выражению Сассюра жизнь знаков в жизни
общества, произвела революцию в гуманитарных науках и философии, поскольку
предложила рассмотрение влияния языковых знаков на психологию человека, его
поведение, ценностные суждения. Этот подход предоставил возможность изучать,
как, с помощью каких воспроизводимых нами структур языковые выражения, речь
субъектов, конструируют социальную и психологическую реальность. Глот Левистрос,
занимавшийся на основе данного подхода исследованием языков приобытных
сообществ, в том числе и тех племен, которые сейчас на земле в таком состоянии
существуют, он в Латинскую Америку ездил, показывает, что первичные структуры,
воспроизводимые в разных условиях, в разных культурах, обществах, они
наполняются разным содержанием, но по своему воздействию на индивидов схожи. Так
родилась основная идея структурализма философии, понимая все в бытии как своего
рода язык, или знаки, или текст. Выстретители этого направления, кстати, с Лизой
Хайдегером, открывают, что языковая реальность, вернее, заложенные в ней
структуры языка, во многом управляют человеческим мышлением. В этой связи есть
еще одно распространенное определение языка, знаковая система. Вообще говоря, во
всем можно увидеть знаки и все рассмотреть как язык. Ведь для нас все что-то
значит. Для нас что-то означают те или иные вещи, жесты, поступки, факты,
проявления, события. И большой вопрос, есть ли тогда вообще что-то, кроме языка,
что-либо непосредственно доступное человеку, кроме знаков? Ведь мы все видим
через призму языка. В широком смысле даже у природы есть язык, в котором,
например, роль слов играет химические соединения, складывающиеся в предложениях,
химические реакции. Символическое измерение человеческого существования
заставляет нас мыслить определенным образом, поскольку мы погружены в среду
структуры языка, идеологии, пропаганды, рекламы, социальных стереотипов. И вот,
например, если какие-нибудь астрологи, спиритологи и прочие мистификаторы хотят
современного человека побудить к ним обращаться, они начинают имитировать
научный дискурс, ссылаются на якобы большую эмпирическую базу, типа наши
практики многим помогли, вставляют псевдонаучные термины, уверенно в форме
теории говорят, например, о влиянии чисел на жизнь человека и что он состоит из
таких-то чакр, там такие-то энергии текут по каналам, их можно прочистить
определенными упражнениями, и это все полезно для здоровья. И человек
покупается, потому что воспринимает псевдонаучные фразы, структурированные в
форме научной теории, с терминологией, законами и как бы эмпирическими
подтверждениями. Но хотя вы понимаете, что это зачастую не количественные данные
и вообще эффект барного. Точно так же своя четкая структура есть, например, о
волшебной сказке, почитайте какого-нибудь Владимира Проппа. И вот если в
рекламном ролике воспроизводится структура сказки, вы проассоциируете продукт с
чем-то фантастически полезным и крутым, и что в конце все будут жить долго и
счастливо. Скажем, какое-нибудь лекарство, и герой вступает в неравный бой с
вирусами, ему помогают клетки организма, и пройдя трудности, они справляются, и
все хорошо. Ну вот мы начинаем верить подсознательно в такое лекарство, как в
сильного героя, который обязательно победит болезнь. Жан Бодрияр, французский
философ-постструктуралист, осмысляет эти моменты с другой стороны и приводит
примеры власти символического над человеческим сознанием в современном обществе
потребления. Зачастую мы покупаем вещи не для непосредственного использования, а
для того, чтобы обозначить ими что-то для другого. Книги стоят на полках не для
того, чтобы их читать, а чтобы гость, взглянув на обложки, составил для себя
определенный образ человека, имеющего такие книги. Мышлина покупается не просто
любая, чтобы ездить, но выбирается в соответствии с тем статусом, который
человек хочет для себя подчеркнуть в глазах других. То же самое касается марки
одежды, которую мы носим, названия ресторанов, в которые ходим, бренды
электроники, которые мы пользуемся и так далее. Причем очень часто все это
выбирается нами бессознательно. Если мы даже не задумываемся, почему захотели
купить кто-то или решили создать себе именно такой-то внешний вид, образ. Но это
может означать лишь то, что нами без нашего ведома завладели образы рекламы или
социальные стереотипы, которым мы желаем соответствовать. Те самые фантазмы и
симулякры, о деструктивности которых предупреждают лакан, жижек и другие
современные мыслители. Так вещи, участвуя в создании симулякров, тоже сегодня
становятся своеобразным языком, говорящим о человеке. Ну вы понимаете, айфон в
кредит или сидеть на хлебе и воде, но ходить в шмотках от Гуччи. Ничего не имею
против этих замечательных брендов. Просто речь о том, что вещи, предметы,
устройства в современном обществе выполняют не только свои непосредственные
функции, но становятся символическим, невербальным языком, который тоже что-то
означает, несет какую-то информацию о владельце вещей и настраивает на общение с
ним в определенной манере. То есть позволяет распознать претензии человека на
определенный статус, его какие-то мировоззренческие установки, социальный класс,
уровень достатка, хобби и предпочтения. Но главный вопрос в том, осознанно мы
это делаем или бессознательно под влиянием стереотипов, рекламы, идеологии. Если
вместо активности нашей собственной мысли в нас начинают бессознательно
действовать такие вот конструкции, нами управляют языковые структуры, которыми
нас программируют политики, соцсети, СМИ, рекламные ролики и плакаты, то это
безусловно печально. Получается, что нашу реальность за нас конструируют, нам ее
навязывают, побуждая поступать определенным образом, как выгодно кому-то. В
связи с этим постструктуралисты старались максимально разобрать и разрушить наши
иллюзии, стереотипы, показать, откуда они берутся, чтобы мы сами выходили к
самостоятельному осмыслению. Так что на самом деле все не так печально в плане
тотальной власти языка над нами. Свобода открывается в пространстве между
словами и вещами в движении осмысления. В этом моменте сходится подавляющее
большинство рассмотренных нами делателей языка. Человека отличает именно
способность поднимать смысл, вернее, творить его. Вы думаете, переводом мы
занимаемся только с какого-то иностранного языка, на наш родной или наоборот с
родного иностранного? Нет, ребята, мы постоянно в общении друг с другом, в
слушании и чтении того, что высказано другими, осуществляем перевод внутри
своего родного языка. Поскольку смысл не совпадает со словами, мы постоянно
должны его воссоздавать в рамках использования слов и восприятия их от других.
Должны каждый раз свои смыслы, которые нам кажутся само собой разумеющимися,
переводить на язык понятный собеседнику и наоборот расшифровывать, зашифрованное
другими. В пределе с каждым человеком общаемся немного по-разному, потому что
переводим свои смыслы каждый раз на языки разных других. Что же касается свободы
или несвободы от языка, здесь повторю, все зависит от того, проделываете ли вы
эту работу по осмыслению, задаетесь ли вопросами по поводу расхожих всеми
принятых вещей, совершаете ли усилия понимания, прежде всего, себя самих, почему
я так поступаю, почему хочу того-то и того-то, нужно ли мне это на самом деле
или мне это навязали со стороны и так далее. Рассматривайте всегда максимально
широко все позиции, все точки зрения, даже если они вам не близки, тогда вы
лучше начнете понимать, что именно и почему вам ближе. Рождайте собственный
смысл, не проглатывайте бездумно то, что приходит к вам из вне.