\section[Предпосылки кризиса классической науки. Философские основания неклассической рациональности]{Социальные, исторические, философские и мировоззренческие предпосылки кризиса классической науки. Философские основания неклассической рациональности}

\subsection{Социально-экономические и политические условия}

% Значит, 19 век, 19 век и вообще период далее 20, начало 20 века для
% многих исследователей, исследователей культуры, я про это говорю, это период
% социоцентризма. Вот помните, у нас был античность, антропоцентризм,
% средневековье, теоцентризм. Ой, простите, что я вам говорю? И не поправляете,
% сидите там меня. Опять никто не слушает? Я сейчас вот всем вот тут вот не знаю,
% что сделаю. Нет, я вам на экзамене сделаю. Слушаем. Так вот, давайте.
% Античность, космоцентризм, да, правильно. Средневековье, теоцентризм.
% Возрождение, антропоцентризм. Дальше новое время, это как натуроцентризм. И вот
% 19 и начало 20 века это социоцентризм, потому что социальные процессы, они
% настолько активно пролили, что невозможно это игнорировать. Все подчинено именно
% трансформации изменениям в социальной жизни. И, соответственно, философия такая,
% и новые науки социальные появляются и так далее. 18-19 век. Давайте вспомним, а
% где же нам искать тот самый социальный слой, там то вот народное единство или
% дискурс или социум, как это ни назовите, вот этих людей ресурсных, вовлеченных,
% но, важно, ранее не учитываемых. Кто будет двигать-то процесс? Я про это уже
% говорила, и мы как бы по сторонам смотрим. Сегодня это, ну, одна группа там
% позавчера, это другая. А вот что было в 19 веке? Это, конечно, США. Вот
% появилась эта страна. Страна действительно энтузиастов, как вы понимаете. Это
% изгои, это авантюристы, решившие построить новый Рим. Да, это действительно так,
% и поэтому не Колизей, а Капитолий, это не просто так. Это холм, да, и так далее.
% Много там не просто так. Ну, то есть мы построим не просто новую страну, где мы
% тихонечко будем жить. Не-не-не, это сразу же вызов. Мы построим новый Рим. И
% причем этот Рим изначально мыслился, не пытались они строить, не какой-то Рим,
% знаете, позднего периода, когда это вот, или вот, как сказать, Рим позднего США,
% я бы так это назвала, когда вот эти крылья Кондора пытаются в своих лапах
% удержать весь мир, мир сопротивляется, он его клюет и так далее. То есть это не
% поздний Рим такой, времен разврата и попыток удержать расползающуюся империю.
% Это Рим ранний. То есть в смысле работоспособности, риска тесной сплоченности и
% презрения к трудностям. Я немножечко сейчас в Рим погружусь, только буквально на
% пару минут, для того, чтобы вы поняли, что такое США раннего периода.
% Удовольствие для римлян, это даже особое слово, удовольствие, это синоним
% соблазна, чувствуете? Ну, слово имеется в латинском языке. Добродетель для
% римляна, это мужество воина, именно виртус, я про него как-то говорила,
% вспомните. Виртус как способность менять реальность. Это мощнейший вообще ресурс
% человека, виртус, да? А позор – это то, что бесславно, безымянно. Даже любовь в
% римском языке – это рассудочность, это не какая-то там страсть или там восторг,
% или что-то такое это слово «выбирать». Вот о чем ранние США. 


19 век и начало 20 века для многих исследователей культуры --- эпоха социоцентризма. Если раньше были античность с космоцентризмом, средневековье с теоцентризмом, Возрождение с антропоцентризмом, новое время с натуроцентризмом, то 19 и начало 20 века --- именно социоцентризм. Социальные процессы стали настолько важны и активны, что их уже нельзя игнорировать. Всё подчинено изменениям в социальной жизни, появляются новые социальные науки и философские направления.

В 18-19 веках возник вопрос: где искать тот социальный слой, народное единство, социум --- вовлечённых, но ранее неучтённых людей, кто будет двигать процесс? Сегодня одни группы, вчера --- другие, а в 19 веке ключевой пример --- США. Эта страна возникла как страна энтузиастов, авантюристов, изгоев, решивших построить новый Рим. Не просто жить спокойно, а бросить вызов. Новый Рим --- это не поздний Рим с его развратом и распадом, а ранний Рим --- работоспособный, сплочённый, готовый к риску и презирающий трудности.
Для римлян удовольствие --- синоним соблазна, а добродетель --- мужество воина, виртус, способность менять реальность, главный человеческий ресурс. Позор --- бесславие, безымянность. Любовь --- рассудочность, выбор, а не страсть или восторг.

\subsubsection{Война за независимость}

% И первое, что мы
% должны тут вспомнить, это война за независимость 70-е годы, да, 70-е, 80-е годы
% 18 века, то есть 1775 и 1783. Это освободительное движение в Северной Америке,
% но тогда еще трудно говорить про государство Соединенных Штатов, но мы сейчас не
% будем это особенно акцентировать. Когда колонии, британские колонии, начали бунт
% против митрополии, против Британии, в ответ это произошло на принятие актов, то
% есть Англия приняла ряд актов, которые подрывали экономическое развитие колоний.
% Ну вот возникает война за независимость. Она сложная, мы не будем там, ну как
% бы, анализировать. В любом случае, в 1776 году была подписана Декларация
% независимости, которая была переработана Томасом Джефферсоном. Мы потом о нем
% еще поговорим. И вот в Декларации было сказано, мы считаем самоочевидным
% следующие истины, что все люди созданы равными и наделены Творцом определенными
% неотъемлемыми правами. Вот, по сути дела, это сказанное Джефферсоном, оно будет
% руководящим для политики, в том числе и в Европе. То есть все-таки я вам пытаюсь
% сказать, что началось США. Это там вот это вот все, как бы, то, что потом на
% Европу перекинулось в 18-19 веках, в конце 18-го и в 19-м. Почему это было
% сказано Джефферсоном? Ну потому что он как бы прямо говорит, да мы тоже люди,
% мы, да, мы колония, мы как бы тут вот, ну, мы тоже люди, и вы будете с нами
% считаться. Вот этот принцип равноправия, тем не менее, был весьма таким
% специфичным, он распространялся только на белых мужчин-собственников. Имеется в
% виду не на индейцев, с которыми встретились европейцы в Северной Америке, то
% есть коренные жители Америки, и уж точно не рабы. Они не включались в
% политическую общность. Тот же Джефферсон был вполне себе работорговцем, да. Но
% почему это стало возможным? Да потому что он следовал на самом деле принципам
% Римской империи, да, принципам, которые озвучивал Цицерон, это римский автор,
% надеюсь, вы знаете, и принципам Джона Локка. Джон Локка – англичанин времён,
% там, 16-17 век, вот это вот, времён этого периода, то есть более раннего. Мы про
% него говорили в прошлый раз, но немного, конечно. Так вот, оба, и Цицерон и Джон
% Локк были вообще-то рабовладельцами. Да, да, и Джон Локк тоже, только в особом
% смысле. Цицерон прямо, естественно, был рабовладельцем, и он считал, что
% рабство, я сейчас цитирую его фразу, рабство обусловлено самой природой, которая
% дарует лучшим людям владычество над слабыми, для их же пользы, имеется в виду,
% для пользы слабых. Ну вот так они считали в древнем риме, понимаете, да. Локк –
% это отец европейского либерализма, отец практически европейской
% действительности, создатель концепции прав человека, но он был крупным
% инвестором в английскую работорговлю. Была такая компания, королевская
% африканская компания, которая на западном побережье Африки захватывала, или
% покупалась у местных жителей, покупались рабы, пленники, да. Они вывозились в
% Южную Америку в колонии, которая в районе вот Карибского бассейна создавала. Эти
% колонии существовались. Колонии именно британские в Южной Америке. Что значит в
% этих колониях было? Ну там на вырученные деньги, вот именно на вырученные с
% рабства деньги, ну вот представьте себе, закупили рабов на западном побережье
% Африки, привезли, в эти колонии продали, да. И вот на эти деньги покупалось
% кофе, табак, какао, сахар, и везли с Европы. То есть там прибыль была
% сумасшедшая. Вначале первая прибыль, потом вторая прибыль. Так что все, что мы
% сегодня любим, да, какао и, что называется, сахар, и табак, и кофе, это, по сути
% дела, построено на самые кровавые деньги. Вот на самые кровавые. Почему? Да
% потому что рабство на плантациях, вот этих карибских плантациях, плантация с
% сахарного тростника, оно было в разы более жутким, жестоким, нежели положение
% рабов в Риме, или даже потом в Северной Америке. Это английское рабство, про
% которое удивительным образом мы почти ничего не знаем. А знаете почему? Потому
% что историография и написана, по сути дела, англичанами. Поэтому давайте
% немножечко переписывать. Английское рабство такое уникальное, в этом смысле
% такое прокси-рабство, да, не на своей территории, как вы понимаете. И это было
% самое жуткое рабство в истории человечества. Но тем не менее, в этот же период
% рождается базовая концепция прав человека. Опять же, озвучиваю, как это
% позиционировалось в те времена. Не сегодня, да, а в те времена. Что имелось в
% виду? Нарушение человеческих фундаментальных прав и достоинства происходит
% тогда, когда с кем-то обращаются таким образом, что по мнению их, имеется в виду
% тех, с кем обращаются, для них было бы лучше не родиться, чем жить такую жизнь.
% Понимаете? Вот так это было. То есть из чего выискивать права человека? Давайте
% посмотрим на ситуацию, в которой человек говорит, мне было бы лучше не родиться,
% чем жить так. Вот из этого родилась концепция прав человека. И по всей видимости
% считалось, что молод рабы, в жизни рабов не было ничего такого, из-за чего
% стоило бы не рождаться. И вот, кстати, я просто вам хочу подчеркнуть, что вот
% это акцентирование рабства именно на северноамериканском рабстве, я его никоим
% образом, что называется, не одобряю, но тем не менее, это ведь тоже британская
% позиция. Они как-то смазали вообще историю собственной королевской африканской
% компании, чтобы вот там рабство, вот там ужас. Третья, хотя, ну, понимаете, что
% было. 

Война за независимость в Северной Америке в 70–80-х годах XVIII века, с 1775 по 1783. Тогда британские колонии начали восставать против метрополии из-за актов, ограничивавших их экономику. В 1776 году была подписана Декларация независимости, подготовленная Томасом Джефферсоном, в которой говорилось, что все люди созданы равными и имеют неотъемлемые права. Этот принцип равенства стал руководящим для политики, в том числе в Европе.

Однако равенство распространялось только на белых мужчин-собственников, исключая коренных жителей Америки и рабов. Сам Джефферсон был работорговцем. Такая позиция базировалась на принципах Римской империи и идей Цицерона и Джона Локка --- последних, кстати, тоже связанного с работорговлей. Цицерон считал рабство естественным, дарованным природой владычеством сильных над слабыми, якобы на благо последних.

Джон Локк, отец европейского либерализма, был крупным инвестором в королевскую африканскую компанию, занимавшуюся работорговлей на Западном побережье Африки. Захваченных или купленных рабов везли в британские колонии Карибского бассейна, где на вырученные деньги закупали кофе, табак, какао и сахар. Все эти товары, которые мы сейчас любим, по сути построены на кровавых деньгах рабства.

Рабство на карибских плантациях было особенно жестоким --- куда хуже, чем в Древнем Риме или даже в Северной Америке. Английское рабство было уникально тем, что происходило не на своей территории, и об этом почти не рассказывают, так как историография писалась англичанами. В то же время в этот период формируется базовая концепция прав человека, которая подразумевает, что нарушение прав происходит, если жизнь так ужасна, что человеку лучше не было бы родиться.

Именно из таких представлений выстроилась концепция прав человека. При этом считалось, что жизнь рабов не была настолько ужасной, чтобы их существование считалось бы недопустимым. Акцент на североамериканском рабстве --- тоже британская позиция, которая смазывает историю королевской африканской компании и ужасы рабства в других регионах.

\subsubsection{Отмена рабства}

% Отмена рабства тоже это же период. Но тут тоже, мы не верим британской
% историографии, лезем, как говорится, глубже в источники и понимаем, что вообще-то
% отмена рабства произошла не в Британии, и вовсе даже не по тем причинам, что в
% Британии. Вначале, ну, рабство отменяли всю человеческую историю, и первые такие
% отмены рабства, разумеется, были в эпоху Судневековья. Первые, кстати, кто
% отменил рабство в эпоху Судневековья – это ирландцы, потому что, ну, вот
% христиан, они были одним из первых христиан в Европе, они отменили сразу же
% рабство. Но мы говорим про вот это рабство, которое в новое время, да, завелось
% активно, то есть второе пришествие рабства, такое более страшное. Так вот, его
% вначале как бы отменили в России, в России, которая в тот период, в конце XVIII
% и XIX веке, активно присоединяла Причерноморье. 1783 год – это присоединение
% Крыма к Римской империи, а потом в ходе русско-персидской войны, это 1804 год,
% Россия запретила вот эти вот невольничьи рынки, которые были очень
% распространены в Причерноморье, в Крыму, в Грузии, ну, вот в нынешней Грузии,
% да, конечно, это была не Грузия, ну, понимаете, да, о чем речь идет. Там
% османское еще рабство было. И вот, когда эти территории вошли в состав
% Российской империи, сразу же, Александр I, запрет невольничьих рынков просто
% жесточайший. Очень долго бунтовали вот эти вот земли, типа, ну, мы так всю жизнь
% жили, будем жить, нет, не будете. Вот, дальше. Ну, и уже в 30-х годах 19 века
% Британия запретила рабство, но в силу экономических причин. Почему? Ну, потому
% что все-таки у них в Британии уже появляются колонии другие, в этих колониях
% выращиваются хлопок, и он оказывается дороже, чем хлопок североамериканский,
% хлопок, который выращивался рабами. Стало быть, рабы это плохо. Или, например,
% сахар из сахарной свеклы, да, стал конкурировать с тростниковым сахаром. И,
% соответственно, тростниковый сахар, который выращивается рабами, тоже плохо.
% Рабство надо запретить. То есть, опять же, смотрите, запрет рабства вроде бы и
% там, и там, все хорошо, но мотивы разные, да, мотивы разные. В любом случае, вот
% эта вот определенная борьба Британии и США всегда существует, до сегодняшнего
% дня очень активно существует. На тот период, что это происходило, в результате
% этой борьбы, я бы так сказала, США отомстили. Они из Америки с любовью отправили
% в Европу революцию, идею революцию. Маркис Лафаэт такой вот был, очень
% интересный человек. Я не писала, я его тут не писала. Ну, в принципе, это не
% важно, просто интересный человек. Ну, запишите, потом поинтересуйтесь его
% жизнью. Знаете, я не знаю, почему фильм до сих пор прям по нему не сняли. Ну,
% ладно. Так вот, богатый французский авантюрист, его, на самом деле, его имя это
% Жильбер де Матье, авантюрист, постоянно курсирующий между Францией и США. И вот
% он активно способствовал тому, чтобы революционные настроения из Америки
% развернулись во Франции. Возникает революция. И возникает вот это вот знаменитый
% европейский документ о правах человека, который является уже образцован для всех
% остальных революций. Но, тем не менее, было кое-что изменено. Если у
% Джефферсона, в общем-то, там право на счастье было зафиксировано, то здесь стало
% право на собственность. Ну, еще было отменено право народного восстания. Вот
% таким образом. Политический процесс очень бурный. И сразу скажу, что основой
% политических процессов вот этого периода является, конечно же, не философская
% проработка темы прав человека. Ее вообще-то, ой, если и начали, то много-много
% позже. То есть возник вопрос, хорошо, а что такое человек, что такое его
% естественное право, это намного позже. Вначале все-таки политику двигала
% риторика. Никакая не философская работа по формированию концептов, понятий и
% тому подобное. А вот теперь давайте подумаем, что такое риторика как основа
% политики. Это важная тема, дорогие мои. Прошу обратить на это внимание. Я не то,
% чтобы даже для экзамена, а вот просто для того, чтобы быть взрослым человеком.
% То есть еще раз, научная база политики нового времени – это не философия, это
% даже не наука, это риторика. А что такое риторика? Риторика, помните, в эпоху
% Возрождения – это пневматическая магия. То есть это манипулирование магом
% фантазмами человека. Естественно, в новое время никто уже не говорит о магии. Но
% тем не менее, по факту, это действительно манипулирование фантазмами человека.
% Почему? Потому что вырабатывается очень конкретный инструментарий. Работы по
% риторике активно писались, мы их читаем и понимаем. То есть опираться, говорят,
% нужно в своей речи на топосы. Топос – это аристотелевское понятие, понятие из
% его концепции риторики. Топос – это общее место. Ну вот, например, я вам говорю,
% красть плохо. Это общее место для нас всех, потому что мы все разделяем. Ну,
% действительно, красть плохо. Или, например, вот такой топос может быть.
% Например, топос в определенном дискурсе. Все бабы дуры. Понимаете, если вы
% берете определенный дискурс, где это общее место, с вами дальше будут согласны
% по многим положениям. Или наоборот, там все мужики-козлы. Я не одобряю эти
% топосы, ни в коем случае не предлагаю вам разделять это общее место. Я просто
% объясняю, как это работает. Вы приходите в компанию обиженных женщин, говорите,
% все мужики-козлы, и все, и вас готовы слушать. Понимаете, вы топос, вы двинулись
% с топоса. Итак, так вот, политика этого периода, территорика в первую очередь и
% двигается с топоса. Например, вот вы сегодня можете услышать в политике, весь
% цивилизованный мир – это вот этот топос. Никто не вдумывается, а почему мужики-
% козлы, а почему бабы дуры, а почему, что за весь цивилизованный мир, да? Просто
% топос, он как бы все, он пробивает дорогу к вашему сознанию, если вы с ним, ну,
% если это тоже разделяете как общее место. И, соответственно, там уже фантазмы,
% да, вы можете рулить фантазмами как угодно. Общее место не анализируется, в него
% не погружаются, из него выводятся рассуждения в каком-то направлении. Вот
% серьезно, это, по сути дела, то же самое заклинание фантазмов. Но, смотрите,
% возрождение было более честным. Оно не называло это каким-то здравым смыслом. Не
% называло это, ну, истинно. Оно говорило, это магия. Мы работаем, мы
% манипулируем, мы честно говорим, что это просто манипулирование человеческим
% сознанием. Вот, сегодня это называется здравый смысл. И, кстати, если вы будете
% внимательны, вот зайдите на сайты, на которых прям выложены методички
% Госдепартамента. Я их анализировала именно с точки зрения, как это работает,
% какие инструменты. Мне, что называется, неинтересно там, что они говорят, но,
% тем не менее, вот как это, то есть вам предлагается вначале найти фразу «мост».
% Так вот, фраза «мост» --- это и есть топос. Например, главная ценность
% человеческой жизни. Кто будет с этим спорить? Никто не будет с этим спорить. Вот
% с этим вы пробиваете дорогу, да, соответственно, сознание. А там уже
% манипулируете чем угодно. Поэтому всегда обращайте внимание не то, что вам
% говорят. Вы люди, интеллектуальные элиты этой страны. Поэтому вы должны обращать
% внимание не на то, что вам говорится, а в первую очередь на то, как вам
% говорится, какие инструменты использованы. Уловили топос. Согласитесь или не
% согласитесь с топосом? Понимаете вы его, не понимаете? Вот в это вот надо
% разбираться. Ну ладно, всё. Это как бы такое философское отступление.
% Возвращаемся к XIX веку. Вот таким вот топосом выступили, конечно же, права
% человека. Никто не знал, что это такое. Все права были разные, но достаточно
% было встать и сказать, там, права человека требуют и всё. Дальше вы слушаете,
% что требует. И тем не менее, что такое право человека, почему это всё-таки
% разные вещи. Вот, например, у Цицерона на него очень сильно опирались. Цицерон –
% право на жизнь, право на безопасность, право на счастье. Вот такая вот была
% интересная, интересная права, да. Улобка – право на жизнь, право на свободу,
% право на собственность. У Джефферсона – право на жизнь, право на свободу, право
% на стремление к счастью. Чувствуете разные очень вещи, поэтому, ну, всегда
% уточняйте, права человека какие? По лобку, по Цицерону, по Джефферсону.
% Серьёзно, очень разные концепции. Вот. В любом случае, это стало инструментами,
% этот топос стал инструментом политики. И сегодня, ну, с тех пор, вернее, как
% говорит Валер Стайн, один из самых известных таких философов политики, ну, он
% просто база. Не то, чтобы я полностью разделяю его позицию, но он база. Его не
% знать, как говорится, если вы занимаетесь политикой, нельзя. Не политикой, а
% политикой, как философией политики. Так вот, в 19 столетии, говорит он, возникли
% три основных течения политической теологии. Вот он так это называет.
% Консерватизм, либерализм и социализм. Вот таким образом. Просто имейте в виду,
% да? При этом он считает, что идеология, которая в тот период развивается,
% помните, либеральная идеология, мы про это говорили, это способ справляться с
% тем, что навалилось на людей времен французской революции. Это нормализация
% революции. Нормализация вот этих вот катаклизмов. А катаклизмы, что называется,
% баловали. Баловали историю людей 19 века, помимо вот этих социальных потрясений,
% замечательные климатические катаклизмы. Мы не будем поддаваться убеждениям, что
% только мы сегодняшние с климатом, что называется, сражаемся. Нет, конечно, это
% были времена, всегда, во все времена, случались такие бурные периоды, именно
% климатического. И вот, например, год без лета знаменитый. У нас даже был доклад
% по этому поводу очень интересный. Это как раз голодные сороковые, когда
% взорвался вулкан там город. Ну, я вам просто зачитаю слова капитана судна,
% Остынской компании, который был свидетелем этого взрыва. И, ну, как последствия
% его. Пепел падает на нас дождем. Общий вид окружающей действительности был
% воистину ужасен, тревожен. К полудню свет, который был еще виден в восточной
% части горизонта, исчез, и непроницаемая тьма окутывала небеса. Тьма оставалась
% настолько густой до конца дня, что я никогда не видел ничего подобного, даже
% самой темной ночью. Было невозможно увидеть даже свою руку, держа ее близко к
% глазам. Ну, понимаете, да, это, конечно, такое было, ну, серьезное, очень
% серьезное извержение, которое, по сути дела, на какое-то время серьезно изменило
% климатическую ситуацию. Был кризис сельского хозяйства. Например, в период
% Великобритании начался период экономической депрессии. Торговля сократилась,
% резкий род безработицы, неурожай, завышение цен на хлеб, заболевания там
% сельскохозяйственных культур, фитофтороз и так далее. Очень, как бы, тяжелая
% ситуация. Народ в этом смысле сочинял страшные сказки. И вот на этом фоне
% родилась замечательная история о Франкенштейне, о докторе Франкенштейне, его
% монстре. И вы понимаете, при чем здесь, почему я там разместила фотографию Мэри
% Шелли. Люди собирались, грелись как могли, ели что могли, что называется,
% придумывали страшные сказки. 

Отмена рабства тоже относится к этому периоду, но мы не верим британской историографии и копаем глубже в источники. Оказывается, отмена рабства произошла не в Британии и не по тем причинам, которые там обычно называют. Рабство отменяли на протяжении всей истории, и первые отмены были еще в Средневековье --- например, ирландцы, одни из первых христиан в Европе, сразу отменили рабство.

Но речь идет о новом, более жестоком рабстве Нового времени. Его сначала отменили в России, которая в конце XVIII --- начале XIX века активно расширялась в Причерноморье. В 1783 году Крым присоединили к Российской империи, а в 1804-м, во время русско-персидской войны, Россия запретила невольничьи рынки, которые были распространены в Крыму и на территориях нынешней Грузии, где еще действовало османское рабство. Александр I жестко запретил рабовладение, что вызвало протесты на этих землях, где это считалось нормой.

В 1830-х годах Британия тоже запретила рабство, но уже из экономических соображений. В их колониях стал дороже хлопок, выращенный без рабства, чем тот, что производился рабами в США. Конкуренция с сахарной свеклой тоже сыграла роль: тростниковый сахар, который делали рабы, оказался менее выгодным. Так что запрет рабства был мотивирован экономикой, а не гуманизмом. Эта борьба между Британией и США продолжается и по сей день. В ответ США "отправили" в Европу идею революции через французского авантюриста Жильбера де Матье, который активно распространял революционные настроения из Америки во Францию.

Во Франции возник знаменитый европейский документ о правах человека, ставший образцом для многих революций. В нем вместо права на счастье, как у Джефферсона, прописали право на собственность, а также отменили право на народное восстание. Политические процессы того времени были бурными, но основывались не на философском осмыслении прав человека --- это началось намного позже. Тогда главным двигателем политики была риторика, а не философия или наука.

Риторика эпохи Возрождения --- это, по сути, пневматическая магия, манипуляция фантазмами людей. Сегодня никто не называет это магией, но по сути это то же самое: с помощью особых приемов создаются обобщенные "топосы" --- общие места, которые воспринимаются всеми как аксиомы. Например, "красть плохо" --- это общий топос, с которым никто не спорит. Или в другом дискурсе: "все мужчины козлы" --- это тоже топос, с которым согласится определенная аудитория. Риторика использует такие топосы, чтобы пробить путь к сознанию и управлять восприятием.

Возрождение было честнее --- оно открыто говорило о манипуляции сознанием. Сейчас же это маскируется под "здравый смысл". Например, в методичках Госдепа первым шагом является найти фразу-топос, вроде "главная ценность жизни человека". Никто с этим не спорит, а дальше можно манипулировать сознанием. Так что нужно обращать внимание не на то, что говорят, а как говорят и какие инструменты используют. Понимаете топос --- значит, контролируете ситуацию.

Возвращаясь к XIX веку, правами человека тоже стали манипулировать как топосом. Никто толком не знал, что это значит, но достаточно было заявить "права человека требуют", и дальше слушать, что именно. Например, Цицерон выделял право на жизнь, безопасность и счастье, а Джефферсон --- право на жизнь, свободу и стремление к счастью. Разные концепции, поэтому всегда уточняйте, о каких правах идет речь.

Этот топос стал мощным политическим инструментом. Как говорит философ Валер Стайн, в XIX веке возникли три главных политических течения: консерватизм, либерализм и социализм. Либерализм, в частности, был способом смягчить последствия Французской революции --- нормализовать революционные потрясения. Причем XIX век был непростым и климатически --- были серьезные катаклизмы, например, "год без лета" после извержения вулкана. Пепел заслонил солнце, наступила тьма, сельское хозяйство пострадало, цены на хлеб взлетели, началась депрессия и безработица.

\subsubsection{Промышленный переворот}

% Тем не менее, смотрите, конец 18-19 века, в Европе
% происходит промышленный переворот. Надо было выживать. Люди очень активно
% подключали машины, уже руками не на работе, чтобы выжить. Начинается этап
% развития экономики, который называют индустриальным. Выделяют две стороны
% промышленного переворота. Еще раз повторю, конец 18-19 век, это кризис сельского
% хозяйства, развитие индустриального общества. Первая фаза – это переход от
% ручного труда к машинному, а затем появление новых областей, областей экономики.
% Во второй половине 19 века возросло значение нефтяной промышленности. Потом
% освоение электричества, кстати, за 50 лет. Насколько, как говорится, трудные
% времена толкают человечество. Новые отрасли промышленности возникли.
% Электрохимия, электрометаллургия. Химия вообще получила бешеный импульс к
% развитию. Началось производство искусственных удобрений, и естественная
% необходимость к этому привела. Синтетических взрывчатых веществ, потому что
% горное дело активно развивалось и так далее. Простите. Теперь социальная сторона
% промышленного переворота. Речь идет о формировании общества, связанного с
% фабричным производством. Фабричным производством – это наемные рабочие и
% собственники средств производства. Ну, это в данном случае буржуазия. Вот.
% Обратите внимание вот на эту ситуацию. То есть кризис традиционного общества в
% своей завершающей фазе. Кризис традиционного общества в своей завершающей фазе.

В конце 18 --- начале 19 века в Европе происходит промышленный переворот. Людям нужно было выживать, поэтому они активно начали использовать машины, уходя от ручного труда. Начался этап развития индустриальной экономики. Промышленный переворот имеет две стороны. Первая фаза --- переход от ручного труда к машинному и появление новых отраслей экономики. Во второй половине 19 века выросло значение нефтяной промышленности, а затем освоили электричество всего за 50 лет --- пример того, как сложные времена стимулируют прогресс.
Появились новые отрасли: электрохимия, электрометаллургия. Химия получила сильный толчок, началось производство искусственных удобрений и синтетических взрывчатых веществ из-за развития горного дела.

Теперь социальная сторона промышленного переворота --- формирование общества, связанного с фабричным производством. Это наемные рабочие и владельцы средств производства, то есть буржуазия. Обратите внимание: это завершающая фаза кризиса традиционного общества.

\subsection{Филисофские основания. Эпоха романтизма}
% И вся эта социально-экономическая и политическая сложность эпохи повлияла на
% формирование особого мировоззренческого феномена, на формирование романтизма.
% Романтизм – это, по сути дела, реакция на глобальную трансформацию. На вот эту
% вот, как сказать-то, с одной стороны, очень оптимистическая реакция, с другой
% стороны, очень пессимистическая. Это конец XVIII века, первая половина XIX.
% Чувствуете, мы пока еще не говорим о новой науке. Романтизм – он в пределах
% классического этапа, но уже кризис. Романтизм – это, значит, идейное и
% художественное явление, выступающее реакцией на мировоззрение, просвещение. Оно,
% как происходит само слово понятие, а слово «роман». Ну, роман как литературное
% произведение. А ведь в романе что главное? Это фабула, повествование о некоем
% случае, о некоей драме, о происшествии, о чем-то особенном, то есть выпадающем
% из фона повседневности. Вот. Поэтому романтизм – это о чем-то особенном, о том,
% что выпало из фона повседневности. Романтики называли себя духовными
% революционерами. Они обещали новую культуру, новую гениальность, новую религию.
% Кстати, они тогда были сатанистами, мама не горюй. Почему? Ну, потому что сатана
% мыслился романтиками как борец. Бунтарь против роли бога. Да, бунтарь против
% роли бога. Можно, конечно, много говорить, с чего это сами романтики взяли.
% Имеется в виду не то, что он бунтарь, а то, что этот бунт вообще-то, ну, как
% сказать, ну, что этот бунт имел именно то смысловое значение, в которое в него
% вкладывалось, вот так вот назову. Но мы про это сейчас говорить не будем.
% Романтика интересует сам пафос вот этой борьбы, чувства, выраженные с особой
% силой. Пафос – это чувства, выраженные с особой силой. Это всегда событие на
% грани, да? Поэтому я немножечко вам картинок просто покажу, чтобы, так скажем,
% проиллюстрировать. Я же искусствовед-то, понимаете, бывшим не бывает, а
% алкоголик. Поэтому картинки – это мое все. Ну вот, 18-й год, 19-го века,
% знаменитая картина Фридрих «Странник над морем туман». На самом деле эта картина
% – это, по сути дела, ну, икона романтизма. Это всегда человек, который стоит на
% какой-нибудь там бурной такой вот природной стихии, и он ее активно там вот как-
% то изучает внимательно. А мы видим лишь его такую вот в некотором смысле фигуру
% напротив, да? Вот он напротив стихии. Конечно же, это врубелевский демон. Можно
% много говорить, что это было связано с самим психическим состоянием, как
% говорится, самого врубеля, но дело-то не в этом. Дело в том, что эта фигура, она
% вот очень такая, хотя стиль – это уже символизм, но неважно, это очень
% романтическая фигура. Вот он, тот самый сатана, который как бы бунтует против
% Бога. Ну, конечно же, это любимый нами Брюлов, который не только нами любим,
% один из покупаемых художников во всем мире, последний день Помпеи. Это когда
% красота катаклизма показывается в полной мере. То есть красота даже страшных
% событий. Плод Медузы. Медуза – это был корабль такой, он потерпел
% кораблекушение, и на плоту там долгое время спасались люди совершенно страшным
% образом. Они там кого-то кушали, они там… Ну, в общем, такие страшные вещи были.
% Но, понимаете, романтики умеют делать из страшного красивое. Из какого-нибудь
% монстра сделан прекрасный литературный персонаж, из какой-то ужасной трагедии
% человеческой сделана красивая картина. Жирико – один из романтиков. Вот сюда же,
% пожалуйста, в Делакруа – «Свобода знаменитая ведущая народ». Здесь как раз
% изображены события Великой Французской революции, то есть то, чем Америка
% заразила Европу через нашего Лафаэта. Не нашего, я просто про него говорила.
% Самое, конечно, для меня, как искусствоведов, это всегда было удивительное, я уж
% не могу не поделиться, пожалуйста, обратите внимание, это курьезность
% романтического пафоса. Ну, тяжело смотреть много романтических картин, потому
% что постепенно ты начинаешь замечать на переднем плане мужика без штанов. У тебя
% возникает вопрос, ну ладно, здесь еще понятно, все-таки, ну, как будто одежда
% потерялась, там что-то на тряпке порвали. Почему он на баррикадах? Что делает на
% баррикадах мужик без штанов в одном носке? Это загадка, дорогие мои. Я
% спрашивала у ведущих искусствоведов, моих учителей. Почему? Никто не ответил.
% Поэтому вот оставляю вас с этой загадкой, надеюсь, что вы подумаете после лекции
% об этом. Я много думала, не нашла ответа. Конечно, это очень курьезные вещи, вот
% надо его замечать. Ну, прекрасная картина Брилова, всадница. Но вы можете себе
% представить, что вот так вот сидеть на лошади, которая вот это исполняет, очень
% нереально. Дорогие мои всадницы, которые, знаю, присутствуют среди нас, ну,
% оцените, да? Ну, вообще не вариант. Или вот, например, еще прекрасная картина
% романтизма. Почему я вам в конце показываю такую картину? Потому что хочу, чтобы
% вы оценили, что романтизм – это все-таки еще и усталость от самого себя, от этих
% вот бурных перемен, от того, что это какое-то бурление, которое, по сути дела,
% ни к чему не приводит. Чистое бурление ради бурления. То есть романтики
% постепенно начали понимать, что вот эти вот рыцари, которыми полны, значит,
% тексты и картины, рыцари, причем любых эпох, они рано или поздно возвращаются
% домой, усталые, такие вот несчастные, ну, по красивой природе зато. Давайте
% теперь как бы про мировоззрение романтизма поговорим поглубже, чем это делается
% в изображениях мужиков без штанов на переднем плане. Значит, романтизм связан с
% такими понятиями, как индивидуализм и субъективизм. Но вы должны понимать, что
% вообще новое время связано с этими принципами, а именно индивидуализм и
% субъективизм. Что такое индивидуализм? Ну, вас об этом могут спросить даже. От
% слова индивидуум, то есть нечто далее неделимое, то есть все, мы делим, делим,
% делим сложную систему на части, на элементы, и вот нечто далее неделимое.
% Кстати, элемент тоже означает слово далее неделимое. Вот. И смотрите, на мир,
% человек нового времени, который в парадигме редукционизма находится, в парадигме
% индивидуализма находится, он живет все тем же редукционизмом. Если вы еще не
% освоили, что такое редукционизм, рекомендую вам это, ну, как бы понять.
% Редукционизм и, наоборот, холизм. Редукция сведения, ну, к элементу, к малому, к
% части. Холизм – это примат, приоритет целого. Вы понимаете, что это существует и
% разная оптика взгляда на реальность. Очень разная. До нового времени по
% преимуществу существовала оптика холистическая. Холистическая. Когда, смотрите,
% индивидуальная жизнь и ее потребности – это как бы только часть. Есть что-то
% больше. Есть что-то больше. Человек, он мыслится, его характеристики, его смысл
% жизни, его значения. Все мыслится из большего. А что такое больше? Это, может
% быть, божественная реальность, может быть, понимание через призму рода, через
% призму народа, через призму традиции, понимаете, или даже семьи. То есть это вот
% так то, что мы связываем с традиционным восприятием реальности. Но я вам
% предлагаю вот эти вот несколько устаревшие уже понятия традиционное общество,
% нетрадиционное общество, ну, их знать, но все-таки больше на философские какие-
% то позиции опираться, а речь идет все-таки об оптике взгляда на реальность. Так
% вот, человек, ну, скажем, холистического склада, это не только люди там
% средневековья, это и современные люди. То есть я прекрасно понимаю этих людей,
% сама к таким отношусь. Это все-таки, ну, то есть я рефлексируюсь, я понимаю, у
% меня оптика вот такая. Я не хочу другую. Я ее понимаю, я ее знаю, я знаю, в чем
% ее ценность, но тем не менее, вот моя такая. Я уважаю людей с другой оптикой.
% Дело, как бы, понимаете, надо всем давать жить свою жизнь. Поэтому, как
% говорится, мы не будем говорить, какая оптика хорошая, какая плохая, но надо ее
% особенности видеть. Так вот, оптика прежних времен, оптика холизма, да, этот
% человек, он из большего всегда. Теперь в новое время возникает иная оптика. Но
% она не то, чтобы возникает, такие люди всегда были, но она как бы получает
% приоритет. Потому что в свое время, помните, как бы вот это вот социальным
% слоем, энтузиастом стали буржуа. А как многие исследователи считают, буржуа это
% в первую очередь психологический тип. Ну, ладно, тут спорный достаточно. Просто
% я это озвучиваю. Так вот, оптика редукционизма. Индивидуальная жизнь, ее
% потребности, это в центре. Вот, то есть ты как бы, ну, как бы, это оптика
% сокращения, понимаете, несколько сжатия зоны видимости. У человека нет оптики,
% как бы, такого, чтобы я, жизнь и посмертие, да, только жизнь. И эта жизнь
% складывается, ну, скажем так, вот из конкретных повседневных, очень актов, они
% важны, крайне важны. Вот, из частных, как бы, род складывается, или народ
% складывается из частных жизней. Они первичные, и они важнее. Род, он как бы,
% вот, это просто результат сложения, да. Тогда, как видите, в холистической
% логике, например, род или народ будет, как бы, ну, как сказать, тем целым, что
% характеризует частное. А в логике редукционизма будет наоборот. целый род народ
% будет таким, каким вот совокупность частного позволит. Вот такие вот разные
% оптики. У человека нового времени как бы нет телескопа, мировоззренческого
% телескопа. Он его как бы убрал в сторону, да. Или ему просто не дают в него
% посмотреть, говорит, что, дурак, что ли, так вообще не делают, нельзя так, ты
% чё, ты чё, дурак, чё. Ну, понимаете, да. Это вот, ну, такая вот идёт, ну, это же
% как бы не просто все, как бы, заживут свою жизнь, вы же ещё и подчиняетесь
% определённым социальным нормам, если подчиняетесь, конечно. Так вот, плюс, какой
% плюс у этой новой оптики? Повышенная активность с точки зрения мироустройства
% вокруг себя. Вот если ты видишь себя, в первую очередь, индивидуализм, да,
% субъективизм, ты будешь постепенно как бы замечать, что у меня здесь непорядок,
% здесь непорядок. Надо бы мироустроиться вокруг самого себя. И Европа этим
% активно занялась. Европа начала, вот, по сути дела, эта новая оптика, она
% позволила заняться, ну, скажем так, вот она, повседневной реальностью,
% повседневной жизнью. То есть чувствуете, да, всегда очень разные вещи. С такой
% оптикой трудно жить, выполняя какие-то задачи большие. Большие задачи. Не знаю,
% крестовый поход точно не отправишь. Да и просто на войну человек вряд ли пойдёт.
% Ну, мы видим это, да. Вот. Или там, он не будет жертвовать своей жизнью. Он
% будет её беречь, потому что у него это всё, что есть. Он больше ничего не видит.
% И от него нельзя требовать, чтобы ты чё? Он будет в воспоминаниях к насилии, да.
% Ну, ничего, что нельзя, куда деваться. Но он-то это, вы его не вдохновите. Он
% будет в воспоминаниях к насилию. Он согласится, может быть, но будет в
% воспоминаниях к насилию. Тогда как человек другой оптики, наоборот, вдохновится.
% Сейчас пойти и пожертвовать свою жизнь ради чего-то более значимого, крестоносец
% или депостовую. Да, пожалуйста. Ради народа, ради Бога, ради народа. Разная
% оптика. Так вот, романтизм – это обострение смены этой оптики. Понимаете?
% Романтизм – это эпоха, когда эта оптика поменялась, и всё вот из взгляда,
% реализуется из центра человека. Роман – переживание чувства постоянно, да? Вы
% читали роман, это переживание чувств по поводу чего-то там произошедшего. Всё,
% всё в мире. Есть лишь повод по поводу для этих переживаний. То есть это так
% называемая романтическая продуктивность. Немножечко она временами зашкаливает, и
% человек начинает переживать уже из-за чего-нибудь, из-за чего вообще не стоит
% переживать, и у него возникает тревожное расстройство. Так вот, это на самом
% деле такое вот последствие романтического видения реальности. Роль субъекта
% предельно усиливается. Атом – это отдельный человек, атом общества – отдельный
% человек, кризис традиционного общества. Я просто сейчас вам накидываю вот эти
% вот смыслы, которые, я думаю, вам и так известны. Но смотрите, что происходит,
% почему вот этот вот путник, вот этот вот крестоносец, такой печальный, усталый,
% и он больше не имеет сил, чтобы что-то вообще предпринять, не только потому, что
% он старый. Я сейчас немножечко, как вам так, как иллюстрацию это описываю.
% Потому что, смотрите, ведь всё ложится на плечи отдельного человека, и надо
% сказать, что за это нам, сегодняшним, все наши тревожные расстройства, они как
% раз связаны с тем, что в эпоху Нового Времени на атом возложили все функции
% человечества. Ведь прежнее традиционное общество, смотрите, ты принадлежишь,
% например, к сословию воинов, ну ты и воюешь. Ты принадлежишь к сословию
% клириков, ну так ты, как говорится, разбираешься в этой теме, а ты вот там
% другое сословие, ну так ты и правишь. Я как бы об этом не думаю. Есть вот люди,
% которые разделили свои роли в обществе, да, общество как целое. А теперь всё
% должно быть в одном человеке совместиться. Человек должен быть образован как
% профессор. То есть помните, высшее образование обязательно для каждого человека,
% кто это придумал, кто это продвинул. Это же не было в качестве нормы никогда.
% Это идея Гюста Конта. То есть образование должно включать науку. Все должны
% знать науку так, как только вот возможно. Да, в итоге её никто не знает. Вы
% потом, когда начинаете работать в науке, вы полностью переучиваетесь. Но тем не
% менее, все должны потратить годы на то, чтобы это было освоено. Человек должен
% быть реализован в профессии, как, я не знаю, там, гений последний. Почему?
% Потому что, ну, ты должен быть достойным буржуа. Ты должен быть успешным
% капиталистом, успешным предпринимателем. А как иначе-то? Ты должен быть воином
% обязательно. Ну, не женщиной в данном случае, мужчиной. Потому что всеобщая
% воинская обязанность сформируется во времена французской Великой революции.
% Понимаете, да? Конец 18-го, начало 19-го века. То есть, давай, давай, ты там и
% высшее образование получил, и в зону профессии. Дальше. Он должен обладать
% полноценным знанием добра и зла в политическом смысле. Потому что
% представительская демократия, ты уже теперь сам выбирай. Ты как бы разбираешься
% во всех политических вопросах и имеешь возможность выбрать достойную власть. Вы
% понимаете, да, насколько это вот, ну, как бы, сложно взять и разобраться сегодня
% у всех политических программах. При том, что вы занимаетесь, например, какими-то
% совершенно другими делами. Ты должен обладать богословским полностью, всем
% ресурсом. Ты должен выбрать, что называется, вот себе религиозное вот это
% состояние, или отказаться от него. Потому что тебе как бы, ну, вот, да, сам иди
% изучай Библию, вот всё, всё сам понимай. То есть нетрудно увидеть, нетрудно
% понять, что такой человек – это утопичный идеал. Но человек не может это всё
% вместить. Закономерно, что возникает не только настроение героизма, да, такой
% тип героического человека, но и тип романтика, который устал. Он устал на его
% плечах всё время мира. Он перестаёт действовать, потому что в этих условиях
% невозможно действовать. Это уход в себя, это в свою тоску уход, уход как бы в
% определённое самобичевание. Я не годен, я не тот, я не этот. Понимаете, да?
% Чувствуете, как в личном, даже в сегодняшнем, отражается историческое. Очень
% люблю такие вещи, и постоянно на них указываю пальцем. Надеюсь, вы что-нибудь
% дозаметили в этих указаниях. Таким образом, романтизм, я обобщаю тему
% романтизма, это и продолжение, и критическая реакция на проект модерна, на идеи
% просвещения. Именно в искусствах, в художественных образах реализуется и
% восторг, и ужас. Осознание того, что какая-то мощная инерция тянет тебя куда-то
% помимо твоей воли. И, соответственно, это реализуется и в люциферианский бунт,
% вы помните, где мы говорили, да? И одновременно в сопротивлении этой силе, и в
% пессимизм, как осознание невозможности, просто ты плывёшь по течению. Вот очень
% богатое время, богатое мировоззрение, связанное с эпохой романтизма. Прошу
% любить и жаловать, что называется. А теперь ближе к науке. Продолжается поиск
% облика науки. Это, что называется, не шутка. Выбор облика науки. Наука по-
% прежнему решает, какой ей быть. Помните, вот вы уже много прошли в истории
% науки, и я вас призываю ещё раз это осмыслить. Наука попробовала себя в качестве
% философии, на турфилософии, на турфилософский период, да? И сказала нет. Она
% сказала нет. Мне нужно что-то другое. Она попробовала себя в одном из видов на
% турфилософии, а именно в герметизме, в магии, в алхимии. Она активно это
% попробовала, эпоху возрождения, помните, да? Это ещё тоже на турфилософский
% период. И в принципе она тоже сказала нет. Не моё. Затем она попробовала себя
% даже в форме религии. Это, например, Рогюстконт и религия как, вернее, наука как
% объект религии, как объект поклонения. И тоже очень быстро сказал нет. Давайте,
% говорит, что-нибудь другое. И вот она продолжает себя искать. И вот наступает
% новое время. 

Вся социально-экономическая и политическая сложность эпохи сформировала особый мировоззренческий феномен --- романтизм. Это реакция на глобальные перемены конца XVIII --- первой половины XIX века, одновременно оптимистичная и пессимистичная. Пока речь не о новой науке, романтизм находится внутри классического этапа, но уже в состоянии кризиса. Это идейное и художественное явление, противостоящее просвещению. Название связано с понятием «роман» --- повествование о чем-то особенном, что выпадает из повседневности. Романтики называли себя духовными революционерами, обещая новую культуру, гениальность и религию.

Романтики часто ассоциировались с сатанизмом, поскольку сатана для них был борцом, бунтарём против Бога. Не в смысле просто бунтаря, а с особым смыслом этого протеста. Главное в романтизме --- пафос борьбы и сильных чувств, всегда на грани события. Чтобы проиллюстрировать, приведу картины: Фридрих --- «Странник над морем тумана», где человек стоит перед бурной природой, символ романтизма. Врубель --- демон, бунтующий против Бога, тоже романтический образ, хоть стиль и символизм. Брюлов --- «Последний день Помпеи», показывает красоту катастрофы. Жирико и Делакруа --- «Свобода, ведущая народ» о Великой Французской революции, передают дух времени.

Романтизм связан с индивидуализмом и субъективизмом --- ключевыми принципами Нового времени. Индивидуализм от слова «индивидуум» --- неделимая часть. Человек нового времени видит мир через редукционизм --- сведение всего к элементам, частям. Противоположность --- холизм, где главное --- целое, а не части. До Нового времени доминировал холизм: человек мыслится из большего --- бога, рода, народа, традиции или семьи. Это традиционное восприятие мира, но и сегодня многие живут в холистической оптике, как и я.

Новое время даёт приоритет редукционизму, с акцентом на индивидуальную жизнь и повседневность. Буржуа как социальный слой и психологический тип стали носителями этой оптики. Редукционизм сужает взгляд, исключая мировоззренческий «телескоп». Человек нового времени видит только жизнь, без посмертия, и сосредоточен на конкретных частных актах, из которых складываются род и народ. В холизме целое определяет частное, а в редукционизме --- наоборот.

Эта новая оптика даёт активность, но усложняет жизнь: невозможно отправиться на крестовый поход или пожертвовать собой ради чего-то большого --- человек бережёт жизнь, потому что видит только её. Романтизм --- это острота смены этой оптики, где всё начинается с субъекта. Роман --- переживание чувств по поводу события, и часто эти чувства чрезмерны, вызывая тревогу. Усиливается роль отдельного человека, атома общества, что ведёт к кризису традиционного общества.

Раньше роли были распределены: воин воевал, клирик --- занимался духовным, правитель --- управлял. Теперь всё должно быть в одном человеке: образование, профессия, политическое понимание, религиозное состояние. Это утопичный идеал, и человек не может это всё вместить. Поэтому наряду с героизмом возникает усталость, уход в себя, тоска и самобичевание --- «я не тот, я не этот».

Романтизм --- и продолжение, и критика модерна и просвещения. В искусстве отражаются и восторг, и ужас, осознание, что тебя тянет куда-то вне твоей воли. Это люциферианский бунт и одновременно сопротивление, пессимизм и бессилие. Очень богатое и сложное время, мировоззрение эпохи романтизма.

\subsubsection{Наука}
% снова  возвращается наука к
% герметизму. Причём в таких острых формах, которые прям трудно даже себе
% представить. Вот про это мы очень мало знаем. Я уверена, даже в учебниках
% философских про это очень мало написано. Но есть исследователи, мы к ним активно
% обращаемся. Мы, что называется, не пренебрегаем важными исследованиями, обращаем
% внимание на всё. И вот один из таких романтиков, учёных был Гёте. Я вам
% показывала, да, его картину, в смысле, его тут вот этот слайд. Ведь Гёте на
% самом деле не хотел, чтобы его запомнили поэтам. Он хотел, чтобы его запомнили
% учёным. Он активно, как бы, в своей задаче видел активное сопротивление
% ньютоновской физики. Поэтому он серьёзно повторял те же самые опыты, те же самые
% эксперименты, которые проводил Ньютон. А какие эксперименты проводил Ньютон?
% Подчеркну здесь, Ньютон не мог провести эксперименты и не проводил связанных с
% теорией гравитации. Там было, по сути дела, математическое исследование и
% расширение принципов галилеевской физики. А вот в оптике Ньютон активно проводил
% эксперименты с призмой. Он смотрел на то, как вот там вот разлагается свет. И
% Георгий воспроизводит ньютовский эксперимент с призмой очень активно. И, как
% говорится, увидел нечто новое. Это удивительно. То есть он, дело в том, что не
% просто посмотрел на призму, он начал смотреть на разные виды света. Например,
% отражение света в зеркале, он пытается преломить, и свет, который прошёл через
% цветное стекло. Цвет в тени, в сумерках, на разных цветных поверхностях. Или
% подносил глазам, или подальше призму ставил. Понимаете? То есть он безумное
% количество экспериментов провёл и обратил внимание на то, что спектр или другая
% форма цвета исчезали при определённых обстоятельствах. То есть, верите,
% сопротивляясь ньютоновской физике всеми силами, он пришёл к несколько иным, к
% несколько иной оптике. Другое дело, что мы-то сейчас развеем ньютоновскую, но
% тем не менее, вот имейте в виду, что Гёте предложил свою оптику. Он говорил, что
% только если на светлой поверхности, которую он рассматривает, нет тени, то
% тогда, да, появится знаменитый спектр. А вот стоит появиться тени, тот цвет
% меняет... Вот это вот, ну как сказать, я ведь сама эти опыты не проводила, я не
% могу ему писать. Я лишь привожу слова Гёте, ну в смысле, вернее, как бы то, что
% он описывал в своих произведениях и, соответственно, то, что мы можем себе
% представить. Так вот, он делает вывод, что цвет --- это степень темноты. Цвет ---
% это не разложение спектра, это степень темноты. А разлагает на спектр, по сути
% дела, человеческое восприятие. Вот человеческое восприятие --- это центр и
% источник цвета. Он говорит, вот если я стою так по отношению к призме, цвет
% появляется, ну, разложение на спектр. Если я стою по-другому, он не появляется.
% Если я призму ставлю там, цвет появляется, цвет не появляется. Чувствуете, моё
% восприятие здесь ключевое. То есть, вот в этом результате с появлением и
% восприятием цвета Гёте установил, что когда человеческий глаз видит цвета, он по
% сути видит свои собственные возможности. Они просто реагируют на раздражителей.
% Что-то похожее на Канте, если вы понимаете, о чём я. Вот цитата из его учения о
% цвете. Можете поинтересоваться. То есть, субъект и объект, говорит он,
% непрерывно взаимодействуют. Нет отдельно, нет субъекта и объекта. Всё, что мы
% знаем, всё наше познание реализуется вот в этой вот связке, состыке объекта и
% субъекта. Объект продолжает, является продолжением субъекта, а субъект является
% продолжением объекта. Это очень интересная мысль, которая на самом деле далеко
% вперёд уходит, если вы понимаете, о чём я. Ну, надеюсь, в конце курса поймёте,
% даже те, кто сейчас не понимает. Дальше. Ну, немало времени Гёте посвятил и
% опровержению механицистического подхода к действительности и механицистического
% образа природы. Надо сказать, что многих в Европе даже в эпоху Возрождения, даже
% в эпоху Нового Времени коробило от выражения «мир машина». В первую очередь, это
% были англичане. Англичане вообще слово «машина» использовали в качестве
% ругательного слова. Да ты машина, это сначала, что ты как что-то нечеловеческое.
% Вот. Ну, сейчас не так, а вот как бы раньше так было. И немцы отчасти, ну, как
% бы, как лебницы, например, ну, да, машина, конечно, может, машина, но другая, не
% та, которую вы имеете в виду. Мир, говорили они, скорее организм. Не машина, но
% организм. А организм чем отличается от машины? Он имеет свою историю, он имеет
% свою эволюцию. Машина такая создана, такая вот живёт. А всё, говорят, вот
% романтики, имеет свою эволюцию. Мир живой и развивается. Отсюда установка
% органицизма. Органицизма. И вот вам некоторые оппозиции, да, по отношению к
% механицизму. Механическое противопоставляется органическому, вернее,
% органическое противопоставляется механическому. Холизм противопоставляется
% редукционизму. Я надеюсь, что вы понимаете. 

% Но, тем не менее, романтическая
% мысль, отказываясь от науки нового времени, она как бы не двигается вперёд
% поначалу. Она двигается назад к возрождению, к флагистону, к пневме, к плероме,
% к теплороду, к седеризму, к огненной перстихии и тому подобное. То есть, к
% натурфилософии. Мы называем это романтическая натурфилософия. Ещё раз подчеркну,
% да, этап науки классический, этап науки классический, но в нём есть место
% натурфилософии, а именно романтической натурфилософии. На основе иенского кружка
% братьев Шлегель, ну, это вот как раз, как, знаете, Платоновская академия в эпоху
% Возрождения, так вот этот иенский кружок для мыслителей эпохи Возрождения. Там
% Шеллинг, Навалис, Гёте, Шлирмахер, Гердер, фон Гумбольд и много-много других
% мыслителей эпохи романтизма, вот они как раз состоят членами этого кружка. И они
% активно изучают природу. Но смотрите, их основными темами становятся уже
% несколько иные предметы, не те, которые изучал Ньютон, Галилей и, например, там,
% Роберт Бойли. Их основными темами становятся электричество, магнетизм,
% психическая жизнь человека. Вот таким образом. Они активно, на самом деле, тоже
% цитируют Гермеса Трисмегиста, и они ищут то, что пытался, о чём пытался говорить
% Гермес, ну не пытался, он говорил о едином. вот этот поиск всеохватной единой
% первой субстанции, то есть они пытаются найти плерому, они пытаются найти
% пневму. Чувствуете, возрождение говорило, что она просто есть. Почему? Ну, в
% конце концов, Аристотель сказал ей, значит, есть. Чего надо ещё по этому поводу?
% А тут-то уже эпоха экспериментаторов, видите, эксперимент уже вошёл в историю
% науки, он вошёл вместе с Галилеем, он вошёл вместе с, вот, ну, со многими-
% многими-многими учёными. И, соответственно, вот это вот новое поколение учёных,
% учёные-романтики, романтические натурфилософы, они пытаются найти эту самую
% пневму, ну, с помощью разных приборов. Помните, да, приборы очень, как бы,
% важная часть новой науки, инструментализация. Помните, технонаука, надеюсь, вы
% это понимаете. 

% Ну, давайте, пример вам приведу, чтобы не быть голословной.
% Риттер всю жизнь искал связь электричества и магнетизма, открыл несколько
% законов. Дальше. Эрстет, надеюсь, вы тоже знаете, он был ученик Шеллинга Гёта,
% он исследует активное электричество. Почему он исследует? Потому что он видит
% именно в электричество эту самую плерому, первую субстанцию, кневму. Он говорит,
% вот она, кневма, мы, наконец, её нашли, его трактат «Дух в природе». В феномене
% гальванизма, вы наверняка слышали, что Эрстет открыл феномен гальванизма, то
% есть животное электричество. Риттер нашёл доказательство, он увидел в этом
% доказательство вот этой вот единой мировой субстанции, единой мировой души.
% Электричество, говорил, он считает, вернее, он считал, что электричество бывает
% металлическим и биологическим. Ну вот он там много ещё чего разработал. В том
% числе он как бы создаёт научную концепцию астрологии. То есть его воздействие
% небесных сил осуществляется уже с помощью электричества. Он говорит,
% электромагнетическое единство Вселенной. Ну посмотрите, как это интересно. То
% есть наука развивается нелинейно. Она не просто вот кто-то там открыл, там что-
% то и пошла, знаете, наука пошла. Да нет, это такие вот блуждания в самой разной
% области. Вряд ли мы бы сегодня так хорошо знали электричество, вообще вряд ли бы
% даже заинтересовалась наука электричества, если бы в своё время такие вот
% товарищи не решили, что ах, вот оно что, пневмоты, есть электричество. Давай его
% активно исследовать. Ну то есть Риттер, тот самый, да, простите, я сейчас про
% Риттера почему-то, но Риттер и РСТ, они открывают, что электролиз,
% электрокапиллярные явления в ртуке, ультрафиолетовые лучи, в поиске пневмы они
% это всё открывают. 

% Пристли, наверняка, очень вам известный учёный. Он был
% учёный-экспериментатор и одновременно священник, униак. Так вот, он свою научную
% деятельность воспринимал как продолжение религиозной борьбы. Он униат, то есть,
% ну, против принципа триединства выступала определённая вот как бы религиозная
% традиция, это униат, ну, это английская традиция. Вот он ищет доказательства
% униатного, то есть единого, ну, как сказать, Бога, без принципа Троицы, ищет его
% в природе, ну, он же тоже как бы не без герметизма. И вот он тоже предложил
% видеть её в электричестве и попутно открыл кислород. Вот так. То есть в
% учебниках вы встретите, что там Пристли открыл кислород, что Риттер открыл эти
% самые ультрафиолетовые лучи, что Эрстет открыл галлиническое электричество, да,
% всё это вы видели. Но теперь вы поймите, что они искали, когда это нашли. Вот
% такие мистические настроения натурфилософов не означает пренебрежение
% экспериментов. Вот всё, что угодно, эксперименты они делали. То есть чувствуете,
% сам эксперимент науку не рождает. Рождает как бы союз определённого стиля
% научной мысли и уже эксперимент. То есть эксперимент плюс стиль научной мысли.
% Потому что, ещё раз повторю, вот вам, пожалуйста, герметическая натурфилософия,
% которая активно экспериментирует. Можно из этого современную науку увести? Нет.
% А из чего-то да, она появилась. Где это? И вот тут я обращаю ваше внимание на
% роль природознацев. 

% Я про них постоянно говорила, упоминала. Пожалуйста,
% вспомните всё, что это говорилось. Вот если вы действительно хотите понять
% историю науки, то без природознацев не обойтись. У меня статья на эту тему
% написана. Я ничуть не рекламирую своё творчество, но если хотите погрузиться в
% эту тему ближе, ищите оболкина становление научного дискурса. Его в списке
% литературы нет, потому что список составлялся тысячи лет назад и вообще не мной.
% Хотя надо бы его исправить, конечно. 

% Давайте как раз обратимся во внимание на
% вот этот, когда возник союз природознацев и вот таких вот учёных. учёных.
% Исследования теплоты. Исследования теплоты начинаются традиционно. Это поиск
% первой субстанции, флагистон, теплород, кневма. То есть герметические
% исследования. Но их надо было в ходе вот этих исследований одно общество решило,
% а надо бы вывести их на практику. Вот эти принципы воплотить в практике, в
% промышленности. Это так называемое лунное общество конца 18 века. Интереснейшая
% история. Она также редко описывается, но поинтересуйтесь. Уж найти сегодня
% несложно. Это было тайное общество. Тогда все общества были тайные. Я не знаю,
% какие-нибудь общества были не тайные. Они собирались естественно в период
% Новолуни, в Black Country. Основателем, один из основателем, активистом этого
% общества был дедушка Дарвина, Разум Дарвин. И вот здесь широко обсуждалась идея
% необходимости выхода тайной науки. Но вы помните, что весь герметизм это тайная
% наука. Так вот, нужно выйти на практическое поле, в промышленность. И вот тут
% включаются товарищи природознатцы, которые смотрят на эти герметические поиски,
% говорят, ну да, ну да, а надо-то вам что сделать? Какую машину построить? То
% есть чувствуете, моряки, технари, разного рода, вот такие вот люди, которым
% раньше пренебрегали, они же не ученые, они же там копошатся в земле, в своих там
% металлах, какие-то инженеры, господи боже мой, кому они интересны. Так вот,
% Уолтон и, вы спросите, Болт и Уолт, крутые, боже мой, сейчас скажу, Мэтью Болтон
% и Джеймс Уатт, наконец-то, Мэтью Болтон и Джеймс Уатт, они были члены лунного
% общества, но они в то же время как бы, имели выход на вот эту среду
% природознации. И они такие, знаете, приземленные достаточно такие, такие, ну,
% как сказать, без этих вот парений в натурфилософских каких-то высях, они
% говорят, нам бы надо назначить сумму за использование паровой машины, но чтобы
% она вошла в общее потребление, нужно как-то заинтересовать покупателей. И вот
% они измерили количество теплоты в соответствии с работой лошадью за единицу
% времени в фута фунта. Да? Помните, лошадиные силы. И дальше в дело шел рекламный
% ход, то есть установка и обслуживание паровой машины осуществлялись бесплатно,
% но взималась стоимость одна треть суммы, которая образовывалась из экономии на
% стоимости корма для лошади. Ну, сейчас я так объясню. То есть, эту работу должна
% сделать паровая машина и лошадь. Сколько нужно кормить лошадь, то есть, сколько
% это стоит, чтобы лошадь сделал эту работу? Вот это стоит столько же. Так вот, мы
% с вас возьмем одну треть этой суммы. Рекламный ход удался. И паровая машина
% пошла, что называется, в народ. Ее стали покупать, а отсюда же возникла идея
% измерения эффективности паровой машины в лошадиных силах. Но это, что
% называется, ладно, самое главное, что из этого возникла идея механического
% эквивалента теплоты. мая, мая, карно, гельм, гольц, джоуль. И вот, когда мы
% читаем эти имена, мы вдруг обращаем внимание вот на что, что романтическая
% натура философия постепенно расслоилась. В ней остались, с одной стороны,
% сторонники чистого герметизма, это Риттер, Окен, Фехнер, Оствальд, А с другой
% стороны, ее ярые противники. Они называли вот эту герметизм, называли досужими
% выдумками, бред и так далее. То есть впервые какие-то ученые назвали герметизм
% бредом.

% Это были, вот они, Фрис, Сентеллер, Майер, Гельмгольц, Больцман. Ну,
% некоторые имена, я уверена, вам известны. Некую среднюю позицию заняли вот
% Рстет, Либих, фон Гульбольд. Если вам интересно вот это расслоение, а оно весьма
% интересно, то обращаю ваше внимание на тексты Порус, такой исследователь Порус.
% Вот он именно выделяет среди ученых этих вот, это расслоение начинается. И
% именно тогда, кстати, появляется понятие оккультизм. То есть наука, вот эта вот
% единая, пока дни дифференцированная среда, наконец-то разделилась на оккультизм,
% все, герметизм пошел своей дорогой. И наука, наука пошла своей дорогой. Вот, по
% сути дела, мы имеем дело, впервые мы имеем дело с появлением научного стиля
% мысли, научный стиль мысли. Понимаете, эксперимент уже был, но требовалось до
% конца, но до конца 19 века вырабатывался вот этот вот научный стиль мысли.
% теперь давайте обратимся конкретно к тем исследованиям. Ага, подождите, значит,
% не к тем исследованиям мы обратимся. Мы сейчас обратимся к кризису научного, не,
% не к кризису научного, к кризису классического мировоззрения, то есть пойдем от
% философии. Можно было бы, конечно, сейчас пойти сразу же в науку, к Больцману,
% да, рассказать, что там Больцман делал, и мы поговорим об этом. Давайте вначале
% поговорим о более широкой палитре, тем более, что нам так удобнее по времени
% сделать. 

Снова возвращается наука к герметизму --- причем в таких острых формах, что трудно себе представить. Про это мало что известно, даже в учебниках философии. Но есть исследователи, к которым мы активно обращаемся и не пренебрегаем важными работами. Один из таких ученых-романтиков --- Гёте. Я показывала его слайд. Он не хотел остаться просто поэтом, а хотел запомниться как ученый. Гёте видел сопротивление ньютоновской физике и повторял его опыты, особенно с призмой.

Ньютон не проводил экспериментов по гравитации, это была математическая работа, а вот в оптике он много экспериментировал с призмой и светом. Гёте воспроизводил эти опыты, но углублялся дальше: исследовал отражение света в зеркале, свет через цветное стекло, цвет в тени и сумерках, ставил призму ближе или дальше от глаз. Он провел огромное количество экспериментов и заметил, что спектр цвета исчезает при определенных условиях. Сопротивляясь ньютоновской физике, он пришел к иной оптике.

Гёте утверждал, что цвет --- это степень темноты, а не разложение света на спектр, которое создается восприятием человека. Цвет зависит от положения наблюдателя и условий освещения. Человеческий глаз видит не внешний цвет, а собственные возможности восприятия, реагирующие на раздражители. Субъект и объект взаимодействуют непрерывно --- всё познание происходит в этой связке. Это идея, очень продвинутая для своего времени.

Гёте много времени посвятил критике механицистического подхода к природе. В Европе эпохи Возрождения и Нового времени многих раздражала идея мира как машины, особенно англичан. Для них «машина» была ругательством --- что-то нечеловеческое. Немцы видели мир как организм, с историей и эволюцией, а не как статичную машину. Отсюда противопоставление механического и органического, холизма и редукционизма.

Но романтическая мысль, отвергая науку Нового времени, сначала двигалась назад --- к натурфилософии: флагистон, пневма, теплород, огненная стихия и прочее. Это называется романтической натурфилософией. Научный классический этап включал место для нее, например, в Иенском кружке братьев Шлегель --- с мыслителями эпохи романтизма, такими как Шеллинг, Навалис, Гёте и другие. Они изучали природу, но уже новые явления --- электричество, магнетизм, психическую жизнь человека. Цитировали Гермеса Трисмегиста, искали единую первичную субстанцию --- плерому, пневму.

Экспериментальная наука с Галилеем и другими уже была в истории, но романтические натурфилософы пытались найти пневму с помощью приборов. Важна была инструментализация, технонаука. Например, Риттер всю жизнь искал связь электричества и магнетизма, открыл несколько законов. Эрстет, ученик Шеллинга и Гёте, изучал активное электричество, считая его проявлением пневмы, писал трактат «Дух в природе». Он открыл гальванизм --- животное электричество. Риттер увидел в этом доказательство единой мировой души, выделял металлическое и биологическое электричество, разрабатывал научную концепцию астрологии, связывая небесные силы с электричеством и электромагнитным единством Вселенной.

Наука развивается нелинейно --- много блужданий в разных областях. Без таких, как Риттер, наука об электричестве вряд ли была бы так развита. Он и другие открывали электролиз, электрокапиллярные явления, ультрафиолетовые лучи в поисках пневмы. Пристли, известный ученый-экспериментатор и священник, воспринимал науку как продолжение религиозной борьбы, искал в природе Бога без триединства, тоже с герметизмом, и открыл кислород.

В учебниках вы видите их открытия, но теперь понимаете, что они искали, когда нашли. Мистические настроения натурфилософов не означали пренебрежение экспериментами. Эксперимент сам по себе науку не рождает, рождается союз стиля научной мысли и эксперимента. Герметическая натурфилософия экспериментировала, но из нее не вышла современная наука. Здесь важна роль природознацев, про которых я часто говорила.

Если хотите понять историю науки, без природознацев не обойтись. У меня есть статья на эту тему --- «Становление научного дискурса» от Оболкина. Рекомендую, если интересно. Союз природознацев и ученых важен, особенно в исследованиях теплоты, традиционно связанных с поиском первой субстанции --- флагистона, теплорода, пневмы. Одно общество решило вывести эти идеи в практику --- Лунное общество конца XVIII века, тайное общество, основанное, среди прочих, прадедушкой Дарвина, Разумом Дарвином.

Они обсуждали необходимость выхода тайной науки в промышленность. Смотрели на герметические поиски и думали, что нужно построить машину. Среди них были инженеры, моряки, технические специалисты, которых раньше не считали учеными. Мэтью Болтон и Джеймс Уатт были членами Лунного общества и связывали герметизм с практикой. Они придумали систему расчёта лошадиных сил, измеряя количество теплоты, необходимое для работы лошади, и брали с пользователей паровой машины треть этой суммы, объясняя это экономией на корме лошади.

Рекламный ход сработал --- паровая машина вошла в широкое употребление, возникла идея механического эквивалента теплоты, которую развивали Майер, Карно, Гельмгольц, Джоуль. Романтическая натурфилософия постепенно расслоилась: с одной стороны остались сторонники герметизма --- Риттер, Окен, Фехнер, Оствальд, а с другой --- ярые противники, которые называли герметизм бредом: Фрис, Сентеллер, Майер, Гельмгольц, Больцман. Среднюю позицию заняли Эрстет, Либих, фон Гумбольд.

Это расслоение подробно анализирует исследователь Порус. Тогда и появилось понятие оккультизма --- герметизм пошел отдельно, а наука --- отдельно. Появился научный стиль мысли, который окончательно сформировался к концу XIX века. Эксперименты были, но нужен был именно стиль научного мышления.

\paragraph{Маркс и Энгельс}
% Речь идет о философии, о кризисе, именно такого кризиса классичности в
% философии, ну, в мировоззрении в целом широко, в философии. И я призываю вас
% обратить внимание вот на нескольких философов, не то, чтобы это были, ну, все,
% да, но это очень важный философ, ключевые, я бы сказала. Давайте, вначале,
% конечно же, движение в сторону материализма, движение в сторону материализма
% очень активное, уже, понимаете, даже не столько какого-то такой в качестве
% борьбы с атеизмом, хотя, конечно, товарищ Маркс очень боролся с клиром, очень
% боролся с религиозностью, штамповал какие-то даже мифы по этому поводу, но
% неважно. Во всяком случае, вот этот материалистический тренд, он появился в это
% время, и не случайно вот эти истории материализма, это был бестселлер 19 века, и
% жизни Иисуса, то есть это материалистическое истолкование религиозных позиций,
% это тоже был трендом вот просто таким, всеобщим, ну, знаете, бестселлер. Так
% вот, философия Маркса начинается с политической борьбы в самой гуще вот этого
% вот развивающегося буржуазного общества. Обращая ваше внимание на его ранние
% труды, например, экономическо-философский рукописи 1844 года. В них он исходит
% из простого, очень простого наблюдения над обществом. Собственник затыкает рот
% правоведу. Ну, он сам был юридически подкован, образован, поэтому он прекрасно
% понимает, что вот этот нарождающийся капитализм, он ставит под сомнение все
% прежние формы легитимности, то есть все, что приносит прибыль, вот это
% легитимно, а то, что там право, неправо, это уже, знаете, закрывает рот, что
% называется. В этом проявляется на самом деле и ограниченность самого Маркса. Я,
% я на время подтверждения его теории, это очень интересная вещь. Смотрите, сам
% Маркс не может выйти за рамки буржуазного мировоззрения. Сам Маркс буржуазен до
% крайности. А в чем это проявляется? Он считает, что все самое главное решается в
% вопросе экономики. Вот экономика это самое главное. Экономика это что? Это, ну,
% есть прибыль, нет прибыли, что называется, ну, как говорится, есть ресурсы, нет
% ресурсов. А вот все другие формы культуры, все другие формы заинтересованности
% человека. Искусство, политика, наука, религия, право, это все что угодно. Все по
% Марксу есть надстройка над экономическим базисом. то есть чувствуете, вообще-то
% он рассуждает как буржуа. Главное, это все-таки экономика. Главное, что мы
% зарабатываем, как мы зарабатываем, почему мы зарабатываем, а как бы все
% остальное это вторично. Но тем не менее, что называется, вот, как бы, ну, ну,
% ну, в общем, сами порассуждайте, насколько Маркс одновременно ограничен этим
% рамками буржуазного мировоззрения, но в то же время, ведь сам факт этой
% ограниченности показывает, что буржуазное мировоззрение, оно, как бы,
% захватывает человека фундаментально. Тезис бытие определяет сознание. Наверняка
% вы знаете, это материалистическая философия, бытие определяет сознание. Только в
% данном случае слово бытие не надо рассматривать так же, как его мысль
% упромедить. Надо по-правильному перевести это слово, видите, перевод-то с
% английского, а в английском языке бытие и существование не различаются. Это
% большая проблема для философии, поэтому надо правильно прочитать это.
% Существование определяет сознание. И все, смотрите, становится понятно. Условно
% говоря, если ты 18 часов в день вкалываешь на работе, где требуется физическая
% сила на износ, то вряд ли ты напишешь философскую концепцию. Ну, согласитесь,
% тривиальный, на самом деле, тривиальный вывод. Правда, ну, вот, как бы, не знаю,
% честно могу сказать, моя личная жизнь вполне себе запровергает. У меня есть
% работа, физическая работа, вроде с дерьмом, конюшня у меня есть. Но, тем не
% менее, как-то я пишу философские работы, в принципе, как бы, он-то, видите, как
% считал, не просто по времени у тебя не будет сил, а у тебя как бы сознание
% перестроится, сознание перестроится, ты просто не сможешь. Не сможешь. Ну, нет,
% сможешь. Можно убирать говно полдня, а потом написать философскую работу. Можно.
% Ну, ладно. Что называется, это мой личный спорт с Марксом, он в экзамен не идет.
% Ладно? Да, договорились. Еще раз подчеркну, производительные силы определяют
% отношения людей. Вот это как бы база марксистского анализа. История представляет
% собой закономерную смену общественно-экономических формаций. в основе вот этого
% изменения противоборства социальных сил, которые Маркс назвал классами. Ну, я
% тут говорю Маркса, я подразумеваю все-таки Маркса и Энгельса. Сейчас не буду
% рассказывать про то, как они там различались, они сильно различались. И Маркс
% активно, особенно активно критикует последнюю из этих формаций, как он говорит,
% последнюю, это капитализм. Когда развитие человека, как личности, подавленной
% эксплуатацией, стихией рынка и процессом роста капитала. В чем как бы истина
% Маркса? Можно долго спорить с тем, что его концепция общественно-экономических
% формаций, она себя, в общем-то, изжила. То есть время, ну, действительность
% показывает, что, ну, не так общество называется. Но, тем не менее, Маркс очень,
% как говорится, никто не сохраняется в философии просто так. И Маркс, это он
% полное право занимает философия, ты прекрасный мыслитель. А почему? Вот, мне
% кажется, вам не стоит забывать о концепции отчуждения. Вот прям подчернитесь и
% красненьким, отчуждение. Я про него немножко скажу, уточните, разберитесь,
% поинтересуйтесь. Так вот, Маркс говорит, смотрите, самая большая проблема
% заключается в том, что продукт труда и сам труд отчуждается от человека в
% условиях наёмного труда и автоматизированного производства. Он много пишет о
% машине, он много пишет о фабриках, он много пишет о наёмном труде. Отчуждение
% труда. А труд-то, говорит, и сделал человека. Труд и есть наше основание быть
% человеком. Помните, труд сделал человек из обезьяны человека, простите, вот. А
% труд в условиях наёмного капиталистического производства оказывается отчуждённой
% реальностью. Вот это отчуждение, это основная категория Марксовой критики. И
% отчуждение может быть не только, вот знаете, на фабриках. Например, отчуждение
% продуктов творческого. Я, например, публикую книгу, и книга перестаёт мне
% принадлежать, она разбирается на такие смыслы, с которыми я могу быть не
% согласна. Правильно? Или там, статья, не важно. Это тоже отчуждение. Но Маркс
% это не критикал, потому что, видите, ну вот это что называется неизбежный
% процесс. Но тем не менее, это тоже категория отчуждения, которая крайне важна. А
% теперь смотрите, вот с чем связано. В чём не классичность Маркса? Ведь Маркс
% очень много пишет про то, что рационализм – это наша сила, наука – это наш
% двигатель и так далее. Но я сейчас обращаю ваше внимание, а потом вы поймёте
% это, например, Больцмана. Человек – это совокупность социальных отношений.
% Человек как элемент – это, по сути дела, то, что принцип редукционизма-то
% разрушает в максимальном понимании. Ведь смотрите, человек, его как атома нет. У
% него нет личного сознания, у него есть только общественное сознание. Он есть
% совокупность социальных сил. Чувствуете? Из целого. Целое отражается в частном.
% Целое отражается в частном. Вот эта холистичность – это и есть интересный такой
% вот кризис уже, ну, как бы, мысленно, нового времени. Вот этого подхода
% редукциониста. Маркс, надо сказать, романтик, конечно. Вот я сейчас немножечко
% отвлекусь, чтобы просто фигура Маркса стала у вас более выпуклой. Она,
% понимаете, советская, как бы, ну, советская наука, советская философия слишком
% затёрлась. А Маркс, конечно, романтик. Он был сатанистом по молодости. Он писал
% стихи такие сатанистские. Ну, тогда, на самом деле, многие это делали. То есть я
% не то, чтобы обвиняю Маркса, просто чтобы он картинку рисует. Думайте, почему
% они такие бородатые с ангельсом. Потому что они решили копировать, одновременно
% пародировать ветхозаветных пророков. Они моду ввели. Это панки 19 века, дорогие
% мои. Просто панки тогда, вот, ну, не брились аналоза, а наоборот, бороды
% отрачены. Вот если вы видите фотографии, где какие-то вычурные бороды или эти
% самые, как их назвать, усы, имейте в виду, моду ввели Маркс с ангельсом. Вот.
% Более того, он писал страшные сказки в духе вот Мэри Шелли, да, Мэри Шелли. Вы
% можете почитать книгу его дочери. Она описывает Маркса домашнего, Маркса как
% папа. Я вас умоляю, слушайте, какие страшные сказки он рассказывал детям, дети
% его обожали. Дети, конечно, прожили страшную жизнь, ну, там, погибло их очень
% много. Они жили постоянной бедности. Маркс же не работал, Маркс же был этим
% содержантом у Энгельса. Вот. За эти труды ему мало платили. И он постоянно в
% бедности жил. И вот он рассказывает своим детям сказку про одного владельца
% игрушечного магазина, который постоянно беден, никак не может продать свои
% игрушки. И поэтому он, значит, ну, как сказать-то, он стал продавать свои
% игрушки дьяволу. Вот. Эти дьяволы, вернее, эти игрушки претерпевались самые там
% страшные переключения. А потом они сбегали из ада и возвращались к кукольнику.
% Вот. По-моему, это, ну, такая, знаете, история. Говорит, мы, говорит, мы, как
% это, дочь его описывает, мы, говорит, боялись страшно. Не знаю, может, поэтому
% мне сколько там двое, по-моему, дочерей покончили с самоубийством у Макса. Ну, я
% не знаю. Ну, как-то, понимаете, то есть вот такой панк. Романтик бунтает.
% Понимаете? Вот. Я как бы прошу вас, ну, поживее воспринимать эту засаленную
% фигуру. Она сложная, она двусмысленная, она никаким образом не какая-то такая
% простая. А, Маркс, не знаю. Не-не-не. Ее нужно погружаться, в тему отчуждения
% надо погружаться.
Активного движение в сторону материализма возникло не столько как борьба с атеизмом, хотя Маркс ярко боролся с религией, сколько как общий тренд. Истории материализма и материалистические трактовки религии были популярны в XIX веке.
 Философия Маркса начинается с политической борьбы в условиях буржуазного общества. В его ранних работах, например, «Экономико-философских рукописях» 1844 года, он замечает простую вещь: собственник подавляет правового эксперта. Маркс, будучи юридически образованным, видит, что капитализм ставит под сомнение старые формы легитимности, где прибыль становится главным критерием, а право отодвигается на второй план. В этом проявляется ограниченность Маркса --- он сам остается в рамках буржуазного мышления и считает экономику ключевой сферой жизни, над которой строится вся культура, политика, наука и религия.

Такой подход --- экономический базис и надстройка --- показывает, что для Маркса главное --- экономика, а остальное вторично. Это ограничение, но в то же время подтверждает, насколько глубоко буржуазное мировоззрение проникает в сознание человека. Материалистический тезис «бытие определяет сознание» нужно понимать именно как «существование определяет сознание», учитывая сложности перевода с английского. Если человек весь день физически работает, вряд ли у него будет время или силы на философию --- это тривиальный вывод.

Для Маркса сознание просто перестраивается под условия существования, и человек не может выйти за их рамки. Базовый принцип марксистского анализа --- производительные силы определяют отношения между людьми. История --- это закономерная смена общественно-экономических формаций, основанная на борьбе классов, как их называли Маркс и Энгельс.

Маркс критикует капитализм как последнюю форму эксплуатации, подавляющей личность. Его концепция формаций устарела, но он остается важным философом. Главное --- концепция отчуждения: продукт труда и сам труд отчуждаются от человека при наёмном труде и автоматизации. Труд --- основа человеческого бытия, он сделал человека человеком, но в капитализме он превращается в отчужденную реальность.

Отчуждение проявляется не только в фабричном труде. Например, творческий продукт, как книга или статья, перестает принадлежать автору, приобретая смыслы, которые могут ему не нравиться. Это неизбежный процесс, но ключевая категория критики Маркса.

Что не классично в Марксе --- он видел человека как совокупность социальных отношений, без личного сознания, только общественное. Это разрушает редукционизм, потому что целое отражается в частном, и человек --- элемент общества. Эта холистичность --- важный кризис мышления нового времени.



\paragraph{Шопенгауэр}
% Его философия называется волюнтаризм от слова воля. Волюнтаризм. По
% Шопенгауэлу, бытие следует соотносить вовсе не с разумом. Ну, то есть, ну, не
% разум, основа всего существующего, бытие и тому подобное. Это так, как это
% рассказывает классическая мысль. А нужно рассматривать основой всего
% существующего некую иррациональную силу чистой воли. от слова хочу. Воля не вот
% в узком смысле слова надо. Не-не-не. Вы понимаете, это как бы в немецком смысле
% слова lust. Вспоминаете, если вы знаете Рамштайн, ich habe kein Lust. Я не имею
% воли, я не имею желания означает. Я не имею воли действовать, я не имею желания
% действовать. Ich habe kein Lust. Ich habe это самое. У меня нет желания. Или
% will. Два слова для, ну, как минимум, два слова для слова желания или воля, или,
% ну, в общем, как бы в немецком языке это более сложное понятие, чем в русском.
% То есть воление означает желать. Так вот, основа жизни это такая вот
% иррациональная сила чистого, безличного желания. Когда вам говорят слить с
% потоком, будьте осторожнее. Уточните, а точно не с потоком желания. Вот,
% смотрите, эту иррациональную силу он и назвал философией в ней. Философию,
% описывающую эту иррациональную силу, Шопенгауэр назвал философией жизни. И,
% пожалуйста, это понятие тоже подчеркните. Философия жизни это многие авторы,
% многие философы, родоначальник Шопенгауэр. И они говорят о жизни как о той
% иррациональной силе, которая нас захватывает. Она подчиняет нас себе. Кому-то от
% этого, ну, временами хорошо, но в основном это вообще-то достаточно такая злая
% сила. Во всяком случае, Шопенгауэр это точно. Так, смотрите, он, кстати,
% опирается в основе своей на Канта и на понятие вещь в себе. Но он нарушает
% запрет Канта и дает объяснение этой вещи себе. Говорит, это воля, мир как воля,
% мир как желание. Вот она, вещь в себе. Некая надсубъективная, не чья-то
% конкретная, а просто космическая, космическая, бессознательная воля. Он говорит
% об этом так, воля волит. То есть не кто-то что-то желает, а воля. Желание
% желает, воля волит. Эта сила заставляет всё существующее чего-то хотеть. Хотеть
% существовать, хотеть жить, хотеть развиваться, хотеть размножаться, хотеть
% достигать целей, хотеть развивать науку, хотеть что-то там выстраивать, хотеть
% политических изменений. Он говорит, это всё воля волит. И это основа жизни как
% страдания. Почему? Потому что чувство неудовлетворения, это и есть основа воли.
% Ведь желание двигается неудовлетворением. И он говорит, что это же источник
% воли, неудовлетворения. Ну разве есть, говорит, удовлетворение, а его нет?
% Удовлетворение одаривает нас в итоге только скукой. Вы достигаете желаемого и
% скука. И вы снова что-то хотите. Вы достигаете и снова достиг, а нет, скука. То
% есть мы постоянно оказываемся в разных потоках вволения, желания, не получая при
% этом ни счастья, ни успокоения. Вся природа, вся человеческая культура – это
% различные проявления воли. В науке это появляется в полной мере. Поэтому брать
% научный образ мысли в качестве какого-то инструмента достижения счастья – худшее
% из-за блуждений. Чувствуете, да? Это вот уже такой антисцентизм в определенном
% смысле пошел. То есть еще раз. Брать научный подход к реальности в качестве
% инструмента достижения счастья – ха-ха, говорит Шопенгауэр, это худшее из-за
% блуждений. Тогда вопрос, а что все-таки нам делать-то, да? Что делать-то по
% Шопенгауэруму? Он и говорит, что надо раствориться в такой форме деятельности,
% которой удается потерять индивидуацию. То есть вот свою как бы отдельность
% удается потерять. Свою собственную субъектность удается потерять. Вот это станет
% трендом мыслей на весь XX век, как потерять свою субъектность. Как бы нам
% раствориться? Но надо сказать, что сам Шопенгауэр это не выдумал. Он взял это из
% буддизма. Он активно интересовался буддизмом, вообще индуистской философией и
% буддизмом в частности. Но он ее трактовал по-своему. И вот Шопенгауэр видит
% возможность раствориться в целом, но при этом как бы оставаться собой только в
% искусстве. Когда человек говорит, он творит, например, поэзию, да, или там
% музыку сочиняет, или там картину пишет, он как бы забывает себя, да? Он активен,
% но он забывает себя. Ну, такой вот как бы выход видит Шопенгауэр. Видимо, нам
% надо всем... Ну, во всяком случае, может, не помешает заниматься такой вот
% формой активности. 

Шопенгауэр называет свою философию волюнтаризмом --- от слова "воля". В отличие от классической философии, которая делает акцент на разуме как основании бытия, он утверждает, что в основе всего лежит иррациональная сила --- чистая воля. Не воля в смысле "надо", а скорее в смысле желания, как в немецком "Lust" или "Will", то есть стремления, влечения. Это не личная воля, а безличная, космическая, бессознательная сила, которая заставляет всё существующее чего-то хотеть: жить, размножаться, развиваться, строить, менять.

Шопенгауэр связывает свою мысль с Кантом, но нарушает его запрет на постижение "вещи в себе" и утверждает: вещь в себе --- это воля. Воля просто волит. Она не принадлежит никому конкретно --- она просто есть и пронизывает всё. Это она толкает к действию, развитию, к стремлению. Однако результатом этого непрерывного стремления оказывается не счастье, а страдание. Желание рождается из неудовлетворённости, и даже когда достигаешь желаемого, приходит скука, и появляется новое желание. Так возникает вечный круг страдания, в котором мы пребываем, не находя покоя.

Все проявления природы, культуры, науки --- это выражения воли. Но сама наука, с её рациональностью, по Шопенгауэру, не может быть инструментом счастья. Он резко критикует веру в науку как путь к благополучию --- это, по его словам, худшее заблуждение. Счастья она не приносит, потому что не решает проблему воли.

Что же тогда делать? Шопенгауэр предлагает раствориться в деятельности, которая позволяет утратить чувство собственной отдельности, субъектности. Это станет важным мотивом философии XX века --- поиск способов забыть о себе. Он берёт эту идею из буддизма, которым глубоко интересовался, хотя интерпретировал его по-своему. Одним из путей он считает искусство. В творчестве --- поэзии, музыке, живописи --- человек способен забыть себя, быть активным и при этом потерять ощущение "я". Это и есть способ временно вырваться из власти воли.

\paragraph{Кьеркегор}
% Дальше. Это тоже философия. В смысле, еще философия жизни,
% она связана, например, с другим, очень интересным, ну, рано очень умершим
% философом Кьерке Гором. Он критикует рационализм за то, что... Рационализм,
% помните, да, вот эта вот ориентация на разум, культ разума. Он критикует эту
% позицию за то, что она заставляет человека видеть только то, что выражено в
% общем, в понятиях, в логических умозаключениях, в каких-то строгих научных
% положениях. И, говорит, сосредоточившись на этих вот всеобщих характеристиках,
% философия потеряла реального человека. А реальный человек, конкретный человек,
% он может говорить только о жизни. Не... Вот это всеобщая-то чистая абстракция. А
% вот жизнь, она конкретная. А в жизни что важнее всего? Переживание. Жизнь, она
% переживается. То есть жить означает переживать. Проживать, переживать – это
% накоренные слова, только разные приставки. И они не меняют суть содержания вот
% этого слова. Философия, говорит, он не имеет права отказываться от переживания,
% как истока всех тем, всех проблем. Поэтому анализ сущности человека должен
% уступить анализу места его существования, экзистенции. То есть Гиркегор –
% основатель экзистенциализма. 

Он критиковал рационализм --- культ разума, ориентированный на логические умозаключения и строго научные понятия. По его мнению, такое мышление заставляет видеть лишь абстрактное и общее, теряя из виду конкретного, живого человека.

Кьеркегор утверждал, что философия, сосредоточившись на всеобщем, упустила главное --- саму жизнь. А жизнь не абстрактна, она конкретна, и суть её --- в переживании. Жить --- значит переживать, проживать, ощущать, и в этих словах не просто общая корень, но и общий смысл. Философия не имеет права игнорировать переживание как источник всех вопросов и тем.

Поэтому, по Кьеркегору, философия должна сместить акцент с анализа <<сущности человека>> на анализ его <<существования>>, то есть экзистенции. Именно Кьеркегора считают основателем экзистенциализма.

\paragraph{Ницше}
% И я бы хотела сосредоточиться на все-
% таки ключевом для конца XIX века философии Ницше. Он очень сложный. Я даже не
% берусь вот как бы вам... Не берусь даже надеяться на то, что я сейчас опишу всю
% философию Ницше, так как... Ну, чтобы вы ее поняли. Единственная моя задача –
% это вот убрать... Извините за вульгаризм, нафиг. Тот образ Ницше-философа,
% который наверняка вы встретите в интернете, в самых расхожих вот таких вот... И
% даже в некоторых учебниках по философии. То есть, когда Ницше просто
% антиморалист, антихристианин, анти, скажем так, анти... Все разумное, вечное и
% доброе. То есть, такой подлец и практически фашист. Глупости, большие глупости.
% Ницше, конечно, нигилист. Вот это, давайте скажем точно. Нигилист от слова такой
% борец. Ну, такой, знаете, нет для меня никаких авторитетов. Я нигилист. Но, а
% кто тогда не нигилист? Тогда все были нигилисты, антихристиане, антиморалисты.
% Ну, помните, я говорила вам, какие книги были бестселлерами. Тогда это модно,
% тогда это нормально. Но на фоне вот этой моды Ницше уделяется как хороший
% философ. Он исходит из уверенности, что слабым человеком, который, ну, не может
% противостоять более сильным, там, социальным или физически даже, всегда владеет
% стыд и желание мести. Как бы, это, вот эта воображаемая месть, то есть всегда
% этот человек слабый, всегда живет воображаемой местью, говорит, Ницше, он будет
% только понять, да, рессентимент. Таким образом, говорит он, и формируется мораль
% рабов. Мораль рабов, то есть мораль сильного, это я действую в своей силу.
% Неважно, какой, да, там, социальный или там, где-то. Вот я действую в своей силу
% или ментальной силы. А мораль рабов, это не иметь возможности действовать из
% сил, потому что нет силы, но при этом жить вот этой вот моралью рабов, да, из
% сентиментов. И вот он говорит, история человечества в одно прекрасное или
% непрекрасное время переворачивается, и мораль рабов становится лидирующей. И он
% это связывает с христианством. Он говорит, что мораль рабов – это христианство.
% То есть это те самые слабые, те самые хилые и убогие, которые решили
% противостоять, как бы сказать, противостоять сильным. Но как они не могут же
% напрямую противостоять, наши слабые, да, поэтому они выдумали некую моральную
% философию, моральные идеи, в которых, например, милосердие – это хорошо, а не
% милосердие – плохо, в которых, например, значит, ну такое, ну скажем, как
% сказать-то, ну подставь вот там щеку, да, вот это вот. Типа это ценно, а вот
% мораль сильных не ценна. Сразу же могу сказать. Конечно, Ницше очень плохо знал
% историю христианства, и, конечно, я думаю, что те первые христиане, которые
% были, а, мученики, вторые были солдатами, в-третьих были строителями новой
% эпохи, они, конечно, посмеялись бы, сказали, что мы-то, это мы-то, да, слабые,
% да? А, это мы слабые, ну ладно. Вот, они бы, конечно, посмеялись, и я вместе с
% ними. Но, понимаете, Анидши-то смотрел на вокруг себя христианство, он ведь не
% на то, он, на самом деле, туда не заглядывал, в этом смысле он слаб, как
% мыслитель, но он силён как мыслитель, анализируя вот то христианство, которое
% вокруг него, а он знал, о чём говорил, он из пастырской среды, у него семья
% была, отец был пастыр. И вот, то есть, понимаете, он критикует, по сути дела, то
% христианство, которое вокруг него, а уж говорить про то, что вокруг него было
% христианство, реально очень сильно отличающееся от того, что в начале эпохи, в
% начале христианской эпохи, ну это, как говорится, очевидно, да? Поэтому Ницше,
% он, как говорится, не останавливается на критике вот этого христианства, ведёт
% дальше и настаивает. Человек, который представляет собой прирождённого
% господина, так вот, если только такой человек обладает состраданием, то
% сострадание имеет цену. То есть, смотрите, если ты, будучи сильным, можешь
% противостоять любому, но просто не тыкаешь его палкой в глаз, не потому что ты,
% что называется, ну, как сказать, не можешь, а потому что ты не хочешь этого
% делать, потому что тебе жаль этого человека. Вот только это сострадание имеет
% силу. То есть не сострадание слабых, а сострадание сильных. Поэтому что? Только,
% ну, как сказать, ну, давайте будем сильными, предлагает он человечеству. Он
% негодует, если под милосердием прячется всего лишь обычная человеческая
% слабость. Разве это не, что называется, это даже уже не критика христианства, а
% это критика фундаментальная? Говорит, если уж ты, говорится, силен, ты можешь
% себе позволить быть милосердным. Если ты слаб, твои милосердия ничего не значат.
% Он настаивает. Если вы хотите быть антихристианами, пожалуйста, будьте, но
% будьте последовательны. Станьте кардинально иными. Уничтожьте мораль как
% таковую. Он ставит эксперимент. А давайте уничтожим мораль как таковую, если она
% основывается на слабости. Я, говорит, вокруг себя не вижу морали сильных. Я не
% вижу сострадания сильных. Я вижу вокруг себя сострадания слабых, которые
% кучкуются в кучки и начинают давить друг друга, называя это христианской
% добродетельной. Он просто не позволяет быть христианином или моральной личностью
% в силу некой привычки или в силу традиции. Или в силу какой-то такой, понимаете,
% ну, типа так принятым. Я с людьми, я с кучкой. Вот это он считает, я знаю, что я
% задержалась по времени, но я онису закончу. А именно это он видел вокруг себя.
% Философ негодует. Среди тех, кто, например, нынче живет в Германии, живут, ну,
% то есть нынче в Германии, живут в свободе... Сейчас я поставлю коротенькую
% цитату, как бы сократить. То есть большинство тех, кто сегодня проникнув в
% свободом мысли самых разнообразных сортов происхождения, прежде всего, среди них
% много таких, у которых трудолюбие, уничтожило все религиозные инстинкты. Они уже
% не знают, для чего нужны религии. Они только с каким-то тупым удивлением как бы
% регистрируют их наличность в мире. Они вовсе не враги религиозных обрядов. Нет,
% они будут выполнять религиозные обряды полностью, если требуется в известных
% случаях. Ну, там много-много больших такие цитаты, да. Вот. Но это равнодушие в
% религии и есть смерть христианства. Вот этот труп христианства, который он
% наблюдал вокруг себя, он и активно осуждает. И он выдвигает свой знаменитый
% тезис. Бог умер. Да? Ну, а только почитайте, пожалуйста, что имеется в виду. Бог
% умер, потому что это мы его убили. Этот мир, говорит он, не заслуживает Бога. Мы
% слабые. Мы используем религию для оправдания собственной слабости. Так пошла
% тогда нафиг такая религия, говорит он. Ну, вот вам и антихристианство. С той же
% требовательностью и непримиримостью он ополчается на моральные устои, которые не
% на религии основаны, а на идее общественного прогресса. Да? Он говорит, ваш
% общественный прогресс – это тоже слабость. Вот он, его цитата. Я ее здесь не
% писала. А нет, кто исследует совесть нынешнего европейца? Та-та-та-та.
% Почитайте. Общественный договор, о котором говорили Лок, о котором говорил Гокс
% и многие другие мыслители, вот для Ницше это слабость, но только обряженные в
% одежды социальной морали. Так вот, Ницше предпринимает безапелляционное
% разоблачение всего того, что в европейской культуре считается разумным, истинным
% по умолчанию или в силу модных положений. Он говорит о том, что… А давайте все
% мы, все идеи доведем до конца. Вот, например, говорит, тогда вошла в моду
% положение, в тот период вошло положение, в моду положение Дарвина о том, что
% человек… Ну, это не положение Дарвина, положение дарвинизма, его сторонников.
% Что человек – это, ну, как бы животное, да? Человек – это обезьяна. Еще раз,
% Дарвин этого не говорил, но тем не менее, вот, и там, его, как бы, последователи
% это активно развивали. Да, давайте, говорит Ницше, доведем до логического конца
% эту установку, не будем останавливаться на полпути. Если человек только
% животное, то речь может идти только о потребностях и реакциях, правильно? Ну,
% давайте говорить только о потребностях и реакциях. Но, говорит, стоп. А почему
% возбуждение сильного, например, реакции хищника, должны быть в каком-то там…
% Вернее, мы должны исключить, ну, как бы, как нечто такое неправильное. А вот
% реакции, например, жертвы хищника должны оценивать как что-то там ценное, как
% моральное. Ну, вот, скажем так, миролюбие. Миролюбие – это, говорит, если мы
% животные, то миролюбие – это будет мировоззрение или, так скажем, реакция кого?
% Зайчика. А миролюбие не будет реакцией волка. А почему, говорит, мы должны быть
% животным, травоядным? А давайте, говорит, будем волками тогда. Почему нет? И
% тогда давайте все-таки право сильного и будет рулить. Мы должны перестать
% притеснять действительно сильные проявления животного начала. Ну, если человек –
% животный, да? Воля к власти – только она источник и основа исцеления от
% дыхлевшего человечества. Воля к власти – единственное, что имеет ценность,
% поскольку в условиях прогнивших в своих основах рациональности – это
% единственное, что можно назвать настоящим. Вот тут можно очень много говорить.
% Но я закончу разговор о Ницше вот какой мыслью. Ницше сделал себя полем битвы
% вот этих вот идей. Он буквально взрывал свой мозг вот этими доведенными до конца
% принципами, понимаете? И этим он великий философ. Не каждый так решится.
% Проговорить до конца все принципы, которые являются общими местами. Общее место,
% помните, да? Ну, как бы все разделяют. Ну, все говорят, действительно, ну,
% человек – это животное. Ну, смотрите, не так, как это говорили святые отцы, это
% животное, призванное стать Богом, а просто животное. Со своими просто реакциями,
% просто импульсами, имеется в виду этими инстинктами, просто там какими-то
% гормональными обстоятельствами. Ну, как, что животное? Так давайте будем,
% понятно, какими животными. А? По-моему, сильно. Почему я говорю, он сделал себе
% полем битвы? Потому что он сошел с ума. И последним, так скажем, вот этим вот
% актом, когда его, значит, сознание буквально разорвало на части, на части
% разорвало, да, на два, на два, на две части. Это было утро 3 января 1989 года.
% Ницше вышел из своей наемной квартиры в городе Турини, и вдруг он видит
% изводчика, который сбивает лошадь, да? Это очень известная такая тема, она и в
% фильмах встречается, и в разного рода. Ну, в общем, это такая известная тема. И
% вот Ницше с криком бросается через площадь, обнимает дрожащую лошадь за шею,
% стараясь защитить её. Это я к чему это говорю? Почему я это рассказываю? Потому
% что, ну, не берите в голову образ Ницше, как человека, который просто как бы
% негодует по поводу милосердия, потому что он сам такой немилосердный. О, нет, в
% нём милосердие живёт ещё, как живёт. Он старается защитить эту лошадь, потом он
% теряет сознание, и после этого оказывается уже в больнице. Его сознание
% разделилось на два, на такую диаду. Он называл себя Дионисом, это то яростное,
% да, дионисийское начало. Он в авторождённом, то есть это имя Диониса. И
% распятым, догадайтесь о комнате. Умер 25 августа 900 года. И всё, что вы знаете
% про то, что Ницше – это вот основоположник нацизма, это, пожалуйста, ну вот не
% надо этого, да. Во-первых, он умер за долгами нацизма, не мог предсказать даже
% там итальянский фашизм, не говоря уж про, собственно, немецкий. Во-вторых, он
% очень критично относился к соотечественникам. Он мало кого любил, но немцев он
% не любил больше всех. И уж точно он бы не одобрил. А во-вторых, там была его
% сестра, которая прибрала все рукописи к рукам. А когда всё-таки, ну она-то
% прожила гораздо дольше, когда всё-таки власти пришли нацисты, я уж не знаю, то
% ли спасая свою жизнь, то ли пытаясь там, ну как говорится, на волне подняться,
% она вот их ей, собственно говоря, пристроила к этому дискурсу новому. Поэтому,
% ну что называется, если уж кого-то обвинять, то в том, что какие-то идеи Ницше
% попали в обойму нацистов, немецких, да, фашистов, это всё-таки сестру его надо.

Ницше действительно нигилист, но не единственный --- в его время таких было много. Это была эпоха нигилизма и антиклерикализма. На этом фоне он выделяется тем, что предлагает целостную философскую позицию. Он считает, что слабым человеком движут стыд и желание мести, и именно эти чувства лежат в основе того, что он называет «моралью рабов». В отличие от морали сильных --- действия из собственной силы --- мораль рабов рождается из бессилия, она питается рессентиментом, завистью и обидой.

По Ницше, в истории человечества наступает момент, когда мораль рабов побеждает --- и он связывает это с христианством. С его точки зрения, христианская мораль была изобретена слабыми, чтобы противопоставить себя сильным. Они не могли победить силой, поэтому объявили смирение, милосердие и отказ от насилия высшими ценностями. Сильная, деятельная мораль обесценивается.

Конечно, Ницше плохо знал историю раннего христианства. Те, кто умирал за веру, кто строил новую эпоху, вряд ли были слабыми. Он не интересовался ранними христианами, а критиковал то христианство, которое видел вокруг себя --- лицемерное, мёртвое, обрядовое. Он сам происходил из пасторской семьи, хорошо знал эту среду и оттого особенно яростно её отвергал.

Ницше утверждает: только сострадание сильного имеет ценность. Если ты силён и мог бы уничтожить, но выбираешь не делать этого --- вот тогда твоё милосердие что-то значит. Сострадание слабого, по его мнению, всего лишь прикрытие бессилия. Он требует радикальности: если уж быть антихристианином --- то быть им до конца, уничтожить мораль, основанную на слабости, и строить новую этику.

Ницше говорит: вокруг него нет морали сильных, есть только мораль слабых, прикрывающихся христианскими добродетелями. Он презирает привычку быть «нравственным» по традиции или из страха быть отвергнутым. Он требует честности --- настоящей, беспощадной честности в философии и в жизни.

Он негодует по поводу того, что в обществе процветает равнодушие к религии. Люди утратили религиозное чувство, но продолжают соблюдать обряды просто из привычки. Такое равнодушие и есть, по Ницше, смерть христианства. Отсюда и его знаменитое: «Бог умер». Но умер он не сам --- это мы его убили. Мы сделали религию удобным прикрытием собственной слабости. Такая религия, говорит Ницше, должна быть уничтожена.

Но и прогресс, основанный на светской морали, он также отвергает. Для него идея общественного договора --- не более чем форма прикрытия той же слабости. Он разоблачает всё, что в европейской культуре считается самоочевидным. Если человек --- просто животное, как утверждает популярный дарвинизм его времени, --- то давайте доведём эту идею до конца: почему мы должны ценить реакции жертвы (миролюбие, покорность), а не реакции хищника (сила, напор)? Почему бы не быть волком, а не зайцем?

Он предлагает волю к власти как основу новой силы, новой морали. Только она, по его мнению, способна исцелить разлагающееся человечество. Рациональность больше не работает, старые ценности мертвы. Нужна новая философия --- философия силы, действия, преодоления.

Ницше сам стал полем битвы этих идей. Он буквально разрывал собственное сознание, доводя до конца любые принятые в культуре установки. Не каждый способен на такую честность. Он не соглашался с тем, что человек --- просто животное, если только это животное не стремится к божественному. Он задавался вопросом: если человек всего лишь животное, то каким животным нам быть?

\paragraph{Фрейд}
% Значит, смотрите, философская мыс XIX века, с одной стороны, формирует
% такие радикально нигилистические, критические настроения, а с другой стороны,
% даёт интерпретацию вот этому стихийному чувству протеста как такового, да.
% Поэтому, вот как это сказать-то, какие-то философы сосредоточились, например, на
% политической активности, какие-то сосредоточились на этой активности внутренней,
% какие-то на мировоззренческой, а были философы, которые сосредоточились в кои-то
% веке на психической жизни и там сопротивлении, которое нам же оказывает,
% собственно, психика. Да? Ну вот мы что-то захотели, а не можем. Или наоборот, мы
% вроде этого не хотим, а делаем. Понимаете, да? То есть оказывается, вот эта вот
% стихия, это неподконтрольная сила, она живёт ещё и в человеке внутри него. Это
% называется психоанализ. На самом деле
% Фрейд вообще-то хотел создать физику психики, и он говорил о том, что психика –
% это машина, да? То есть в духе вот человек – машина, по сути дела, но уже не
% часы, а тепловой двигатель. То есть она накопит, а потом выдаёт работу. И эта
% работа абсолютно автономна от нашей воли. И поэтому очень важно понимать, что
% психоанализ – это было всё-таки, с одной стороны, как бы, ну, вроде, это же
% другое, это же не так философия, но он очень в духе того времени. Далее Юнг
% развивает сферу бессознательного, ну, не только Юнг, не только, как говорится, в
% таком, ну, там очень много последователей у Фрейда, но вы чувствуете,
% открывается целая область знания. Целая область знания. Она вообще-то называется
% наукой, по этому поводу есть сомнения, я бы назвала это искусством, но я не
% навязываю эту точку зрения, вполне возможно называть это наукой. Всяка, в общем,
% в этом случае Фрейд очень хотел, чтобы это была наукой. И сегодня действительно,
% ну, психология, как минимум, она существует в научном дискурсе совершенно как бы
% спокойно. Но психоанализ, вот тут у меня, конечно, с вами, насчет того, что это
% наука. Это прекрасное искусство, и оно может кому-то очень нравится, кому-то
% может не нравится. Ну, чувствуете, да? Скорее-то, ну, опять, это как бы, это,
% опять же, личное мнение, да? Но чувствуете, в целом, я сейчас обобщаю этот наш
% разговор, вот, в целом, искусство того периода, оно называется модернизмом. Вот
% помните, был модерн, а вот модернизм. И опять же, не только потому, что я
% искусствовед, а в силу того, что это очень показательно, искусство всегда очень
% показывает, что происходит с обществом, что происходит с этим миром. Вот я вам
% покажу примеры модернизма и объясню, как это связано с наукой. То есть,
% понимаете, искусство предчувствует то, что будет потом в науке и, в конце
% концов, в обществе, в каждой голове. Так вот, модернизм, основные характеристики
% его какие? Это отказ в той или иной мере от образа, ну, то есть, реалистичного
% восприятия мира. Это стремление к отвлеченным, или так, по-другому сказать, к
% абстрактным образом. Они передают не столько внешнюю суть предмета восприятия,
% сколько его преломление восприятия. Чувствуете, как вот этот цвет у Гёте. Цвет –
% это не что-то объективное, а это скорее результат нашей способности видеть. Свет
% и тень. Так и тут. Вот, например, импрессионизм от слова впечатление. В этих вот
% мазках, которые сливаются в фигуры, их можно увидеть, а можно не увидеть. Я как
% сейчас помню, мы сидим с моим спутником в Эрмитаже напротив картины Клода Мане.
% Ну, у него много подобных картин, там, значит, пруд изображен. Я искусствовед. Я
% изучала и тысячи раз смотрела или в репродукциях, или в подлинниках
% импрессионистов. Я была, знаете, таким очень самотвёрстным искусствоведом, очень
% люблю до сих пор эту сферу своей деятельности. Вот. И я не видела того, что
% вдруг увидел другой человек. Он говорит, ой, смотри, я там ещё один человек где.
% А вон там женщина, смотри, и вдруг я понимаю, что у него из мазков сложилась
% фигура там, что женщина. Да, ты что, сверлила? А, смотри, ты там женщина. А там,
% смотри, лодка. И вот, то есть, и так несколько раз. Я была в шоке. То есть,
% действительно, так это работает, понимаете? Какое-то сознание вдруг в реальности
% вычленило, ну, в данном случае, в картине, вычленило какой-то образ. Он говорит,
% на, смотри, и вот я вижу. Это вот импрессионист. Чувствуете, да? Вроде бы, как
% сказать, всё есть уже в картине, но на это надо ещё увидеть. Или вот внутреннее
% состояние, которое тоже рождается в этой эпохе экспрессии. Ну, понимаете, не без
% этого. Постимпрессионистический Ван Гог или его крайние такие проявления, как
% экспрессионизм. А тут просто обращаю внимание на то, какой техникой это
% написано. Особенно Ван Гога хорошо смотреть в подлинниках. У нас, слава тебе,
% Господи, в Эрмитаже много Ван Гога прекрасная вещь. Вы просто проникнетесь этой
% энергетикой, этой экспрессией, которая рождается. Или вот вам экспрессионизм,
% да? Мунг знаменитый. Понимаете, да, о чём я говорю? А вот сейчас я вам выхожу
% ближе к абстрагированию. Конечно же, его начинают разного рода кубисты, разного
% рода, как сказать, такие вот художники, разрывающие аковреализма, так это
% назовём. Что здесь имеется в виду? Смотрите, вот, например, этот самый, Господи,
% я сижу и вспоминаю автора. Ну, я уверена, вы знаете, скажите, самый известный
% кубист. Пикассо, слава тебе, Господи, спорта. Так вот, это, значит, портрет,
% один из портретов Пикассо. И все долгое время удивлялись, почему люди,
% изображённые на этих портретах, себя узнают. Ведь Пикассо всё сделал для того,
% чтобы реалистичность была разорвана, рассыпана на части. Почему же? И не только
% сами они себя узнают, но и те, кто их знает, тоже узнают. Да, это же он. Вот. А
% как это добивается Пикассо? А таким образом, он сохранял абсолютно как бы
% неизменными базовые пропорциональные позиции лица. Ну, например, там расстояние
% от носа до губ, расстояние между глаз, там расстояние, там, ну, понимаете, такие
% вот базовые пропорциональности. математические он сохранял, и, оказывается,
% этого хватает. То есть абстрактные категории, какие-то математические пропорции
% могут, оказывается, тоже являться фактором восприятия реальности. Понимаете? И
% вот я как бы перевожу ваше внимание, к чему я это всё рассказываю, к принципам
% абстрагирования. Ну, вот уж совсем абстрактное искусство, супрематизм Малевича.
% Пожалуйста, не рассказывайте мне про то, что это плохое искусство, это хорошее
% искусство. Надо просто его понимать, и надо вот так вот встать перед красным или
% чёрным квадратом и объяснить, что это такое. Мне это удавалось не единственное.
% Вообще-то, каждый раз мне удавалось показать, что это хороший художник. Ну, вот,
% пожалуйста, ещё Кандинский, или это Миро, не помню. Беспредметность, да?
% Абстракция. И вот давайте теперь возвращаемся к науке. И мой тезис, который, я
% надеюсь, вы мне вернёте на экзамене, усиление роли абстрагирования в научном
% знании. Чувствуете? 

Философия XIX века, с одной стороны, породила радикальные критические, нигилистические настроения, а с другой --- дала осмысление самому чувству протеста. Одни философы сосредотачивались на политической активности, другие --- на внутреннем сопротивлении или мировоззренческом конфликте. Впервые внимание обратили на психическую жизнь, на то, как психика сама по себе сопротивляется нашим желаниям: что-то хотим, но не можем; или не хотим, а делаем. Оказалось, что иррациональная стихия существует и внутри человека --- это и стало предметом психоанализа.

Фрейд хотел построить «физику психики», где психика представлялась как машина --- не часы, а тепловой двигатель: она накапливает энергию и выдает работу независимо от нашей воли. Несмотря на то что психоанализ не вписывался в классическую философию, он соответствовал духу времени. Юнг и другие продолжили разрабатывать понятие бессознательного, и в итоге сформировалась целая область знания. Можно спорить, наука это или искусство, но Фрейд видел в этом именно науку.

\section{Кризис понятий классического естествознания}
% Вначале в искусстве всё, мы больше реальностью как бы не
% занимаемся, мы занимаемся абстракциями. А потом это проявилось в науке. Что
% такое абстрагирование? Это поиск тождества многообразия, поиск того, чем это
% многообразие объединено. И, конечно же, математика на самом деле это будет самый
% первый кандидат на процесс абстрагирования. Почему? Потому что вот перед нами
% пять предметов, которые вообще ничем не связаны. Ну, сами придумайте, какие. Ну,
% вообще ничем не связаны. Но они связаны. Чем? Какой абстракцией можно описать
% их, ну, как сказать, как нечто целое, как нечто схожее? Конечно, математика. Их
% пять. Понимаете? Мы их можем посчитать. И вот математика это одна из высших форм
% абстрагирования. И одновременно самая самая нам, так сказать, привычная. Да?
% Числа. Они нечто общее и абстрактное для самых различных объектов. Люди оценили
% способность чисел упорядочивания их восприятия, мира, отвлекаясь от
% индивидуальных особенностей предметов, ну, с самого начала человеческой
% культуры. И вот я вам Гейзенберга, которого, ну, работу, в смысле, Гейзенберг
% много писал философских работ, вот цитаты за мной из работ. С точки зрения
% современной математики отдельные числа не так важны, как сама операция счета.
% Освоив счет, люди сделали решающий шаг в сферу абстракции. Абсолютно, да, что
% называется, ну, такая понятная мысль и справедливая. Но при этом, смотрите,
% Гейзенберг же говорит, всегда можно убедиться в том, что математические формулы
% правильные, но не в том, существуют ли в действительности объекты, к которым они
% могли бы относиться. Вот как. То есть, смотрите, математика, да, она может быть,
% как говорится, она хорошо описывает реальность, но не факт, что нашу. Не факт,
% что ту, которая существует. Вот. И поэтому существуют как бы, ну, вот, плюсы и
% минусы вот этого абстрагирования. С одной стороны, мы можем очень многое описать
% и объяснить, а с другой стороны, мы рано или поздно столкнемся с ситуацией,
% когда мы не сможем убедительно сказать самим себе, что мы все еще описываем
% реальность. Дальше. Отмечаем еще один важный момент. Абстрагирование связано с
% конструирующим мышлением. с конструируем. То есть мы как бы конструируем
% реальность, понимаете? Вот. И тенденции вот этой вот абстракции в искусстве, как
% мы понимаем, они перешли границу, за которой вот искусство уже ну, как бы
% потеряло связь с действительностью, да? Ну, вот я вам показала последнюю работу,
% покажу еще раз. Вы можете без названия определить, что это? Ну, конечно, никогда
% вы не определите. Все, что угодно, да? Вот. И в данном случае как бы связь с
% реальностью, она потеряна, она осознательным художником потеряна, но к этому
% шел. Надо было сделать. Ну, ему так хотелось. А вот в науке-то мы бы не очень
% хотели, да, чтобы наука разорвала с реальностью, чтобы полностью ушла от нее.
% Вот поэтому, смотрите, надо понимать, надо понимать, раз искусство предчувствует
% такую возможность, то мы должны тоже, как ученые, должны понимать, что наши
% конструкты, идеальные, абстрактные конструкты, могут потерять связь с
% реальностью. Как бы всегда чувствовать эту опасность. Ну и вот, наконец-то, мы
% пришли к науке нового, к новому типу, к неклассической, но опять же, вначале,
% как сказать, пройдемся по пути, вот скажем, кризиса в науке. Мы до этого
% говорили о кризисе мировоззрения, о появлении неклассической философии,
% философии жизни, а теперь вот о появлении неклассической науки, как бы опять
% немножечко отступаем в 19 век, даже в начало 19 века. Так вот, середина 19 века
% уж точно, это как раз процесс, период кризиса классической науки. Конец 19 века
% и начало 19 века это уже не классическая наука. Пожалуйста, не путайте, да? Вот,
% классическая наука это новое время, середина 19 века это кризис классической
% науки. Конец 19 и начало 20 это новый этап вне классической науки. Это схема,
% это, что называется, вот та самая абстракция, но она еще имеет какую-то связь с
% наукой, с реальностью, поэтому, пожалуйста, не пренебрегайте этой схемой. Уже в
% середине 19 века начался процесс осознания пределов в классической физике.
% Процесс был болезненный, поскольку вера в ньютоновскую физику была огромна. Но,
% тем не менее, смотрите, я вам уже рассказывала, что ньютоновская физика очень
% многих вещей просто не описывала, очень многих явлений. Например, она не
% описывала электричество и магнетизм. Правильно? Правильно. Но люди-то хотели ее
% описывать, тем более, что они искали эту пневму. И надо было работать с этим вот
% феноменом, феноменом электричества и магнетизма. РСТ обнаружил связь между
% электричеством и магнетизмом. Помните, мы это говорили, обрадовался, что нашел
% пневму, но другие исследователи, другие мыслители, они, в общем-то, отнеслись к
% этому по-другому. А что это за другие мыслители? Вот те, которых отнесли уже,
% как бы, как сказать, к негерметистам. Помните, да, мы говорили о расслоении
% среди ученых? И вот одним из таких негерметистов был Майкл Фарадей. Он поставил
% опыт, который подтвердил то, что изменяющееся магнитное поле вызывает
% электрическое поле. Надо сказать, что он и ввел понятие поля, потому что он
% просто не мог в терминах классической физики назвать то, что он обнаружил. Ему
% пришлось, как сказать-то, отставить в сторону термины прежней науки,
% сформировать новое. Это и называется кризис того понятия материи, которым
% располагала прежняя наука, классическая наука. Единственное, что было, была
% проблема для Фарадея, он был из бедной семьи, он не получил фундаментального
% образования, он был, знаете, таким вот экспериментатором, природознатцем. То
% есть он очень хорошо мог работать с природой, с инструментами, но у него не
% хватало математических знаний, вот этих вот теоретических знаний совершенно. И
% он не смог свои открытия представить в математической форме, но книгу-то издал
% «Экспериментальные исследования в области электричества». И на его работу
% обратил внимание Джеймс Клав Максвелл, который происходил из совсем других
% кругов, он получил великолепное образование, он был этим вундеркиндом, он там в
% 10 лет поступил в Нинбургский университет, и как раз математические знания были
% его ключевым интересом. И вот он знакомился с трудами Фарадея и сразу же заметил
% проблему в тех интерпретациях силовых линий, которую тогда разделяли большинство
% учёных. Тогда какой был принцип основной? Принцип дальнодействия, помните?
% Дальнодействия. И он распространялся на представление о физических,
% электрических флюидах. Ну, флюиды, вот принцип дальнодействия. А Макселл
% предпочёл видеть силовые линии, пронизывающие пространство. Он назвал их трубки
% тока, ввёл в рассуждения научные такие представления, как скорость потока и
% разница давления. Ну, он просто аналогия с гетродинамикой. Электрический
% потенциал, разность напряжения, вот это вот всё. В итоге он смог перевести идеи
% Фарадея на язык математики. И он написал статью о Фарадеевых силовых линиях,
% причём выслал её самому Фарадею. Он был сначала, который, вернее, Фарадей был
% ошарашен совершенно эти сложности математического аппарата, но всё-таки
% разобрался. Человек-то умный был, да. И, ну, вообще-то, пришёл в восторг, да.
% Сказал, что, ну, вот поэтому мы говорим Максу Фарадею. При жизни Максвелла не
% обнаружилось то, что он предсказывал. Он заявлял, что свет это электромагнитная
% волна, предсказал существование других электромагнитных волн. При жизни
% Максвелла их не обнаружили, а сейчас, как мы знаем, их достаточно, ну, то есть
% мы знаем большое количество электромагнитных волн и оптического спектра. Таким
% образом, понятие поля существенно скорректировало прежнее понятие материи.
% Прежняя классическая система понятий и математических абстракций уже не
% удовлетворяла исследователей. Понимаете, словаря не хватает. классическая
% физика, физика Галилея Ньютона не дала всего словаря, которым можно было
% описывать природу, описывать действительность. И в первую очередь описывать
% такие явления, как электрические, магнитные и электромагнитные явления. 

\subsubsection{ЭМ} XIX век --- время кризиса классической науки. Классическая наука --- это эпоха Нового времени, а середина XIX века --- уже начало её кризиса. К концу века появляются первые формы неклассической науки. Это важный этап: он связан с осознанием границ ньютоновской физики, которая, несмотря на успехи, не могла объяснить, например, электричество и магнетизм.

Исследователи искали объяснение этих явлений. Фарадей --- один из тех, кто пошёл за пределы классической физики. Он показал, что меняющееся магнитное поле вызывает электрическое, и впервые ввёл понятие поля, потому что не мог использовать старые термины для описания нового явления. Классическая физика не имела нужного понятийного аппарата. У Фарадея, не имевшего теоретической подготовки, не получилось оформить идеи в виде математической теории, но его книга «Экспериментальные исследования в области электричества» произвела впечатление.

Её прочитал Джеймс Клерк Максвелл --- представитель другой среды, с блестящим образованием и сильной математической подготовкой. Он увидел в идеях Фарадея новые возможности. Вопреки тогдашней доминирующей идее дальнодействия, он представил силовые линии как пронизывающие пространство --- «трубки тока». Через аналогии с гидродинамикой он ввёл понятия потока, давления, потенциала и разности напряжения, переведя идеи Фарадея на математический язык. Он написал статью о силовых линиях и отправил её самому Фарадею, который, хотя и был сначала ошарашен сложностью, в итоге пришёл в восторг.

Максвелл предсказал, что свет --- это электромагнитная волна, и существует множество других таких волн. При его жизни это не было подтверждено, но позже теория нашла экспериментальное подтверждение. Понятие поля радикально изменило представление о материи: прежняя классическая система понятий оказалась недостаточной. Для описания природы, особенно электромагнитных явлений, словаря классической физики уже не хватало --- нужны были новые абстракции и новая логика.

\subsubsection{Статфизика}

% Второе,
% как бы, кризис, второй проблемой стал статистическая физика. Вот она тоже
% столкнулась с тем, что, ага, а где наша статистическая физика? Вот она.
% Совершенно невозможно, что называется, описывать классическим словарем, словарем
% классической науки, некоторые новые понятия. Статистическая физика изучает
% явления макроскопического такого свойства, да, то есть систем, состоящих из
% очень большого количества одинаковых частиц, например, газ, да. То есть,
% смотрите, классическая физика могла исследовать такие автономные частицы, их
% взаимодействия. А вот эта вот физика, термодинамика, статистическая физика вышла
% на явления, которые, ну, нельзя, как бы, они в реальности, нам нельзя выделить,
% особенно тогда нельзя было выделить одну молекулу газа, да. Мы имели дело
% только, учёные, люди имели дело только с большими объёмами. И вот Роберт Бойл
% показал, что отношение между давлением и объёмом газа можно так описать, как
% результат множества ударов отдельных атомов о стенку сосуда. Ну да, а как это
% описывать, какими понятиями? И, понимаете, вот термодинамика не складывалась из
% тех понятий, которые располагала прежняя физика. поэтому возникла как бы новый
% подход физики, когда большое число частиц в макроскопических телах описывается в
% новых статистических закономерностях поведения тела. И это поведение, как
% выяснилось, достаточно быстро в широких пределах не зависит от конкретных
% начальных условий, от точных значений начальных координат скоростей частицы.
% Если вы поняли, что такое классическая физика, нюденовская физика, то вы поняли,
% что это вообще-то ссора. Мы поссорились, называется. То есть, если классическая
% физика, физика Лапласа, ну, в смысле, физика Лапласа, Лапласовский детерминизм
% говорит о том, что всё зависит от конкретных начальных условий, мы можем
% просчитать, если мы будем демонами, да, особыми такими, мы могли бы знать, зная
% начальные положения, конечное состояние системы с точностью, если мы знаем
% начальные координаты скорости, то вот статистическая физика, имеется
% термодинамика и тому подобное, говорят уже, а вот нет, не можем. Учёные были
% вынуждены согласиться с тем, что процессы, происходящие в таких системах,
% описываются макроскопическими величинами. То есть это такие понятия, которые
% неприменимы к атому, например, то есть к индивидууму, к отдельной молекуле или
% атому, неважно. Например, что это за понятие? Давление. Понимаете? Давление –
% это характеристика, которая описывает только, как работает ансамбль, как
% работает какая-то, ну, коллективная такая вот форма существования, то есть много
% молекул, да, или температура, да, мы можем, ведь когда мы говорим о температуре,
% мы же не говорим о температуре атомов, мы говорим о той характеристике, которая
% связана, то есть на атомном уровне это движение, а вот на определённом другом
% уровне это температура или концентрация, то есть вот эти все понятия, они
% расширили представление физики о реальности, уйдя тем самым от только
% классического, от классических абстракций, только от классических абстракций. 

Статистическая физика тоже столкнулась с проблемой: как описывать новые явления, если классический язык физики уже не справляется? Она изучает макроскопические явления в системах, состоящих из огромного числа одинаковых частиц --- например, газов. Классическая физика могла исследовать отдельные частицы и их взаимодействия, но в термодинамике и статистике речь идёт о совокупностях, где невозможно выделить одну молекулу и проследить её поведение. Учёные имели дело с большими объёмами вещества, а не с отдельными атомами.

Когда Роберт Бойль установил зависимость между давлением и объёмом газа, он показал, что это можно объяснить ударами множества частиц о стенки сосуда. Но какими понятиями это описывать, если старая физика такими не оперировала? Так появился новый подход --- описывать поведение систем через статистические закономерности. Выяснилось, что макроповедение таких тел не зависит строго от начальных условий --- конкретных координат и скоростей частиц, в отличие от ньютоновской физики, где детерминизм был основой.

Статистическая физика как бы вступила в конфликт с классическим детерминизмом Лапласа. Если раньше считалось, что зная начальные условия, можно просчитать всё поведение системы, то теперь стало ясно: нет, нельзя. Учёные были вынуждены принять, что описывать такие системы надо через макропараметры --- такие, которые неприменимы к отдельной частице. Давление, температура, концентрация --- всё это свойства ансамбля, а не отдельного атома. Эти понятия расширили представления физики о мире, выведя её за рамки чисто классического подхода.

\subsubsection{Теория относительности}
% Ну
% и вот мы теперь приходим напрямую уже к неклассической физике, мы приходим к
% теории относительности, к теориям относительности Альберта Эйнштейна, который
% вот эти пределы классической физики и её панихейной системы уже совершенно
% определённо обозначил как вчерашний день. Но при этом, смотрите, никто из них,
% вот из этих людей, которые здесь перечислены, они не хотели вообще-то разрушать
% механику Ньютона, они её хотели обобщить. Но вот в ходе этих процессов обобщения
% возникает новый подход к реальности, новый тип науки и новые, соответственно,
% даже виды науки. Специальная теория относительности, это 100, да, специальная
% теория относительности разработана Эйнштейном, хотя ранее к ней пришёл Планкаре.
% Ну, Планкаре – это вообще особая история в истории науки, где, как говорится, он
% только не сказал новое слово, но проблема была в том, что он никогда новые вот
% эти свои идеи не дорабатывал. Поэтому существует даже в научном дискурсе такой
% спор. Эйнштейн или Планкаре – это типа за кого, да? Ну, я в нём не буду
% участвовать, если позволите. Поэтому дальше про Эйнштейна. Работа 1905 года к
% электродинамике движущихся тел. Она была основана на преобразовании Лоренца,
% если вы… Ну, как бы я сейчас не буду сильно погружаться. То есть, по сути дела,
% речь шла, на самом деле, просто о расширении Галилеевской системы представлений.
% Но вот эти координаты, вне эти преобразования Лоренца позволяют преобразовывать
% пространственно-временными координатами при переходе от одной инерциальной
% системы к другой. То есть, они описывают такие эффекты, как замедление хода
% времени и сокращение длины быстро движущихся тел. Да? Вот. Ну, каких быстро?
% Быстро, то есть, приближающихся к скорости света. Ну, а скорость света, как вы
% понимаете, она ключевая как в специальной теории относительности, так и в общей
% теории относительности. Поэтому, первый вывод, который был сделан Эйнштейном,
% это фундаментальное такое следствие специальной теории относительности, вывод о
% том, что вещество не может двигаться со скоростью света. Вещество, скорость
% вещества, любого атома, всегда меньше скорости света. Еще раз повторюсь,
% несмотря на эти вот такие неожиданные какие-то, достаточно интуитивно
% неожиданные выводы, все-таки специальная теория относительности опирается на три
% постулата Галилея, они, которые всего лишь несколько расширены за счет
% преобразования Лоренца. И к ним добавлен еще один постулат о постоянстве
% скорости света во всех системах отчета. Еще раз повторяю, вообще Эйнштейн, или
% там, тем более там какой-нибудь, какой-то, да, здесь тут никто, вот здесь, и
% честно, не хотел вообще опровергать классическую механику. Все к ней относились,
% ну, как-то, это, ну, база физики, да, но все равно получается так, что механика
% остается классической позади. Ньютон считал, что пространство абсолютно, а
% Эйнштейн показал, что пространство это такой сложный физический объект с
% определенными качествами. Оно способно как бы растягиваться и искривляться, и
% сжиматься, да, таким объектом становится и время. Ньютонская концепция
% предполагала, что течение времени одинаково во всей Вселенной. Отрезки времени,
% условно говоря, минута, она что здесь минута на Земле, да, что где-нибудь в
% другой части галактики или вообще Вселенной. Да, часы не в состоянии сказать
% нам, насколько быстро вот эти интервалы времени идут, да, но количество этих
% интервалов можно совместить в любой там, во всех системах, в любой системе,
% которая расположена там в галактике, здесь, в поле тяготения, не в поле
% тяготения и так далее. А Эйнштейн показал, что это не так. Он сказал, что эффект
% замедления времени и даже отставания часов, он заметен при скоростях близких к
% скорости света. Время замедляется тем сильнее, чем выше скорость. Ну вот, отсюда
% гипотетический вывод. Время останавливается, когда достигнута скорость света.
% Время останавливается, когда достигнута скорость света. Он мысленный эксперимент
% назвал, что будет, если двигаться вместе параллельно с лучом света. Такой
% мысленный эксперимент и описывает как вот остановку времени. То есть, смотрите,
% почему это вот ключевое положение недостижимость скорости света? Потому что сама
% этот постулат о недостижимости скорости света порождает категорию причинности,
% оставляет ее неизменной. Ну смотрите, принцип причинности опирается на что? на
% существование времени. Причина раньше следствия. Не одновременно причина раньше
% следствия. Если времени нет, то нет и причины. Нет и следствия. Все
% одновременно, все здесь и сейчас. А если мы достигаем скорости света, то времени
% нет. Поэтому вот почему так важен, ну во всяком случае для Эйнштейна, принцип
% скорости света. Принцип, не скорости света, а принцип недостижимости скорости
% света. А если мы допустим возможность сверхсвестной скорости, то, конечно, мы
% имеем возможность путешествовать во времени. Опять же вспоминаем фильм
% Интерстелл. С понятием пространства, как вы понимаете, естественно, связаны не
% только наши там скорости, разумеется, и так далее. Кстати, я сейчас выпускаю из
% внимания, а вы, пожалуйста, сами поинтересуйтесь, принцип эквивалентности
% скорости и движения в поле гравитации. Эйнштейновский знаменитый принцип,
% знаменитый его лифт, тоже мысленный эксперимент с лифтом, да. Пожалуйста,
% поинтересуйтесь, почему. Потому что, конечно, в поле гравитации мы скорости
% света не достигаем, но как позднее там возникла идея о черных дырах, где
% гравитация настолько мощна, что в ней тоже все вот эти эффекты сокращения
% объемов и растяжения времени вплоть до его остановки, они как бы выполнимы. Это
% вот принцип эквивалентности. А теперь к геометрии. Геометрии, да. Эксклидом были
% сформулированы положения такой геометрии, которая интуитивно воспринимается нами
% как единственная достоверная. Ну, она вот геометрия действительного мира. А это
% мир как бы плоскость, мир такая, ну, как сказать, емкость или доска, на которой
% все происходит. А вне классической науке такое положение вещей ставится под
% сомнение. Причем, ну, вина за это лежит, конечно же, на самом эвклиде. И нет,
% это не вина, это вина в кавычках, разумеется. Это победа гениальности эвклида,
% потому что один из, одно из пяти аксиом, которые он формулирует в своих началах,
% один из постулатов похож на теорему. То есть требует, как бы, чтобы, ну,
% доказательство было. Она звучит так. Предположим, что имеется прямая линия,
% точка вне ее. Тогда через эту точку можно провести одну и только одну прямую и
% параллельную первой. И как бы хочется доказать. И эти занимались математикой.
% Немецкий математик Гаусс, это 18-19 век, ну, в смысле, его деятельность,
% конечно, 19-го, теперь около Индии приходится. Он понял, что эвклидовая
% геометрия это только геометрия плоскости. Но мы же можем представить себе
% геометрию искривленного ландшафта, искривленной плоскости, искривленной
% реальности, да? Например, геометрию шарика на Дутова. И вот Гаусс, рассмотрев
% последствия перенесения эвклидовой геометрии на искривленную поверхность, только
% в двух измерениях, он заметил, что пятый постулат вот этот перестает быть
% справедливым. Невозможно провести ни одну параллельную другой линию. И сумма
% сторон треугольника оказывается не 180 градусов, а больше. И так далее. Римман
% успел создать неэвклидовую геометрию, хотя прожил только 40 лет. Он обобщил
% геометрию Гаусса на три измерения. Вот понимаете, геометрия Гаусса это еще
% плоскость, два измерения. А Римман обобщил на три измерения. То есть речь идет
% об искривлении трехмерного пространства с положительной кривизной. Это полностью
% порвало с принципом наглядности. Вон, вспомните, мы говорили про это
% абстрагирование. Это уже то, что невозможно представить. Это уже разрыв с
% реальностью, с представимым. Ну, конечно, математика это не может
% останавливаться, естественно. И Римман это не интересовало. Можно представить
% его построение, нельзя, неважно. Он дал способ выполнять расчеты и предсказания
% именно для искривленной трехмерной поверхности. Ну, хотя там сохранялась связь с
% эклидовой геометрией. То есть в малых областях, в малых очень областях, в малых
% приближениях эклидовая геометрия вполне справедлива. Другая не эклидовая
% геометрия это независимо только от друга разработанные Николай Лобачевский и
% Иоанн Шболиев. Их интересовала математическая возможность, когда через точку,
% расположенную в непрямой, можно провести бесконечно много параллельных линий.
% Бесконечно много параллельных линий. Эта геометрия описывает пространство
% трехмерного пространства с отрицательной кривизмой. Так вот, Эйнштейн рассмотрел
% возможность вот такого искривления физического пространства, трехмерного
% искривления физического пространства, потому что геометрия позволила ему это
% делать. И он в этом смысле опирался на геометрию Риммана и Клиффорда. Но
% Эйнштейн это интересовало не сами геометрические свойства пространства, а то,
% как это связано с гравитацией. С гравитацией. И он предложил новую теорию
% гравитации, просто обобщая теорию, специальную теорию относительности. И вот тут
% мы выходим в принцип эквивалентности. Смотрите, наблюдатели, которые летят ли в
% лифте, идущим вверх, с ускорением, равным ускорению земного тяготения, не
% почувствуют отличие от действия земного тяготения. То есть инерционное поле,
% создаваемое ускорением, будет точно такое же, как и поле гравитации, если вы
% скоростью уравновесите. Он сосредоточился на этом феномене, на этой идее. И
% сделал вывод о том, что луч света, который искривляется, проходя у края солнца,
% это вот связано с искривлением пространства. То есть не луч вот там отклоняется,
% его не притягивает каким-то образом солнце, как вот у Ньютона, да, помните?
% Притягивает с помощью принципа дальнодействия. Просто так вот, ну, просто вот
% притягивает и все. А надо сказать, что Ньютоновская идея дальнодействия всегда
% очень не нравилась Эйнштейну. Не потому что там, там, алхимия или еще что-то
% такое. Нет, просто, ну, как бы незакончено. Наука некрасива, потому что. Вот. И
% гравитационное взаимодействие, как предлагает Эйнштейн, это не тяготение, вот
% это вот, да, вот, как сказать, там, симпатии взаимные, да, просто притянуло и
% все. А потому что, условно, как будто Вселенная знает вот все, что происходит,
% вот, помните, чувство лица Бога и так далее. А потому что масса искривляет
% пространство. Он приводил образ, когда сын ему объяснял, что такое его теория.
% Он говорил, выставь себе резиновую пленку и на нее положили апельсин. И вот, как
% бы, до этого все, что лежало ровно на пленке, начали скатываться туда, в эту
% ямку. Но только надо это представить в трехмерном измерении. Двухмерный,
% трехмерный. Солнце искривляет пространство вокруг себя, как любая материя. Но,
% понимаете, масса огромная. И в этом искривлении движение любой другой материи
% идет по самому кратчайшему пути, просто по геодезической линии. Рейнштейн
% сформулировал большинство физических теорий, общей теории относительности,
% однако долгое время не мог сформулировать их математически. Он просто не знал
% подходящего математического аппарата. И вот тут ему очень существенно помог его
% друг Марсель Гроссман. Он был как раз специалистом-математиком в недавнем, на
% тот период недавнем, в недавней области математики, недавно возникшей области
% математики. Она как раз и развивалась на представлениях о искривленных
% пространствах. Вы понимаете, что вот абстракция полностью вступила в свои права.
% Искривленные трехмерные пространства сидят математики и спокойно занимаются этим
% темами. И тут это понадобилось физикам. То есть вроде бы то, что имеет какую-то
% вообще отвлеченные какие-то такие вот абстрактные характеристики, физикам
% понадобилось тензорное исчисление. Так вот, Гроссман познакомил Эйнштейна с
% этими работами, Риммана, Кристофеля и Ричи. И вскоре под руководством Гроссмана
% Эйнштейна владеем тензорным исчислением, приступил к формулированию уравнений,
% которые описывают идеи общей теории относительности. Вот таким образом
% выведенные эти уравнения позволяют определить, насколько и как именно искреблено
% пространство у данной массы или там внутри массы. И он там показывает, что время
% в сильных гравитационных полях течет медленнее. Он, кстати, сделал вывод о том,
% что и доказал, что его уравнение сводится к уравнениям Ньютона, когда вот это
% вот, значит, скорости небольшие или поле гравитации недостаточно мощное. Ну, в
% смысле, искребление пространства и времени. Вот. Но он не смог решить ряд
% уравнений. Ряд уравнений, которые бы закончили его теорию. Понимаете? Ну, потому
% что это особое, как бы создать уравнение, решить несколько разной вещи, да? За
% него это сделал астроном Кларл, Карл, простите, Шварцшидт. Вот он последний у
% меня в списке. И овите, как много. Поэтому делать слайды я уже просто, ну, не
% стала, извините, да, их лица. Ну, вот. В это время Шварцшидт был добровольцем на
% русском-немецком фронте, да? Он умирал в госпитале в это время. Но вот, знаете,
% какая сила заинтересованности, какой математик. Он успел найти до смерти на свои
% решения для сферического распределения масс, которое послал Эйнштейну. Эйнштейн
% очень был рад, ну, вот, всё решение есть. И очень удивительно, что из этого
% решения выходило одно следствие, которое начало беспокоить Эйнштейна. При
% достаточно высокой концентрации масс пространство искривлялось так сильно, что
% область внутри вот этих вот искривлений определенного радиуса казалась как бы
% отрезанной от остальной Вселенной, пространство становилось похожим на
% бутылочное горлышко или на воронку, а на другой стороне как будто бы имелось её
% зеркальное отражение. Чувствуете, да? Куда наука запрыгнула от человека, от его
% мысли глубины, неописуемой, человек, который в это время умирал в госпитале.
% Вот. Эйнштейн назвал этот феномен пространственным временным мостиком. Вот мы
% переходим с вами к следующей. Очень такой сложной науке чувствовать и общую
% относительность прямо скажем непроста. 

Специальная теория относительности, оформленная Эйнштейном в 1905 году, основывалась на преобразованиях Лоренца. Хотя ранее к подобным идеям приходил и Анри Пуанкаре, он не доводил свои концепции до завершённой теории, из-за чего в научной среде до сих пор идут споры, кому принадлежит приоритет. Однако Эйнштейн сумел построить стройную теорию, расширив галилеевскую систему представлений. Преобразования Лоренца позволяют описывать переход между инерциальными системами и объясняют эффекты сокращения длины и замедления времени при движении тел с околосветовыми скоростями.

Ключевым стало положение о недостижимости скорости света для любого материального объекта. Эйнштейн показал: вещество не может двигаться со скоростью света, его скорость всегда меньше. Несмотря на парадоксальность выводов, теория опирается на расширенные принципы Галилея и включает постулат о постоянстве скорости света во всех инерциальных системах. Пространство и время в ней рассматриваются как взаимосвязанные, изменяющиеся под действием движения и гравитации.

Ньютон считал пространство абсолютным, а течение времени --- одинаковым везде. По Эйнштейну же, и пространство, и время --- физические объекты, способные искривляться и растягиваться. При высоких скоростях, близких к световой, течение времени замедляется, а на пределе --- может остановиться. Его мысленный эксперимент о движении со скоростью света иллюстрирует именно остановку времени, из чего следует: если скорость света достижима, то исчезает причинность, так как без времени невозможно различить причину и следствие.

Именно принцип недостижимости скорости света сохраняет причинность. Его нарушение открыло бы возможность путешествий во времени. Здесь вспоминается принцип эквивалентности --- равенство эффектов ускорения и гравитации. Эйнштейн показал: если в замкнутом лифте вы движетесь с ускорением, равным ускорению свободного падения, вы не отличите это от действия гравитации.

Отсюда и идея: сильное гравитационное поле искривляет пространство и влияет на течение времени. Например, луч света, проходящий рядом с Солнцем, отклоняется не из-за "притяжения", как считал Ньютон, а потому что пространство вокруг искривлено. Гравитация по Эйнштейну --- это не дальнодействующее притяжение, а следствие искривления пространства массой. Он иллюстрировал это примером с резиновой плёнкой и апельсином: массивное тело деформирует плоскость, и объекты "скатываются" к нему по кратчайшему пути --- геодезической линии.

Формулировать математически общую теорию относительности Эйнштейну помог его друг Марсель Гроссман, специалист в новой области математики --- геометрии искривлённого пространства. Именно тензорное исчисление, разработанное на базе трудов Римана, Кристофеля и Риччи, позволило Эйнштейну вывести уравнения, описывающие поведение пространства и времени в присутствии массы. Эти уравнения показали, что в сильных гравитационных полях время замедляется, а при слабых условиях сводятся к классическим ньютоновским законам.

Однако сам Эйнштейн не смог найти точные решения для этих уравнений. Это сделал астроном Карл Шварцшильд, находясь при смерти в госпитале во время Первой мировой войны. Он нашёл решение для сферически симметричного распределения масс и отправил его Эйнштейну. Вывод оказался неожиданным: при высокой плотности массы пространство искривляется настолько, что внутри определённого радиуса оно отрезается от остальной Вселенной --- возникает нечто вроде воронки или бутылочного горлышка с зеркальным отражением по другую сторону.

Эйнштейн назвал это пространственно-временным мостом. Так наука пришла к понятию чёрных дыр и мостов Эйнштейна–Розена. Общая теория относительности остаётся сложной, требующей глубокого понимания, но именно она заложила основу современных представлений о гравитации, времени и пространстве.

\section{Роль квантовой механики}

% А квантовая механика, как мы все знаем,
% ещё не простее. В начале XX века понятие материи претерпевает существенную
% трансформацию. Существенную трансформацию, причём это касается как
% онтологических, так и гносиологических позиций. Напомню, онтологические позиции,
% это то, что касается того, ну  I end what for, which isabilir, which is OHT,
% onthology, which is accomplished. А есть гносиологические позиции, ещё раз
% напомню, гносиология, учение о познании, науках о познании, и соответственно,
% вот эти изменения касаются того, как мы познаём реальность. Квантовая механика
% внесла существенные изменения в том, как мы с помощью каких-то подходов к
% реальности познанём реальность. И здесь, конечно, ключевая фигура – это Макс
% Планк. Опять же, человек, который совершенно не хотел конца классической физики,
% классической механики, он всего лишь хотел исправить один из серьезных
% недостатков классической теории. А я напомню, с этого все и начинается. Как
% говорится, достаточно ковырнуть в каком-то там миропредставлении, казалось бы,
% целостном и на века существующем, какой-то один элемент, и все посыпется. И я
% сознательно напоминаю вам о первом таком ковырянии, простите, да, которое
% происходило в Средневековье. Схоласты ковырнули концепцию Аристотеля, его
% физику, сковырнули и создали, ну, просто на основе такого умозаключения и
% отчасти наблюдения, о том, что воздух действует как сопротивление. Он говорит,
% да нет, ну воздух же действует. Мерпанские схоласты, напоминаю.
% Там теория импетуса родилась, из которой потом теория импульса. А вот Макс
% Планк, он, по сути дела, по сути, ну, как сказать, изучал излучение. И вдруг у
% него так получилось, что оно излучается порциями. Их назвали кванты. Он так их
% сам и назвал. Но они непрерывно, ну, кванты, да, то есть порции, а как бы
% дискретно. Вот есть континуально, это волны, континуально связаны. А есть
% дискретно порциями. Но только за несколько лет до него было доказано, что свет
% имеет волновую природу. Представляете, как он расстроился? Ну, она же волновую
% природу имеет. И хотя сам Планк только заделывал дыры в одном из уравнений
% классической теории, тем не менее получилось, что свет одновременно и частицы и
% волна. Вот так вот. Корпускулярно-волновой дуализм. Корпускулярно-волновой
% дуализм. Запоминайте эти фразы, потом там в слайде будет. В 1923 году. В 1923
% году. Видите самое длинное имя? Луи Виктор Пьерре Монде Брой. Как вы понимаете,
% не зря у него такое длинное имя. Он был французским принцем. Ну, то есть вот
% этих кровей, да, аристократических принц. Королевская династия. Конечно,
% французы уже не имели короля, но тем не менее династия существует. Так вот, он
% показал, что взаимодействие электронов с излучением легче всего понять, что
% электроны ведут себя то, как частицы, то, как волны. Более того, он вёл
% представление о том, что свет может иметь двойственную природу. В смысле, что не
% только свет может иметь двойственную природу, но и вещество, вообще любое
% вещество. Любая материальная система имеет двойственную природу. Она и корпуску,
% ну, как бы материальная такая вот, да, отдельная изолированная сущность. И
% волна. И он предложил достаточно простую формулу. Чем больше масса объекта, тем
% больше длина его волны, то есть меньше частота. Поэтому, например, мы не
% замечаем частоту такого объекта, как Земля или Солнце. Понимаете, слишком
% большой объект, частота просто очень-очень маленькая. Для мезомира волновые
% свойства материи, говорит он, практически не проявляются, но они есть. А вот для
% микромира длина волны очень мала, соответственно, частота очень высока, и мы
% видим вот эту вот осцилляцию. То есть вот это волновые-то свойства. Песня,
% песня, песня. Эти идеи явились основой диссертации Дебройля. Но она выглядела
% настолько абсурдной, настолько дикой, что комиссия предпочла бы, конечно,
% послать его куда подальше, отвергнуть диссертацию даже на стадии рассмотрения.
% Но он-то ведь королевского происхождения, да, он же аристократ высших-высших-
% высших слоев. Диссертацию было нельзя ни принять, ни опровергнуть. И комиссия
% пришла свалить это трудное дело на эксперта, так частенько бывает, да. И
% экспертом стал Эльберт Эйнштейн. А, как вы понимаете, Эльберт Эйнштейн далеко не
% тривиальный мыслитель. Он любил всякие такие вот, ну, парадоксальные идеи. И он
% был впечатлен идеей Дебройля, поддержал ее полностью. И уже спустя 4 года
% американские физики Клинтон Девинсон и Лестер Джеммер экспериментально доказали,
% что электроны обладают волновыми свойствами. Они обнаружили картину дифракции
% электронов. А позднее было показано, что частицы любого вида дают такую же
% картину. То есть вещество действительно обладает волновыми свойствами.
% Математическую форму такого представления о корпусулярно-волновом дуализме. А я
% подчеркиваю, это прям уже резкий разрыв с прежней концепцией материи. Прежняя
% концепция материи какая? Корпускулы, которые просто взаимодействуют. Вот это и
% есть наши атомы, молекулы, которые взаимодействуют. И вот наша материя. Но
% оказывается, материя еще обладает характеристиками поля, характеристиками волны.
% И совсем странно, характеристиками волны частицы. Математическую форму таким
% представлением придали независимо друг от друга Шривенгер и Гейзенберг. Вернер
% Шривенгер и Гейзенберг. Правильно, с немецкого. Ну, мы привыкли к Гейзенбергу,
% поэтому я буду так говорить. Хотя, ну, прекрасно понимаю, что, конечно, не
% Шривенгер, а Шривенгер и Гейзенберг. Гейзенберг. Да и не Эйнштейн, а Айнштейн,
% конечно же. Сама знаю немецкий язык, поэтому он говорил. Но буду говорить, как
% мы привыкли. Так вот. И они включили вот эти вот вероятностные представления,
% статистические представления, статистический подход, включили в это понимание,
% вот этого дуализма. Напомню, что такое вот эти вот статистические или
% вероятностные причинности. Известно, сколько людей погибнет в ближайший год, да?
% Но невозможно заранее предсказать, кто погибнет. Так и квантовая теория. Она
% позволяет предсказать, что три атома из 10 в ближайшие 10 минут распадутся,
% потерпят радиоактивный, претерпят радиоактивный распад. Но она не даёт
% возможности узнать, какие им... Волна, которая Дебройлем понималась в физическом
% смысле, вот какая-то волна, да, физическая, у Шривенгера и Гейзенберг становится
% волной вероятности. То есть вот там, где волна имеет максимальную амплитуду, для
% нас это означает максимум обнаружить там частицу, то есть проявление
% корпускулярности в материи. А там, где волна имеет минимальную амплитуду, мы
% обнаружим лишь что? Ну, дифракционную картину. Волны в этом обнаружены. То есть
% квантовая теория оказывалась статистической. И она, получается, накладывала
% определённые ограничения на наше знание о поведении объекта. Мы можем знать, ну,
% лишь вероятности. Это и есть копенгагенская интерпретация квантовой механики.
% Копенгагенская интерпретация квантовой механики. А Гейзенберг предлагает принцип
% неопределённости. Нильс Бор предлагает принцип дополнительности. О них,
% пожалуйста, чуть полнее там сами проработайте. Ну, это, в общем, довольно
% известная вещь, но, тем не менее, если вам они очень известны, то проработайте.
% Это как раз более полное воплощение принципа корпускулярно-волнового... Ох, у
% меня там пропущена буковка, но не важно. Корпускулярно-волнового дуализма.
% Сейчас сделаем вот так. Корпускулярно-волновой дуализм. Копенгагенская
% интерпретация как раз, ещё раз подчеркну, сосредотачивается над вероятностных
% толкований. Ну, чуть-чуть про принцип неопределённости. Смотрите, на атомном
% уровне, говорит он, имеется некая размытость. Ну, нельзя одновременно измерить
% импульс или координату частицы. При измерении импульса нарушается положение
% частицы. Почему это происходит? Ну, просто представьте себе, что вы изучаете там
% какую-то, ну, какой-нибудь там электрон. А чтобы его увидеть, да, ну, как
% сказать, там, измерить, понять его состояние, нужно же фотономос... Ну, как,
% минимум один фотон направить вот в сторону... Ну, в смысле, ну, чтобы он
% отразился, этот фотон от чего-нибудь, да, как вы понимаете. А представьте себе,
% для таких вот этих областей, материи, для таких вот маханьких объектов, что
% означает удар фотона? Ну, всё, положение изменилось. То есть, как минимум из-за
% этого. Получается, как сказать, мы можем либо одно, либо другое узнать. Вот эта
% вот неполнота знаний в системе нормально, говорит Этель Резинга. Это нормально.
% Мы должны с этим, ну, просто согласиться. Вот. Нильс Бо. Он развивает этот
% принцип неопределенности. Он говорит, электрон ведёт в себя то, как частица, то,
% как волна, и эти аспекты дополняют друг друга. То есть, существует только по
% отдельности. И всё зависит от способа измерения. Положение и импульс частицы
% зависят от того, как мы их измеряем. А дальше возникла большущая проблема с
% прохождением, с опытом, с экспериментом прохождения электрона сквозь две щели.
% Наверняка вы все об этом наслышаны. Если не наслышаны, пожалуйста, те, кто, ну,
% как-то раньше не интересовался этой квантовой механикой, ну, найдите в интернете
% этот опыт, его описание, его визуализацию, и поймёте, о чём речь. То есть на
% экран, расположенный позади, щели, две щели, направляется пучок электронов.
% Когда щель одна, картина такова, что можно говорить, ну, вот, что электроны
% проходят через неё в качестве частицы. Проходят, проходят все. Просто там, ну,
% как бы на экране обнаруживаются вот эти вот элементы, элементы ударов, да,
% скажем так, ну, картины ударов. То картина выглядит... А дифрация --- это, ну, как
% сказать, дифрукционная картина --- это след волны. Такие там, ну, как сказать,
% надо было картинку-то вам всё-таки нарисовать, ну, в смысле, не нарисовать,
% показать, но я-то всё-таки, видите, общаюсь с учёными, уже много надеются, что
% вы все прекрасно понимаете, о чём речь. Я тут вообще философ, а вы все
% естественные, всё, ничего не знают. Так вот, если уменьшить интенсивность пучка
% вплоть до одного, то интуитивно, да, так скажем, с точки зрения здравого смысла
% мы можем предположить, что электрон пройдёт через одну щель, через ту или через
% другую. И мы обнаружим картину чего? картину вот этого удара, да, ну, как бы
% результат удара в качестве каких-то вот его отражений. Но есть и щели две, на
% практике вдруг при достаточно долгом проведении опыта показана и фракционная
% картина, то есть волна. Электрон --- это волна, которая размывается перед тем, как
% попасть в установку, как бы проверяя, какая щель открыта. И вот на основе этого
% эксперимента развелось огромное количество других интерпретаций квантовой
% механики. Их прям много. И они на самом деле множатся. Одна из самых известных ---
% это многомировая интерпретация Хью Эверетта. Про неё можно много говорить, но,
% во всяком случае, сегодня её можно даже посмотреть в разных фильмах. Это
% концепция многих миров. И он говорил, что это спасает на самом деле причинно-
% следственную цепочку, потому что уже нет какой-то такой вот проверки, как это
% электрон проверяет. На самом деле, электрон совершенно спокойно проходит через
% одну щель. Но мы же видим только это в одной реальности, в одном мире, в одной
% вселенной. А в другой вселенной пошёл через другую щель. И вот эта дифорционная
% картина, то есть часто повторяющийся опыт, это некое последствия эхо
% многомирности, многомировой интерпретации. Когда он это рассказал, конечно,
% очень сильно покрутили пальцем вискар. Вообще он даже заболел из-за того, что
% его многомировую интерпретацию, мультиплей, его как бы не восприняли. ни физики,
% ни нефизики, не восприняли. Он заболел из-за этого. Но тем не менее, его
% учитель, Джеймс, да, Джеймс, он как раз взял и эту тему развил. И сегодня на
% основе этой многомировой интерпретации создаётся новая физика, это называется
% цифровая, физика цифровой реальности. Ну ладно, это уже другой разговор, это
% современность, и физика далека, прям скажем, полноценных. Это концепция
% математической вселенной. Она далека от своего, как говорится, воплощения
% полноценного, но книги выходят с завидной регулярностью. Надо сказать, что всё
% это результат деятельности аспирантов одного учёного. Вот Эверетт был аспирант
% этого, и вот его другие аспиранты, они вот так продолжают эту линию. Но другие
% интерпретации, пожалуйста, попытайтесь сами выяснить, это невозможно, чтобы я
% сейчас их все озвучила. Дальше. Ну, конечно же, квантовая механика всегда
% пытается сформировать какую-то атомную модель. То есть попытки соединить принцип
% наглядности и математически экспериментальный, какой-то вот, каких-то
% математически экспериментально найденных истин в отношении микромера. Одна из
% самых первых, это модель Томпсона, пункт Пудинг с изюмом. Посмотрите картинки,
% найдите. Я не стала даже их тут как бы восстанавливать. Не потому что это
% неинтересно, что это вы сами прекрасно найдёте, хоть какая-то, может быть, для
% вас станет попытка что-то будет самостоятельное изучение, потому что на самом
% деле, не знаю, как в других группах, в моей группе довольно прискорбная работа
% происходит с материалом лекционным. материалом лекционным площадь вообще тяжело,
% прям вообще. Надеюсь, в остальном группах не так. Надеюсь, это я плохая. Это я
% плохо преподаю. Да. Так. Сейчас, секундочку, у меня тут опять небольшие проблемы
% с материалом, в смысле с техникой, это нормально. Это ненормально, дорогие мои.
% Сейчас, секунду, у меня просто заделись компьютер, который дальше-то нам
% предлагает. Никак. Ладно. Сейчас, значит, дорассказываю с этим самым, без каких-
% то там помощников. Я вам предлагаю только два варианта эволюции моделей атома.
% Планетарная модель Боро-Рейзенфорда, Боро-Рейзенфорда. Есть другие модели. Мы
% будем очень рады послушать. То есть это вам как бы, понимаете, даже по поводу
% пантеромеханики, пожалуйста, на экзамене озвучивайте, рассуждайте по поводу
% любых. Вот холистических концепций можете, или там интерпретацию Нейман.
% Пожалуйста. Цифровую там, как угодно. Просто умейте рассказать про то, как эти
% концепции кардинально меняют представление реальности. Или модель атома тоже.
% Пожалуйста, на этом основе. Но я, конечно, особенно хочу подчеркнуть все-таки
% роль, опять же, инженерного знания. Потому что без всех этих вот, ну как бы
% теоретические, теоретические позиции, все хорошо. Но крайне важно, что наука
% выходит на создание новых инструментов анализа. 

% Помните, инструментализация
% науки продолжается. рентгенология, рентгенография, она настолько серьезно
% продвинула науку, насколько вообще возможно. Потому что даже все то, что мы
% знаем сегодня в биологии, фундаментальной биологии, например, о гене, да, о
% генетике, это же возможно только потому, что рентгенография этих молекул была
% осуществлена. То есть вот как бы от Санкриг они почему смогли вот эту мануку,
% потому что у них были наглядные рентгены, ну как наглядные, смутные, конечно,
% тем не менее, рентгены, вот эти вот рентгеноснимки. Это очень важно. Про это
% поговорите. Другие инструменты науки, которые были связаны с открытиями в сфере
% в сфере неклассической науки. 

В начале XX века представление о материи радикально меняется --- как с точки зрения того, что такое материя (онтология), так и с точки зрения способов её познания (гносеология). Эти изменения касаются не только сущности вещей, но и того, как мы их исследуем. Квантовая механика изменила наше понимание реальности и подходы к её изучению.

Ключевая фигура здесь --- Макс Планк. Он вовсе не стремился разрушить классическую механику, а всего лишь пытался устранить один из её недостатков. Изучая излучение, он пришёл к выводу, что энергия испускается не непрерывно, а порциями --- квантами. Это противоречило господствовавшему на тот момент волновому представлению о свете. Получилось, что свет --- это одновременно и волна, и частица. Так родился корпускулярно-волновой дуализм.

В 1923 году Луи де Бройль предложил идею, что не только свет, но и любое вещество обладает волновыми свойствами. Он вывел простую формулу: чем больше масса объекта, тем меньше длина его волны. Волновая природа макрообъектов почти не проявляется, но для микромира она становится заметной. Эти идеи легли в основу его диссертации, которая поначалу казалась слишком радикальной, но была поддержана Эйнштейном.

Через несколько лет Дэвиссон и Джермер экспериментально подтвердили волновые свойства электронов, зафиксировав дифракционную картину. Позже стало ясно, что все частицы проявляют подобные свойства. Это означало серьёзный разрыв с прежней картиной мира, где материя понималась как совокупность взаимодействующих корпускул. Теперь она обладает ещё и волновыми характеристиками.

Математическое обоснование этим идеям дали независимо друг от друга Шрёдингер и Гейзенберг. Шрёдингер предложил волновое уравнение, где волна отражает не физические колебания, а вероятность обнаружения частицы. Там, где амплитуда волны максимальна, вероятность обнаружить частицу выше. Так квантовая механика стала статистической теорией.

Это статистическое понимание легло в основу копенгагенской интерпретации, где Гейзенберг сформулировал принцип неопределённости: нельзя точно знать одновременно координату и импульс частицы. Причина в том, что само измерение влияет на измеряемый объект. Нильс Бор добавил к этому принцип дополнительности --- электрон ведёт себя и как частица, и как волна, в зависимости от способа измерения.

Особую роль сыграл опыт с двумя щелями. При прохождении электронов через одну щель они ведут себя как частицы, а при двух --- создают интерференционную картину, как волны. Даже если пускать по одному электрону, сохраняется волновой рисунок. Это вызвало множество интерпретаций квантовой механики. Одна из самых известных --- многомировая интерпретация Эверетта, согласно которой электрон проходит через обе щели, но в разных вселенных. Эта идея сначала была отвергнута, но позже легла в основу новых направлений, включая цифровую физику.

Существуют и другие интерпретации, с которыми стоит ознакомиться самостоятельно. Что касается моделей атома --- от первых попыток вроде "пудинга с изюмом" Томпсона до планетарной модели Бора–Резерфорда --- все они стремились объединить наглядность и математическую строгость в описании микромира. Сегодня особенно важно понимать, как инженерное знание и теоретическая физика взаимодействуют в формировании новых представлений о реальности.

\section{Основные черты неклассической КМ}

% Так вот, неклассичность науки теперь в целом. Мы
% говорим с вами здесь только о физике, но это же несправедливо. Наверняка,
% неклассичность проявлялась в других науках. Конечно, она проявлялась в меньшей
% степени и вообще, вот знаете, как Маркс создал свою концепцию общественных
% формаций, только опираясь на, по сути дела, английское общество, и оно с трудом
% было применимо там, вот где-то там для русского общества, для там какого-то еще,
% а что-то может для восточных обществ. Ну, так вот, удел этих самых абстрактных
% моделей, которые к одним более-менее подходят, а к другим уж вообще не подходят.
% Но ведь на самом деле и наша концепция вот этого деления на классическое, не
% классическое, там, на т.д. Она тоже абстракция, мы же это не скрываем, да? И
% она, естественно, ну вот к физике хорошо подходит, потому что на основе ее и
% создана, да? А вот, например, к биологии уже хуже, но тем не менее. Вот
% смотрите, что в биологии не классичного в это же время? Конечно же, это
% исследование именно генетическое. Вот генетика это яркое проявление, по сути
% дела, не классичности. Начинается с Дарвина это, тем не менее. Смотрите, что,
% почему? У Дарвина не классичность проявляется в какой форме? А в то же самое,
% что у Маркса. Вот поразмышляйте по поводу своих наук, а я вам пример дальше, вы
% просто почувствуете. Смотрите, как Маркс сказал, что, по сути дела, ну, нет вот
% этого атома человека, как и отдельного сознания, а есть вот такой холистический,
% да, подход. То есть, сознание человека это некая, ну, как бы, констилляция
% социальных взаимодействий и, соответственно, вот отражение в индивидуальном
% сознании всяких разных социальных сил. то есть, коллективное измерение субъекта.
% То же самое у Дарвина. Если бы мы говорили в терминах классической науки,
% рассматривали исключительно в терминах классической науки теории Дарвина, теории
% эволюции, ну, Дарвина, теория эволюции, то мы бы были вынуждены сказать, вот
% сейчас, простите меня, биологи, но мы были бы вынуждены сказать, что когда-
% нибудь какая-нибудь обезьянка родила человека. Серьезно. И это не просто пустые
% слова, потому что в биологии очень долго шел спор, а когда происходит смена
% видов. И, например, такие философы, как мыслители, ученые, как Сент-Илерс, отец
% и сын, они даже новую науку развили, называется тератология, то есть наука о
% бродах. Они говорили о том, что виды новые возникают благодаря родствам
% эмбриона. Вот эмбрион подвергся каким-то там воздействию, ну, урод родился, так
% вот это вот и есть, ну, это мысль 18 века. Так поэтому вот как бы это и есть
% новый вид. Но, правда, не все новые виды так сохраняются, а только те, которые
% потом выжили и когда-то там... То есть когда-то какое-то вот уродство, какой-то
% обезьянки на уровне эмбриона позволило осуществиться тому, что все-таки родился
% человек. Да? Понимаете, да? Современная теория эволюции так не мыслит. А как она
% мыслит? А оказывается, единица эволюции это популяция, это не особь, популяция.
% То есть чувствуете коллективные измерения. Так же, как в термодинамике мы не
% рассуждаем уже об отдельной молекуле, мы рассуждаем только об ансамбле. И
% поэтому у нас такие понятия, как температура, давление, концентрация и так
% далее. Так и тут. Все, мы в терминах индивидуального, редукционистского такого
% подхода рассуждать о теории эволюции не можем. Единица эволюции популяции. Вот
% такие подходы и изменения в разных науках возникают. Вот этот кризис, несмотря
% на то, что он очень показательный на материале физики, тем не менее, возникает и
% в других науках. подумайте, посмотрите, обсудите с преподавателями. Один из
% таких форм вашего, подсказка в ходе этого рассуждения, я предлагаю обратить
% внимание, а с помощью каких методов изучаются, какие-то там знания возникли в
% этих науках. Если эти методы связаны с той же самой рентгенографией, то как
% минимум мы твердо можем быть уверены, что это уже не классичность. Или,
% например, появляется очень много математических абстракций, математического
% моделирования в этой науке, ну, значит, это тоже уже не классичность. И теперь
% сам, как бы, обращая ваше внимание на последний слайд и последнюю нашу тему, это
% не классическая научная картина мира, надеюсь, мы быстренько это все общим и
% расскажем. Значит, ну, первое, смотрите, я немножечко, может быть, не в том
% порядке пойду, как здесь, потому что в разные времена писались лекции,
% презентации создавались, неважно. Давайте первое, значит, системность.
% Системность и некая универсальность теории, то есть возникают эти теории,
% которые снова претендуют на универсальность, но возникает проблема объединения.
% И одна из важнейших вот таких принципов, как сказать, объединения порождает
% философские исследования. Философские исследования. Как бы наука снова
% становится отчасти философией. Пример. Эйнштейн и Бор. Спор Эйнштейна и Бора.
% Ведь о чем спор? Помните, играет ли Бог в кости? Эйнштейн говорит, Бог не играет
% в кости, Бор говорит, что, ну, как бы он не говорит это прямо, ну, вроде как
% играет. О чем речь? Пожалуйста, не говорите, что речь идет о каком-то там,
% чистом случае и тому подобное. Нет. На самом деле, Эйнштейн спорит с Бором о
% том, что статистическое понимание, что такое играть в кости, это означает, как
% сказать, ну, заниматься той деятельностью, которую может описать только теория
% вероятностей. Он говорит, Бог мыслит линейной причинностью. То есть Эйнштейн в
% некоторых вопросах оставался очень классичным. Он как бы одной наглой был во
% вчерашнем дне. И говорит, нет, Бог, это линейная причинность. А Бор спорит с
% ним, говорит, нет, Бог, это вероятностная причинность. И, пожалуйста, вот
% вероятностный детерминизм в нашем слайде можно совершенно как бы, ну, интересно
% и, в принципе, достаточно легко описать с помощью Бора, с помощью спора
% Эйнштейна и Бора. Вот. Еще один момент. Многовариантность моделей
% действительности. Почему это возникает? Да потому что надо все объединить. Надо
% же, наконец, создать какую-то такую похожую картину мира, которая была времена в
% исторической физике. Вот там все было объединено. В исторической концепции не
% подкопаешься. Там метафизика соединяется с физикой, физика с биологией,
% биологией, биология снова с метафизикой, космос, человек, социальная структура
% общества, экономика. Все срастается, понимаете? Все вот не подкопаешься. Правда,
% все чисто умозрительно. И начнешь только, говорится, практически проверять и
% выяснять, что даже камень падает не так, как описывает Аристокий. Но им-то было
% и не надо. И них эмпирическое знание стояло далеко на заднем плане. Вот. А нам-
% то надо. Наша современная наука это наука эмпирическая, экспериментальная и так
% далее. Нам нужно, чтобы это все с реальностью вот так вот состыковалось. И
% приходится платить таким вот, скажем, ну, какими-то нецельностью картины мира.
% Эйнштейн очень вот по этому поводу переживал. Смотрите, его уравнение такое, ну,
% если записать его в максимально упрощенном виде, будет выглядеть так. Тензор А и
% тензор Б. То есть, тензор А равно тензору Б. Тензор А описывает кривизну
% пространства, чувствуете, это поле, а тензор Б материю, которая вызывает
% искривление. Это самое простое объяснение. Чувствуете, там материя, вызывающая
% искривление, а здесь поле. И будучи мыслителем, ну, как бы ориентированным на
% красивое, на полноту и системность и целостность, он говорит, ну, ведь должно
% быть либо только материальное, либо только полевое. И он склоняется к полевому.
% Первая часть, говорит, он, ну, в смысле, та часть, тензор Б, я не помню, какая-
% нибудь, какая-нибудь часть, это электромагнитное поле, не связанное с
% гравитационным, да, мы даже полевую-то не можем связать. Вот есть гравитационное
% поле, а есть электронное, простите, электромагнитное. И он поставил вопрос, не
% является ли вся реальность, вот эти все взаимодействия, проявлением одного и
% того же поля. Ну, как вот это происходит с объединением электрического и
% магнитного взаимодействия. Вот, он начал, в 25-м году начал работать над единой
% теорией поля. но в своих работах он игнорировал плантовую механику, почему?
% Потому что она опирается на статистическое понимание причинности. Нет, говорит,
% вот пока вы линейную причинность не встроите в плантовую механику, я вообще эту
% науку за, ну, как бы, уже окончательную, за полностью реализовавшуюся считать не
% буду. Вот таким образом. Это как бы, то есть единая картина мира, нет, она
% постоянно ищется и из этого предлагается много вариантов, многовариантная
% концепция действительности. Дальше, ну, линейный детерминизм, вероятностный мы
% объяснили, потребность, системность, универсальность, ну, отчасти я тоже это
% объяснила, да? проблема субъект объектной связи. Начиная с Гегеля, фуху, с
% Гегеля, с Гёте, надеюсь, вы помните, о чём я говорила, или там записали, или
% потом как-то так зафиксируете, исходя из презентации, но, в общем-то, ситуация
% такая, что уже Гёте сказал, субъект объект связан. Не просто свет распадается на
% на эти самые, на спектр, ну, белый свет, да, на спектр, а способность человека и
% ситуации, ну, то есть, можно создать такую ситуацию, когда человеческое
% восприятие увидит это, когда человеческое ситуация, когда человеческое
% восприятие реализует это вот распадание на спектр. То же самое в квантовых
% экспериментах. Можете вспомнить про Каташи и Эрингера, да? То есть, наше
% измерение, наше наблюдение, по сути дела, влияет на реальность. Как это
% понимать? Либо как, ну, как бы, как статистически, то есть, я пока не знаю, но
% вот откроются и узнают, либо как-то чуть ли не магически, то есть, в современных
% некоторых интерпретациях, квантовой механике, очень много говорится про наша
% способность воздействовать на реальность. Мысли, например, да, мы можем
% воздействовать на реальность. Чувствуете, это прям прямой путь обратно в теорию
% пневматической инфекции. Как бы, надо понимать, что происходит в науке. Когда
% вам говорят, ну, это научно, же основа на квантовой механике, вы говорите, да,
% на квантовой механике и еще возрожденческой теории магии. Когда мы действительно
% можем влиять на реальность силы и мысли. Конечно, как всегда, в такие вот
% переломные какие-то такие процессы, как это, не процессы, а переломные для науки
% эпохи, это один из хороших таких, да, очень важных фигур в истории науки. Речь
% идет о, не говорю, пытаюсь вспомнить, значит, Томас Кун, Томас Кун говорил о
% научной революции и периоды нормальной науки. В периоды нормальной науки ничего
% особо не происходит, накапливается на материал, обосновывается вот такое. А есть
% периоды революционные, один из таких был рождение классической науки, но ведь не
% классическая наука, это тоже революция. Так вот, в эпоху вот этих революций
% всегда обостряется проблема случаев. Как нам понимать причинность случаев и
% вероятность? На этом основан спор Нильса Бора Эйнштейна, Альберта, да, на этом
% основано очень многое. Принцип причинности в новой науке, вот в неклассической,
% не отменяется, но опять же обостряется. Как понимать вот то обстоятельство, что
% все-таки вероятности рулят, вероятности рулят. то есть, что мы можем сказать, я
% завершаю свою лекцию, что наука в тот период, в период не классической науки,
% пережила те условия и оснастляла те проблемы, которым был задан большой кризис.
% Но кризис не в смысле ой, все плохо, я вообще слово кризис не рассматриваю как
% что-то, что плохо. Почему? Ну, потому что это не только моя особенность, потому
% что само слово, что означает кризис. Кризис – это переход. В самом слове кризис
% нет никаких аллюзий на какие-то страдания, ничего подобного. Кризис – это просто
% переход. Вот, наука не классики – это в нормальном смысле слова кризис. Переход
% к пост не классики, переход к исследованию хаоса, случая, вероятности, ну, вот,
% просто в ее в остроте. И к совершенно новому образу реальности. Мы тут сразу же
% столкнемся с вами уже с категориями цифровой реальности и тому подобное. Вот. Но
% если как бы завершать нашу лекцию вспоминая романтизм, то я вам предлагаю
% Гейзенберга статью, прочитать ее и осмыслить «Картины природы о Герте и научно-
% технический мир». То есть Гейзенберг это тот, кто, собственно, создавал
% квантовую механику, кто двигал, а он, знаете, был кто среди других
% представителей квантовой механики лидер? Отец, его отцом считали. Ну, кто-то был
% там, не знаю, блудный сын, кто-то был там борец, кто-то был какой-нибудь там
% этот самый диссидент, а вот он был отцом. И поэтому ему как бы, ну, он посчитал
% необходимым и возможным всё-таки обобщить общие достижения неклассической науки.
% И вот этот, как бы, ну, вывод, он отчасти повторяет всё-таки некоторую такую,
% ну, с настроением такое было. Ну, во всяком случае, это не просто какая-то
% оптимистичная картина, да, а всё-таки это картина такого некоторого вопроса, мы
% точно правильно идём. А мы вообще туда вот ходили, в этот наш крестовый поход, и
% мы вот возвращаемся, такие отчасти усталые и немножечко печальные, почему-то,
% потому что мы скорее поставили ещё больше проблем, чем решили прежние. Вот. Но
% это не какой-то пессимистичный вывод, это просто я вас хочу познакомить с
% мировоззренческими такими идеями самих творцов новой науки, неклассического
% этапа.

Новая наука не предлагает одной всеобъемлющей картины, как в античности или классической эпохе, где всё было связано: космос, человек, общество, физика, метафизика. Сейчас --- наоборот: эмпирическая, фрагментарная наука стремится к объединению, но не достигает былой цельности. Эйнштейн пытался преодолеть этот разрыв, предложив уравнение, где тензор кривизны пространства равен тензору материи. Он искал единую теорию поля, где всё сводится к полю, не признавая вероятностную причинность квантовой механики. Он хотел, чтобы за всем стояла линейная, непрерывная причинность --- а значит, рассматривал квантовую механику как неокончательную.

Идея многовариантности моделей возникает из стремления к универсальности. Но каждая такая модель остаётся неполной, предлагая лишь один из возможных взглядов. Отсюда --- расхождение, кризис цельной картины мира. Сюда же относится проблема субъекта и объекта. Ещё Гёте указывал: не только свет распадается на спектр, но и восприятие человека формирует саму возможность увидеть этот спектр. Так и в квантовой физике: измерение влияет на результат. Возникают интерпретации, в которых сознание воздействует на реальность --- почти как в старой магии. Это повод быть критичным: когда говорят, что нечто «основано на квантовой механике», надо спросить --- на какой именно её интерпретации?

Во времена научных революций, о которых писал Томас Кун, всегда обостряется вопрос случайности и причинности. Классическая наука рождалась в такой эпохе --- и неоклассическая тоже. Принцип причинности не отменяется, но понимается иначе: вероятности становятся не случайной погрешностью, а центральным инструментом описания. И это вызывает глубокий пересмотр научных оснований.

Кризис здесь --- не упадок, а переход. Само слово означает поворотный момент, а не катастрофу. Наука этого периода прошла через напряжённую трансформацию: от линейного, устойчивого образа мира к реальности, где правят случай, хаос и вероятность. Она вышла на новые горизонты --- вплоть до цифровой реальности.

И если подытожить с интонацией, близкой к романтизму, стоит вспомнить статью Гейзенберга «Картины природы у Гёте и научно-технический мир». Он, один из отцов квантовой механики, попытался обобщить суть неклассической науки. Его взгляд --- не торжествующий, а задумчивый. Вопрос не в том, правильно ли мы всё сделали, а в том, куда мы вообще пришли. Революция поставила больше вопросов, чем сняла. Но именно это и есть суть научного пути --- не в окончательности, а в готовности к следующему шагу.