Наука эпохи Возрождения. 

\section{Социально-историческая характеристика эпохи Возрождения}

Что это за термин «возрождение» и откуда он взялся? Его ввел еще в
15 веке Вазари в своих жизнеописаниях наиболее известных художников,
скульпторов, архитекторов. Это книга середины 16 века. И вот он говорит о том,
что ренессанс – это итальянское возрождение. Можно, конечно, многие слова было
использовать, но он именно ренессанс использовал от слова «реново» — возрождать, восстанавливать, воскрешать. 

То есть приходится признать, что что-то умерло, что необходимо было воскресить. 
И итальянские
мыслители того времени пытались воскресить тот Великий Древний Рим, который был для
них, ну, их, собственно говоря, национальными предками. 

Мыслители предлагают восстановить
в первую очередь язык Рима — чистую латынь, не испорченный вульгаризмами. 

Некую науку античного Рима — образы архитектуры, изобразительного искусства и так далее. 

И сам город Рим.

% Вплоть до того, что вот, так бы сказать, то, что мы потом в 20 веке встретим в
% качестве итальянского фашизма, да, это же было тоже заложено в Боковозрождение.
% Не в смысле вот все эти ужасные вещи, которые потом фашисты творили, а в смысле
% самой, как бы, идеи и даже самого слова. Тогда тоже слово, ну, как бы, слово
% фашизм, вообще слово фасции, да, фасция – это не только вот то, что в организме
% у нас есть связки, это еще и такие специальные образы изобразительного
% искусства, там, где стрелы, вы знаете, в древнем Риме, я не разместила вам, к
% сожалению, но ничего страшного. В древнем Риме часто встречались такие образы,
% когда стрелы связаны особой такой, ну, не знаю, какой-то повязкой. Вот это
% фасция, значит, это повязка, то есть это принцип единства, принцип соединения. И
% в Возрождении первое мы начали заговорить про важность вот этих фасций, то есть
% единство с древностью, со своей там историей, ну, то есть корни вот этого
% национализма итальянского, который потом, ну, как бы, ну, скажем, приобретет
% несколько иные черты. 

% Да, я уже не, ну, труднопростительные черты, назовем это
% так. Ну, вот они, в принципе, Возрождение, да, были. И самое главное, вот почему
% я заговорила про фашизм-то, потому что 

итальянцы в тот период Возрождения
отталкивались от другого народа. То есть мы будем великими, потому что мы, как
бы, уберем из нашего прошлого, из нашего окружения кого-то, понимаете, другие
народы. 

А что это были за другие народы? Они так и назывались в итальянском
Возрождении народы готики. Готики, тогда готика было ругательное слово. Слово от
имени, не от имени, от названия, самоназвания германских племен, готы, да. Ну,
готы, это были такие элитные военные подразделения у германцев. И,
соответственно, вот всё, что связано с готами, это итальянцев Возрождения.
Бесило, назовём это прям таким простыми словами. Почему? Потому что везде было,
во всех текстах вот это вот случалось. Нас обокрали варвары. То есть Древний Рим
уничтожили вот эти германцы, да, эти вот плохие.

Фу, интересная история, да, вся
вот возвращает, прикольно даже. Ну вот, нас обокрали варвары. Вот этот вот
звучал лейтмотив, и надо вернуть своё прошлое. Это, надеюсь, понятно. Для других
народов Европы это понятие носило, конечно же, несколько иной смысл. Те же самые
немцы ничуть не возражали против того, что в своё время германцы захватили Рим,
естественно.

Национальное самосознание и обоснование культурного вот этого
прошлого оказывается не на первом месте у других народов. Имеется в виду у
немцев, будущих немцев, у французов там, ну вот и так далее, да. В германских
землях прозвучал у художника Альберта Дюрера. Речь идёт вот об этом художнике.
Вы наверняка знаете его вот эту меланхолию, знаменитую литографию. Так вот, у
него начало фигурировать понятие «дивидер вахм». То есть дословно это снова
просыпаться, снова стоять на страже.

Для северного возрождения
характерно беспокойство о том, что надо возродить христианство, древнее
христианство, то первичное.

То есть вот это
вот вам надо отчётливо понимать. Италия – это возвращение своей национальной
древности. А для других народов Европы это возрождение, ну понятно, античных
традиций, но античных-то каких? Античных, как ранних христианств. 

Научный
оборот, термин «возрождение» ввёл французский историк Мишле. Это середина XIX
века. До сих пор существуют споры. Возрождение было только вот в Европе или
других народов, другие географические области, другие народы, цивилизация-то
коснулась. Ну, я не буду сейчас в этот спор погружаться. Существует много точек
зрения, можете придерживаться любой, что называется. 

Ну, также спор ведётся о
том, а нужно ли вообще возрождение выделять из эпохи Средневековья. Потому что,
например, для Северного возрождения это не так актуально. Ну, мы не видим здесь
прям принципиального различия, какого-то кардинального между Средневековым,
например, искусством и искусством возрождения. Тоже есть смысл в этом. Но мы
выделяем и мы характеризуем эпоху возрождения вполне себе такой простой,
хронологической рамкой. XV-XVI век. Опять же, это не единственная рамка в
истории искусства. Это XIV-XVI век. Некто, кто-то говорит XV-XVII век. В общем,
есть варианты. Мы придержимся XV-XVI именно потому, что с позиции истории науки
это более рационально. Понятно, да? То есть вот то, что касается историографии,
в первом вопросе её немножечко нужно затронуть. 

\subsubsection{Социально-исторические предпосылки кризиса Средневековья}

Европа в этот период переживает
сильнейший кризис. То есть это был такой социальный стресс. Начался он ещё в XIV
веке, то есть в эпоху позднего Средневековья. И достаточно упомянуть только
Столетние войны, да? Это когда Англия и Франция больше, чем сто лет воевали.
Почти перманентно. 

% Чтобы вам было понятно, о чём речь, вот феномен Жанны Дарк
% связан со Столетней войной. И надо сказать, что в её смерти, так просто к слову,
% конечно, позорную роль сыграли университеты, а именно Парижский университет. Вот
% так вот. То есть как бы они, по сути дела, приговорили Жанну Дарк. Потому что им
% была поставлена задача провести экспертизу её богодухновенности. Ну и они по
% политическим причинам вот такую экспертизу провели, что, мол, вот так вот.
% Ладно. Это вот то, что спритшествовало. 

Но второй момент. Происходит закат рыцарства. 

Ну вот видите, тут картиночку даже я нарисовала, что просто наглядно
было. Крестьяне забивают рыцаря. И это вот как раз, собственно говоря, весь
14-15 век. Это происходит то, что в Средневековье не было. Мы не знаем в
Средневековье ни одну крестьянскую войну. Конечно, какие-то выступления были.
Возможно, я абсолютно легко допускаю, что пару-тройку десятков рыцарей-крестьяне
замочили. Ну, за какие-нибудь плохие там, плохое отношение и так далее. Но это
все, что называется, частные моменты, которые всегда случаются. 

Крестьянских
войн не было. Но не было и поводов. Мы говорили о том, что, в принципе, все
достаточно социально было, ну, как сказать, гармонизировано. Не было особых поводов возмущаться. 
Что же произошло в 14 веке? Дело вот в чем. В
предшествующую эпоху, дело было в Средневековье, как вы понимаете, рост
населения был большой. И он создал проблему безземелья. Ну, то есть уже не
хватало обработанных земель для крестьянства. И поэтому, собственно говоря,
крестьяне-то со своих наделов и не уходили. Даже, например, если их рыцарь, ну,
был жлоб, если он, условно говоря, там, с оброком лютовал или с какой-нибудь
рентой слишком уж дорогую брал, ну, особо уходить было некуда, потому что
обработанных земель, имеется в виду, там, лес вырублен, да, там, земли спаханы.
Ну, в общем, это сложный процесс. Их было недостаточно для того количества
людей, которое постепенно, ну, вот, как бы появилось. А я еще раз говорю, рост
населения отмечают. Следователи был значительный, причем среди крестьянства.

Вот. Но в 1346-1348 году случилась серьезная эпидемия, черная смерть, да, чума,
великая чума и так далее. И демографическая ситуация резко изменилась, то есть
людей стало в разы меньше. Естественно, крестьянство тут, ну, как бы в первых
рядах, люди умирали просто деревнями целыми, как вы понимаете. И в этих условиях
средневековая система вот этой вот вассальной зависимости дала сбой. Почему?
Потому что земли-то появилось много, ну, люди-то умерли. И крестьяне начали уже
уходить от тех, ну, например, там, тех сеньоров, которые, ну, там, плохо себя
вели, да, ну, что-то еще. Или просто на лучшие земли где-то там росло что-то
лучшее. Ну, то есть это понятно вполне. А поскольку они не были никакими там
крепостными, никто это, в принципе, им не мог запретить. Но ведь никого это и не
радовало. Представьте себе, вы живете, вы сеньор, да, и у вас все крестьяне
взяли и ушли. А где взять новую? Ведь вассальная зависимость означает, что люди
пришли, принесли вам, ну, своего рода там, крестьяне клятву верности, ну, то
есть поклонились, и сказали, все, ты нас защищаешь, а мы тебя кормим. А кто, кто
ведет, да, что называется? И в этих условиях дворянство начинает лютовать во
всех практически странах. То есть, например, значит, с какого начну? С Франции,
да? Здесь вышли указы, указы по поводу того, что дворянство уже имело
возможность закреплять крестьян на земле, ну, законодательно, то есть, ну,
условно говоря, держать силы. Держать силы. То есть нужно было силовыми
политическими методами заставить крестьян работать, ну, вот на этого, там,
сеньора. Более того, было разрешено налог, не налог, простите, аренду делать
какую угодно дорогую. И в середине 14 века все эти проблемы, которые связаны
именно с закреплением крестьян на земле, силовыми методами. Более того, крестьян
нужно было, они уже стали считаться беглыми, их можно было поймать, и уже тех
вот пойманных крестьян можно было превращать практически в рабов. То есть эти
новые указы, они в обилии вышли по всей Европе, эти новые указы разрешали к ним
относиться как к рабам. Понимаете, да? Ну, естественно, это... А крестьяне были
свободолюбивы, они, знаете, не обыкнутся за шубанным народом. В том-то и
проблема, что они-то, что называется, жили-то по своей воле, работая на этих
самых, на сеньора. Ну, как бы, ну, не то чтобы им там очень нравилось, но тем не
менее, это же было как бы их решение. А здесь это не их решение, это уже
вынужденная какая-то ситуация, да еще и в более худших условиях. И крестьяне
поднялись массово. 

Первое, 14 век, Жакерия, Жакерия во Франции. Они штурмовали
рыцарские земли, убивали всех подряд. Дворян, дворянских слуг, там женщин,
детей. Сожжено более 100 замков. То есть это очень масштабное выступление. В
конце концов, дворянам удалось собрать ополчение и выступить против Жаков,
Жакерия, да? В июне 58-го года восстание было разбито. Но с тех пор, как бы,
крестьяне поняли, как надо сражаться. И вот в Англии восстание у Улта Тайлера,
когда парламент принял закон о том, что батраков, то есть наемных крестьян и
городских чернорабочих можно было заставлять работать за абсолютно нищенскую
зарплату, которая была до Великой Чумы. И более того, любой дворянин мог
схватить беглого крестьянина и заставить, опять же, работать на себя. Народ
оказывал сопротивление, парламент приходилось снова повторять эти требования,
снова и снова выпускать указы. 1481 год. К этому добавился сбор чрезвычайного
налога на военные нужды, шла война, и в ответ поднялось большущее восстание. То
есть крестьяне дошли до Лондона, они зашли в Лондон, потому что им открыли
ворота городская, ну, горожане такие, имеется в виду, такие же там, наемные
рабочие. Восстание возглавил Уот Тайлера, если я еще не сказала, вот,
пожалуйста, запишите Уот Тайлера. Они дошли до Лондона, и в ходе переговоров с
королем, то есть они встретились с королем, Уот Тайлер схватил луж короля за
усы, и его убили. Его убили, и это поднялось просто народное возмущение до
предела. Стотысячное крестьянское войско столкнулось с сорокатысячным войском
рыцарей. Ну, понимаете, да, масштаб. Ни одна война в эпоху Средневековья такой
масштабной не была. Вот это было уже, что называется, всерьез. Завязалась битва,
да. Восстание было в итоге подавлено. Все-таки, ну, сорок тысяч военных,
профессиональных, да, и сто тысяч вот как бы просто людей, которые просто, ну,
как бы разозлились. И что характерно, и что нужно запомнить после этих
восстаний? Почему я так подробно о них говорю? Потому что, понимаете, изменилась
не только система вассальной зависимости, не только, ну, скажем так, поссорились
внутри социальных систем разные слои населения. Важны были последствия
идеологического характера. А они до сих пор, вот они вот тогда были заложены, а
до сих пор работают. Речь идет вот о чем. Еще один лидер восставших крестьян был
священник Джон Болл. Он, интересно, там интерпретировал некоторые богословские
идеи, и постепенно все это трансформировалось у него в очень простую идею. То
есть, это как бы его идея, да, вот чувствуете его идея. Христос заповедовал всем
быть бедными. Ну, то есть вот это вот такая священная бедность. Возникла идея
священной бедности. А кто же нарушает вот эту идею священной бедности? Конечно
же синьор, конечно же богатые люди. Имеется в виду не только рыцари, но и
священство, там, короли, ну, в общем, все, кого-то там... Так вот, убивать
богатых богоугодно. Вот эта идея, она таким образом как-то укрепилась хорошо в
сознании людей, что они не чувствовали какой-то двойной тут, ну, так, двойных
стандартов. То есть, смотрите, когда богатые люди убивают бедных, это плохо,
когда бедные убивают богатых хорошо. Я думаю, вы чувствуете здесь какая-то
система двойной морали. То есть, еще раз, Джон Болл подчеркивал, что вот это
вот... Ну, то есть расправляться с любыми там богатыми людьми, скажем так, вот
такими способами восстанавливать какое-то попранное свое там достоинство или
честь, или там, не знаю, экономическую, несправедливость, это богоугодное дело.
Ну, разумеется, это, если мы так чуть-чуть сравним, то, конечно, это
антиевангельская история и даже антиветхозаветная, хотя там вообще ничего не
сказано, не про какую там... вообще речь идет о бедности, богатности, но
неважно. Вот. Но появилась новая система, еще раз, двойных стандартов. Угнетение
или такие же расправы со стороны богатых – это грех, безбожное дело, а точно
такие же расправы со стороны бедных – это богатый год. крестьянские восстания 14
века надолго утвердили в сознании европейцев, то есть сегодняшнего дня, что
проливать свою и чужую кровь за социальное уравнивание – это по-настоящему
крестьянское дело. И вот в Германии это, что называется, развернулось в полную
силу. Германия, когда Священная империя, состояла из разнородных областей, и
надо сказать, что в Германии вот этот народный протест, он имел более такой
национальный характер, причем потому что все высокие слои населения это были
немцы в этой Священной империи, а вот слои более низкого статуса – это были
Чехи, потому что тогда Чехословакия, Чехия входила в Священную империю. И
выступление известно под именем «Гусицкие войны». Ну вот, собственно говоря, я
вот сейчас поменяю слайд, потому что здесь рассказано уже про это. «Гусицкие
войны». Гус был проповедником, ну магистр Пражского университета, вот он в Праге
выступал с проповедями, и был очень популярен, на его выступления собирались,
очень, скажем, много людей поддерживали его, и эта популярность ему, что
называется, вышла боком, потому что папство объявило его еретиком, и стало как
бы обвинять людей в гусицкой ереси. Он был обвинен, ой, простите, в английской
ереси, вот так это тогда называлось, английская ересь. Ну, я сейчас не буду
углубляться, почему именно английская, сейчас не в этом. И потом это уже было
позднее, когда протестанты появились. Просто вот английская ересь тогда как-то
была связана с идеями протестантизма, но не в немецком варианте, пока еще там.
Ну, есть специфика, она нам не так важна. В итоге Яна Гуса убили, сожгли, и это
подняло, опять же, народный пункт до неимоверных таких вот масштабов. Огромные
массы восставших собрались на горе Табор, Табориты, вот они так назывались, и
они основали даже свое государство, которое назвали, ну, конечно же,
христианским, естественно. То есть сопротивление Таборитам, с одной стороны,
папа назвал делом церкви, а сами Табориты назвали это делом христианским в
полной мере. Ну, то есть чувствуете, вот все, это разложение христианского
идеала. То есть не только Dansha Ш besser ими 276, но это еще и разложение
христианского идеала. Никто не понимает, а что означает быть христианином. Папа
говорит, надо вот это сделать, то есть уничтожить Таборитов. Табориты говорят,
надо уничтожить в том числе папу. То есть получается какая-то совершенно
неотчетливая картина, а что есть в христианстве. В итоге рыцари терпели очень
много поражений. И только в 1434 году табориты потерпели поражение в битве при
Липанах. Знаменитая такая была битва. 

Так вот, еще раз повторюсь. Все народные
выступления эпохи Возрождения шли под религиозным лозунгом. Лозунг
восстановления Божьей правды. Происходило смешение религиозного и социального
начала. Когда сегодня говорят люди, что главное в христианстве это социальные
идеи, идеи социального равенства, этики. То есть социального взаимодействия. Не
взаимодействие человек и Бог, а взаимодействие между людьми. То это прямое
наследство эпохи Возрождения времен христианских войн. И вообще это ситуация,
которая ведет прямую к секуляризации. Очень хочу, чтобы вы это понятие поняли.

Секуляризм. Это антирелигиозная доктрина, согласно которой движение к лучшему
обществу, социальные, политические реформы не должны основываться на религиозных
установках. То есть чувствуете, вначале как бы религиозный взгляд сместился на
человека и взаимоотношения в обществе. А потом стало понятно вполне себе такие
вещи простые. А простите, а при чем здесь вообще религия? Ну мы разберемся и
сами тут, как говорится, да, тут политика имеет значение, экономика и так далее.
А религия здесь при чем? Чувствуют лишние. Ну то есть действительно, это очень
простой процесс был. Долгий, но простой. Стоило убрать из религиозных отношений
вот это главный ось человек-бог, собственно говоря, она, это осталось просто
присказкой, такова божья воля. Ну это можно сказать по любому полу. Чувствуете?
Ну вот просто. Или там, это божья правда. Все. По сути дела это секуляризм. Само
слово секулярный, секулум, секулум, это древнеримское слово, означало столетний
цикл. А имеется в виду, что время от времени императоры, имеется в виду
древнеримские, да, императоры означали, ну см назначали такой вот секулярный
год. Имеется в виду, что это когда мы все сначала начинаем, все сначала говорим.
Вот теперь все по-новому пойдет. Вот эта идея секуляризма, а вот теперь всё по-
новому, она как раз зародилась, то есть антирелигиозная, принципиально
антирелигиозная, она зародилась как раз из этого нового мировоззрения.
Чувствуете, прям в пределах религии, это что такое? Дальше. Второй момент. Вот
крестьянские войны, я надеюсь, вы что-то там усвоили. Дальше. 

Кризис именно
рыцарства, самого рыцарства. Дело в том, что рыцари ещё раз подчёркивают, и
неустанно подчёркивают, а рыцарское сословие. Но в 15 веке появляется
огнестрельное оружие. Ну, конечно, это вот смешно ещё огнестрельное оружие, но
тем не менее оно уже в обилии появляется. Его изобрели арабы и китайцы.
Соответственно, армия изменилась, сам состав армии серьёзно изменилась. Уже
артиллерия имела наибольшее значение и наёмная армия. Ведь что такое рыцарская
армия? Это небольшой отряд, где каждый, значит, бьётся за собственную доблесть,
дисциплина хромает, потому что каждый... То есть управлять, смотря на,
практически невозможно, потому что, собственно, сражение – это поединок личных
доблестей. Вот каждый сражается просто в силу своей личной добленности. Нет
особой стратегии, нет особой тактики. Главное – просто реализовать это своё
умение сражаться. Тогда, как представляете, с оружием такую ситуацию, да, это же
хаос. Нужно выстроиться в линеечку, нужно одновременно залп сделать и так далее.
Поменять дальше раз для заряда. Нужно было выступить, нужно было, ну, то есть
поменяться, да, там строй меняет свою, ну, то есть вот эти вот положения меняет.
Это всё требует дисциплины. Совсем другой подход. А вот такая большая наёмная
армия, то есть война принимает характер, когда, по сути дела, не какие-то там
шахматные партии, да, там разыграла, была разыграна, и победитель тот, кто
сделал более хитрый ход. А на самом деле, ну, вот война приходит к тому образу,
который мы имеем сегодня. Чем больше противника убито, тем больше, тем, ну, как
бы, вернее твоя победа. То есть просто мясорубка. То есть сражения стали просто
мясорубками. Нехитрыми какими-то вот ходами и поединками личной доблести, когда,
в принципе, можно было и не будучи даже раненым, понять, что вот доблесть за
этим человеком. Понимаете? Не говоря уже о том, что это небольшие, по сути дела,
армии сражались. А здесь они и большие, и действительно, это артиллерия, а любое
ранение это в условиях, где нет антибиотиков, сами понимаете, практически
стопроцентная смерть. Вот такая вот ситуация получилась. Мясные уже вот войны,
они как раз в эпоху возрождения. Но куда деваться рыцарскому сословию? Оно же
воюет, а оно не нужно. Оно не нужно королям, по большому счету, с этого времени,
потому что королям нужны просто деньги на наемную армию. Конечно, рыцари в
наемную армию не шли. Более того, считалось позорным использовать
дальнестрельное оружие. Они с трудом выносили-то луки, они даже из луков не
стреляли, только англичане стреляли из луков. То есть только мечом можно
сражаться и копьем. То есть в личном контакте. А вот это вот издалека стреляет,
но вот эрр-цари, считай, ниже своего достоинства. Более того, это же было даже
до 19 века. В Германии, я просто сейчас сходу вспомнила, в Германии, когда армия
меняла свой состав, то есть конницу убирали, и появлялись моторизированные
войска, танки первые появлялись, машины. В общем, то очень трудно было офицеров
заставить пересесть на какие-то машины, которые стреляют издалека. Одно дело
конница, да, вот мы скачиваем, и вот ближний бой, по сути дела. А издалека
стреляют из пушек, это настолько стрёмно, даже в 19 веке было. И они соглашались
переходить только в авиацию. Вот этот как бы рыцарский дух, он был настолько
силён, что он даже до 19 века-то подлился. В армию нанимали, конечно же, ну тех,
кто из более низких сословий. Рыцарям делать нечем. В этих условиях они либо
начинают заниматься просто бандитизмом, создают просто банды. Банды, которые
просто грабят. Это уже не та средневековая война со всякими хитроумными штуками,
там, клятвами, договорами и тому подобное. Это просто обычный грабёж. И одна из
форм таких грабежей, кстати, было пиратство. Его тоже замутили рыцари. Ну вот,
это примерно так происходит в эпоху средневековья разложение феодального идеала.

\subsubsection{Реконкиста и формирование испанской абсолютной монархии}

И что такое реконкиста? Значит, вы
должны помнить из прошлой лекции или просто знать это, что испанцы, Испания
почти полностью была под властью мавров. Мавры — это арабы, мусульмане. Но
маврыми даже называли ещё и евреи, которые там тоже переселились. В общем,
Испания под властью мавров, под властью мусульман. И Испания не сдавалась.
Северные области Испании оставались в руках христианских монархов. Но тем не
менее, спустя какое-то время реконкиста остановилась. Реконкиста — это
возвращение себе земель, освобождение её земель от мавров. Но где-то в 14-13-14
веке реконкиста в конце уже немножко остановилась, приостановилась. И это
вызвало хаос в Испании. Потому что, как и везде, как и всегда, общество
консолидируется борьбой. И Испания была одной из самых консолидированных
общественных систем. Войной с маврами. Но когда она остановилась, начало всё
рассыпаться. Гранды — это имеется в виду феодалы испанские — объявили войну
королям. Там начались междоусобицы жуткие. Крестьяне перестали подчиняться.
Сказали, что одно дело, когда мы воевали вместе под руководством Грандов, это
да. А вот другое дело, просто что, работать на Грандов? Нет. Другое. Это как бы
всё достаточно было хаотично. И на фоне вот этого хаоса выходит на передний план
истории Фома Тарпимада, настоятель одного из монастырей, духовник, наследница
Костильского престола. Изабелла. И этот вот человек умудрился сделать, конечно,
Испанию великой. Сделаем Испанию снова великой. Это Тарпимада. Он сделал это с
помощью испанской инквизиции. В первую очередь. То есть он возглавил процесс
объединения Испании вокруг королевской власти. К тому же очень удачный брак
случился. Брак по любви, по большой. Когда супруги были такими, как сказать,
поддерживали другое во всём, усиливали другое. В общем, вот это вот единство,
когда два супруга, король и королева, и Фома Тарпимада сделали Испанию
величайшим игроком на какое-то время, величайшим игроком европейской политики. И
более того, конкиста развернулась под управление полную силу. То есть
чувствуете, что Тарпимада – это жесть на самом деле. Это такая инквизиция,
которая, конечно же, не знала итальянская инквизиция. Сравниться могла только
потом уже всякие разные… У протестантов были вот эти вот жестокие такие суды.
Они не назывались инквизиторские, но они были тоже инквизиции по своим фактам.
Вот. И надо сказать, что, конечно, религия здесь была дело второе. Ну, понятно,
что официально было возглавлено движение против вот таких неистинных христиан.
Что имелось в виду? Ну, многие арабы и евреи хотели остаться там, где они живут,
то есть в Испании, на Иберийском полуострове. Они хотели там остаться, но
остаться можно было только христианинам. И они принимали христианство, ну, так
скажем, возможно, допустить, что для видимости. И вот именно их веру, типа мы
будем испытывать их веру, но когда мы читаем документы, инквизиция, конечно, в
меньшей степени обращала внимание на этих новых крещенных христиан, а гораздо в
большей степени обращала внимание на всяких там бунтующих грантов. Ну, то есть
вот королевская власть усиливалась за счет инквизиции, в первую очередь. В итоге
Испания стала страной, которая серьезно повлияла на эпоху великих географических
войск. Вот еще одно понятие, которое вы должны очень хорошо понять и разобраться
с ним. 

\subsubsection{Эпоха великих географических открытий} 
Надо сказать, что самой крайней
точкой европейской земли на западе являлась земля королевства Португалия. Земля
королевств даже Португалия, но не было одного королевства, королевства
Португалия. Далее простиралось море мрака, и туда, в общем, не плавали. Не
плавали в Атлантический океан, потому что особые течения, особые ветра, с ними
невозможно было справиться с прежним типом кораблей. Но португальцы создали
каравеллу, судно с треугольным парусом и рулем. Вот все, что мы видим в фильмах
про пиратов, это вот каравеллы. То есть они были маневренны. И вот португальцы,
то есть они могли идти против ветра, вот их самая главная фишка. Португальцы шаг
за шагом учились плавать против ветра, они освоили вот эту технику. Куда они
плавали? Они плавали вдоль западного побережья Африки. Тогда Португалия не была
еще очень сильна, она вот изобрела ту самую каравеллу. Плавая вдоль западного
берега Африки, они достигли зеленого мыса на том берегу, как вы понимаете, это
южная точка, в 1445 году. Португальцы купцы, португальские купцы, которые
вкладывали, кстати, деньги вот в эти походы морские, они искали же прибыли, и
прибыль очень быстро нашлась. Они начали отлавливать население Африки, негров, и
продавать их на вот яморках Лиссабона. Так что европейская работорговля
началась, конечно же, не с США, она началась с португальцев. И один из первых
географов эпохи возрождения Паула Тосканелли, вот его карта, говорил, что
поскольку земля имеет форму шара, то отплыв из Португалии на запад, можно
достичь Китая примерно за месяц плавания, по его расчетам примерно за месяц
плавания. Его письмо португальскому монарху в том числе прочитал Христофор
Колумб, купец из Генуи. Он загорелся идеей реализовать проект Тосканелли. Это
было, знаете, ну вот вроде бы да, как бы он купец. Вот это один из тех купцов,
которые были, ну, что называется, предпринимателями. Не купцы-лавочники, а
купцы-негацианты, то есть предприниматель. Это один из тех ликов капитализма
будущего, которые, ну, вот они даже до сих пор сохранилось, когда, ну да, вроде
я поехал-то за специями, то есть мне, конечно, деньги важны, но риск важнее,
важнее успех предприятия. Конечно, Колумб в этом смысле, понятно, что он до
конца жизни думал, что он открыл Индию, и понятно, что там еще многие всякие
вещи, но это великая фигура, именно великая с точки зрения, давайте рискнем, а
давайте попробуем. Он загорелся, в общем, этой идеей. Португальский король Жоан
II отказался поддержать Колумба, ну, денег на это нет, конечно же, и Колумб
обратился к испанской короне, а испанская корона в то время уже сильная. Она
может себе позволить такое путешествие. Португальцы тогда еще этого, ну, не
могут себе позволить. И вот он получает в 1492 году титул «Адмирал моря-океана»,
«Адмирал моря-океана», как видите, да? Вице-короля всех земель, которые будут
ему открыты, им открыты, и вот он отправился через океан на запад. Открытие
Америки, как вы понимаете. Всю эту историю с названием, что там Америка в
Веспучии, я рассказывать не буду, она широко известна. Если неизвестно,
пожалуйста, поинтересуйтесь, почему Америка названа Америкой. В ответ на
экспедицию Колумба португальский король уже другой, Мануэл I, отправил в
плавание вокруг Африки. Ну, потому что стало немножечко как бы, ну, так завидно.
Ну, завидно, конечно, не с точки зрения, знаете, как космическая гонка. Нет,
тогда еще вот такие вещи не работали. Было, что называется, завидно. Ведь
драгоценности пошли оттуда, всякие разные из Новой Индии, то есть из Америки.
Золотишко пошло, какие-то там интересные вещи. А они поплыли, португальцы
поплыли вокруг Африки в ответ, что называется. И вот плавание Колумба,
Васкадагамы, Магеллана – это и есть эпоха великих географических открытий. Это
морская гонка. Она была, ну, в такого, с одной только стороной вопросов
технологического престижа, а во вторую – это все-таки, ну, поиск богатств. То
есть люди ехали, чтобы найти богатство. Это купеческие авантюры. Очень
интересное время, очень опасное, очень интересное. Но я вот уже упомянула про
два ликокапитализма. Да вот смотрите, первый ликокапитализм – это когда опасно,
и деньги – это вот мерило удачи в этом опасном предприятии. И на самом деле
такой капитализм сегодня существует. Это вот тогда, когда для человека важно не
деньги заработать, а типа получится, не получится. Ну, я не буду сейчас как бы
погружаться. А есть другой ликокапитализма, который как раз… Как бы найти
побольше бабла, только чтобы безопасно. И вот финансовый капитализм, который
завьется позднее, намного позднее. Это вот как раз другой ликокапитализма.
Потому что, ну, знаете, с точки зрения философии, крайне глупо кричать на всех
углах и на всех заборах, писать капитализм – это дерьмо. Это называется
«большого ума не надо». А вот чуть-чуть больше ума требуется увидеть, ну, в
любом феномене что-то, что, ну, как бы неоднозначно плохое. Никогда не бывает
ничего однозначно плохого. Вот серьезно. И поэтому я вас призываю к тому, чтобы
вы видели корни таких даже самых неприглядных, возможно, каких-то моментов,
видели во всем корне, как это проявлялось, как зарождалось, почему так. Люди же
не делают ничего просто потому, что хотят какого-то говно сотворить. Извиняюсь,
я не одобряю всякие разные вульгаризмы в речи педагога. И это просто оборвалось.
Так вот, капитализм – это после эпохи Возрождения, но основы, как мы понимаем,
здесь. И на основе этого первого лика капитализма, я почему его упомянула?


\subsubsection{Культ Фортуны}

Потому что очень важно рассказать про культ Фортуны. Очень такой момент, крайне
важный для эпохи Возрождения. Тема Фортуны. Фортуна – это слепой случай, слепая
удача, такая древнеримская богиня, но речь уже даже не о богине, речь идет о
неком явлении. Фортуна как слепой случай, который может либо тебя убить, ну или
там, не знаю, разрушить твою жизнь, либо вознести тебя на какие-то там
социальные вершины. Фортуне в Боговозрождении посвящены трактатам. Она постоянно
упоминается в деловой переписке, в письмах личного характера. То есть для
Возрождения было характерно призывы активно противостоять власти Фортуны. То
есть Возрождение при всей моей личной нелюбви к этой эпохе, ну вы не обязаны ее
разделять ни в коем случае, да, но тем не менее, это шикарная эпоха там, где
речь идет о каком-то вот вызове, вызове по отношению к любым ограничениям. И
первое это ограничение, это же судьба, фортуна. И представляете себе, кучу людей
поставила свою цель, а как можно победить фортуну? Вопрос. Если вы сейчас
внимательны, то вы поймете, как ее в итоге победили. Ее отменили. Принципами
предельного линейного детерминизма. Но скажу вам просто, как человека-философ,
такое себе. Это не победа. Вот возрожденцы еще предлагали победу, а то, что
потом в новое время, ну опять же, это моя оценка. Фортуна для меня стала
единственным богом, говорит один из флорентийских купцов, вот этих вот первых
авантюристов, Банакорса Петти. Ну не надо его запоминать, дело не в этом. То
есть некая окончательная причинность всегда, конечно же, остается за богом. Тут
они по-прежнему религиозные люди. Но вот внутри реальности, внутри своих
действий мы боремся с фортуной. То есть, смотрите, в некотором смысле новая
элита, купеческая новая элита, она появляется в противовес старой элите, имеется
в виду рыцарству. Она как бы становится более актуальной. Но она по духу еще та
же самая рыцарская история. То есть, средневековые рыцари жаждали воевать, так
скажем, с демонами, с какими-то там такими силами инфернального характера, ну
или между собой, как минимум. А купцы Ренессанса тоже хотели бы воевать, но с
фортуной. И только позднее купцы уже предпочитали с этой фортуной договориться.
Ну ладно. Возрождение видит фортуна одновременно и добрым, и злым началом. И так
формируется, на самом деле, антологическая основа для морально-нейтральной
оценки действительности. Ну, всех, что называется, под властью фортуны ходим.
Причём совершенно неважно, как вы воспринимаете фортуну. Может, богом, а может,
просто такой антологической характеристикой. Онтология, вспоминаем, дисциплина,
которая занимается вопросами бытия из самых общих свойств сущего. Ещё один
момент, связанный с культом фортуны. Христианская этика во все времена, имеется
в виду с древних времён и вот до этого времени, не поощряет рост богатства, рост
прибыли. Была в экономическом смысле концепция справедливой цены. Мы сейчас её
не будем разбирать, но вы поинтересуйтесь. Ростовщичество вообще не одобряется
ни в коем случае, потому что ростовщичество, чем занимается ростовщик? Он
продаёт время. Он дал деньги, но он же не деньги продаёт, продаёт время, а время
принадлежит богу. Соответственно, что делает ростовщик? Ну, извините меня,
просто ворует у бога. Но дело как минимум неблагодарное. Но в эпоху Возрождения
ссылка на риск подвергнуться коварству фортуны, это становится достаточным
основанием для оправдания прибыли. То есть, смотрите, я рискую, я рискую жизнью,
я поехала на корабле и вообще непонятно куда, в какое-то море мрака. Я даже не
знаю там вообще, я даже не знаю, земля круглая или нет, доплыву ли я. Ну так,
доподлинно-то я же не знаю, до меня же никто не плавал. Я подвергаюсь риску
просто получить от фортуны так, что мне мало не покажется. Но в итоге я получаю
прибыль, если я всё-таки выигрываю. Согласитесь, это типа, ну, достаточное
оправдание для прибыли. Вот это первое, ну, как бы первый такой
мировоззренческий шаг в сторону капитализма. Ну, ещё с таким человеческим лицом,
назовём так. Это ещё не капитализм курильщика, это ещё так. Ну и, кстати, во
власти фортуны, как что интересно, там не здоровье, не какие-нибудь там
деторождения, а всегда это богатство, политическая реальность и социальное
положение человека. Вот чем она там фортуна занимается. Ну, вы понимаете, да,
что как раз здесь в основном там авантюристы, вот они так, ну, в хорошем смысле
этого слова, работают. За 30 лет плаваний «Каравелл» экономическая карта Европы
изменилась до неузнаваемости. 

Силовые линии поменялись, новые города появились.
Это вот Лиссабон, Португалия, да. Ну, не то, чтобы они новые были, Лиссабон
очень старый город, но он стал портовым важнейшим городом. Амстердам, это уже
выше, да, севернее Европы. Антверпен, всё это Нидерланды. Постепенно зарождается
капиталистическое общество, которое со стороны вот экономических трансформаций
связано с появлением мануфактур. То есть это большие мастерские, где
производственный процесс поделен на десятки мелких операций. А, соответственно,
вот эта машинизация ручного труда, имеется в виду машинизация пока в кавычках,
машин еще нет. Мануфактура. Смотрите, ману от слова мануальный, да, ручной.
Ручная фактура, ручное производство. Самую простую работу делали женщины-дети,
более сложные подмастерия, мастера руководили общими работами,
производительность труда в мануфактурах была гораздо выше, чем в мелких цеховых
мастерских. Где мануфактуры были? Конечно, в тех самых бургах, про которые, ну,
мы с вами говорили, в средневековые города, которые возникают уже в позднем
средневековье. Поэтому, конечно... В конце концов, понимают, где проходят
торговые пути и осваивают мореплавание. Ну, не говорю, что прям тут же они с
лошадейно перескочили на корабли, но тем не менее, да. Разбойничье ремесло
оказалось столь выгодным, что вот этих вот обедневших дворяний все больше и
больше становятся пиратами. Джек Воробей, это, конечно, простите, капитан Джек
Воробей, это, конечно же, 17-18 век, но тем не менее, его вполне можно было
представить и в веке 16. Так вот, в 1578 году, 16 век, да, пиратский капитан
Фрэнсис Дрейк нашел дорогу в Тихий океан, ограбил порты на всех побережьях,
захватил корабли, перевозившие золото с американских рудников. Надо сказать, что
это золото обеспечило, собственно говоря, Испании, а потом отчасти Португалии.
Ну, Португалии другое золото, не американское. Ну, в общем, вот эти морские
державы на долгое время стали самыми крутыми политическими игроками благодаря
американскому и другому золоту. Ну, вот им приходилось бороться с такими новыми,
значит, этими проблемами, имеется в виду пиратства. Добыча была выгодным делом,
и вот Дрейка, как вы, наверное, знаете, Елизавета, королева Елизавета I,
получив, конечно же, свою долю прибыли, произвела ворцовское звание, удостоила
невиданных почестей. Ну, так было всегда с Англией. Вот что, как говорится, кто
гадит, так это англичанка, что называется, тут даже не приходится сомневаться.
Ведь по большому счету вот это вот при всей нашей любви к пиратам,
кинематографическим, как мы понимаем, да, всё-таки пиратство загубило, загубило
на корню вот этот первый лик капитализма. Потому что после успеха Дрейка, после
такого одобрения королевской власти, началась эпоха пиратства. 

\section{Итальянское Возрождение. Феномен ренессансного гуманизма}

Начинаем второй вопрос. Он сложнее. Речь идёт об
итальянском возрождении. Надеюсь, мы с вами осилим всю ту как бы новизну
понимания итальянского возрождения, которое я вам буду предлагать. Я не буду вам
рассказывать про Леонидовича, какой он был великий, какие замечательные Рафаэли
были в то время. А вы обратите внимание, как только начнёшь говорить по
возрождению, так вот в обычном смысле, ой, прекрасный Рафаэль. Ну да, неплохой
художник. Я вам говорю, как искусствует. Неплохой художник. Леонардо не закончил
ни одной картины. Он гораздо интереснее, как изобретатель. Вот. Понимаете, не с
художников мы будем с вами начинать, а, конечно же, с мыслителей. Феномен
ренессансного гуманизма. Итальянские мыслители, напоминаю, считают свои задачи
создать основы для нового совершенного общества, которое бы повторяло старое
общество, которое они видят совершенным. Имеются в виду древние гимны. Пытаясь
восстановить этот золотой век Римской империи, они обращаются к периоду
античности. И, собственно говоря, вот что значит вернуться в прошлое? Значит,
что значит исследовать это прошлое и вернуть в настоящее? Это работа с текстами.
Да, всегда работа с текстами. Это только во вторую очередь рисование картин и
выстраивание каких-то там сооружений. А вначале это язык. То есть они поставили
пиксовую цель очистить латинский язык от варваризмов, то есть от германского
начинственного следия. Вот. И во вторую очередь это социальные процессы. То есть
они пытались вернуть античные города. Конечно, Рим это античный город, древний
Рим, но он же слишком особенный. Он был как бы уникальностью всех остальных
городов. А другие города это были маленькие государства. Помните полис?
Древнегреческий полис. И вот в Италии возникает такая тенденция развивать такой
тип социальной структуры, то есть города как маленькие полисы. И тем более, что
в Италии это позволял исторический процесс. Дело в том, что Италия того периода
это такое поле перманентной битвы, перманентной раски между германскими
императорами и папами. А когда они между собой воевали, они воевали постоянно,
всю практически историю Средневековья, между ними были терки. Так вот, чтобы,
как говорится, переманить на свою сторону города горожан тех или иных, то папа
дает какую-то свободу горожанам, то какой-нибудь король. И вот так постепенно
образовалось достаточно много городов, которые не принадлежали ни папству и ни к
какому-нибудь там императору Священной Римской империи. Вот лавируя так вот
между обеими сторонами, многие города смогли освободиться от внешнего контроля.
И вскоре, за исключением вот Неапольского королевства, весь Апенинский
полуостров, то есть вся современная Италия, был разделен на множество мелких
городов-государств республиканской формы правления, о которой мы поговорим чуть
позже. То есть они были почти полностью, почти, но все-таки независимы как от
императора, так и от папы. И вот эти республиканские режимы, что они из себя
представляли? Они представляли себя, ну, такую... Республика — это, значит,
такой политический строй, где горожане выбирают себе правителей. И они выбирали
из тех, кто, собственно говоря, им обещал много всяких плюшек. Ничего не
меняется, вы же понимаете. Именно плюшек, каких-то развлечений, каких-нибудь
там, я не знаю... Ну, то есть не серьезных социальных трансформаций, а каких-то
вот таких вот... Ну, в общем, приехал, ну, раздала печеньки, да? Ну, а кто мог
раздавать печеньки? Раздавать печеньки могли банкиры, могли какие-нибудь там
ростовщики, ну, вот такое вот, да? И, соответственно, политическими лидерами
таких коммун, городов коммун, городов республик, чаще всего становились, ну,
либо военачальники, которые захватили этот город в очередной битве, либо вот
именно банкиры. И в одном из таких городов, до этого довольно неприметно,
Флоренция, в одном из таких городов власть захватили Медичи. Это банкиры, и они
были одними из самых крутых банкиров, потому что им было даровано право от папы
взимать десятину церковную. Ну, то есть они всегда при деньгах, вообще, какая бы
там, какой военачальник бы ни пришел, они всегда при деньгах. Поэтому это были
одни из крупнейших банкиров Европы. Банк был уполномочен собирать эту церковную
десятину и отчасти распределять так, как хочет, собственно, этот банк. И вот
Козима Медичи, один из таких банкиров, стал страстным поклонником Платона и
создал на своей вилле, которая потом была подарена Марсилио Фичино, вот, на этой
вилле создал академию, платоновский такой кружок по интересам. Ее называли еще
платонической семьей, потому что, ну, это не было официальным учреждением,
никакими там, это не было похоже даже на академию Платона, которая была в
античности. Это такой был, потому что это не учебное заведение, да, это место
такое общение по поводу платонических идей. И спонсорство Медичи позволило
состояться многим изысканием, написанию многих текстов, многим переводом и так
далее. Расцвет Флорентийской академии приходится как раз на 15 век, на конец 15
века. Примерно в это же время, 1450-е годы, 60-е годы, из Византии бегут,
активно в Европу бегут ученые люди, ну, вообще люди бегут, и ученые. Почему?
Надеюсь, вы помните, захват турками. Турки уже захватили Константинополь и,
собственно, осваивают всю Византийскую империю, постепенно, вот, ну, как бы,
выдавливая людей. Ну, не то, чтобы всех, но имеется в виду, что большинство,
конечно, стремилось куда-то бежать в христианские земли. И вот много таких вот
ученых, монахов, философов, они приезжают в Италию и находят себе, ну, занятия в
таких вот местах, типа Платоновской академии. Они привозят с собой труды
Аристотеля, Платона, но уже на греческом языке, уже не на испанском, которая, на
испанском, на арабском, которые привезены из Испании, а именно на греческом.
Почему они только сейчас привезли? Потому что, напоминаю, раскол был между
Востоком и Западом, да, и не общались они долгое время из-за раскола церковного
католичества, православия. И эти тексты, как бы, подстегнули интерес к
античности еще больше. 

И формируется вот это вот понятие гуманизма. Так вот,
гуманизм в античном смысле, простите, в возрожденческом смысле не имеет
никакого, имеет в ней крайне-крайне косвенное отношение к тому, что мы понимаем
под гуманизмом. Для нас гуманизм – это буквально перевод слова «гуманный».
Гуманный – значит, человечный. Гуман, да? Человечный гуман. То есть относись по-
человечески к другим людям, к животным, к природе. То есть гуманный – это
означает человечный. Для гуманистов в возрождении речь не идет о какой-то
человечности, не о какой-то доброте, милосердии. Речь идет о гуманитарных
предметах, гуманитарных дисциплинах. Слово «гуманист» было произведено по
принципу других терминов, а именно законник, например, легист, итальянское слово
«легист», знаток канонического права, канониста, художник, артиста, от слова
«арт» добавляется субфикс и так далее. И гуманист – это учитель грамматики,
риторики, поэзии, истории, философии, то есть гуманитарных дисциплин. Все они
работают с текстами. То есть первый смысл гуманиста это установка культурная,
согласно которой тексты латинской и греческой древности являются главными и даже
единственными факторами формирования подлинной культуры. Вот это, пожалуйста,
прочувствуйте. То есть это установка, согласно которой подлинной культурой
подлинную культуру нужно осуществлять хорошими текстами. Вопрос, конечно, какие
тексты будут хорошими, но неважно. То есть тексты играют решающую роль в
формировании человека этой культуры. Марсилия Фичино, один из лидеров этого
процесса, немножечко позднее вам скажу, говорит о своем времени так. Это,
несомненно, золотой век, который вернул свет свободным искусствам до того почти
уничтоженным. Грамматики, красноречию, живопись, архитектуре, скульптуре,
музыке. Он считает, в духе эпохи, считает, что все, что было до него, вот это
вот готическое. Вот. Гуманисти видели в своем геле миссию совершенствования
человека посредством гуманистической литературы, которую они же, собственно,
писали, которую они же, собственно, переводили и так далее. Обращение к античной
классике, как они были уверены, улучшает человеческую природу. Наверное, вы тоже
слышали такое от родителей, от учителей, вообще, в принципе, читай классику. Ну,
как ты, ну, ты еще, ты не человеком, ты не сможешь быть, если не прочитаешь всю
классику. Ну, мы же понимаем, что классика не делает нас людьми. Нет, не делает.
Она помогает мыслить, потому что, ну, прочитай вот, даже, то, как пишут люди в
девятнадцатом веке, не говоришь, какой он семнадцатый, и тем более пятый, ты
просто, ну, это как новый язык, а новый язык, он, по-моему, расширяет рамки
твоего, твоей компетенции интеллектуальной. Но сказать, что все эти тексты
наполнены такими смыслами, что ты приходишь просто, да, вот оно, и все, и ты, и
ты совершенства. Ну, это же нет. Но идея откуда? Кто придумал? Гуманисты. Что
чтение классики, работа с классикой, это и есть задача подлинного образования.
Ну, не то, чтобы я хочу ее сейчас разоблачить, уничтожить, просто я вам
предлагаю понять, откуда ноги растут. У нашей современности ноги растут в
основном из возрождения. Как вы понимаете, почему слово гуманный, гуманист
разный? Вы понимаете, гуманным может быть даже необразованный человек, да, то
есть, я хочу еще раз подчеркнуть различия смыслов. Гуманным может быть какой-
нибудь человек даже, ну, простите, с отклонениями в развитии, может быть
гуманным. А для возрожденческих мыслителей это полный нонсенс, он невозможно.
Если у человека мозги не работает, он не прочитал достаточное количество
классики, он никакой не гуман, не гуманный, не гуманист. То есть, с помощью
литературы мы совершенствуем человека. И вот теперь самый главный вопрос. Вот
этот совершенный человек, они называли его гомо-виртуозу. И это не виртуозный
исполнитель чего-либо, виртуоза, от слова виртус. Виртус — это некий дар,
которым обладают вообще-то только воины в античном смысле. Если я сейчас вас
немножечко в слово виртус погружусь, чтобы вы просто как бы прочувствовали дух.
В свое время в последние цицерон, а цезик пишет о своих похождениях, о своих
военных достижениях. И в случае рассказывает про одного, не двух военачальников,
которые ограбили своих подчиненных. Его военачальники ограбили своих
подчиненных, и он их сказал, но я не стал их наказывать, потому что они обладают
великим виртусом. То есть, да, они подлецы, они ограбили солдат, но они обладают
великим виртусом. способность сражаться с реальностью. То есть, подчинять
реальность своей власти. Когда, знаете, в состоянии какого-то боя, в состоянии
высшего напряжения, они как будто бы переломывают ход битвы только тем, что
появились в этой битве. И все, и меняется сражение с реальностью. Это и есть
виртус. От этого слова потом виртуальный пошло. Нет, это не то, что сегодня
имеется в виду. И это не означает добродетель. Да, добродетель, добродетель
воина. Это особая способность. Ну и вот, вот что имеется в виду примерно под
этим вот гомовиртуозу. Это человек, способный подчинять реальность своим, своей
власти. Но как они это понимали? И вот тут надо сразу сказать, они понимали это
магически. Магически. 

Мы переходим к теме герметизм, как одна из основных
характеристик эпохи Возрождения. Что такое герметизм? Я тут не все вам
рассказываю пока, потом уточню. Герметизм это религиозно-философское течение,
которое родилось примерно во 2-3 веке нашей эры. И оно сочетало там элементы
греческой философии, вавилонской астрологии, персидской магии, египетской
алхимии и чего-чего только еще нет. То есть это эзотерические или, как сегодня
можно сказать, оккультные практики. Слово оккультный появилось только в 19 веке,
а вот слово герметичный вы знаете, герметичный значит закрытый. А откуда это
слово произошло? Конечно, от слова гернес. Гернес это тот самый бог античный,
который посредник между богами и людьми, который бог знания, ему соответствует
планета Меркурий. но еще это и легендарный автор всех, как говорится, вот основы
всей магии. Имеется в виду трактат этот изумрудная скрижаль. Вот я тут
специально вам такую фоточку поставила изумрудная скрижаль, которая, видите,
брелочек, да. К чему это я говорю? Что, конечно, это все тайные знания, но
удивительно, что тайные знания, как правило, можно купить в подземном переходе
во все времена. Астрология это всю жизнь тайное знание, всю жизнь, это
бесконечно тайное эзотерическое знание, все знают астрологию, или хотя бы просто
окину. То же самое, вот все эти тексты, они были как бы тайными, но вот как ни
странно, с тайной ничего не получилось. Гермет тресмегист еще раз, мифическая
фигура, она была связана с древнеегипетским богом Тотом, ну, почитайте,
послушайте где-нибудь, про это много сказано. Проблема вот, почему возрождение
легко купилось на то, чтобы просто прославлять Гермеса тресмегиста, а ведь это
же чистое язычество, чистейшее язычество, магия, это же не совсем христианство,
но в этом трактате было написано, что его автор, имеется в виду Гермес
тресмегист, он предок еще и Моисея, то есть Моисей сам учился у Гермеса
тресмегиста, и гуманисты говорят, ну, так слушайте, он этот древнее Моисея, но
не просто древнее, он как бы, ну, как сказать, там, Моисей идет по его стопам,
то есть это, это как бы протохристианство, протохристианство, и на долгие года
христиане, имеется в виду и священство, и Папа Римский, там, они таки
задумались, а в смысле, а может, а как, если это протохристианство, ну, то есть,
понимаете, а точно ли надо запрещать магию тогда? В 17 веке, 1614 году, Исакка
Слабон доказал, что герметические тексты, вот этот корпус герметикум, в составе
всех, которого в 17 трактатов, имеет вообще не такое древнее происхождение, они
созданы во 2-3 веке, то есть, гораздо позднее, чем была написана Библия, гораздо
позднее, чем были написаны даже, ну, не все книги Нового Завета, но часть книг
Нового Завета, родились тексты среди гностицизма, я не буду сейчас уточнять, что
такое гностицизм, но там самое главное, это дуализм духовного и материального,
причем, материя, это чистое, неразбавленное зло, а духовное, это чистое,
неразбавленное благо. Так вот, все время гностицизм потерял, ну, как бы, очень
много своих сторонников, вот таких вот лобовых столкновений с христианством, то
есть, во 2-3 веке гностицизм как бы проиграл в таких вот, ну, как сказать,
слабовые столкновения, в прямых дискуссиях, в прямых дискуссиях. И была
выработка, на свой вопрос, такая обходная тактика, то есть, был написан тексты,
которые, ну, как бы, стали древнее библейских, вот так вот. И причем, не просто
древнее типа в какую-то сторону уходящий, а ведущий к Библии. Итальянское
возрождение заворожено герметизмом, заворожено изумрудной скрижалию всеми вот
этим вот позициями. И в 1488 году даже в Сиенском соборе была создана мозаика, я
ее, к сожалению, вам не дала, но можете посмотреть с изображением Гермеса и
надписью Гермес, Меркурий, трижды величайший, потому что Гермес, Трис, Мегист
это Гермес, трижды величайший. Гермес, трижды величайший, современные Моисеи.
Ну, тогда они почитали, что современные. И вот научной базис итальянского
возрождения это герметизм плюс натурфилософия поздней античности. Вот это и есть
наука возрождения, дорогие мои. Это не то, что вы понимаете под наукой сегодня.
Это магия. Поэтому, когда вас спросят о роли герметизма в Богу возрождения,
можно смело отвечать, да это все для герметизма. И чтобы вы поняли, как с этим
быть, ведь это сейчас разрывает немножечко тот шаблон, который заложен в
школьном образовании. Вот вам тексты вот здесь вот или сфотографируйте, или
потом возьмете в этой самой. Это уже классические тексты, которые надо изучить,
если вы хотите в эту тему погрузиться. Можете мне на слово поверить, а можете
вот их прочитать. То есть, к чему-то я говорю, к тому, что называть товарищей
возрожденцев всяких Джордана Бруна, того же Марсилио Фичино и тому подобное
учеными, ну знаете, такое уже не носят. Такое уже не носят. Вот давайте все-таки
понемножечку посовременнее прочитайте и будет вам счастье. Как минимум на
экзаменах, если вопрос встретится. Ну так вот, так называемый антропоцентризм
возрождения, слышали про такое? Да, обязательно слышали, а если не слышали,
запишите. Антропоцентризм, то есть по аналогии, смотрите, античность,
космоцентризм, средневековье, теоцентризм, возрождение, антропоцентризм. То есть
вокруг человека. Но какого человека? Человека, который может подчинить себе
реальность. И можно, конечно, подчинять реальность, ну типа, вы знаете, ну как-
то так вот, когда ты в море мрака отправляешься на корабле, а можно сесть в
какой-нибудь комурочке и начать колдовать. Ты же тоже пытаешься подчинить себе
реальность. И вот Платоновская академия, даже монахи в Платоновской академии,
они занимаются в то время магией. 

Марсилио Фичино, вот давайте я вам немножечко
о нем скажу, он был такой очень интересный человек, он как бы говорил, что
совершенство человека связано не с какими-то там, ну не с красотой, как в
античности, и не с приближением к Богу, как в Средневековье, а именно с
магическими силами. А он был руководитель Платоновской академии. Сам он был
такой, знаете, весьма неказистый, он был горбатым, заикался, ну в общем, так ему
как-то, ну он как бы под идеального человека с точки зрения античности, ну не
получалось, не годился. но он же был прекрасным магом, в том смысле, что он сам
об этом писал, что у него не плохо получается. Вот теперь немножечко, хотя у нас
сейчас заканчивается скоро, но я думаю, что я за 10 минут уложусь, речь идет о
том, что такое магия возрождения. Вот это очень важно. То есть еще раз повторяю,
вот Марсилий Фичина прямо говорил, если бы у человека были инструменты и
материал, он бы сам сотворил мир. Чувствуете? Это вообще-то как бы сомнительное
с точки зрения христианской религии утверждение. И это сомнительное с точки
зрения научного современного подхода утверждения. Почему? Ну потому что
современная наука не приемлет магию, ну в силу того, что она просто, это
волшебство. А вот христианство не приемлет магию, потому что это
антикреационистская позиция. В смысле ты сам бы сотворил реальность. Чувствуете,
да? И с той другой стороны как бы христианство и наука современная против
магизма возрождения. Поэтому очень осторожно заявляйте, что эпоха возрождения
вот это как раз такой расцвет научного знания. еще раз, что такое магия
возрождения? Чтобы объяснить, что это такое, я обращусь, вот давайте вначале я
обращусь к знаменитому тритемию, тритемию, как он себя называл, и объясню, что
здесь имеется в виду. Потому что магия может быть разная, понималась, вот есть
природная магия тритемии. Это значит обращение к демонам и ангелам. Вот прям так
вот просто. Нужно подчинить их своей воле, а они уж там сделают. Они же имеют
связь, вернее, какую-то там власть над материей. Они же даже в Евангелии названы
космократоры. Космократоры, то есть владельцы космоса. Они владеют вот этой вот
всей реальностью материальной. Соответственно, их надо просто себе подчинить, а
они уж там, как говорится, выполнят твои просьбы. И вот пример, прекрасный
пример того, как тритемии предлагает реализовать общение на расстоянии. Ну,
согласитесь, очень мило. Люди уже об этом думали, но думали именно магически. И
вот он пишет, значит, что нужно взять, передать какому-нибудь своему другу
послание. Ты изготавливаешь фигурку этого друга из воска, значит, наносишь
определенные письмена. Я сейчас не буду вам прям рецепт давать, просто, значит,
объясняю технологию. Наносишь, значит, эти письмена на фигурку, а затем читаешь
вот этот текст. Он его пишет. речь идет вот о чем. Слушай меня, Арифейль.
Арифейль. Это ангел звезды Сатурна. Силой всемогущего Бога, ну, тут я не
комментирую, как это вообще связано. Силой всемогущего Бога, повинуйся мне, я
тебе повелеваю, посылаю тебя властью образа твоего передать Эн, сыну Эн.
Следующее послание. Залагается послание. Точная, верная тайна, не опустив ничего
насчет того, что я хочу передать ему и что я доверил тебе. В имя Отца, Сына,
Святого Духа. Аминь. Это, конечно, что-то, но он был настолько знаменит, это,
знаете, Маск этого самого 15 века. Он знаменитейший был. Он просто это было
культ тритемии был в 15 веке. Есть вопрос. Послания-то ведь не доходили, а как
он все-таки вот это вот вопрос. Почему это работало, хотя это не работало? Ну,
ладно, это другой вопрос. Теперь более как бы это один вид магии. Она называется
демоническая магия. Поняли, почему, да? И я так хочу сказать, гуманисты такой
старались не заниматься, ну, потому что они люди интеллекта, они знают науку.
Тритами был монахом. Монахом и этим аббатом, монастыря. Потом, кстати, был, по-
моему, епискотом. Такая вот интересная история, да? А вот наши гуманисты, они же
ученые. Ну да, хорошо, Фичина тоже монах. Но, тем не менее, как бы, это уже,
называется, его, там, такая, не связанная с профессией. делая, а по профессии-то
они, они, именно, работа с текстами, что же они включают? И вот здесь другая
магия, что включается в эту магию? Это магия, которая называется пневматической,
и главный ее адепт, это, конечно, Джордана Брунна. Джордана Брунна, человек,
который не имеет наук никакого отношения. То, что он говорил в отношении
бесконечности космоса, гелиоцентризма, это было исключительно магическое
происхождение. Вот сам он о себе так и говорил. То есть, понимаете, называть его
ученым и отказывать ему в статусе мага, это просто издевательство над смертью,
над, вернее, над, ну, когда, знаете, о мертвых плохо не говорят. Вот не сказать,
что он был магом, это означает очень обидеть Джордана Брунна, потому что он во
всех своих текстах пишет «Я на ланец», он так себя называет, «Джордана Брунна
ланец», «Я тот, кто заклинает материю», «Я тот, кто учит магии», «Я тот, кто…»
Другое дело, что в нашей советской, значит, образовании никто не знал, что писал
Джордана Брунна. Его самое знаменитое произведение о связи, связи, или о связи
как таковой, она в 2018 году была переведена на русский язык, опубликована, и
книги Яйц тоже не были опубликованы в советское время, то есть, да, сегодня мы
это все знаем, сегодня мы прочитали эти книги, там, вот эти вот, а все
астрологические трактаты «Изгнание торжествующего зверя», математики, в смысле,
пифагорейской математики, безмерным нечисленным и так далее. И вот трактат
«Самые итоговые», где он все обобщает о связях как таковых, это разговор вот о
чем, разговор о пневматической магии. Это опора на платоновскую натуру,
философию, развитую стойками. Смотрите, что такое? Кратенько. Есть четыре стихии
земля, огонь, вода, воздух. Помните, да? По-другому скажу, земля, вода, воздух,
огонь. А дальше квинтэссенция, пятая стихия, пятая, это, сущная, ой, простите,
да, пятая сущная, пятая, пятый элемент, вот, пятый элемент. Это субстанция, с
которой состоят звезды, это пневма, пневма. Это и платон, и аристотник, про это
говорят, и стойки, и неоплатоники. Эта пневма очень тонкий эфир, тоньше воздуха
во много-много раз, но она пронизывает весь космос. И самое главное, что она
соединяет две по тем представлениям абсолютно разнородные сущности, душа
духовная и тело материальное. Вот сегодня мы не особо задумываясь о этой
психофизической проблеме, а тогда это был вопрос вопросов, как соединяется душа
и тело. Мысль это духовная, как она рождается в мозге, мозг это материя. Так
вот, ответ был дан такой, в эпоху еще античности, и это подхватили в эпоху
возрождения. Пневма соединяет, это все-таки материя, она очень тонкая, и она как
бы соединяет духовное и телесное. То есть происходит вот это взаимопроникает
туда и туда, пневма их соединяет. Дальше. Пневма и есть условия нашего познания
человеческого. Каким образом? Когда мы с вами смотрим на что-либо, вот я сейчас
смотрю на монитор, и по мнению этих натурфилософов, из моих глаз выходит пневма,
но она же все проникает. Она получает некий толчок от меня, но я же куда-то
направила взгляд. Пневма, значит, как бы пошла туда, такое движение пневмы,
движение тонкого эфира, она оттолкнулась от этого самого, от ноутбука и пошла
обратно. Она же поплыла, пошла обратно, как бы, как волны, да, поплыла обратно,
попала мне в зрачок, по зрительному нерву попала мне уже в область, ну, там, не
мозга, а вначале средняя часть, где происходит перерождение пневмы в фантазм. То
есть я, мое сознание работает с фантазмом монитора, то есть или с фантазмом
дерева, на который я смотрю, или с фантазмом там, любого другого предмета или
другого человека. И вот этот фантазм уже имеет, он уже духовной природы, да,
такой оттиск на моей пневме, которая там, да, и вот она получает, она как бы
заполняет либо мозг, либо сердце у разных авторов по-разному, и это и есть
основание моего познания. То есть я могу познавать мир и вы тоже по мнению
античных и возрожденческих мыслителей благодаря тому, что пневма все
проникающая. Вот она и осуществляет связь как таковую, связь связей, оков и
оков. И магия должна быть, говорит он, ну как бы руководить этим процессом. еще
раз, вот смотрите, в нашем организме, по мнению стойков, есть такое начало
господствующее, оно называется гегемоникон, или гегемоникон, по-другому, да,
который вбирает в себя потоки пневмы отовсюду, из всех органов чувств.
Гегемоникон собирает из всех органов чувств потоки пневмы, словно, как говорят
они, словно паук в сердцевине паутины, расположившись в центре тела, он
собирает, вбирает в себя всю информацию, то есть отпечатки на пневме,
передаваемые от чувств. Эти умопостигаемые образы, то есть фантазмы, передаются
разуму. Разум, у него духовная природа, может постичь только духовная,
нематериальная. Соответственно, вот эти отпечатки на пневме, фантазмы, с ними
имеет дело разум. Но напомню, это же астропневма, это же как бы то, из чего
состоят звезды, поэтому все изменения в мире звезд, в космосе, все отражается на
человека, все на человеке. Это античная основа астрологии. То есть античность
уже не считает, что это боги. Сатурн, что это бог, Венера бог, Меркурий бог и
так далее. Но, видите, все равно астрология работает, потому что теперь это
астропневма, которая до нашего напрямую пронизывает нас. я сейчас вначале
расскажу про пневматические инфекции, чтобы стало понятнее, а потом вернусь к
Джордана Бруно. 

Что такое пневматическая инфекция? Вот шамб было понятно, о ней
много говорит Марсилий Фичино, а вам таки будет понятно. романтическая любви, в
первую очередь, чей-то образ, например, образ какой-то женщины, про мужчину они
не говорят, то есть про женскую влюбленность почему-то никто не рассказывает. Но
вот мужчина взглянул на женщину, и вот он, этот образ, но как-то слишком сильно
отпечатался на его пневме. это и есть пневматическая инфекция. Сильный толчок на
пневму. И вот слишком сильный толчок на пневму оказывается грубо в печатном
сознании, слишком грубо, слишком активный фантазм. То есть всё, у человека как
бы сознание заполнилось этим фантазмом, другие-то уже не проникают, и он
начинает, что называется, страдать. Страдать, потому что, а почему он страдает?
Здравый смысл его претерпевает столь сильные изменения, это я цитирую Василия
Фичина, что он, человек, становится не способен никакому занятию. Если кто-
нибудь заговаривает с ним, он едва слышит, он инфицирован, его воображение
инфицировано. А раз это так, это болезнь, это истощает пациента, и правда не
слабеют только его глаза, потому что они получают пневму. Изображение женщины
проникает в человека через взгляд на нее, посредством оптического нерва он
передается в сознание, где формируется вот этот фантазм. Превратишься в фантазм,
он заполняет три сердечных желудочка и вызывает расстройство, расстройство
разума. И вот глава его о любви, которая называется романтическим, вот он пишет,
что получается вот этот взгляд на женщину, которая по неосторожности
инфицировала человека, это убийство его души, она виновата. Единственное, чем
она может отплатить, ну как-то, она может вместить в себя фантазм этого
несчастного влюбленного. Ну то есть ответим взаимностью. Тогда его как бы душа
переходит в неё, и её душа в него, потому что его собственное нет, он заполнен
ею. И тогда все выдохнули, а становится, б становится, а все живы, что
называется. А если этого не происходит, знаете, что дальше происходит? Пневма с
кровью вытекает из глаз бедного влюбленного. Да, потому что пневма это ещё и
состав крови, кровь, это огненная стихия, огненная стихия ближе всего Кастро,
стихия. Соответственно, всё, кровь из глаз, все умерли. Вот, вы понимаете, да,
то есть выйдет научное обоснование, вот такое строгое, логическое всё. И
одновременно нет. Почему нет? Потому что никто не ставил эксперимент.

Теперь
Джордана Бруно, я к нему возвращаюсь, я сейчас подольше расскажу, потом мы
отдохнём. Дело вот в чём. Джордана Бруно говорит о том, что магия должна
использоваться для дела вообще-то. Вот это все эти заражения, конечно, лечить
влюблённых, полечим, это дело, да. Вот в своём практике о связи всеобщем он
говорит о том, что магия в первую очередь дана, чтобы манипулировать большими
массами людей. Вот для чего должна быть магия. То есть мы вот этими фантазмами,
говорит он, да, мы умеем ими заражать, и нужно заражать большие массы людей,
чтобы исправить общество. То есть пишет один из исследователей, говорит, так
забавно, что человек, которому буквально поклоняются анархисты около памятника,
вот около этого памятника собираются ежегодные встречи анархистов, это человек,
который ратовал за всеобщее подчинение благодаря магии. Это же вот какая-то
просто какие-то перевертуши истории, это изумительно. Так вот, он говорит о том,
что вот эта связь связи, это эрос в самом широком смысле этого слова, потому что
он говорит, все, что мы любим от физических удовольствий, от совершенно
неожиданных вещей, в том числе богатства, власть, и тому подобное, все это
представляет собой всемирное тяготение. Мы тянемся к чему-то, а стало быть, этим
можно манипулировать. Да? И вот цитата. «Через действия магии, говорит он,
происходит через непрямой контакт, через звуки и образы, которые воздействуют на
зрение и слух. Проходя через отверстия органов чувств, они сообщают воображению
влечения или отвращения, наслаждения или годливость. Зрение и слух всего лишь
вторичное отверстие, через которое охотник за душами а он и есть охотник за
душами. Аниматор, он винотар, маг может устанавливать связи, подкладывать
приманки. Это трактор связи, связи. Позаботься о том, чтобы не превратиться из
мастера фантазмов в их инструмент. То есть он советует магу, понятно, что он
должен обязательно ну как бы манипулировать фантазмами, но ни в коем случае
чтобы самого заражения не было. Истинный маг должен приводить в порядок,
корректировать, направлять фантазию, распоряжаться ею по собственной воле.
Гораздо легче манипулировать многими людьми, чем отдельным человеком. Ну вы
поняли, о чем речь? Вы представляете, о чем идет речь? Это о современных
средствах массовой коммуникации, о пропаганде и тому подобное. Но, кстати,
пропаганда рознь. Мы должны тоже понимать, что не всякая пропаганда магия. Вот,
например, очень хорошо известная нам пропаганда, пропаганда есть в каждой
стране, это обязательно. Это норма. Так вот, пропаганда может быть разной. У нас
обычно умалчивают, вы же знаете, да? Вот я в советское время жила очень хорошо,
знала, что у нас просто мы ни о чем не знали. Мы не знали, что что-то взорвалось
где-нибудь, что где-нибудь там какая-нибудь военная операция журнеть еще. Мы
ничего не знали. У нас не было ни аварии, не было маньяков, никого не было. Это
вам подтвердят даже ваши родители. У нас все было хорошо. То есть традиционно
наша такая пропаганда это всегда умалчивание. То есть ложь, которая, это просто
недоговаривание. И совсем другой тип пропаганды это европейская пропаганда. И мы
это сейчас просто видим воочию. Создание фантазмов, с помощью которых мы
управляем реальностью, помещая их в чужие головы. Да, сегодня никто не говорит о
пневме, но все, что говорил Джордана Бруно, вот только исключить там слово
пневма и там всякие детали, да, все это можно совершенно спокойно применять в
современной медийной политике. Современной западной. То есть они занимаются
магией. Мы занимаемся умалчиванием, имеется в России, да, а Европа занимается
чистой магией, бруновской магией. Ну вот. И заключение, чтобы уж совсем, как
говорится, закончиться темой, искусство памяти очень еще развито у гуманистов.
Что такое искусство памяти? мнемотехника, это манипулирование уже собственными
фантазмами. То есть не только фантазмами других людей, ну, либо там
пневматическая магия, либо магия с помощью фантазмов, как у Джордана Бруно, это
способность накладывать образы на любое содержание, понятийное или нарративное.
Вот когда ты это освоишь, говорят, они пишут много трактатов, у самого Джордана
Бруно много, и исследуется это очень активно, тоже можете полюбопытствовать. Вот
если на последовательность понятия или какого-то нарратива рассказа наложить
последовательность образов, то таким образом вы подчините себе свои же
собственные фантазмы. это вот тоже работает, но они только это воспринимают как
магию. Вот так вот. И ещё, конечно же, всё, что пишет Джордана Бруно, это не
только манипуляции общественным сознанием, это ещё и психоаналитика. Я просто
была в изумлении, что берёшь, читаешь Лакана, а после Лакана читает Джордана
Бруно и понимаешь, что это об одном и том же. Разной лексикой, но об одном и том
же. То есть вот эта вот мысль, она, мысль возрождения, она вообще никуда из
европейского социума не девается. Они по-прежнему занимаются магией, но только в
разных формах. Это удивительно, просто удивительно. На этом давайте сейчас
остановимся, передохнём. 10 минут, сейчас без 20, значит, без 10 мы
возвращаемся. вы слышно. Спасибо. Даже чуть лучше, чем прямо перед перерывом
было. Да? Что я сделала? Не знаю. Не знаю, просто, может быть, это отдохнули. Да
я орала, тут вы не поверите как. Ладно, давайте дальше. Значит, что важнее,
сейчас я секундочку вернусь к этому самому, что важнее для науки из того, что мы
с вами, я тут много пропускаю, пропускаю там мирандолу. Почему? Потому что, ну,
вы просто теперь должны понять, что речь о достоинстве человека, знаменитая, это
всё-таки, ну, по большей части о человеке-маге. Ну, это просто вы уже поняли, я
не уточняю. Смотрите, ну, не все впитывают вот эти идеи магии так вот
полноценно. Но темнее на всех влияет. И вот вы должны почувствовать, куда
девалось и почему куда-то девалось вот это превосходная компетенция схоластов
рассуждать строго логически. Да, схоласты, конечно, ну, как бы огромный минус
схоластики тем, что она сосредоточилась исключительно на комментаторской
деятельности, то есть общий интерес это труды аристотия по всем, по всем наукам,
а исключительно богословский интерес там ещё аристотию плюс Библия, ну, то есть
для богословов, богословских факультетов. А вот остальные все факультеты это
аристотия, вот, понимаете, да? и бесконечное комментирование трудов Аристотеля
или комментаторов Аристотеля, вот так вот. Но, тем не менее, и это в эпоху
Возрождения вызывает, ну, негодование, отвращение, они говорят, надо полностью
изменить, то есть отказываемся от схоластики, но не только потому, что мы не
любим схоластику, но и потому, что мы же люди маги. Вот. Отказ от риторики,
простите, от логики, это означает плюс в пользу риторики, ведь риторика чем
берёт? эмоциональной убедительностью. Это и есть заражение фантазмами. Когда мы,
да ещё и плюс, если харизма есть, да, чувствуете, мы заражаем мысль, мы её
инфицируем, как вот было рассказано про инфицирование в романтической, в такой
болезни, как романтическая любовь. И в принципе, даже вот такие немагически
ориентированные мыслители, как Лоренцо Валло, они-то всё равно ориентируются на
такой подход. Поэтому в работах, например, в становлении диалектики философии,
он вдохновлён именно риторикой, он ратует за приобретение такой компетенции,
когда не логика будет силой убедительности, а именно риторика, то есть
антиаристотельский подход. Он рассказывает очень много про принципы организации,
фраз, про предложения и так далее, так далее, так далее. Огромный, ну таким
важным моментом, плюсом, можно так сказать, его работы заключается в том, что он
говорит, всё-таки убеждает человек, всё-таки убеждает человек и как бы его вот
эта вот особая сила, когда он убеждён, особенно в том, что говорит, вот он
искренне убеждён в этом, и это заразно, вот это заражает. Это для античных,
именно для возрожденческих мыслей, для этого магия, она вот в чистом виде, как
она и есть. Вот. А пневматическая магия. Но он говорит, что это должно стать
искусством, это не должно стать просто способностью одного человека, это должно
стать способностью всех, все должны заражать идеями с помощью риторики. Хорошо
это или плохо, слушайте, ну вот я бы сказала, для науки нет, для политики нет.
Но что делать, мы живём в условиях, когда это уже случилось, это изменилось.
Поэтому, дорогие мои, мы с вами как бы просто принимаем это как факт, когда
логическая система аргументации сменилась на риторическую в эпоху Возрождения. И
вот вам для, как говорится, понимания, вот просто представьте себе, какое
высказывание могли сказать схоласты в эпоху Средневековья, а какое именно вот
как бы уже переводчики схоластов в эпоху, например, Возрождения или позднее. Ну
вот первое, просто прочитайте его по-быстренькому, и вы поймёте, что это,
конечно, логика, это чистая схоластика. И мы потеряли эту компетенцию, я точно,
я не умею так мыслить. Я и хотела бы, но я не умею. Дочитая схоластов, у меня
просто мозги в трубочку сворачиваются. Это дико сложно. Это как будто бы,
знаете, мы, я вот в своей группе это говорила и повторю всем уже остальным, это
как будто бы ты мыслишь, как будто бы сразу же кодишь. Вот сразу же пишешь код,
так вот логически мыслишь. Я так не умею. Одна математик мне сказала, что она
так умеет. Не всегда она умеет. Но она математик, она сказала, это
профдеформация. И вот смотрите, B под буквой B, не умножая сущности без
надобности, известная фраза, её нигде не встретишь в тексте Акама. Это вот такой
перевод, чувствуете, вот это уже риторика. Просто это призыв. Это риторика. Это
уже мы могли сказать только позднее. Вот таким образом. 


И ещё одно достижение в
эпохе Возрождения это гелиоцентрическая система. И с этим тоже не всё так
просто. Совсем не просто. Гелиоцентрическая система это представление о том, что
солнце является центральным небесным телом, его вращаются планеты, а не
наоборот. Сразу же скажу, гелиоцентризм был известен ещё в эпоху античности,
эпоху Средневековья. Все прекрасно знали про концепции гелиоцентризма. Это были
соревнующиеся концепции. Геоцентризм, гелиоцентризм. И была ещё третья. Я сейчас
просто сижу, вспомнил, но я не могу её вспомнить. Но она как-то вообще не
утвердилась, не закрепилась. Может, когда-нибудь вернётся. Ладно. Но, в общем,
из этих двух, пока остановимся на двух концепциях, победил в эпоху античности и
Средневековья ход, который предлагает Аристотель, а за ним Птоломей. Это
аристотельско-птоломеевская система. То есть там, где Земля, есть центральное
небесное тело. И почему оно победило? Да потому что эта концепция всё объясняла.
Земля есть та самая свалка, на которую всё сваливается. Вот она есть центр, на
которую всё сваливается. Всё оказывается на Земле в итоге. И причём это свалка в
худшем смысле этого слова. То есть Аристотель никоим образом не возносит вот это
центральное положение как какой-то статус. Нет, это свалка. Он так и пишет.
Понимаете? То есть центральное положение для античных мыслителей не означает
центрального с точки зрения статуса. Да никоим образом. И вот через какое-то
время там возникает потребность переосмыслить роль Земли. Понимаете? Для
товарищей гуманистов, коими являются тот же самый Джордана Бруно, который
ратовал за гелиоцентрическую систему, аристотельская система не устраивает тем,
что он как бы забыл солнце. А солнце по Бруно — это видимый бог. Не в смысле бог
христианский, а тот самый языческий бог, за которого он, собственно, и сражался,
за которого он потом и погорел в буквальном смысле этого слова. Потому что он
действительно, ему предложили отречься от его магических и языческих идей, он
сказал, нет, я хочу, чтобы моя душа воспарила и вернулась в небесную обитель. И
в поднебесной обитель он не имел в виду рай там христианский, он имел в виду
астро-пневму. То есть чтобы личность и душа освободилась от материальности тела,
и только тогда пневм вернётся туда. Ему сказали, ну окей, давай. То есть не то,
чтобы мне не было жалко человека, по-человечески жалко, но с другой стороны,
слушайте, ну при чём здесь пострадание за науку? Это не страдание за науку.
Никаким образом. По-человечески, да, мы можем его, как говорится, ну
попечалиться о нём, но памятник, как великому учёному, за это ставить крайне
глупо. Так вот, возвращаюсь к гелиоцентризму. Большую часть внимания
гелиоцентризма это всё-таки связано с магией в эпоху Возрождения. Солнце – это
особая разрежённая субстанция, в ней гораздо больше вот этой пневмы, чем в
земле. Земля – это какой-то камень, как Платон и Ристотель говорит. Поэтому,
конечно, идея Коперника получила своё одобрение, и тут началась борьба, самое
главное, потому что бороться пришлось не между геоцентризмом и гелиоцентризмом,
а внутри представления о гелиоцентризме. То есть такие люди, как Коперник,
хотели, чтобы этому был предан статус математики. Он говорит, я поставил Солнце
в центр космоса, тогда именно так вот, потому что это математически обусловлено.
Мы, наконец-то, вот эти вот все движения планет смогли как-то посчитать. А тот
же Джордана Бруно его ругает. Он говорит, да, он сделал великое открытие, но он
не понял его значения. Настоящее значение того, что Солнце теперь в центре, это,
наконец-то, вернулась та древняя наука, астрология, вся, которая связана с тем,
что Солнце – главный бог, и дополнительные боги, ну, как бы более-менее важные
боги, например, бог Сатурн, бог Меркурий, бог Венера, бог этот сам, они вот
крутятся вокруг него. И вся истинная борьба заключалась в том, чтобы
гелиоцентризму придать тот статус, который мы сегодня имеем в виду,
математический. Коперник только в нескольких местах говорит о каком-то такой
божественности Солнца. И многие исследователи считают, что это скорее, ну, вот,
дань моде, потому что иначе, говорит, и вообще никто читать не стал. Понимаете,
да? То есть, мало того, что вот это открытие было, да, собственно говоря,
открытия-то как такового, ну, не было в чистом виде, по факту, потому что
гелиоцентризм был известен. Математическое обоснование, вот это открытие, и
второе статус – это преодоление вот этого мистика-магического понимания
гелиоцентризма. Я специально на этом обращаю ваше внимание, это важно. Позднее,
значит, Иоганн Кеплер добавил к этому математическое, к этой разработке, и еще
астрономические таблицы, имеется в виду, это уже эмпирическое подтверждение, да.
Ну, а далее, Галио Галилей, но это мы поговорим в следующей лекции. То есть,
вот, интересное очень достижение. Но тут же, конечно же, включается вопрос, ну,
такой философский. Человек, говорят некоторые следователи, в эпоху Возрождения
сильно сгрустнул, потому что он потерял свое центральное положение. Это тоже
означает не читать текстов людей, которые, собственно говоря, создавали эпоху
Возрождения. Например, не читать Николая Кузанского, который особенно
подчеркивал, а это великий мыслитель, вот серьезно, вот он просто, я специально
даже про него много не говорю, потому что надо говорить очень много. Но он не
очень, как сказать, возрожденческий, в том смысле, что он выбивается из ряда. А
моя задача все-таки рассказать о типичном Возрождении. Так вот, Николай
Кузанский в силу своей гениальности, в силу своей удивительности был скорее
атипичным представителем Возрождения. Но тем не менее, чуть-чуть скажу. Вот он,
пожалуй, единственный противостоял этой позиции, что, ну как бы, он противостоял
аристатилизму. Почему? Потому что не считал Землю каким-то низким во всех
смыслах местом. Никакой не свалкой. Он и говорил о том, что, сейчас почитаю
выдержку его, из его. Земля благородная звезда со своим светом. Ну, не
прикапывайтесь к тому, что это звезда, сейчас неважно. Своим теплом, своим
собственным влиянием, которое отличается от света, тепла и влияния других звезд.
И человек, как главное существо на этой планете, говорит он, это богатство
возможностей, поскольку все сотворенное обладает виртуальным бытием. То есть,
ну, тут другая уже часть его концепции. Но неважно. То есть, не было в
возрождении вот этой вот идеи о том, что нас сместили на край Вселенной, теперь
не все вращается вокруг нас, все конец. Да потому, что любой, кто так думал, что
мы в центре Вселенной, он думал по-аристотелевски. А думать по-аристотелевски о
Земле, это означает видеть ее свалкой, ничтожным элементом мироздания. Поэтому,
как ни крути, гелиоцентризм, наоборот, возвысил статус человека. Понимаете? И
почему Николай Кузанский говорит, что человек в чем-то похож на Бога, то есть,
приближается к образу и подобию? Потому что человек обладает способностью,
угадайте, к чему? К виртусу. Опять виртуальность. В работе о возможности бытии
Кузанец, это его как бы, ну, такой более короткий, Николай Кузанский или
Кузанец. Поясняет. Смотрите, Бог единственное, что, единственное, что не
виртуально. То есть, Бог в том смысле, что у него потенциальное совпадает с
актуальным. Виртуально это возможно. Может быть, а может не быть. Может такое,
может не такое. а Бог говорит, это где всякое могу, как сказать, это хозяин
могу. В другой работе он пишет, сейчас скажу, какой, найду, чтобы сейчас не
перепутать ничего. Не помню, не написала. Но в общем, в этой работе он говорит,
каждый мальчишка знает о том, как важно слово могу. я могу это сделать я могу и
он переводит пример как это для мальчишек важно и потом говорит что всякое могу
любого человека том числе любого мальчишки она коренится великом могу бога бог
это и есть чистое неразбавленное могу я могу это сделать то есть это мощи
могущества и соответственно в чем величие этого могу что я могу сделать это
солнце таким могу и другим знаете все я даже про это говорила епископ темпе о
том что всемогущество бога связано с тем что реальность может быть любой и
николай кузнанский продолжает эту мысль бог может сделать реальность другой
любой и вот это могу это есть самое главное это есть божественность чистом виде
и отчасти именно этим располагаем мы мы можем сделать то что в принципе ну как
бы не было и вдруг стало ну например как техническое изобретение вода вверх же
не течет не течет а мы можем сделать так с помощью технических всяких штучек и
уловок что она потечет вверх понимаете да чем говорится то есть статус человека
не потерял он с из гелиоцентризма ничего и он средневековья ничего не терял то
есть так что всегда был высоким весьма высокий он только сейчас кстати ниже
плинтуса опустился а во все эти эпохи он очень высок а вот каждая могу берет
свою силу от самого абсолютного неограниченного совершенного могу мощнее
которого нельзя ничего ощутить его выобразить не помыслить потому что в нем
возможность всякого могу бог есть само по себе могу проявляющихся в разных и
отличающихся по виду модуса бытия то есть понимаете быть и это виртуально потому
что виртуально может таким быть а может другим а бог единственный не виртуально
поэтому совсем не странно что итальянские гуманисты чувствовали за собой вот
такое моральное право они сильно это сильная позиция человека вот он
антропоцентризм как в мерзком своем воплощении когда и чё хочу то врачу да как и
в великом они имели такой некой моральное право выдвигать свою позицию которую
они увидели ну как бы основной для того чтобы добиться социальной справедливости
там и того и другого и третьего но тем не менее не будем забывать все таки о
корнях вот этих вот магических которые пришли вместе с интересом контичности а
теперь северное возрождение я тут немножечко вам картинок на накидала из
изобразительного искусства по сушь совсем чтобы нет чтобы не лишать вас этого
леонард да винчи давид и рокио когда терга то милата просто прочувствуйте вот
этого антропоцентризм в искусстве он может проявляться так когда это просто либо
огромность огромность ведь давид это мальчик библейский персонаж мальчик
маленький небольшой только не мальчик да ну или гаттемилата это один из
наемников которые вот там во флоренции и еще один момент все-таки эпоха
возрождения это момент . историческая когда родилась портретная живопись явно
что неспроста то есть очевидно что все-таки портреты появляются тогда когда
статус человека усиливается 

\section{Северное возрождение. Роль и сущность движения Реформации. Специфика гуманизма Северного Возрождения}

теперь северное возрождение крайне важное отличие от
южного есть некое сходство но ухо гораздо менее уловима если честно чем различие
начну эти период возрождения отчетливо про иранского идеала обмирщение
католической церкви похожие явления кстати были характерны не только для европы
но и в россии похоже было ну немножко не в это время но все равно видимо какое-
то кризисное состояние которое проявляется в виде обмирщения а вот на западе в
вытекании это был не обмирщение обмерзение там такая мерзость строилась в текане
что просто вот когда читаешь об этом думаешь но то какой-то высшей степени
цинизу начать с того что примерно 90 процентов священников не верили в бога
остальные творили куда больше мерзости было дело что называется при дворе
римского папы шла отчаянно борьба за кардинальские мантии за дележ доходов то
есть ну как бы это стало карьерой просто карьерой причем такое прибыльный
достаточно до 1378 году две группы кардиналов одновременно возвели на престол
даже двух пап 14 век время в иньоне соперники придали другу анафемии началась
еще и борьба между двумя папами в общем беспредел роковую роль для католичества
сыграла практика индульгенции индульгенция от латинского терпеть позволять вот
это интересно название появилось в 11 веке но вообще-то ведет свое начало
индульгенция от практик так называемых разрешительных грамот то есть это грамоты
которые выдавались каким смыслом это было древней церкви то есть это было и
византии это было и на западе что за разрешительные грамоты разрешительные
грамоты избавляли человека от так называемых духовных тяга но имеется ввиду вот
если ты совершил да потом ты пришел раскаяться сам грех смывается исповедью
рассказали но после исповеди все она остается состоянием этого стыда
угнетенности определенные какой-то духовной тяготами называется и священники
предлагали мирянам значит скажем так покупать вот эти грамоты для того чтобы эти
деньги шли на благотворительность то есть ты как бы сделал какое-то плохое дело
а потом ты помог кому-то например выжить до кому-то там все сиротские приюты все
были на церкви да все вся благотворительность только церковь не то не больше не
занимался но на это же нужны деньги вот такие вот грамоты были они
просуществовали недолго потому что на византии не укрепились а вот на западе да
они укрепились и постепенно они стали чем-то совершенно иным можно было купить
грамоту которая избавляла тебя от духовных тягот за убийство собственных
родителей подороже просто понимаете да то есть смотрите тактика тебе грехи
индульгенциями не отпускались это все равно таинство исповеди но дополнительно
вот это вот плюс ко всему было учение о сверх должных заслугах это чисто
католическое учение больше ни в одной церкви его не существует а именно заслуги
христа и богоматери они имеют как бы избыток они слишком это некая благодать
которая изливается из переливается и и и как бы улавливает католическая церковь
и она уже может распределять посредством этих индульгенций и сверх должные
заслуги среди тех кому посчитает нужным дать можно же и продать все должные
заслуги да ну в общем такой очень сомнительный богословский конечно аргумент так
вот пакством и так все недовольны особенно недовольна германия там постоянно
идет борьба германского императора и пакства поводов огромное количество именно
там проявилась 

религиозная ситуация 16 века которая получила название реформации
от латинского реформацию преобразование исправления массовое сложное по
содержанию которое не только религиозная но и социально-политическая как это
примерно происходило кратенько непосредственно началом явилась деятельность
мартина лютера я вот его здесь сейчас найду чтобы можно было посмотреть
фотографии здесь он фотографию что сказал маркетингютер и милтон но сейчас пока
только промахте на лютера он был сыном рудокопа который отучился в университете
причем на богословском факультете что говорит о том что он был весьма талантлив
окончить университет он был проповедником потом преподавал в тенгемском
университете в 1511 году по делам церкви он отправился в рим и пробыл там
несколько месяцев сначала был просто преисполнил благовения потому что ватикан
вот он а затем сменила острое разочарование в окружении святого престола все
продавалось и покупалось папа принимал участие в пирах карнавалах и чтобы
получить деньги на строительство храма святого петра вот этого знаменитого каун
лев 10 продал 30 кардинальных кардинальских должностей и послал во все страны
европы продавцов интургенций которые могли как бы ну вот избавить от духовных
тяга за любые преступления за любые и вот вернувшись в 517 году значит в ней но
не был раньше но в 517 году 1517 году он написал тезисы свои знаменитые в
которых утверждалось что грешника может спасти не интеллигенция но лишь искренне
раскаяния вера и так далее и он прибил их на ворота виттенгерского
виттенбергского храма это был известный такой жест который призвал дискуссии ну
вот как бы это делаешь это знаете как сегодня сообщение написать полемика
состоялась на диспуте с профессором и компьютер пошел еще дальше и уже утверждал
что вообще церковь может существовать и без папы истина не в папских буллах а в
самом священном писании гримский папа конечно же не мог сдержать не простить
такую крамольную мысли послал в германию буллу отлучающие лютера от церкви но
лютер которого в этом время уже активно поддерживали студенты горожане и
виттенберга разорвал царскую буллу бросил ее в костер от начала реформации к
этой ситуации вот этого бунта народного сразу же подключились рыцарство немецкое
которое помните ну надо же пограбить они дошли даже до ватикана и грабили собор
святого петра при этом они просто говорили что ну все мы разрушаем папские
владения и забираем имущество себе подключились не только как бы ну вот такие
бандитские все-таки будем считать силы по другому это не назвать потому что эти
по всей земле по всей германии начали отжимать владение этих самых владения
церкви вы понимаете какие-то богатство очень серьезные дальше вот эту тенденцию
подхватил и другие религиозные деятели в частности николс что и штубнер начали
совершать обряд перекрещения отсюда понятие анабаптизм и мы понимаем что как бы
не из лютера исходят все современные обилия огромное разнообразие протестантских
каких-то там вот модусов религии ведь у них нет принципа церкви такого единого и
соответственно ну как бы любые варианты возможные они осуществляются вот но
сразу же вот прям буквально сразу же вот пошли такие вот свои понимания а как бы
нам жить без папы римского но в то же время вот вполне себе как бы религиозно к


обязательно стоит упомянуть иоганна гутенберга
почему его потому что он сам дожил в 15 веке и он достаточно давно он создал
станок книгопечатный но книгопечатание как бы ну не задалось а вот именно
движение протестантов движение реформаторское то есть лютер то по сути дела
войне лютер простите гутенберг к этому отношению же не имел но его станок имел к
этому отношению скажем популярность печатанию книг принесло именно
реформаторское движение первое напечатанная книга была библия потом там значит и
салтырь ивангельские тексты а в 516 году был издан с текстом и были тексты
лютера изданы в обилии вот и дальше издаваясь то есть понимаете да речь как бы
не столько даже о самом открытии но мы не умоляем заслуги но надо сказать что
это тема сложная кто-то говорит что китай все-таки забыл книгопечатание но
именно европейская ситуация вот соединились несколько факторов позволила
книгопечатанию встать на ноги просто это имеем ввиду ну а реформация
разрастается сам лютер а подключается конечно же крестьянство она в это время
тоже там крестьянские воины были в таком ну вяло текущем режиме а тут они
обострились и они уже крестьяне начали соответственно на на дворян нападать
дворяне значит испугались ну вот нового всплеска крестьянских войск крестьянских
войн обратились к папе римскому папа римский заключил договор наконец заключили
договор с германским императором потому что всех уже напрягал эта ситуация и
германский император предложил вернуть все награбленные церкви на что конечно
немецкое рыцарство и все остальные кто скак нет мы против мы ничего не будем
возвращать так вот протестанты это против вот этого указа императора понять
исторически как что сложилось мы против указа императора который говорит все
остановитесь давайте все вернем назад тем самым поднялись значит еще раз
выступление еще более новую чё сказал еще более активные выступления еще более
разрушительные и на долгие долгие годы все вся его пополыхал в религиозных
войнах сам был крайне уже стареньким человеком к началу этих вот воин
религиозных он вообще ну как бы закончил жизнью постоянном страхе почему потому
что крестьяне ему не простили что он стал на сторону дворян вот в этой вот
всеобщей войне всеобщем хаосе он стал на сторону дворян и лютер сам писал что он
боится появляться в деревнях просто упью ну а князья значит там протестанты
заключили между собой союз они включились в активную войну с католиками а сам
лютер говорил на уже смертно-мадридовый лучше бы я все оставил как бы ну понятно
что он видит тот тоже неуправляемую хаотичную как бы ситуацию которая 

началась с
эти вот тезисов на соборе на дверях собора виттенберге контр реформация что это
такое сопротивление этому процессу римской церкви со стороны католической церкви
ну первым делом конечно тридентский собор был созван и началась реформа
католической церкви изнутри там запрет на многие злоупотребления призывы к
очищению церкви но трудно сказать насколько все это помогло если мы точно знаем
что дургенции сегодня не продаются но избавила ли это католическую церковь и
всех остальных злоупотребление трудно сказать что нам важно с точки зрения
истории развития науки конечно это создание ордена и иисуса то есть иезуиты его
создал бывший военный игнатилло его он создал монашеский орден по принципу
военной организации то есть приказы исполнялись безоговорочно распределение
заданий и все вперед так вот что такое вперед вперед это означает огромная
миссионерская деятельность и иезуиты разъехались по всему миру куда только могли
довести вот эти вот новые корабли эти новые каравеллы по всему миру буквально и
дело даже не столько в том что это распространение христианства и кстати
улучшение нравов потому что именно они приехав на многие земли пришли в ужасе от
того что там происходило начиная от каннибализма и кончание изнасилования
маленьких детей жуть просто происходило и они ну конечно это огромная сослуга
иезуитов том что они прекратили эти практики долгими ущиваниями прекратили эти
практики то есть люди сами отказались от этого но конечно все таки речь шла о
такой вот серьезной серьезном прощупывание возможности для того чтобы наладить
здесь свою власть власти европы над другими регионами это тоже не стоит забывать
а еще не стоит забывать что иезуиты создали самое лучшее образование в европе
иезуитские школы иезуитские университеты до сих пор это самое европе лучшее
образование не протестантские именно иезуитские их это очень много значит значит
что еще надо знать что вот 

противостояние между католиками и протестантами после
вот этих событий реформации конд реформации привело к 30-летней войне которой
чуть-чуть мы поговорим дальше с точки зрения мировоззрения что такое
протестантизм как новая религия это резкая и очень яростное неприятие
католических моментов моментов связанных с католической религией но для того
периода для протестантов католическая религия связана ну вот тесным образом с
магией понимаете да они же не говорят о католической религию как мы сегодня
видим или как она была там в пятом веке нет они видят магию и в италии магия
везде куда не плют и они начинают обвинять вообще католическую религию в магии
поэтому они отказываются от всех таинств они говорят что вы просто буквально
везде в свою магию внедрили хотя маги это только возрождение для древней церкви
она совершенно не свойственно никаким образом но тем не менее видите вот ну как
бы что видно что вижу то и осуждаю и именно поэтому джордан обронов был конечно
подвергнут резкой критики когда он приехал в оксфорд и читал там лекции его
приняли очень плохо всего магией его после буквально просто изгнали и полетели
письма везде во всем инстанциях посмотрите чё эти католики творят в ответ на это
естественная инквизиция ну смысле папский престол крайне разозлился на самого
этого джордана потому что куда ты поперся этим протестантам поперся они там
заменьше что называется сжигают ну вот он историю вы уже помните протестанты
настолько были не как сказать непримиримы ко всем вот этим элементам магии что
действительно самые серьезные сожжения были не святой палаты осуществлены а все-
таки протестантами жгли как не в себя в 16 17 веках тогда вот эти вот когда
затем на говорят что инквизиция сожгла столько-то ведьм все до видим там тысячи
женщин сожгли в этот момент я думаю там ни один папа переносла к обо потому что
не инквизиция сжигал и ведь сжигали протестанты они враги понимаете это для нас
мы сегодня разницы нет те религии там или для людей которые долетят религии что
там что там но не важно а для них так представляете себе в мысли это же вообще
не мы сделали это бы сделали враги наши 

ну что еще нужно сказать о католите
простите о протестантизме ну надо конечно упомянуть кальвина кальвина жанна
кальвина почему потому что вот именно с ним связана этот этот тенденция
непримиримости протестантской непримиримости отвержение всех таинств и главное
вот это социальная жизнь то есть именно из его вот этих вот традиции понимание
сути реформаторства была отбожена вся мистическая часть религии она превратилась
просто в социальное учение то есть есть христианская община ну да мы вроде как
там что-то причащаемся но это ведь они даже и причащаются не смысле это таинство
не смысле это какая-то вот ядро христианство суть а в смысле просто ну как бы по
в память о тайной вечере то есть всякую мистику отбросили но религия без мистики
не живет и поэтому можно совершенно однозначно сказать что протестантские общины
они стали социальными социальными институтами в первую очередь во вторую
религиозными и сегодня это вот буквально по всему миру заметно лютеранской
общины это еще скажем так вот религиозное что-то а вот уже те которые
евангелические кальвини от кальвинистских вот таких вот мотивов это чисто
социальная группа это же однажды вот у нас в екатеринбурге тоже я одну девушку
спросила просто спасибо а как вы у вас там как причастие как она вообще ну я не
знаю а чё причастие просто собираемся того чтобы обсудить наши дела и помочь
друг другу ну вот у меня у кого нет квартиры кто-то поможет ее снять подешевле у
кого-то например там еще какие-то проблемы жизненные но кто-то может понимаете
это вот социальное единство а вот это это уже не важно ну я не говорю что тут я
сейчас не девушку меняет уж это тенденция общая которая началась скальпера
теперь давайте вернемся к науке наука это важно нас интересует она значит как же
там все таки было с этими возрожденческими тенденциями в северном возрождении не
скажу что совсем без алхимии но невозможно такой был не скажу что совсем без
магии тоже невозможно было и в качестве примера приведу конечно же парацельса
одним из один из самых знаменитых врачей и алхимиков но это вот чувствуется в
этом такая северная возрожденческая такая вот знаете прошивочка почему потому
что отталкиваясь от холостов говоря что мне вот ваша эта логическая мудрость не
интересно аристотель вообще ерунда я ведь пойду но он ведь он не кинулся к
платону платоником и стойком он как бы ни в сторону неоплатонизма кинулся он
пошел в народ он поехал путешествовать и начал интересоваться лечебными всякими
мероприятиями которые осуществляли люди в народе то есть всякие банщики эти
самые цирюльники и вот они тоже лечили людей он стал них узнавать хита методы
травницы бабки травницы вот у них и на самом деле стал хорошим врачом но как маг
как алхимик он хотел создать великую науку прославиться как великий алхимик из-
за этого он предположил ему создал концепцию что вот эти вот астральные это
элементы они связаны напрямую с химическими элементами то есть ртуть связано с
меккурием да сера связано а я не помню с кем связано села в общем он переписал
скажем так концепцию алхимии и начал лечить химическими элементами и включил то
с точки зрения понимаете астрологическое ну то есть условно говоря вот я
заболела до приходит ко мне пора цельсию не дай бог конечно такое было но
случилось вот приходит воспаление воспаление это много огня много огня это
соответственно марс надо минимизировать действие масса действия другой силы а
другая сила на меня связано с меркурием и курят артуть поэтому он дает мне ртуть
но как мы понимаем иногда помогала большинство умирала но помогал а почему все
таки помогала потому что он один из первых это все яд да это я но главная доза и
он вот это вот искусство фармацевтическое которое связано сегодня во первых с
химическими препаратами ведь понятно мы в аптеке купим травку но все-таки
современная фармацевтика это неорганические соединения но они все связаны с
дозой до обязательно искусство дозирования поэтому как ни крути мак он или не
мак алхимик не алхимик но парацельс заложил основания для фармацевтик для всей
фармацевтики и не органической медицины ну вот то что мы сегодня с вами покупаем
опять же хорошо это или плохо но скажу спасибо и низкий поклон парацельс
иронизирую но совсем чуть-чуть а все-таки для северного возрождения гораздо
важнее были интенции не магического действия не магического исследования природы
понимаете почему я надеюсь уже да во первых неприятие протестантами всего что
связано с мистикой таинствами тому подобное на во вторых просто видимо в силу
здравого смысла и в третьих но у них то это не было их как бы национальной
традиции ведь для как бы для итальянцев вот платона и их стоит это почти вот
такие родня почти а для северной для германии для какой-нибудь там или даже для
северной франции но это не так уж и прямо родное породнее можно найти 

поэтому на
северной европе гораздо полнее и масштабнее проявились идеи природознатцев я уже
про них говорила краешком буквально прошлась краешком внимание когда рассуждала
о средневековье кто такие природознации это те кого мы обычно не учитываем когда
рассуждаем о науке это люди которые не получили университетского образования но
они работали реально и много изобретали это ювелиры это красивщики одежды это
моряки это вот те же самые банщики цирюльники да они же все работают с
природными элементами и многие закономерности были ими хорошо изучены и мы о
природознатцев еще потом поговорим а сейчас я просто хочу одного из них
упомянуть это агрикола великий георгиус агрикола он написал 12 книг о металлах и
вот я просто здесь перечисляю потом обработки в смысле почитаете смотрите все
таки наук это не только физика химия биологии мы так привыкли когда наука физика
химия биологии но обратите внимание для современного состояния науки сколько
сказала инженерная дела на об этом мм-м и просто особенно сегодня вот цифровые
инженеры то что же инженеры но что же машины это знание машин а первые машины
это горные машины в первую очередь то есть если по большому счету посмотреть вот
так то основание современной науки заложила горное дело потому что физика химия
это вот туда это в астрологию тогда долго шло это ушло в магию они долгое время
занимались к и пневмы а горняки занимались непосредственно вот вот наша
реальность мы ей занимаемся то есть я очень поспорил что горное дело плетется
или там например рудознатство плетется в конце науки хвосте а впереди всей науки
в современном смысле этого слова физика не не не не если конечно вы не считаете
магию и наук синонимами поэтому особенно прошу вас обратить внимание на
природознатство и вот пример очень такой показательный это георгиус агрикова
очень образованный человек но он сознательно говорит будем заниматься заниматься
делом и он действительно очень хорошим зла делал он не просто ходил там этими
машинами занимался большими он смотрел как люди на них работают он видел как
нужно улучшить условия труда рабочих как много заболеваний связаны с
производством вредным и он первый начал исследовать эту тему и незаслуженно он
низко ставится в противовес тому же парацельсу например или тому же я не знаю
какому-нибудь там ну кому еще в общем обратите внимание пожалуй теперь ну
гуманитарная мысль то есть гуманисты тоже были гуманисты в северном возрождении
гуманисты конечно же именно в северном возрождении связаны с протестантизмом
понимаете да основная тема на тот период для них это богословие протестантское
это философские основания каких-то там сложных богословских вопросов и так далее
и вот здесь можно упомянуть филиппа милптона конечно у него была немецкая
фамилия черная земля да это швактный швакцин то есть это филипп швакцин но тогда
была мода они же гуманисты все равно нужно перевести свои фамилии на греческий
язык и взять эту новую фамилию я кстати подозреваю что моя собственная фамилия
это тоже был когда-то перевод какой-то русской фамилии простой русской фамилии
на греческой а если обол это мелкая греческая монета то соответственно это будет
по-русски это что-то вроде копейкина или или как грошева вот так вот если была
такая мода но я немножечко знаю этой фамилии это не моя мужа фамилия но неважно
я знаю немножко об этой семье там действительно были дворяне и поэтому
подозреваю что они все время там что такое же удумали были какие-нибудь просто
копейкина но не важно так вот ладно вот смотрите значит филипп не ухтон философ
гуманист о чем он занимается перевод из греческого и так далее но что он
переводит если господа итальянские возрожденцы итальянские гуманисты переводили
гермеса трисмегиста и других вот этих вот магов то он переводит святых отцов
патристику он переводит святого фанасия великого переводит василия великого то
есть вот эти греческие отцы церкви 4 века 4 5 века и надо сказать что делает он
это сознательно пытаясь объединить протестантизм и православие долгое время
считалось что ну им стоит объединиться но силу того что все таки протестантизм
пошел просто в сторону социальной этической вот такой вот доминанты конечно
православные ну вот как бы просто православные мыслители из как с которыми
происходил диалог они приветствовали вот это возвращение истинных этих вот ну
публикации на и на текстов отцов церкви древности да но они не могли принять вот
эту позицию такого исключения всякой мистики всякой то всякого всяких всех
таинств из церкви ну а как тогда все теряется понимаете да то есть вот этот союз
он как бы сближение было но разошлось еще на лотон очень известен тем что
возможно он есть основоположник того что сегодня называется вера терпимостью он
очень много он вообще очень хотел чтобы католики протестанты помирились все это
прекратилось очень понятное желание и он под это подводил все-таки философские
основания писал тексты в которых доказывалось что истинная христианская позиция
это вера терпимость чувствуете да то есть по христиански терпеть то с чем ты не
согласен христианский терпеть мусульманство они не кричать что вы там все там
или там пытаться сходить и убить там побольше и так далее и мы должны быть
истинными христианами а стало быть примириться с инакомыслящими но тем не менее
все-таки очень многое он отвергал будучи протестантом например гелиоцентризм
коперника он отвергал именно как вот как сказать как вот эту вот идею в которой
он видел магизм тот самый он как бы не да не слишком хорошо знал коперника он
просто прочитал там все-таки гимн во славу солнца божественности солнца мы все
его это отвратило но как математика как математические как бы гений который
коперник продемонстрировал да вот это он одобрял вот таким образом 

ну и еще один
известный очень человек очень важные для северного возрождения раз в
роттердамский он выступал против древнего благочестия наверное яростней всех а
древнее благочестие то есть имя герметизма однажды его кто-то пытаясь уважать он
был очень известен назвал его разман трижды величайшим имеется ввиду что вы
чувствуете аналогию герметрижды величайший разум и что величайший разум нас он
был очень терпи таким терпеливым и спокойным человеком он сплел и даже что-то
там не увечья нанес то есть его посмели оскорбить сравнением с этим вот термесом
трисмегистом вот он призывал вернуться к отцам церкви переводил также и она
златоуста герой аназианзина и других патристов но в общем то вот он очень близок
в этом смысле меланхтону то есть сочувствовал говорил да церковь нужно
реформировать но все-таки вот эти религиозные политические распри он никак не
мог поощрять его самая известная книга это похвала глупости то есть вот она тоже
этика терпимости этика терпимости которая тоже еще одна традиция европейской
северной такой вот североевропейской мысли и еще секундочку ты еще когда я был
североевропейской я подразумеваю все таки несколько генамию и германию север
франции то есть вот эти протестантские страны но он усилил вот этот
рационалистический подход к религии усилил его и вот в итоге как бы но это вот
как бы да это еще один путь к секуляризму то есть конечно разом этого не хотела
бы истинно верующим человеком но тем не менее он заложил это снова заключение о
северном возрождении я вам покажу несколько образцов искусства чувствуете
различие да вот реально очень отличается от южного возрождения но не менее
прекрасно замечательная работа босха несения креста просто прочувствуете идею
вот он пишет вне он изображает то как за человеческими самыми разными порогами
скрывается собственно приятного почти по почти вот как бы все его уже закрыли
собственно сама суть то вроде все вместе с христом идут в одну сторону
чувствуете не с христом ну и конечно главное место тут занимает в качестве
какого-то уродца кто вот это вот кто да тут самый маг и этот волшебник вот то
самое итальянское возрождение которое они резко критиковали питер брегель
старший великолепный художник сейчас кстати выставка и босха и брегеля но их
копий ночь хороших копий происходит в гостинице сесть вот если если мы лет уже
нет или закончился не знаю я уходила на не очень замечательно в смысле
замечательно видно почему потому что на любой иллюстрации не видно детали а у
них такая детализации так много смысла понанесет просто поразительно ну вот
смотрите несение креста вы видите где этот крест ген нанесения это специфика
именно северного возрождения показывать как религиозное погружено в реальную
жизнь и нужно особый талант чтобы вычленить из просто жизни то что самое важное
но здесь на переднем плане богоматик с учениками которые вот оплакивают да
ситуацию события но самого христа чувствовать очень трудно увидеть ну и конечно
же знаменитая меланхолия дюрера о которой можно говорить три часа той
дополнительной лекции и вы увидите здесь все вы увидите здесь и разочарование то
есть вот этот груз на ангел от мистики а это вот пифагорейский квадрат дальше вы
увидите здесь все все элементы герметизма в полной мере и вот эта задумчивость и
почему меланхолия потому что меланхолия это тоже одно из заболеваний которое
увеличилось и активно было распространено и как бы она была связана с духом
времени наступала наступала время черной печали черной тоски и вот дюрер его
уловил она будет в новое время но дюрер уловила показал 

\section{Формирование европейской социально-политической мысли в эпоху Возрождения}

Республика.

надо понимать, что основы этого строя были заложены в трудах мыслителей
возрождения; в трудах тех, кого мы называем представителями гражданского гуманизма.
То есть главное — это принцип общего блага принцип общего блага чести они видели
где республики общее дело дело народа то есть форма государственного управления
готовность ли и выбирает те органы которые будут осуществлять власть республики
могут быть разные но самое главное что вот они это форма народного управления
долго дело сонам понимаем бывает олигархические республики как та же самая
флоренца была олигархическая республика в этой республике монархически даже
когда парламент это республиканское форма управления а вот есть еще и монархии
но самое главное что здесь дело в том что по социально-политическому мысли мысли
пошла по пути идеализации идеализации то есть они взяли за основу вот этот
порядке в итальянских городах коммунах которые были далеки от идеального там
какого-то состояния но мыслители решили что ну да это как мне не полноценное
воплощение но если вы еще немного подумаем мы напишем некое идеальное состояние
а потом выстроим какую-то социальную общность на основе вот этой вот улучшенной
идеи это огромное заблуждение всей европейской мысли что можно выстроить
реальность согласие с идеей я практически уверена что вот это вот заблуждение
которое легло в основу хорошей по сути дела задачи задача хорошая лучше
социальную жизнь с помощью размышлений с помощью научного на но метод был избран
идеализации то есть превращение реальности в идею а потом попытка идеи воплотить
в реальность а так это не получается и конечно же в обращаем как бы в попытке
вот это вот создать идеальные образы государственного управления идеальных
республик они обращаются к иистотелю платону к цицерону и так далее ну надо
сказать что гораздо больше северное возрождение обращается к алистотелю и
цицерону ну то есть к не платону вот так вот скажем уже платон для них это вот
так это вот не был тонизм это магия это вот там и в основном они предпочитают
конечно же северное выживание поняли да идеализации там подобное 

единственный
кто противостоит этой тенденции никола макеавель его очень долго и много ругают
зря вот сегодня это видно со всей отчетливостью значит никола макеавель историк
государственной деятель которую по праву называют главным политическим
мыслителем этой эпохи и чё только про него не не не говорили он значит работал с
цезарем борджи работал на него борджи это один из ну вот этих олигархов которые
в другом городе италии правил он был конечно очень таким тяжелым правителем
весьма развращен весьма жестоким но тем не менее успешным на него кстати на
борджио же работал леонардо да винчи известный и он работал на него в качестве
военного инженера и самым самым лучшим изобретением в сути единственным
изобретением успешным леонардо да винчи был колесовый замок то есть вот это вот
механическая часть пистолета которая возвращает возвратный механизм и он
произвел революцию просто революцию но не в военном деле в начале а произвел
революцию в городской по городском бандитизме поэтому леонардо да винчи такое
макеавели около года находился при дворе цезаре борджи и был принцип который
вольно толкует как цель оправдывает средства но нет такого для нам у макеавели
он пишет другое пишет всегда о том что определенная цель требует определенных
средств чувствуете совсем другое то есть средства должны соответствовать цели а
какая цель у государства и вот он впервые говорит реальный он заложил основы
реал политики реальная политика то есть политика которая выстраивает государства
какую-то социальную общность не из красивой или хорошей правильной идеи а исходя
из реальности и он совершенно справедливо говорит что в реальности работы только
причинно-следственной цепочки ну да мы конечно люди станут со свободной воли там
очень важно для нас но вот все так же работает как знаете как вода течет по
самому пути наименьшую сопротивление и вы не сможете это поменять поэтому ваша
задача познать эти русла а если вы и покладываете новые то покладывать их
законами причинно-следственных развития событий вот что важно он раздел то что
вы не из-за личного опыта политика рождается не от красивых и правильных идей но
от столкновения интересов и он пишет он создает концепцию государственного
интерес хорошая политика возможно как власть одного говорит он но того кто
подходит для этой деятельности цезарь боджио был конечно в этом смысле силен
весьма силен но не идеален великие политики говорит он те кто обладает великим
вирту сам опять виртуальность опять этот самый виртуз политика не мериться
добродетелю говорит он мы святы не обязательно хорошей политики политика это
дело тех кто умеет как сказать работать с реальностью выкапывать новые русла для
реальных процессов контролировать не противостоять невозможно вы не можете
противостоять причин с этим с почком вы можете их перенаправлять для макиавелли
вирту это тоже сражение с фортуны естественно и этого умения обращать
обстоятельства в свою пользу к галлипу он рассматривает законы истории как
неизбежный ряд причин наследственных цепочек и необходимость совпасть с
реальностью совпасть с реальностью совпасть и оказаться вот ситуации когда
разнонаправленные векторы интересов ты можешь ими управлять это великий виртуз
поэтому говорит он для государя нет иных моральных ориентиров кроме интересов
государства поэтому он предлагает кстати но чтобы религия небольшой такой скажем
поклонник религиозности он говорит но и надо оставить для народа а вот государь
может быть свободе должен даже уже не личность он должен быть свободен даже от
христианской морали он уже не личность чувствуете вот такая реальность он
государственный интерес он воплощение государственного интереса макеавелли
конечно циничен с этим не приходится спорить но в отличие от макеавелли
политическая мысль остальная очень часто забывает про то что реальность не
подчиняется идее и поэтому она либо часто обманывают либо обманывается 

и вот
одна из форм такого самообмана или обманом не знают это утопизм утопия мы знаем
это слово у слова утопия вместо которого нет это литературный жанр описывающий
какое-то идеальное общество но не дающий как бы обоснование как это общество
будет работать понятие возникло из произведения томаса мора знаменитый
английский общественный деятель который не принял англиканство имеется ввиду
когда генрих 3 очень захотел жениться на не болея на знаете эту историю и вот в
результате несогласия папа разорвалась папой и ввел англиканство но томас мор ну
не смог пресекнуть ему как владыке церкви англиканской церкви ну и
соответственно было безглавлен это действительно героическая смерть ее почему-то
спорят гораздо реже чем героическую смерть этого самого джордана бурна хотя она
была действительно в ней больше смысла что и было но видимо дело в том что его
то принципы были связаны не с верностью магии а верностью католичества была
издана его книжечка 1516 году а почему книжечка нужно назвать золотая книжечка
столь же полезная сколь и забавная и в ней написывают утопию особо такой город в
котором как бы ну вот все хорошо вот топе я частной собственности есть
общественное производство по типу семейного 6000 семей семьи от 10 до 16
взрослых я сейчас могу тут много перечислять дело не в этом самое главное что
чему посвящена утопия это делу социального равенства социальное равенство
главный главный принцип говорит он идеальное государство и вот тут начинается
интересные вещи равенство то какой арифметическое или геометрической потому что
это как раз два ну понятия которые все мыслители различали всегда со времен
античности есть равенство геометрическое то есть по достоинству пропорционально
пропорционально достоинству вот ты получаешь соответственно столько-то ресурсов
только-то благ а если арифметическое просто поровну поделить и томас мор все-
таки говорил о том что социальное равенство основано на принципе геометрического
то есть по заслугам чувствуете но не появляется здесь никакого все живут
одинаково тогда как другие концепции тоже используют принцип равенства и кричат
о том что все должно быть равенство и предлагает помните дальше аликова взять
все и определить арифметические принцип правенства и отсюда вы должны понимать
когда кто-то кричит о равенстве а политика очень часто кричит о равенстве надо
всем испрашивать какое вы имеете ввиду вот томас мор определенное это показал
этим цена его работу но получилось ли у нового агксовать идеальное государство
но мы сегодня считаем что это антиутопия да вот эти вот фильмы которые они
только книги они очень похожи 

но еще больше в этом в эту лепту внес конечно тамаза
компонелла с его городом солнца интересный такой персонаж истории он был монахом
общественным деятелем но за подготовку заговора в очередной там с очередными
элигархическим правлением 27 лет провел тюрьме и вот в начале самом начале 17
века он пишет трактат город солнца в нем дается картина которая потом легла в
основу коммунизма принципа коммунизма еще одна хорошей идеей на которую надо
воплотить было жизнь срочно вот что это за идея идея значит идеального
устройства за счет тотальной симметрии и стандартизации всего и вся солярий это
жители города солнце имеет один рост одинаковую одежду я не знаю как это
остается добиться одного роста в определенное время суток определенного цвета за
исключением черного цвет должен был как бы выбираться соответствии с
астрологическими данными глобальный коллективизм человек всегда в отряде солярии
не живут поодиночке казняться преступников тоже вместе забивают камнями общий
жены люди соединяются ну в сексуальном смысле сообразно астрологическим
рекомендациями конечно их соединяет никакая там любовь а правитель правитель
говорит как кому и с кем для чего чтобы было правильное потомство чтобы толстые
соединялись худыми получалось что среднее высокие с низкими от в мой один рост
наверно вот никаких там особых чувств естественно но это идеальный город в
результате получается дети не знают своих родителей они воспитываются
государственных учреждениях городом управляет метафизик и он нарекает кстати
каждого ребенка по имени эта должность переходе к тому кто оказывается самым
мудрым самым способным к проблеме это конечно метафизик как мы понимаем речь
идет о каком-то идеальном обществе это опять антиутопия но почему это все время
позиционируются самими авторами как идеальное общество задаем вопрос который
отвечать на ответить на которую совсем не сложно они создают картины идеального
общества и это идеальное с точки зрения тех тем управляют а с точки зрения того
кто управляет то если они пишут свои трактаты для того кто соберется управлять
общество они подсказывают удобнее всего управлять теми кто стандартизова кто вот
вот это вот это вот это вот это вот это вот все чувствуете То есть очень
странно, когда люди, которые, ну понимаете, не собираются быть правителями,
подключаются к этому общественному какому-то движению и бегут, и рискуют своей
жизнью, или даже кладут на алтарь своей жизни, на алтарь утопий, которые даже
созданы не для того, чтобы они хорошо жили, а просто чтобы было удобно править.
Это удивительно. По какому-то странному выверту сознания принципы, описанные в
утопиях, вот это коммунистические, социалистические, какие-то там социально-
нациалистические, много таких, они оказываются идеями, которые вдохновляют
народные массы. Массам-то зачем это? Ведь это же не для них написано, не для их
жизни, это написано для правителей, чтобы управлять. Не иначе, как
пневматическая магия какая-то, а? Не кажется вам? Ну я, конечно, шучу, но
знаете, в каждой шутке. Ну а непосредственно исследователи Кампанелло знают, что
он жил в существовании этой утопии и видел себя правителем города. Ну, конечно.
Да, он сидел в тюрьме, но он был астрологом, и он рассчитал, что в 34 года будет
Великая конюнция планет 24 декабря 1603 года. И, ну, по его натальной карте
выходило, что он будет освобожден в этот день, и все, и вот он возглавит, и все
будет, как бы он станет этим великим метафизиком. Но не получилось. Заканчиваю
нашу лекцию. 
