\section{Различные типы знания. Специфика научного знания и его критерии}  

\subsection{Сущность знания и познания}

\subsubsection{Понятие знания}
 
Знание – это \textit{проверенный практикой} результат познавательной деятельности, 
форма социальной и индивидуальной памяти. Знание выступает в виде усвоенных понятий, законов,
принципов и так далее.  

Знание можно разделить на две большие группы: 
\begin{itemize}
    \item \textit{знание-умение} (практическое знание, «знание как»). Имеет специфическую проверку (н-р, можно уметь или не уметь кататься на велосипеде, но нельзя
кататься на велосипеде истинно или ложно);
    \item \textit{знание-информация} ("знание что") --- о наличии у предметов каких-то свойств, знание о закономерностях, знание о последовательности действий в той или иной ситуации. Для него характерна проверка на истинность.  
\end{itemize}


Любое знание нуждается в
объективации. В продуктах труда, в технологиях, в социальных институтах, в
произведениях искусства, в текстах. Даже когда мы говорим об индивидуальном
знании, то, которое содержится только в нашей памяти, оно объективируется там в
понятиях, в образах, в воспоминаниях и в других формах.  После своей
объективации знание может быть передано другому человеку.  

Субъект знает некий предмет, если соблюдаются следующие условия:
\begin{itemize}
    \item условие истинности предмета;
    \item условие убежденности/приемлемости;
    \item условие обоснованности.
\end{itemize}


\subsubsection{Понятие познания}

\textit{Познавательная
деятельность} --- это сознательная деятельность субъекта, направленная на
приобретение информации об объектах и явлениях реальной действительности. Это
единство чувственного восприятия, теоретического мышления и практической
деятельности.  

В результате познания формируется
познавательный образ --- образ, возникающий в сознании при непосредственном или
опосредованном взаимодействии субъекта с объектом.  

Познавательные образы делятся на: 

\begin{itemize}
    \item чувственные образы:
    \begin{itemize}
        \item элементарные (ощущение, восприятие и представление);
        \item синтетические (чувственные образы, стереотипы, художественные образы, модельные представления);
    \end{itemize}
    \item рациональные образы (теоретическое восприятие особенностей класса предметов 
через систему всеобщих и необходимых свойств этого класса, выраженных
в понятии. ):
    \begin{itemize}
        \item элементарные (понятия, термины);
        \item синтетические (совокупность понятий, связанных 
единым смыслом или единой политикой их применения в теории или в гипотезе.)
    \end{itemize}
\end{itemize}

\subsection{Особенности научного знания} 

\subsubsection{Системность} 

Наука --- это иерархически организованная система знаний, которая делится на отдельные дисциплины. Изменение значимого элемента в этой системе приводит к постепенному пересмотру всей структуры научного знания (например, переход от физики Ньютона к физике Эйнштейна).

\subsubsection{Объективность} 

Наука стремится изучать предметы и явления так, как они существуют в объективной
реальности. Поэтому знание можно проверять на практике, в эксперименте, в
наблюдении и другими способами. 

\subsubsection{Культурно-историческая обусловленность}

Наука зависит от периода и общества, в котором она развивается. Общие характеристики науки существуют, но они абстрактны. Наука исторически меняется, и современная её версия сильно отличается от науки прошедших эпох, несмотря на общую основу.

\subsubsection{Всеобщность}

Научное знание изучает общие законы и свойства предметов, стремясь формулировать универсальные законы, применимые в любой точке Вселенной. (Хотя и существует гипотеза о возможных исключениях во Вселенной, она остается неподтвержденной.)

\subsubsection{Необходимость}

В научном знании фиксируются системообразующие стороны явлений. 

\subsubsection{Дисциплинарная принадлежность}

Научное знание можно распределить по различным дисциплинам, что представляет собой горизонтальную структуру. Вертикальная структура связана с иерархией уровней. Междисциплинарное знание возникает на стыке дисциплин, но часто приводит к созданию новых дисциплин, таких как физическая химия или эволюционная психология.

\subsubsection{Подтверждение научным сообществом}

В 17-18 веках, когда институт науки только формировался, распространенной практикой было приглашение ученых или уважаемых людей для демонстрации экспериментов, чтобы подтвердить результаты. Если эксперимент не удавался, ученый мог быть подвергнут остаркизму. Сегодня существуют институциональные механизмы подтверждения научных результатов, такие как рецензирование и редактирование статей, а также защита диссертаций, которые обеспечивают коллективное признание научных достижений.

\subsection{Научные конструкты и требования к ним}

Объект
исследования, который чаще всего наука изучает, является не самим объектом, а
является конструктом.
\textit{Научный конструкт} --- это умозрительное построение, вводимое гипотетически
и создаваемое по правилам логики. 

К ним выдвигаются следующие требования:
\begin{itemize}
    \item возможность логических операций над конструктами как языковыми выражениями;
    \item множественность связей между конструктами в рамках некоего целого;
    \item устойчивость конструктов (то есть постоянство значений в различных контекстах);
    \item экстраполируемость конструктов (то есть возможности их максимально широкого использования помимо породивших их ситуаций);
    \item согласованность выражений конструктов с установленными закономерностями;
    \item простота конструктов.
\end{itemize}

\subsection{Сравнительная характеристика
научного и обыденного знания}

\begin{longtable}{|>{\raggedright\arraybackslash}p{7cm}|>{\raggedright\arraybackslash}p{7cm}|}
\hline
\textbf{Обыденное знание} & \textbf{Научное знание} \\
\hline
\endfirsthead

\hline
\textbf{Обыденное знание} & \textbf{Научное знание} \\
\hline
\endhead

непрофессиональное, неспециализированное жизненно-практическое, повседневное знание & Продукт специализированной, профессиональной формы человеческой деятельности \\
\hline
не имеет строгого концептуального, логического, системного оформления. & носит теоретический, концептуальный характер; отличается системной организацией \\
\hline
не требует для своего усвоения специального обучения и подготовки & требуется специальное обучение для овладения ими\\
\hline
констатация явлений, связей и отношений & ориентировано на поиск закономерностей \\
\hline
может иметь субъективный характер. & в идеале должно быть объективным, доказательным, точным. \\
\hline
предмет всегда нагляден и доступен восприятию & включает в себя системы абстрактных объектов. \\
\hline
дается преимущественно как типичное, понимаемое по аналогии. & имеет творческий характер\\
\hline
осуществляется познание единичных ситуаций и явлений & носит универсальный характер. \\
\hline
решает конкретные, сиюминутные жизненные проблемы & выходит за рамки практической заинтересованности \\
\hline
спользует для фиксации знания устную разговорную традицию &  использует письменную традицию \\
\hline
\end{longtable}


Проблема демаркации научного и обыденного знания остаётся нерешённой, потому что обыденное знание часто предшествует научному. Наука возникает на базе первичных представлений о свойствах объектов, с которыми человек взаимодействует в повседневности. Кроме того, наука использует естественный язык, схожий с тем, который применяется в обыденном знании, хотя стремится к большей формализации. В истории были попытки создать строго формализованный научный язык, свободный от недостатков естественного, но полностью это не удалось. Исключение составляет, возможно, математика, которая может быть рассмотрена как формализованный язык. Научный язык и язык обыденного знания связаны через общие понятия и вложенные в них смыслы.



\section{Основные классы научного знания и их дисциплинарная организация} 

\subsection{Проблема классификации науки}

Научное знание может быть классифицировано на определенные
элементы. Проблема классификации наук заключается в том, что в
результате этой классификации необходимо раскрыть взаимосвязи между элементами
науки и выработать представление о ее некой целостной структуре. 

В истории философии были неоднократные попытки классифицировать научное знание. 
Рассмотрим некоторые из них.

\subsubsection{Классификация Аристотеля (384-322 гг. до н.э.)}

Аристотель делил знание на: 

\begin{itemize}
    \item \textit{теоретическое}, где познавательная деятельность ведется ради самого познанияю. И философия, и наука относятся к теоретическому знанию, занимаясь познавание из
    отстраненной позиции, пытаясь созерцать явления во всей полноте. 
    \item \textit{практическое}, где познание ведется для выработки идей для поведения человека; К практическому знанию относятся те отрасли, которые
    вырабатывают определенные принципы поведения человека в жизни (повседневной, политической, и т.д.). Например, этика и политика.
    \item \textit{творческое}, где познание осуществляется для достижения чего-либо
    прекрасного. К творческому знанию относятся виды деятельности, направленные на
получение чего-то прекрасного. Например, риторика, поэтика, живопись.
\end{itemize}

В основании классификации положен принцип цели
познавательной деятельности. Данная классификация не приближает к
пониманию специфики науки, определенного класса знания. 

\subsubsection{Классификация Фрэнсиса Бэкона (1561-1626 гг.)}

Науки делятся на: 
\begin{itemize}
    \item \textit{исторические} --- описание фактов (как гражданских, так и естественных);
    \item \textit{теоретические}, или философия в широком смысле слова;
    \item \textit{поэтические} --- литература, искусство, поэзия. 
\end{itemize}

В основании классификации Бекона лежит принцип опоры на определенные
интеллектуальные способности: память, разум или воображение. 
При выборе интеллектуальной способности возникает
класс научного знания. Однако внутри класса трудно провести дальнейшую
спецификацию. 

\subsubsection{Классификация Георга Гегеля (1770-1831 гг.)}

Исходя из своей системы диалектики, он делит знания на три раздела: 
\begin{itemize}
    \item \textit{логика}: учение о бытии, учение о сущности и учение о понятии;
    \item \textit{философия природы}: механика, физика и органическая физика;  
    \item \textit{философия духа}: изучение субъективного духа (человека), объективного духа (общества) и абсолютного духа (Бога). 
\end{itemize}

Система является стройной, и внутреннее деление на дисциплины близко к современной классификации. 

\subsubsection{Классификация Огюста Конта (1798-1857 гг.)}

Им предложена следующая классификация:

\begin{itemize}
    \item математика (вкл. механику);
    \item астрономия;
    \item физика;
    \item химия;
    \item биология;
    \item социология.
\end{itemize}
При движении от математики к социологии
увеличивается сложность предмета исследования: геометрические фигуры, числа, их
взаимодействие наиболее простой предмет исследования, а социальные процессы ---
наиболее сложные. 

В обратную сторону происходит
увеличение абстрактности предмета: у социологии предельно конкретный предмет
исследования, у математики --- предельно абстрактный. 

Предметы исследования в данной классификации достаточно изолированы друг от друга. 

\subsubsection {Классификация Фридриха Энгельса (1820-1895 гг.)} 

\begin{itemize}
    \item механика;
    \item физика;
    \item химия; 
    \item биология; 
    \item социология. 
\end{itemize}

Предметы не изолированы, а
дисциплины означают диалектический переход от одной формы движения материи к другой.
Т.о., социальная форма движения
материи включает в себя остальные формы движения материи, но ими
не определяется по своей природе. 


\subsubsection{Классификация В.И. Вернадского}

Вернадский разделял науки на две категории:
\begin{itemize}
    \item науки, законы которых охватывают всю реальность;
    \item науки, законы которых свойственны только Земле.
\end{itemize}

Эта классификация имеет практическую ценность в рамках концепции Вернадского о развитии планеты и общества. Для Земли характерны уникальные явления, такие как жизнь и разумный человек, которые не найдены в других частях Вселенной. Поэтому науки, изучающие человека и общество, являются уникальными для Земли. Напротив, такие науки, как физика и химия, охватывают всю реальность. 

\subsubsection{Современная классификация наук} 

В современности деление происходит по трем аспектам: предмет исследования, особенности методологии и критерии научности в данной отрасли.
\begin{itemize}
    \item математические науки;
    \item естествознание;
    \item гуманитарные науки;
    \item социальные науки;
    \item технические науки.
\end{itemize}
 
\subsection{Фундаментальное и прикладное научное знание} 

Разделение наук на фундаментальные и прикладные --- еще один вариант классификации, который широко используется, в том числе для повседневного описания науки. 

\subsubsection{Фундаментальная наука}

Это область научного познания, включающая теоретические и экспериментальные исследования
основополагающих явлений. 

Фундаментальная наука характеризуется концептуальной универсальностью и всеобщностью во времени и пространстве. Ее выводы применимы ко всем системам и условиям в любой точке Вселенной и касаются основополагающих явлений, таких как строение материи, энергия, фундаментальные взаимодействия и эволюция Вселенной. 

\subsubsection{Прикладная наука}  

Направлена на интеллектуальное обеспечение инновационного процесса, как основы развития современной цивилизации. Прикладная наука обеспечивает конкурентное преимущество.

Прикладные научные исследования начали развиваться в 19 веке благодаря деятельности Юлиуса фон Либиха, который систематически использовал науку в интересах промышленности, начиная с химической отрасли. 

\subsection{Дисциплинарная структура науки}

Научная дисциплина --- это форма систематизации научного знания.
Она формируется на основе общности объекта исследования, методов, идеалов и норм исследований.  

Для появления научной дисциплины важно не только осознание общности вот этих
элементов, но и появление институциональных форм существования этой научной
дисциплины. 
Кроме этого, в дисциплинарной структуре науки постоянно происходит интеграция и
дифференциация научных дисциплин. 
Дифференциация подразумевает собой появление дисциплин с все более узким предметом
исследования. 
Интеграция, наоборот, означает то, что предмет исследования
расширяется.


\section{Уровни научного познания и соответствующие им методы и формы знания}
 
\subsection{Эмпирический уровень научного познания}

В результате непосредственного или опосредованного контакта с
предметом исследования ученые получают знания об определенных событиях и явлениях. 

\subsubsection{Формы эмпирического знания}

\texttt{Научный факт} --- это форма эмпирического научного знания, в которой фиксируется
некоторое конкретное явление или событие. 
Чтобы факт мог называться научным, необходимо соблюсти следующие требования:
\begin{itemize}
    \item отношение к определенной предметной области науки; 
    \item содержательное описание процедуры и  обстоятельств фиксации события;
    \item усредненность результатов наблюдений и измерений; 
    \item воспроизводимость в научной деятельности  других  исследователей;  
    \item соотношение  с  некоторой  совокупностью, системой родственных или схожих фактов;
    \item теоретическая нагруженность (понятия, с помощью
    которых факт формулируется, соотносятся с определенной научной теорией, и в
    рамках этой теории обладают определенным смыслом).
\end{itemize}

Научный факт имеет определенную структуру, состоящую из трех
компонентов:
\begin{itemize}
    \item \textit{перцептивный} компонент --- это чувственный образ, который возникает в
    результате восприятия явления;
    \item \textit{материально-практический} компонент --- это
    совокупность приборов, инструментов и действий с ними, используемых для
    установления факта;
    \item \textit{лингвистический} компонент --- это высказывание, формулирующее факт.
\end{itemize}


\texttt{Эмпирическое обобщение} --- это форма знания, которая фиксирует в
себе некую внешне проявляющуюся причинно-следственную связь. Например, если мы 10 раз уронили бутерброд с маслом, и он каждый раз падал маслом вниз, мы можем сформулировать эмпирическое обобщение, что бутерброды с маслом, вероятно, будут падать маслом вниз. Причины этого явления нам пока неизвестны, но мы зафиксировали внешнее проявление причинно-следственной связи: уроненный бутерброд всегда падал маслом вниз.

\subsubsection{Методы эмпирического знания}

Среди эмпирических методов выделяют наблюдение, эксперимент, сравнение, описание и измерение.

\textit{Эксперимент} --- это метод эмпирического научного
исследования, который подразумевает активное и целенаправленное вмешательство в
протекание изучаемого процесса, в том числе в специально созданных и
контролируемых условиях. 

Научное \textit{наблюдение} предполагает фиксацию некого внешнего изменения того или иного процесса. 

\subsection{Теоретический уровень научного познания}

Теоретическое познание описывает идеальные объекты,
которые, в отличие от реальных объектов, характеризуются конечным числом свойств. 

\subsubsection{Формы эмпирического знания}

К ним относятся проблема, гипотеза, закон и теория.

\texttt{Научная теория} --- это форма теоретического научного знания, содержащая целостное
отображение закономерных и существенных связей определенной области действительности.

Научная теория представляет собой высокий уровень организации научного знания, \textit{включая гипотезы, концепции, закономерности и законы}, которые составляют теорию. Теория --- это квинтэссенция теоретического знания, объясняющая явления в широкой предметной области. 

Например, теория всемирного тяготения не ограничивается падением яблока, а охватывает общие принципы взаимодействия материи. Теория эволюции описывает общий процесс эволюции, а не изменения отдельных видов животных.

Компонентами научной теории являются: 
\begin{itemize}
    \item исходные основания --- фундаментальные понятия, принципы, законы, аксиомы,
    которые являются отправной точкой теоретического исследования объекта;
    \item идеализированный объект --- модель существенных свойств и связи
    изучаемых предметов;
    \item логика теории --- совокупность определенных правил и способов доказательства;
    \item философские установки, социокультурные и ценностные факторы (н-р, элементы картины мира самого ученого);
    \item совокупность утверждений, выводимых из этой теории в качестве следствий.
\end{itemize}

\subsubsection{Методы теоретического знания}

Среди теоретических методов выделяют формализацию, аксиоматический подход, гипотетико-дедуктивный метод, обобщение и идеализацию. 

\textit{Гипотетико-дедуктивный метод} заключается в создании системы дедуктивно связанных между
собой гипотез, из которых выводится утверждение о возможных эмпирических фактах. Далее, в эксперименте или в наблюдении отыскиваются спрогнозированные факты. Если они
обнаружены, то гипотеза косвенно подтверждается.  

\subsection{Метатеоретический уровень научного познания}

На метатеоретическом уровне научного познания вырабатываются идеи и принципы, 
с помощью которых происходит обоснование представлений научной картины мира. 
Они служат одним из условий включения научных знаний в культуру соответствующей 
исторической эпохи. 

\subsubsection{Формы метатеоретического знания}

К ним относятся научная картина мира, научная парадигма, общенаучные принципы и философские основания науки.

\texttt{Философские основания науки} --- это
система философских идей, которые используются для обоснования различных
аспектов научной деятельности. Рассмотрим некоторые основания науки:

\begin{itemize}
    \item \textit{онтологические} --- принятые в данной науке представления о строении окружающего мира;
    \item \textit{гносеологические} --- содержат представление о критериях истинности знания;
    \item \textit{логические} --- содержат представление о той логике, которой нужно оперировать для доказательства;
    \item \textit{аксиологические} --- содержат представление о главных ценностях науки (н-р, получение нового знания, пользы для человека, и т.п.);
    \item {методологические} --- представление о том, с помощью какой методологии мы можем получать истинное научное знание (эксперимент, математическое моделирование, и т.п.).
\end{itemize}

\subsubsection{Методы метатеоретического знания}

К ним относятся методологическая рефлексия и философские методы.

\textit{Философские методы} --- методы познания, которые являются сугубо философскими. 
Например, диалектический метод --- метод рассмотрения
развивающихся явлений через диалектическую логику единства и борьбы
противоположностей, перехода количества в качество и отрицания отрицания.

\textit{Методологическая рефлексия} --- это исследование оснований собственной научной
деятельности с точки зрения общих принципов познавательного процесса. 